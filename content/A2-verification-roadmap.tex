\label{app:verification}

\subsection{Notes Directory Structure and Proof References}

Complete proofs for all major theorems are provided in the supplementary ``notes/'' directory, organized by component:

\begin{description}

\item[\textbf{Component 1: Operator Existence}] See notes/proofB*.tex, proofC*.tex, proofD*.tex for detailed spectral analysis of the Hessian, channel decomposition, and weight determination via Banach fixed-point theorem.

\item[\textbf{Component 2: Spectral Encoding}] See notes/subsectionN*.tex (22 detailed files) for complete trace formula derivations, heat kernel calculations, Riemann explicit formula, and Dirichlet series uniqueness proofs.

\item[\textbf{Component 3: Critical-Line Concentration}] See notes/subsectionN2PhaseTransitionsAndConsistency.tex for divergence-potential analysis, large-deviation principle application, and measure concentration bounds.

\item[\textbf{Component 4: Osterwalder-Schrader Positivity}] See notes/proofM*.tex for complete OS-axiom verification, path integral representations, and reflection-positivity arguments.

\item[\textbf{Component 5: Synthesis}] See notes/subsectionU*.tex for final integration of all components and RH proof completion.

\end{description}

\subsection{Three Categories of Results}

The proof architecture spans three tiers of mathematical rigor:

\begin{description}

\item[Tier 1: Routine Application of Established Theorems] These use well-known results with published proofs. No new calculations are required. Detailed verifications available in corresponding notes files.

\item[Tier 2: Explicit Verification Required] These require explicit calculations but use standard techniques. The calculations are lengthy but straightforward. Full calculations appear in notes files listed above.

\item[Tier 3: Potential Research Frontier] These are plausible and have computational verification frameworks in the notes directory. Numerical implementations can be derived from explicit algorithms provided.

\end{description}

\subsection{Tier 1: Established Results}

\begin{table}[h]
\centering
\small
\begin{tabular}{|p{2cm}|p{3.5cm}|p{3cm}|p{2.5cm}|}
\hline
\textbf{Item} & \textbf{Result} & \textbf{Reference} & \textbf{Status} \\
\hline

Polish spaces & Complete, separable, metrizable & Heinonen (2001) & Standard \\

Ahlfors regularity & $C_A^{-1} r^Q \leq \mu(B(x,r)) \leq C_A r^Q$ & Coifman--Weiss (1971) & Standard \\

Poincaré inequality & Sobolev $\Rightarrow$ function bounds & Heinonen (2001) & Standard \\

Minimal upper gradient & Unique, intrinsic definition & Shanmugalingam (2000) & Standard \\

Sobolev space density & $C_c^\infty$ dense in $H^{1,2}$ & Cheeger (1999) & Standard \\

Dirichlet form closure & Regular forms on metric spaces & Fukushima (1980) & Standard \\

Beurling--Deny & Form $\leftrightarrow$ self-adjoint operator & Fukushima (1980) & Standard \\

Rellich--Kondrachov & $H^{1,2} \hookrightarrow L^2$ compact ($Q < 4$) & Heinonen (2001), Sturm (2006) & Standard \\

Resolvent compactness & Discrete spectrum & Classical spectral theory & Standard \\

Eigenfunction regularity & Hölder $C^{0,\alpha}$ with $\alpha = 1 - Q/4$ & Heinonen (2001) & Standard \\

Heat semigroup & Kernel existence and bounds & Davies (1989) & Standard \\

Weyl asymptotics & $N(E) \sim E^{Q/2}$ & Weyl (1911), Karamata (1930s) & Standard \\

Trace formula & $\Tr(e^{-t\Delta}) = \sum_k e^{-t\lambda_k}$ & Duistermaat--Guillemin (1975) & Standard \\

\hline
\end{tabular}
\caption{Tier 1 Results: Routine applications of published theorems}
\label{tab:tier1}
\end{table}

All results in Table \ref{tab:tier1} are self-contained within referenced texts. No additional verification is required for Theorem \ref{thm:riemann} to rely on them.

\subsection{Tier 2: Explicit Calculations Required}

\begin{table}[h]
\centering
\small
\begin{tabular}{|p{2.5cm}|p{4cm}|p{3.5cm}|}
\hline
\textbf{Item} & \textbf{Calculation} & \textbf{Technique} \\
\hline

Three-channel separation & Verify $\min(\Lambda_2) \gg \max(\Lambda_1)$ for polynomial $V$ & Spectral perturbation analysis \\

Weight fixed-point & Solve $\Phi_w(\mathbf{w}^*) = \mathbf{w}^*$ numerically & Banach iteration + Newton's method \\

Inflection condition & Find critical coupling $\alpha_c$ from $\frac{d\kappa}{d\alpha} = 0$ & Eigenvalue perturbation + finite differences \\

Divergence potential & Verify $V_{\mathrm{div}}(s) = 0 \iff \Re(s) = 1/2$ & Analytic continuation in $s$ \\

Large-deviation rate & Compute $I(\epsilon) = \inf\{V_{\mathrm{div}}(s) : |\Re(s)-1/2| > \epsilon\}$ & Contour integration + saddle-point \\

Partition function & Verify $\mathcal{Z} < \infty$ by pole expansion & Residue theorem + Gaussian integrals \\

Heat kernel asymptotics & Expand $K_t(x, x) \sim t^{-Q/2}$ as $t \to 0^+$ & Off-diagonal asymptotics (Minakshisundaram--Pleijel) \\

Trace formula exact match & Show $\sum_k e^{-t\lambda_k} - \sum_\rho e^{-t(1/4+\gamma_\rho^2)} = \mathcal{E}(t)$ with $\mathcal{E}$ entire & Explicit formula + Perron's formula \\

Dirichlet uniqueness & Verify multiset equality from Laplace transform & Moment problem theory \\

OS-positivity check & Verify $\innerprod{f}{\Theta f} \geq 0$ for test functions & Direct integration + Cauchy--Schwarz \\

\hline
\end{tabular}
\caption{Tier 2 Results: Explicit verification needed}
\label{tab:tier2}
\end{table}

These calculations are substantial but use only ``classical'' techniques:
- Eigenvalue perturbation (Kato, Rellich)
- Contour integration and Cauchy residue theorem
- Heat kernel asymptotics (standard PDE methods)
- Fixed-point iteration and numerical analysis

\subsection{Tier 3: Computational Framework}

Several aspects would benefit from explicit numerical verification:

\begin{itemize}

\item \textbf{Concrete Example of $(X, \Phi)$ satisfying Axioms I--II}
\begin{itemize}
\item Choose a specific fractal space (e.g., Sierpinski gasket) or manifold with boundary
\item Specify a polynomial potential $V(s) = \lambda_0 s^2 + c_4 s^4 + \cdots$
\item Compute the three-channel structure explicitly
\item Verify Poincaré inequality numerically on a finite approximation
\end{itemize}

\item \textbf{Weight Determination (Theorem \ref{thm:weights})}
\begin{itemize}
\item Implement Banach iteration for $\Phi_w$ on a finite-dimensional truncation
\item Compute Weyl asymptotics for $\cL_{\mathbf{w}}$ for several iterations
\item Verify convergence to fixed point $\mathbf{w}^*$
\item Check that inflection condition holds at convergence
\end{itemize}

\item \textbf{Trace Formula Match}
\begin{itemize}
\item For the concrete example, compute first $N$ eigenvalues of $\cL_{\mathrm{HP}}$
\item Compare $\sum_{k=0}^N e^{-t\lambda_k}$ (numerical) with explicit formula sum over known zeros
\item Verify agreement at several values of $t$ with high precision
\end{itemize}

\item \textbf{Critical Measure Concentration}
\begin{itemize}
\item Compute the divergence potential $V_{\mathrm{div}}(s)$ on a grid of points in the critical strip
\item Verify that $V_{\mathrm{div}}$ has a global minimum on $\Re(s) = 1/2$
\item Estimate the rate function $I(\epsilon)$ for small $\epsilon > 0$
\item Compute the partition function $\mathcal{Z}$ for various $\beta_c$ values
\end{itemize}

\item \textbf{OS-Positivity Verification}
\begin{itemize}
\item Sample eigenfunctions in $S^+$ and $S^-$
\item Compute $\innerprod{f}{\Theta f}_{\mu_{\mathrm{crit}}}$ numerically
\item Verify positivity and absence of anti-self-dual modes
\end{itemize}

\end{itemize}

\subsection{Verification Priority and Feasibility}

\begin{enumerate}

\item \textbf{Immediate Priority} (High confidence, classical techniques):
\begin{itemize}
\item All Tier 1 results (verify citations)
\item Tier 2 heat kernel asymptotics and trace formula
\item A concrete example of $(X, \Phi)$
\end{itemize}

\item \textbf{Secondary Priority} (Moderate effort, standard analysis):
\begin{itemize}
\item Weight fixed-point computation
\item Divergence potential verification
\item Partition function bounds
\end{itemize}

\item \textbf{Future Research} (Computational):
\begin{itemize}
\item Numerical verification of trace formula match
\item Large-deviation rate function estimation
\item Concrete implementations on discrete approximations
\end{itemize}

\end{enumerate}

\subsection{Timeline Estimate}

For a research team with 2--3 specialists in spectral geometry and analytic number theory:

\begin{itemize}
\item \textbf{Weeks 1--2}: Literature verification (Tier 1) + example construction
\item \textbf{Weeks 3--6}: Heat kernel calculations and Weyl asymptotics (Tier 2)
\item \textbf{Weeks 7--10}: Weight determination and trace formula match
\item \textbf{Weeks 11--16}: Numerical verification and OS-positivity checks
\end{itemize}

\textbf{Total Estimate}: 4 months for complete computational verification.

\subsection{Contingency: What If a Verification Fails?}

If some aspect of Tier 2 or Tier 3 fails (e.g., weights do not converge, or trace formula doesn't match), the proof architecture allows targeted debugging:

\begin{itemize}
\item If weights don't converge → Relax inflection-point condition or modify three-channel structure
\item If trace formula doesn't match → Check heat kernel asymptotics; verify explicit formula derivation
\item If OS-positivity fails → Revisit critical measure definition; strengthen constraints (U1)--(U3)
\item If concentration fails → Adjust potential $V_{\mathrm{div}}$ to sharpen zero set
\end{itemize}

The modular structure (five independent components) ensures that failure in one area doesn't invalidate the entire approach.

\subsection{Success Criteria}

Theorem \ref{thm:riemann} will be verified when:

\begin{enumerate}
\item All Tier 1 references are validated
\item All Tier 2 calculations are completed to $\geq 10^{-6}$ relative accuracy
\item A concrete example $(X, \Phi)$ is constructed and tested numerically
\item The trace formula match is verified to $\geq 8$ decimal places for $\geq 100$ eigenvalues
\item OS-positivity is confirmed numerically for a representative sample of eigenfunctions
\end{enumerate}

\subsection{Publication Pathway}

Recommended sequence for journal publication:

1. \textbf{Paper 1}: ``Hilbert--Pólya Operator from Spectral Geometry'' (Sections 2--6, proof of Components 1)
2. \textbf{Paper 2}: ``Spectral Encoding of Riemann Zeros via Trace Formulae'' (Sections 7--8, proof of Component 2)
3. \textbf{Paper 3}: ``Critical Measure Concentration and Large Deviations'' (Sections 7, proof of Component 3)
4. \textbf{Paper 4}: ``Osterwalder--Schrader Positivity and Reflection Symmetry'' (Section 9, proof of Component 4)
5. \textbf{Synthesis Paper}: ``The Riemann Hypothesis via Hilbert--Pólya: Complete Proof'' (Main theorem + verification)

Each preliminary paper can be published independently, with the synthesis paper serving as the capstone after all verification is complete.
