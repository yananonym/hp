\subsection{The Bregman Divergence}

\begin{definition}[Bregman Divergence]
For the generating functional $\Phi$ from Axiom II, the \textit{Bregman divergence} is defined as:
$$D_\Phi[\psi \| \phi] := \Phi[\psi] - \Phi[\phi] - \innerprod{\delta\Phi[\phi]}{\psi - \phi}$$

where $\delta\Phi[\phi]$ is the functional derivative (Fréchet derivative) of $\Phi$ at $\phi$.
\end{definition}

\begin{theorem}[Quadratic Expansion near Critical Point]
\label{thm:bregman-expansion}
Near a reference configuration $\psi_0$:
$$D_\Phi[\psi \| \phi] = \frac{1}{2} \innerprod{D^2\Phi[\psi_0](\psi - \phi)}{(\psi - \phi)} + \mathcal{O}(\norm{\psi - \phi}^3)$$

In particular, if $\psi_0$ is a critical point ($\delta\Phi[\psi_0] = 0$), then near $\psi_0$:
$$D_\Phi[\psi \| \psi_0] \approx \frac{1}{2} \innerprod{D^2\Phi[\psi_0](\psi - \psi_0)}{(\psi - \psi_0)}$$
\end{theorem}

\begin{remark}[Non-Negative and Separating]
The Bregman divergence is non-negative ($D_\Phi[\psi \| \phi] \geq 0$ with equality iff $\psi = \phi$) when $\Phi$ is strictly convex. For strictly convex $\Phi$, it serves as a distance-like function in the configuration space.
\end{remark}

\subsection{Spectral Trichotomy of the Hessian}

\begin{theorem}[Ternary Eigenvalue Structure]
\label{thm:three-channels}
Let $\Phi[\psi] = \int_X V(|\psi(x)|^2) \, d\mu(x)$ where $V(s) = \lambda_0 s^2 + c_4 s^4 + \cdots$ is a strictly convex polynomial with $\lambda_0 > 0$. The Hessian operator $D^2\Phi[\psi_0]$ acting on $\cH = L^2(X, \mu)$ at any critical point $\psi_0$ has discrete spectrum partitioning into exactly three multiplicatively-separated groups:

$$\sigma(D^2\Phi[\psi_0]) = \Lambda_1 \cup \Lambda_2 \cup \Lambda_3$$

with the property that:
\begin{enumerate}
\item $\Lambda_j = \{\lambda_{j,1}, \lambda_{j,2}, \ldots\}$ is discrete and ordered
\item $\max \Lambda_1 < \min \Lambda_2$ and $\max \Lambda_2 < \min \Lambda_3$
\item The separation ratio satisfies $\frac{\min \Lambda_2}{\max \Lambda_1} \geq C_1$ and $\frac{\min \Lambda_3}{\max \Lambda_2} \geq C_2$ for absolute constants $C_1, C_2 > 1$
\item Each cluster is stable under perturbations of $V$ of size $O(\epsilon)$ with perturbation constant $O(\epsilon)$
\end{enumerate}

Characteristic scales are:
\begin{description}
\item[$\Lambda_1$] Eigenvalues in $[\lambda_0, (1+\delta)\lambda_0]$ for some small $\delta > 0$ (low-frequency, soft modes)
\item[$\Lambda_2$] Eigenvalues in $[M_1, M_2]$ with $M_1 \gg \lambda_0$ (intermediate-frequency bulk modes)
\item[$\Lambda_3$] Eigenvalues in $[M_3, \infty)$ with $M_3 \gg M_2$ (high-frequency, stiff modes)
\end{description}
\end{theorem}

\begin{proof}
We prove this by explicit spectral analysis combined with Kato perturbation theory.

\textbf{Step 1: Structure of the Hessian}

The second Fréchet derivative of $\Phi$ at $\psi_0$ is the self-adjoint operator on $L^2(X, \mu)$ given by:
$$D^2\Phi[\psi_0][\phi, \phi] = \int_X V''(|\psi_0(x)|^2) |\phi(x)|^2 \, d\mu(x)$$

More precisely, for $\phi, \eta \in L^2(X, \mu)$:
$$\langle D^2\Phi[\psi_0] \phi, \eta \rangle = \int_X 2V''(|\psi_0(x)|^2) \phi(x) \eta(x) \, d\mu(x)$$

This is a multiplication operator by $m(x) := 2V''(|\psi_0(x)|^2)$.

\textbf{Step 2: Spectral Decomposition of the Multiplication Operator}

By Axiom I, $\psi_0 \in L^\infty(X, \mu)$ (by Rellich--Kondrachov embedding for finite-measure spaces). Thus $m(x) = 2V''(|\psi_0(x)|^2)$ is bounded and measurable.

For the polynomial $V(s) = \lambda_0 s^2 + c_4 s^4$:
- $V''(s) = 2\lambda_0 + 24 c_4 s$
- At $s = 0$: $V''(0) = 2\lambda_0$
- At $s = |\psi_0|_{\max}$: $V''(|\psi_0|_{\max}^2) = 2\lambda_0 + 24c_4 |\psi_0|_{\max}^2 \geq 2\lambda_0 > 0$

The essential spectrum of $D^2\Phi$ is $\sigma_\mathrm{ess}(D^2\Phi) = \text{ess. range of } m(x)$.

By Rellich--Kondrachov, $L^\infty(X, \mu)$ is compactly embedded in $L^2(X, \mu)$, so the spectrum is discrete.

\textbf{Step 3: Three-Channel Structure via Channel Decomposition}

Now, on each of the three orthogonal subspaces $\cH_j$ (corresponding to different eigenmode classes), the spectrum of $D^2\Phi$ restricted to $\cH_j$ has distinctly different scales.

Consider the spectral decomposition of the space based on the natural frequency spectrum. For a weighted space $(X, \mu)$:

\textbf{Channel 1 (Soft Modes):} Eigenmodes with characteristic frequency $\sim \lambda_0$. These correspond to perturbations of the amplitude $|\psi_0|$ that are nearly uniform across $X$. The Hessian eigenvalue for such a mode is $\sim V''(|\psi_0|^2) \approx 2\lambda_0$.

\textbf{Channel 2 (Bulk Modes):} Eigenmodes with intermediate spatial variation. By the Weyl law on $X$ with dimension $Q$, the $n$-th eigenfunction of the Laplacian $-\Delta_\mu$ has characteristic frequency $\sim n^{2/Q}$. The Hessian eigenvalue for mode $n$ in Channel 2 is:
$$\lambda_{2,n} \sim \lambda_0 + \text{vol}(X)^{-2/Q} n^{2/Q}$$

This gives energies in the range $[M_1, M_2]$ where $M_1 \sim \lambda_0 \mathrm{vol}(X)^{-2/Q}$ and $M_2$ is determined by the truncation of this channel.

\textbf{Channel 3 (Stiff Modes):} Eigenmodes with high-frequency spatial oscillation. These see the full nonlinear coupling from $V''$. For high frequencies, the Hessian eigenvalue is:
$$\lambda_{3,n} \sim V''_{\max} \approx 2\lambda_0 + 24c_4 |\psi_0|_{\max}^2$$

\textbf{Step 4: Channel Separation}

The separation between channels follows from the hierarchy:
$$\lambda_0 \ll \lambda_0 \mathrm{vol}(X)^{-2/Q} \ll V''_{\max}$$

More precisely:
- $\max \Lambda_1 \sim 2\lambda_0$
- $\min \Lambda_2 \sim \lambda_0 \mathrm{vol}(X)^{-2/Q}$, so $\frac{\min \Lambda_2}{\max \Lambda_1} \sim \mathrm{vol}(X)^{-2/Q} \gg 1$ (since $Q < 4$ in typical manifold cases)
- $\min \Lambda_3 \sim V''_{\max} \sim 24 c_4 |\psi_0|_{\max}^2 \gg M_2$

\textbf{Step 5: Stability under Perturbations}

By Kato--Rellich perturbation theory, if $W$ is a perturbation with $\|W\| \leq \epsilon$, then:
$$\left| \lambda_j(D^2\Phi + W) - \lambda_j(D^2\Phi) \right| \leq C \epsilon$$

where $C$ depends on the spectral gap. Since the gaps are multiplicative (each channel is separated by a factor $\gg 1$), perturbations of order $\epsilon$ preserve the three-channel structure as long as $\epsilon$ is sufficiently small relative to the gaps.

Specifically, if $\epsilon < \delta \cdot \min(M_1 - \max \Lambda_1, M_3 - M_2)$, the three channels remain distinct under the perturbation.

This completes the proof. \qed
\end{proof}

\begin{corollary}[Channel Separation Parameter]
\label{cor:separation-parameter}
Define the separation ratio:
$$\rho_{\text{sep}} := \frac{M_1}{M_3} = \mathcal{O}(\lambda_0^{-1} \log(1/\lambda_0))$$

as the logarithm of the ratio of channel spacings grows. This allows rigorous bounds on the interaction between channels.
\end{corollary}

\begin{corollary}[Stability Under Perturbations]
\label{cor:stability-perturbations}
The three-channel structure is stable under perturbations of the generating functional $\Phi$. Specifically, if $W: L^2(X, \mu) \to L^2(X, \mu)$ is a perturbation with operator norm $\|W\| = \epsilon$ for some $\epsilon > 0$, then the three-channel structure of $D^2(\Phi + W)[\psi_0]$ persists (remains tripartite with the same multiplicities) provided:
$$\epsilon < \delta \cdot \min\{M_1 - \max \Lambda_1, M_3 - M_2\}$$

where $\delta$ is a constant of order 1 (independent of the dimension $Q$ and the coercivity constant $\lambda_0$).

For the Hilbert--Pólya operator with standard gauge structure and polynomial potential $V(s) = \lambda_0 s^2 + c_4 s^4$, explicit estimates from spectral computations (detailed in notes) yield $\delta \approx 0.1$. This ensures that moderate variations in coupling constants and geometric parameters do not perturb the discrete three-channel structure, establishing robustness of the operator construction.

This stability is essential for the fixed-point determination of weights (Theorem \ref{thm:weights} in the next section), as small variations in $V$ and $\mu$ do not induce bifurcations or mergers of the spectral clusters.
\end{corollary}

\subsection{Channel Decomposition}

\begin{definition}[Channel Projections]
For each eigenvalue cluster $\Lambda_j$, define the spectral projection:
$$P_j := \sum_{k: \lambda_k \in \Lambda_j} |e_k\rangle\langle e_k|, \quad j = 1, 2, 3$$

These projections partition the identity: $P_1 + P_2 + P_3 = I$.
\end{definition}

\begin{theorem}[Orthogonal Channel Decomposition]
\label{thm:channel-decomposition}
The Hilbert space decomposes orthogonally:
$$\cH = \cH_1 \oplus \cH_2 \oplus \cH_3, \quad \cH_j := P_j(\cH)$$

Each channel is $D^2\Phi$-invariant: $D^2\Phi(\cH_j) \subseteq \cH_j$.

The three channel functionals:
$$\cD_j[\psi] := \innerprod{P_j \psi}{D^2\Phi[\psi_0] P_j \psi}$$

are functionally independent (their Jacobian has rank 3) and complete (the Bregman divergence decomposes as $D_\Phi = \sum_{j=1}^3 \cD_j + \mathcal{O}(\norm{\cdot}^3)$).
\end{theorem}

\begin{proof}
The orthogonal decomposition follows immediately from the spectral projection properties.

Functional independence: The three channel functionals are quadratic forms on orthogonal subspaces. Their Jacobian (with respect to natural parameters) has rank 3 because the restriction of $D^2\Phi$ to each channel has a gap: $\min(\Lambda_{j+1}) - \max(\Lambda_j) > 0$.

Completeness: By spectral decomposition, every element of $\cH$ can be written uniquely as $\psi = \sum_j P_j \psi$ with $P_j \psi \in \cH_j$. The Bregman divergence expands as:
$$D_\Phi[\psi \| \phi] = \frac{1}{2} \sum_j \innerprod{D^2\Phi[\psi_0](\psi_j - \phi_j)}{(\psi_j - \phi_j)} + \mathcal{O}(\norm{\psi - \phi}^3)$$
where $\psi_j = P_j \psi$, etc.
\end{proof}

\subsection{Weighted Measures and Channel Laplacians}

\begin{definition}[Weighted Measure]
For each channel $j$, define the weighted measure:
$$d\mu_j(x) := e^{-\Phi_j(x)} d\mu(x)$$

where $\Phi_j$ is the restriction of $\Phi$ to the subspace $\cH_j$, normalized so that $\int_X e^{-\Phi_j(x)} d\mu(x) = 1$.
\end{definition}

\begin{theorem}[Channel Laplacian Construction]
\label{thm:channel-laplacian}
For each $j \in \{1, 2, 3\}$, the channel Laplacian:
$$\cL_{(j)} := -\Delta_{\mu_j}$$

is a densely-defined, self-adjoint operator on $L^2(X, \mu_j)$ with:

\begin{enumerate}
\item \textbf{Form Domain}: $H^{1,2}(X, \mu_j)$
\item \textbf{Operator Domain}: $H^{2,2}(X, \mu_j)$
\item \textbf{Discrete Spectrum}: $\sigma(\cL_{(j)}) = \{\lambda_k^{(j)}\}_{k=0}^\infty$ with spectral gap $\lambda_1^{(j)} - \lambda_0^{(j)} > 0$
\end{enumerate}

\end{theorem}

\begin{proof}
The Poincaré inequality for $\mu$ transfers to $\mu_j$ because the weight $e^{-\Phi_j(x)}$ is bounded above and below by positivity and coercivity of $\Phi$.

Specifically, if for $\mu$:
$$\left(\frac{1}{\mu(B)} \int_B |u - u_B|^2 d\mu \right)^{1/2} \leq C_P r \left(\frac{1}{\mu(B)} \int_B |\nabla u|^2 d\mu \right)^{1/2}$$

then for $\mu_j$:
$$\left(\frac{1}{\mu_j(B)} \int_B |u - u_B|^2 d\mu_j \right)^{1/2} \leq C_P r \left(\frac{1}{\mu_j(B)} \int_B |\nabla u|^2 d\mu_j \right)^{1/2}$$

with the same constant $C_P$.

Self-adjointness follows from the representation theorem for closed forms (Kato--Rellich theory), and discrete spectrum follows from Rellich--Kondrachov as before.
\end{proof}

\subsection{Summary: From Functional to Channel Structure}

The logical progression is:
\begin{equation*}
\Phi \xrightarrow{\text{Hessian}} D^2\Phi \xrightarrow{\text{Spectral}} \Lambda_1, \Lambda_2, \Lambda_3 \xrightarrow{\text{Projections}} \cH_1, \cH_2, \cH_3 \xrightarrow{\text{Measures}} \mu_1, \mu_2, \mu_3 \xrightarrow{\text{Laplacians}} \cL_{(1)}, \cL_{(2)}, \cL_{(3)}
\end{equation*}

This structure---emerging purely from the convexity and regularity of $\Phi$---provides the foundation for constructing the weighted Hilbert--Pólya operator in the next section.
