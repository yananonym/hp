\subsection{Axiom I: The Polish Metric Measure Space}

\begin{axiom}[Minimally-Equipped Polish Space]
Let $(X, d_X, \mu)$ be a metric measure space satisfying:

\begin{enumerate}
\item[(I.i)] \textbf{Polish Structure}: $X$ is complete, separable, and metrizable with compact closure.
\item[(I.ii)] \textbf{Borel Measure}: $\mu$ is a Borel probability measure on $X$.
\item[(I.iii)] \textbf{Ahlfors $Q$-Regularity}: There exists $C_A > 0$ and $Q \in (2, \infty)$ such that for all $x \in X$ and $0 < r < \mathrm{diam}(X)$:
$$C_A^{-1} r^Q \leq \mu(B(x, r)) \leq C_A r^Q$$
\item[(I.iv)] \textbf{Poincaré Inequality}: There exists $C_P > 0$ such that for all $u \in H^{1,2}(X)$ and balls $B = B(x, r)$:
$$\left(\frac{1}{\mu(B)} \int_B |u - u_B|^2 \, d\mu\right)^{1/2} \leq C_P r \left(\frac{1}{\mu(B)} \int_B |\nabla_{\min} u|^2 \, d\mu\right)^{1/2}$$

where $u_B = \mu(B)^{-1} \int_B u \, d\mu$ and $|\nabla_{\min} u|$ is the minimal upper gradient.

\end{enumerate}

\end{axiom}

\begin{remark}[Emergent Properties from Axiom I]
From Axiom I alone, several deep properties follow:
\begin{enumerate}
\item The measure $\mu$ is automatically inner and outer regular (Ulam's theorem)
\item Ahlfors regularity implies the \textit{doubling property}: $\mu(B(x, 2r)) \leq 2^Q \mu(B(x, r))$
\item The space admits a Cheeger differentiable structure with cotangent fiber dimension at most $Q$
\item For compact Sobolev embeddings, necessarily $Q < 4$ (see Theorem \ref{thm:dimension-necessity})
\end{enumerate}
\end{remark}

\subsection{Axiom II: The Strictly Convex Generating Functional}

\begin{axiom}[Configuration Space Structure -- Quartic Potential]
Let $\cH = L^2(X, \mu; \bbC^n)$ be the Hilbert space of square-integrable sections. There exists a generating functional:
$$\Phi[\psi] := \int_X V(|\psi(x)|^2) \, d\mu(x)$$

where the potential $V$ has the specific form:
$$V(s) = \frac{\lambda_0}{2} s^2 + \frac{c_4}{4} s^4$$

with parameters $\lambda_0, c_4 > 0$.

This functional satisfies:

\begin{enumerate}
\item[(II.i)] \textbf{Strict Convexity}: For all $\psi, \phi \in \cH$ with $\psi \neq \phi$ and $t \in (0,1)$:
$$\Phi[t\psi + (1-t)\phi] < t\Phi[\psi] + (1-t)\Phi[\phi]$$

\item[(II.ii)] \textbf{Positive-Definite Hessian}: The second functional derivative satisfies:
$$\innerprod{D^2\Phi[\psi_0] h}{h} \geq 2\lambda_0 \norm{h}_{\cH}^2$$
for all $h \in \cH$ and some $\lambda_0 > 0$.

\item[(II.iii)] \textbf{Analytical Structure}: The Hessian's second derivative is:
$$V''(s) = 2\lambda_0 + 24c_4 s$$

ensuring that the functional is smooth and has the required spectral structure.

\end{enumerate}

\end{axiom}

\begin{remark}[Why Quartic Potentials?]
The choice of quartic potential $V(s) = \frac{\lambda_0}{2}s^2 + \frac{c_4}{4}s^4$ is not arbitrary. It encodes the minimal structure required for the three-channel decomposition (Theorem \ref{thm:three-channels}):

\begin{itemize}
\item The quadratic term $\lambda_0 s^2$ produces soft modes (Channel 1)
\item The quartic term $c_4 s^4$ creates intermediate bulk modes (Channel 2)
\item The cross-coupling between these terms generates stiff high-frequency modes (Channel 3)
\end{itemize}

Higher-order polynomial terms (degree 6, 8, etc.) would create additional fine structure in the spectrum but do not fundamentally alter the three-channel decomposition. Thus, the quartic case is canonical and sufficient for the RH proof.
\end{remark}

\begin{remark}[Physical Interpretation]
Axiom II defines a classical field configuration space with $\Phi$ as the energy functional. The strict convexity and positive-definite Hessian ensure that the functional has a unique global minimum, providing a natural reference state around which to expand perturbations.
\end{remark}

\subsection{Dimensional Necessity}

\begin{theorem}[Dimensional Necessity]\label{thm:dimension-necessity}
Under Axioms I--II, if the Laplacian has discrete spectrum with Hölder-continuous eigenfunctions of exponent $\alpha = 1 - Q/4 > 0$, then necessarily:
$$Q < 4$$
\end{theorem}

\begin{proof}
By the Rellich--Kondrachov theorem for metric measure spaces, the Sobolev embedding $H^{1,2}(X) \hookrightarrow L^2(X)$ is compact when $X$ is compact and supports a Poincaré inequality with $Q < 2 \times 2 = 4$ (the critical dimension). 

For $Q \geq 4$, the embedding is continuous but not compact, which precludes the operator from having a discrete spectrum. The eigenfunction regularity constraint $\alpha = 1 - Q/4 > 0$ further restricts to $Q < 4$.
\end{proof}

\begin{corollary}[Critical Dimension]
The constraint $Q < 4$ emerges from mathematical consistency alone, independent of any physical considerations. This bounds the effective dimensionality of the underlying space to less than 4.
\end{corollary}

\subsection{Minimal Domain Requirements}

\begin{definition}[Sobolev Space Domain]
For Axioms I--II to yield a well-defined operator theory, the domain of the Sobolev space is:
$$H^{1,2}(X; \bbC^n) = \left\{ u \in L^2(X, \mu; \bbC^n) : \int_X |\nabla_{\min} u|^2 \, d\mu < \infty \right\}$$

with norm $\norm{u}_{H^{1,2}}^2 := \norm{u}_{L^2}^2 + \int_X |\nabla_{\min} u|^2 \, d\mu$.
\end{definition}

\begin{theorem}[Density of Smooth Functions]\label{thm:density-smooth}
The space $C_c^\infty(X; \bbC^n)$ of smooth compactly-supported functions is dense in $H^{1,2}(X; \bbC^n)$.
\end{theorem}

\begin{proof}[Proof Sketch]
The density follows from Cheeger's theory \cite{cheeger1999differentiability}: the combination of the Poincaré inequality and Ahlfors regularity yields a measurable differentiable structure. Mollification using heat kernels on this structure yields smooth approximations. Truncation to compact subsets gives compactly-supported approximants.
\end{proof}

\subsection{Summary of Axiomatic Setup}

Axioms I--II provide:
\begin{itemize}
\item A geometric substrate ($X, d, \mu$) with rigorous regularity properties
\item An energy functional $\Phi$ with well-defined Hessian and spectral structure
\item Automatically emerging properties: doubling measure, Cheeger differentiable structure, dimensional bound
\item Domain and density properties sufficient for spectral theory
\end{itemize}

These axioms are not chosen arbitrarily; they represent the minimal mathematical structure required for Hilbert--Pólya-type constructions. Section \ref{sec:verification} will specify concrete examples of $(X, \Phi)$ satisfying both axioms.
