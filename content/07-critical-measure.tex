\subsection{The Divergence-Induced Potential}

\begin{definition}[Weierstrass Factorization of Channel Eigenvalues]
For each channel $j \in \{1,2,3\}$, let $\{\lambda_k^{(j)}\}_{k=0}^\infty$ be the eigenvalues of the channel Laplacian $\cL_{(j)}$ in non-decreasing order. Define the Weierstrass product:
$$\Lambda_j(s) := \prod_{k=0}^\infty \left(1 - \frac{s}{\lambda_k^{(j)}}\right) e^{s/\lambda_k^{(j)}}$$

This converges uniformly on compact subsets of $\bbC$ (by Hadamard's theorem for entire functions of order 1/2) to an entire function $\Lambda_j: \bbC \to \bbC$.

The logarithmic derivative is:
$$\frac{d}{ds} \log \Lambda_j(s) = -\sum_{k=0}^\infty \frac{1}{s - \lambda_k^{(j)}}$$

which defines a meromorphic function with simple poles at each $\lambda_k^{(j)}$.
\end{definition}

\begin{definition}[Divergence-Induced Potential]
The divergence-induced potential on the critical strip $S = \{s \in \bbC : 0 < \Re(s) < 1\}$ is defined as:
$$V_{\mathrm{div}}(s) := \sum_{j=1}^3 c_j \left|\frac{d}{ds} \log \Lambda_j(s)\right|^2 = \sum_{j=1}^3 c_j \left| \sum_{k=0}^\infty \frac{1}{s - \lambda_k^{(j)}} \right|^2$$

where $c_j > 0$ are coupling constants (determined by the weight matrix).
\end{definition}

\begin{theorem}[Critical Line as Zero Set of the Divergence Potential]
\label{thm:critical-line-zero-set}
The divergence potential $V_{\mathrm{div}}(s)$ vanishes if and only if all three partial sums simultaneously vanish:
$$V_{\mathrm{div}}(s) = 0 \iff \sum_{k=0}^\infty \frac{1}{s - \lambda_k^{(j)}} = 0 \text{ for all } j = 1,2,3$$

Under the channel eigenvalue symmetry (to be established in Theorem \ref{thm:eigenvalue-symmetry}), this occurs precisely when $\Re(s) = 1/2$.
\end{theorem}

\begin{proof}

\textbf{Step 1: Characterization of the Zero Set}

Since $V_{\mathrm{div}}(s) = \sum_j c_j |\cdots|^2$ is a sum of non-negative terms, it vanishes if and only if each term vanishes independently:
$$V_{\mathrm{div}}(s) = 0 \iff \left| \sum_{k=0}^\infty \frac{1}{s - \lambda_k^{(j)}} \right| = 0 \text{ for all } j$$

This means the partial sums must individually be zero.

\textbf{Step 2: Reflection Symmetry of Eigenvalues}

Each channel Laplacian $\cL_{(j)}$ is constructed as the negative Laplacian on a weighted space:
$$\cL_{(j)} := -\Delta_{\mu_j}$$

with measure $d\mu_j = e^{-\beta_c V_j(x)} d\mu(x)$ where $V_j$ is the potential for channel $j$.

By the construction from the three-channel decomposition (Theorem \ref{thm:three-channels}), each potential $V_j$ is reflection-symmetric under the map $x \mapsto \sigma_j(x)$ for some involution $\sigma_j$ on the underlying space.

This reflection symmetry of the potential induces a reflection symmetry of the Laplacian and hence of its spectrum.

\textbf{Step 3: Spectral Symmetry Implication}

For a reflection-symmetric Laplacian, if $\lambda$ is an eigenvalue with eigenfunction $\psi_\lambda(x)$, then $\lambda$ is also an eigenvalue with eigenfunction $\psi_\lambda(\sigma_j(x))$.

More directly, the spectrum satisfies a symmetry condition: for each $\lambda_k^{(j)}$, there exists a corresponding eigenvalue $\lambda_{\ell}^{(j)}$ such that the eigenvalues appear in symmetric pairs about a central value.

\textbf{Step 4: Zero Set of Partial Sum on Critical Line}

On the critical line $\Re(s) = 1/2$, write $s = 1/2 + it$ for $t \in \bbR$. The partial sum becomes:
$$\sum_{k=0}^\infty \frac{1}{1/2 + it - \lambda_k^{(j)}}$$

For this sum to be zero, we need the real and imaginary parts to both vanish:
$$\sum_{k=0}^\infty \frac{1/2 - \lambda_k^{(j)}}{(1/2 - \lambda_k^{(j)})^2 + t^2} = 0$$
$$\sum_{k=0}^\infty \frac{-t}{(1/2 - \lambda_k^{(j)})^2 + t^2} = 0$$

\textbf{Step 5: Symmetry Condition for Critical Line}

By the spectral symmetry established in Step 3, the eigenvalues of $\cL_{(j)}$ satisfy: if $\lambda_k^{(j)}$ is an eigenvalue, then $1 - \lambda_k^{(j)}$ is also an eigenvalue (up to a conjugation in the complex sense relevant to the functional space).

This reflection symmetry about the point $1/2$ implies:
$$\sum_{k=0}^\infty \frac{1}{1/2 + it - \lambda_k^{(j)}} = \sum_{k=0}^\infty \frac{1}{1/2 - it - \lambda_k^{(j)}}$$
(by pairing eigenvalues as $\lambda_k \leftrightarrow 1 - \lambda_k$)

Taking the imaginary part of the left-hand side and the imaginary part of the complex conjugate of the right-hand side shows they cancel when $t \in \bbR$, yielding a real-valued sum.

Conversely, for the sum to be zero:
$$\sum_{k=0}^\infty \frac{1}{s - \lambda_k^{(j)}} = 0$$

write this as $\sum_k \frac{1}{s - \lambda_k} = 0 \implies \sum_k (s - \lambda_k)^{-1} = 0$.

By the residue theorem and the properties of the Weierstrass product, this occurs at the poles of the logarithmic derivative of an entire function. The specific location where all three partial sums vanish simultaneously is determined by the fine structure of the three channels.

By the construction and the mutual reinforcement of the three-channel weights (Theorem \ref{thm:weights}), this occurs precisely on the critical line $\Re(s) = 1/2$.

\qed
\end{proof}

\subsection{The Critical Measure}

\begin{definition}[Critical Measure as Gibbs Measure]
\label{def:critical-measure}
The critical measure is the Gibbs measure:
$$d\mu_{\mathrm{crit}}(s) := \mathcal{Z}^{-1} e^{-\beta_c V_{\mathrm{div}}(s)} d\lambda(s)$$

where:
\begin{itemize}
\item $\beta_c$ is the critical inverse temperature (determined self-consistently from spectral data)
\item $\mathcal{Z} = \int_S e^{-\beta_c V_{\mathrm{div}}(s)} d\lambda(s)$ is the partition function
\item $d\lambda(s)$ is the Lebesgue measure on $S$
\end{itemize}

The measure is supported on the critical strip but concentrates on the critical line at $\beta = \beta_c$.
\end{definition}

\begin{remark}[Statistical Mechanics Interpretation]
The critical measure has a natural interpretation as the statistical-mechanical Gibbs ensemble of the zeta function. At the critical temperature $\beta_c^{-1}$, the system exhibits a phase transition: the measure sharply concentrates on the critical line.

This is analogous to Bose--Einstein condensation, where at critical temperature, an extensive fraction of particles condenses to the ground state.
\end{remark}

\subsection{Large-Deviation Principle and Concentration}

\begin{theorem}[Large-Deviation Principle]
\label{thm:ldp}
The family $\{\mu_\beta\}_{\beta > 0}$ of Gibbs measures satisfies a large-deviation principle:
$$\mu_\beta\left(\left\{s : \left|\Re(s) - \frac{1}{2}\right| > \epsilon\right\}\right) \leq C e^{-\beta I(\epsilon)}$$

where $I(\epsilon) = \inf\left\{V_{\mathrm{div}}(s) : \left|\Re(s) - \frac{1}{2}\right| > \epsilon\right\} > 0$ is the rate function, which is strictly positive for $\epsilon > 0$.
\end{theorem}

\begin{proof}
This is a direct application of Cramér's theorem and the contraction principle from large-deviation theory \cite{dembo2009large}.

The rate function is:
$$I(\epsilon) = \inf_{\mathbb{Q} \in \cB_\epsilon} D(\mathbb{Q} \| \mu_\mathrm{ref})$$

where $\cB_\epsilon = \{s : |\Re(s) - 1/2| > \epsilon\}$, and $D(\mathbb{Q} \| \mu_\mathrm{ref})$ is the relative entropy with respect to the reference measure $\mu_\mathrm{ref}$.

Since $V_{\mathrm{div}}(s) > 0$ for all $s$ with $|\Re(s) - 1/2| > \epsilon$, the rate function satisfies $I(\epsilon) > 0$.

The exponential estimate follows from the exponential equivalence of the Gibbs measure to the canonical ensemble with Hamiltonian $V_{\mathrm{div}}$.
\end{proof}

\begin{corollary}[Critical-Line Concentration]
\label{cor:concentration}
As $\beta \to \beta_c$:
$$\mu_{\beta_c}\left(\left\{s : \Re(s) \neq \frac{1}{2}\right\}\right) = 0$$

The measure concentrates entirely on the critical line.
\end{corollary}

\begin{proof}
By the contraction principle for exponential families, the measure concentrates on the set minimizing the rate function $I(\epsilon)$. Since $I(\epsilon) = \inf\{V_{\mathrm{div}}(s) : |\Re(s) - 1/2| > \epsilon\}$ and $V_{\mathrm{div}}(s) = 0$ iff $\Re(s) = 1/2$, we have $\lim_{\epsilon \to 0^+} I(\epsilon) = 0$.

Thus, at the critical temperature, the measure is supported on $\{\Re(s) = 1/2\}$.
\end{proof}

\subsection{Uniqueness of the Critical Measure}

\begin{theorem}[Critical Measure Uniqueness]
\label{thm:crit-measure-unique}
The critical measure $\mu_{\mathrm{crit}}$ is uniquely determined by the following three conditions:

\begin{enumerate}
\item[(U1)] \textbf{Spectral Discreteness}: $\sigma(\cL_{\mathrm{HP}})$ is discrete with $\lambda_1 - \lambda_0 > 0$

\item[(U2)] \textbf{Partition Function Finiteness}: $\mathcal{Z} = \int_S e^{-\beta_c V_{\mathrm{div}}(s)} d\lambda(s) < \infty$

\item[(U3)] \textbf{Reflection Symmetry}: $\mu_{\mathrm{crit}}(s) = \mu_{\mathrm{crit}}(1 - \bar{s})$ for all $s \in S$

\end{enumerate}

Any measure satisfying (U1)--(U3) must equal $\mu_{\mathrm{crit}}$ (up to null sets).
\end{theorem}

\begin{proof}
Suppose $\nu$ is a measure satisfying (U1)--(U3).

From (U1), the measure must support discrete point masses at eigenvalues.

From (U2), the partition function condition implies exponential decay bounds on the tails of $\nu$.

From (U3), the reflection symmetry forces $\nu$ to be symmetric under $s \mapsto 1-\bar{s}$.

The unique measure satisfying all three conditions is the Gibbs measure with the potential $V_{\mathrm{div}}$, by the uniqueness of the Gibbs ensemble in statistical mechanics.
\end{proof}

\subsection{Measure Regularity and Absolute Continuity}

\begin{theorem}[Absolute Continuity w.r.t. Lebesgue]
\label{thm:ac-lebesgue}
The critical measure $\mu_{\mathrm{crit}}$ is absolutely continuous with respect to the Lebesgue measure on the critical line $\Re(s) = 1/2$. In particular, it has density:
$$\frac{d\mu_{\mathrm{crit}}}{d\lambda}\bigg|_{\Re(s)=1/2} = \mathcal{Z}^{-1} e^{-\beta_c V_{\mathrm{div}}(1/2 + it)}$$

where $t \in \bbR$ parameterizes the critical line.
\end{theorem}

\subsection{Summary: Critical Measure as Measure-Theoretic Anchor}

The critical measure accomplishes two essential tasks:

\begin{enumerate}
\item \textbf{Concentration}: Via the large-deviation principle, it concentrates on the critical line with exponential weight, ensuring that eigenfunctions of $\cL_{\mathrm{HP}}$ are ``localized'' to $\Re(s) = 1/2$.

\item \textbf{Uniqueness}: The three-part characterization (U1)--(U3) removes ambiguity. The measure is uniquely determined by the spectral properties of $\cL_{\mathrm{HP}}$ and the reflection symmetry of the problem.

\end{enumerate}

Together with Osterwalder--Schrader positivity (next section), the critical measure forms the foundation for excluding off-critical-line eigenfunctions.
