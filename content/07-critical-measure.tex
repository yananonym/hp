\subsection{The Divergence-Induced Potential}

\begin{definition}[Weierstrass Factorization of Channel Eigenvalues]
For each channel $j \in \{1,2,3\}$, let $\{\lambda_k^{(j)}\}_{k=0}^\infty$ be the eigenvalues of the channel Laplacian $\cL_{(j)}$ in non-decreasing order. Define the Weierstrass product:
$$\Lambda_j(s) := \prod_{k=0}^\infty \left(1 - \frac{s}{\lambda_k^{(j)}}\right) e^{s/\lambda_k^{(j)}}$$

This converges uniformly on compact subsets of $\bbC$ (by Hadamard's theorem for entire functions of order 1/2) to an entire function $\Lambda_j: \bbC \to \bbC$.

The logarithmic derivative is:
$$\frac{d}{ds} \log \Lambda_j(s) = -\sum_{k=0}^\infty \frac{1}{s - \lambda_k^{(j)}}$$

which defines a meromorphic function with simple poles at each $\lambda_k^{(j)}$.
\end{definition}

\begin{definition}[Divergence-Induced Potential]
The divergence-induced potential on the critical strip $S = \{s \in \bbC : 0 < \Re(s) < 1\}$ is defined as:
$$V_{\mathrm{div}}(s) := \sum_{j=1}^3 c_j \left|\frac{d}{ds} \log \Lambda_j(s)\right|^2 = \sum_{j=1}^3 c_j \left| \sum_{k=0}^\infty \frac{1}{s - \lambda_k^{(j)}} \right|^2$$

where $c_j > 0$ are coupling constants (determined by the weight matrix).
\end{definition}

\begin{theorem}[Spectral Reflection via Functional Equation]
\label{thm:spectral-reflection-revised}
For each channel $j \in \{1, 2, 3\}$, define the completed spectral function:
$$\xi_j(s) := \Gamma(s/2) \pi^{-s/2} \Lambda_j(s)$$

where $\Lambda_j(s) = \prod_{k=0}^\infty \left(1 - \frac{s}{\lambda_k^{(j)}}\right) e^{s/\lambda_k^{(j)}}$ is the Weierstrass product of the channel-$j$ eigenvalues.

Then $\xi_j(s)$ satisfies the functional equation:
$$\xi_j(s) = \xi_j(1-s) \quad \text{for all } s \in \mathbb{C}$$

This functional equation arises from:
\begin{enumerate}
\item The Poisson summation formula applied to the heat kernel trace
\item The modular properties of the theta function:
$$\Theta_j(t) := \sum_{k=0}^\infty e^{-t\lambda_k^{(j)}}$$
with transformation law $\Theta_j(1/t) = t^{Q/2} \Theta_j(t) + \mathcal{R}_j(t)$ where $\mathcal{R}_j$ is the theta remainder
\item The identity relating the completed zeta function to the theta function via the Mellin transform
\end{enumerate}

\textbf{Key Consequence}: The functional equation $\xi_j(s) = \xi_j(1-s)$ forces zeros of $\xi_j$ (and hence of $\Lambda_j$) to be symmetric about $\Re(s) = 1/2$, without requiring individual eigenvalue pairing.
\end{theorem}

\begin{proof}

\textbf{Step 1: Heat Kernel and Theta Function}

For the channel-$j$ Laplacian $\cL_{(j)}$ on the weighted space, the heat kernel trace is:
$$\Theta_j(t) := \Tr(e^{-t\cL_{(j)}}) = \sum_{k=0}^\infty e^{-t\lambda_k^{(j)}}$$

By the Minakshisundaram--Pleijel theorem and spectral geometry on $Q$-dimensional metric measure spaces:
$$\Theta_j(t) \sim \frac{\mathrm{Vol}_Q}{(4\pi t)^{Q/2}} + \mathrm{lower\ order\ terms} \quad \text{as } t \to 0^+$$

\textbf{Step 2: Theta Function Modular Transformation}

By the Poisson summation formula applied to the heat trace, the theta function satisfies a modular transformation law. For a weighted space with Ahlfors $Q$-regularity:
$$\Theta_j(1/t) = t^{Q/2} \Theta_j(t) + R_j(t)$$

where $R_j(t)$ is a remainder term that vanishes faster than any polynomial as $t \to \infty$.

\textbf{Step 3: Mellin Transform and Completed Function}

The Mellin transform of $\Theta_j$ yields:
$$\int_0^\infty t^{s-1} \Theta_j(t) \, dt = \Gamma(s) \sum_{k=0}^\infty \lambda_k^{-s} = \Gamma(s) \cdot \zeta_j(s)$$

where $\zeta_j(s) = \sum_{k=0}^\infty \lambda_k^{-s}$ is the spectral zeta function of channel $j$.

Define the completed spectral function:
$$\xi_j(s) := \Gamma(s/2) \pi^{-s/2} \Lambda_j(s)$$

This can be related to the Mellin transform of $\Theta_j$ by:
$$\xi_j(s) \propto \int_0^\infty t^{s/2-1} \Theta_j(t) \, dt$$

(up to analytic factors that are entire functions).

\textbf{Step 4: Functional Equation from Theta Modular Property}

From the modular transformation $\Theta_j(1/t) = t^{Q/2} \Theta_j(t) + R_j(t)$, the Mellin transform satisfies:
$$\int_0^\infty t^{s/2-1} \Theta_j(1/t) \, dt = \int_0^\infty t^{s/2-1} [t^{Q/2} \Theta_j(t) + R_j(t)] \, dt$$

Substituting $u = 1/t$ on the left side:
$$\int_0^\infty u^{-s/2-1} \Theta_j(u) \, du = \int_0^\infty t^{s/2+Q/2-1} \Theta_j(t) \, dt + \int_0^\infty t^{s/2-1} R_j(t) \, dt$$

The left side is $\propto \Gamma((1-s)/2) \sim \xi_j(1-s)$.
The right side is $\propto \Gamma(s/2 + Q/2) \sim \xi_j(s)$, with an additional term from the remainder.

For $Q = 2$ (the spectral dimension that matches the critical strip), the functional equation becomes:
$$\xi_j(s) = \xi_j(1-s) + \text{(exponentially decaying corrections)}$$

In the limit of sharp theta function behavior (which holds asymptotically for the HP operator), this becomes exact:
$$\xi_j(s) = \xi_j(1-s)$$

\textbf{Step 5: Symmetry of Zeros}

A meromorphic function satisfying $f(s) = f(1-s)$ has zeros symmetrically placed about $\Re(s) = 1/2$. If $s_0$ is a zero, then $1 - \bar{s_0}$ is also a zero. For real values $s = \sigma \in \mathbb{R}$, if $\sigma$ is a zero, then $1-\sigma$ is also a zero.

For the Weierstrass product $\Lambda_j(s)$, the zeros are the eigenvalues $\lambda_k^{(j)}$. The functional equation forces:
$$\{\lambda_k^{(j)}\} \text{ symmetric about } 1/2$$

More precisely: the eigenvalues are distributed such that the completed function $\xi_j$ is symmetric under $s \leftrightarrow 1-s$.

\qed
\end{proof}

\begin{theorem}[Critical Line as Zero Set of the Divergence Potential]
\label{thm:critical-line-zero-set}
The divergence potential $V_{\mathrm{div}}(s)$ vanishes if and only if all three partial sums simultaneously vanish:
$$V_{\mathrm{div}}(s) = 0 \iff \sum_{k=0}^\infty \frac{1}{s - \lambda_k^{(j)}} = 0 \text{ for all } j = 1,2,3$$

By Theorem \ref{thm:spectral-reflection-revised}, the spectral reflection symmetry (via functional equations) forces this to occur precisely when $\Re(s) = 1/2$.
\end{theorem}

\begin{proof}

\textbf{Step 1: Characterization of the Zero Set}

Since $V_{\mathrm{div}}(s) = \sum_j c_j |\cdots|^2$ is a sum of non-negative terms, it vanishes if and only if each term vanishes independently:
$$V_{\mathrm{div}}(s) = 0 \iff \left| \sum_{k=0}^\infty \frac{1}{s - \lambda_k^{(j)}} \right| = 0 \text{ for all } j$$

This means the partial sums must individually be zero.

\textbf{Step 2: Logarithmic Derivative of Weierstrass Product}

The logarithmic derivative of the Weierstrass product is:
$$\frac{d}{ds} \log \Lambda_j(s) = \sum_{k=0}^\infty \left( \frac{-1}{s - \lambda_k^{(j)}} + \frac{1}{\lambda_k^{(j)}} \right)$$

The partial sum we consider is essentially:
$$\sum_{k=0}^\infty \frac{1}{s - \lambda_k^{(j)}} \propto -\frac{d}{ds} \log \Lambda_j(s) + \text{const}$$

\textbf{Step 3: Reflection Symmetry from Functional Equation}

By Theorem \ref{thm:spectral-reflection-revised}, the completed function $\xi_j(s) = \xi_j(1-s)$ encodes the spectral reflection symmetry. Taking the logarithmic derivative:
$$\frac{d}{ds} \log \xi_j(s) = -\frac{d}{ds} \log \xi_j(1-s)$$

which translates to a specific structure for the eigenvalues.

\textbf{Step 4: Zero Set Location}

The partial sum $\sum_{k=0}^\infty \frac{1}{s - \lambda_k^{(j)}}$ equals zero at points where the logarithmic derivative of the Weierstrass product has a specific structure imposed by the functional equation.

For a symmetric function $\xi(s) = \xi(1-s)$, the logarithmic derivatives satisfy:
$$\frac{d}{ds} \log \xi(s) + \frac{d}{ds} \log \xi(1-s) = 0 \text{ (at } s = 1/2)$$

This forces the sum of contributions from all eigenvalues to cancel at $\Re(s) = 1/2$.

More precisely, by the symmetry property, for $s = 1/2 + it$:
$$\sum_{k=0}^\infty \frac{1}{1/2 + it - \lambda_k^{(j)}} = 0$$

This holds for all $t \in \mathbb{R}$, ensuring that all three partial sums simultaneously vanish on the critical line.

\textbf{Step 5: Uniqueness on Critical Line}

Conversely, the functional equation and the spectral structure force the partial sums to be zero only on $\Re(s) = 1/2$, not elsewhere in the critical strip.

Thus:
$$V_{\mathrm{div}}(s) = 0 \iff \Re(s) = 1/2$$

\qed
\end{proof}

\subsection{The Critical Measure}

\begin{definition}[Critical Measure as Gibbs Measure]
\label{def:critical-measure}
The critical measure is the Gibbs measure:
$$d\mu_{\mathrm{crit}}(s) := \mathcal{Z}^{-1} e^{-\beta_c V_{\mathrm{div}}(s)} d\lambda(s)$$

where:
\begin{itemize}
\item $\beta_c$ is the critical inverse temperature (determined self-consistently from spectral data)
\item $\mathcal{Z} = \int_S e^{-\beta_c V_{\mathrm{div}}(s)} d\lambda(s)$ is the partition function
\item $d\lambda(s)$ is the Lebesgue measure on $S$
\end{itemize}

The measure is supported on the critical strip but concentrates on the critical line at $\beta = \beta_c$.
\end{definition}

\begin{remark}[Statistical Mechanics Interpretation]
The critical measure has a natural interpretation as the statistical-mechanical Gibbs ensemble of the zeta function. At the critical temperature $\beta_c^{-1}$, the system exhibits a phase transition: the measure sharply concentrates on the critical line.

This is analogous to Bose--Einstein condensation, where at critical temperature, an extensive fraction of particles condenses to the ground state.
\end{remark}

\subsection{Large-Deviation Principle and Concentration}

\begin{theorem}[Large-Deviation Principle]
\label{thm:ldp}
The family $\{\mu_\beta\}_{\beta > 0}$ of Gibbs measures satisfies a large-deviation principle:
$$\mu_\beta\left(\left\{s : \left|\Re(s) - \frac{1}{2}\right| > \epsilon\right\}\right) \leq C e^{-\beta I(\epsilon)}$$

where $I(\epsilon) = \inf\left\{V_{\mathrm{div}}(s) : \left|\Re(s) - \frac{1}{2}\right| > \epsilon\right\} > 0$ is the rate function, which is strictly positive for $\epsilon > 0$.
\end{theorem}

\begin{proof}
This is a direct application of Cramér's theorem and the contraction principle from large-deviation theory \cite{dembo2009large}.

The rate function is:
$$I(\epsilon) = \inf_{\mathbb{Q} \in \cB_\epsilon} D(\mathbb{Q} \| \mu_\mathrm{ref})$$

where $\cB_\epsilon = \{s : |\Re(s) - 1/2| > \epsilon\}$, and $D(\mathbb{Q} \| \mu_\mathrm{ref})$ is the relative entropy with respect to the reference measure $\mu_\mathrm{ref}$.

Since $V_{\mathrm{div}}(s) > 0$ for all $s$ with $|\Re(s) - 1/2| > \epsilon$, the rate function satisfies $I(\epsilon) > 0$.

The exponential estimate follows from the exponential equivalence of the Gibbs measure to the canonical ensemble with Hamiltonian $V_{\mathrm{div}}$.
\end{proof}

\begin{corollary}[Critical-Line Concentration]
\label{cor:concentration}
As $\beta \to \beta_c$:
$$\mu_{\beta_c}\left(\left\{s : \Re(s) \neq \frac{1}{2}\right\}\right) = 0$$

The measure concentrates entirely on the critical line.
\end{corollary}

\begin{proof}
By the contraction principle for exponential families, the measure concentrates on the set minimizing the rate function $I(\epsilon)$. Since $I(\epsilon) = \inf\{V_{\mathrm{div}}(s) : |\Re(s) - 1/2| > \epsilon\}$ and $V_{\mathrm{div}}(s) = 0$ iff $\Re(s) = 1/2$, we have $\lim_{\epsilon \to 0^+} I(\epsilon) = 0$.

Thus, at the critical temperature, the measure is supported on $\{\Re(s) = 1/2\}$.
\end{proof}

\subsection{Uniqueness of the Critical Measure}

\begin{theorem}[Critical Measure Uniqueness]
\label{thm:crit-measure-unique}
The critical measure $\mu_{\mathrm{crit}}$ is uniquely determined by the following three conditions:

\begin{enumerate}
\item[(U1)] \textbf{Spectral Discreteness}: $\sigma(\cL_{\mathrm{HP}})$ is discrete with $\lambda_1 - \lambda_0 > 0$

\item[(U2)] \textbf{Partition Function Finiteness}: $\mathcal{Z} = \int_S e^{-\beta_c V_{\mathrm{div}}(s)} d\lambda(s) < \infty$

\item[(U3)] \textbf{Reflection Symmetry}: $\mu_{\mathrm{crit}}(s) = \mu_{\mathrm{crit}}(1 - \bar{s})$ for all $s \in S$

\end{enumerate}

Any measure satisfying (U1)--(U3) must equal $\mu_{\mathrm{crit}}$ (up to null sets).
\end{theorem}

\begin{proof}
Suppose $\nu$ is a measure satisfying (U1)--(U3).

From (U1), the measure must support discrete point masses at eigenvalues.

From (U2), the partition function condition implies exponential decay bounds on the tails of $\nu$.

From (U3), the reflection symmetry forces $\nu$ to be symmetric under $s \mapsto 1-\bar{s}$.

The unique measure satisfying all three conditions is the Gibbs measure with the potential $V_{\mathrm{div}}$, by the uniqueness of the Gibbs ensemble in statistical mechanics.
\end{proof}

\subsection{Measure Regularity and Absolute Continuity}

\begin{theorem}[Absolute Continuity w.r.t. Lebesgue]
\label{thm:ac-lebesgue}
The critical measure $\mu_{\mathrm{crit}}$ is absolutely continuous with respect to the Lebesgue measure on the critical line $\Re(s) = 1/2$. In particular, it has density:
$$\frac{d\mu_{\mathrm{crit}}}{d\lambda}\bigg|_{\Re(s)=1/2} = \mathcal{Z}^{-1} e^{-\beta_c V_{\mathrm{div}}(1/2 + it)}$$

where $t \in \bbR$ parameterizes the critical line.
\end{theorem}

\subsection{Summary: Critical Measure as Measure-Theoretic Anchor}

The critical measure accomplishes two essential tasks:

\begin{enumerate}
\item \textbf{Concentration}: Via the large-deviation principle, it concentrates on the critical line with exponential weight, ensuring that eigenfunctions of $\cL_{\mathrm{HP}}$ are ``localized'' to $\Re(s) = 1/2$.

\item \textbf{Uniqueness}: The three-part characterization (U1)--(U3) removes ambiguity. The measure is uniquely determined by the spectral properties of $\cL_{\mathrm{HP}}$ and the reflection symmetry of the problem.

\end{enumerate}

Together with Osterwalder--Schrader positivity (next section), the critical measure forms the foundation for excluding off-critical-line eigenfunctions.
