\subsection{Compact Resolvent and Discrete Spectrum}

\begin{theorem}[Resolvent Compactness]
\label{thm:resolvent-compact}
For $Q < 4$, the resolvent $(I + \Delta)^{-1}: L^2(X) \to L^2(X)$ is compact.
\end{theorem}

\begin{proof}
By the Rellich--Kondrachov theorem for metric measure spaces \cite{heinonen2001analysis}, the Sobolev embedding 
$$H^{1,2}(X) \hookrightarrow L^2(X)$$
is compact when $X$ is compact and supports a Poincaré inequality with dimension parameter $Q < 4$.

The resolvent maps $L^2 \to \Dom(\Delta) \subset H^{1,2} \hookrightarrow L^2$ compactly. More explicitly, for $f \in L^2$, the solution $\psi = (I+\Delta)^{-1}f$ satisfies:
$$\norm{\psi}_{H^{1,2}} \leq C \norm{f}_{L^2}$$
so $(I+\Delta)^{-1}$ maps bounded sets in $L^2$ to bounded sets in $H^{1,2}$. The embedding $H^{1,2} \hookrightarrow L^2$ is compact by Rellich--Kondrachov, completing the proof.
\end{proof}

\begin{corollary}[Discrete Spectrum]
\label{cor:discrete-spectrum}
The spectrum of $\Delta$ is purely discrete:
$$\sigma(\Delta) = \{0 \leq \lambda_0 < \lambda_1 \leq \lambda_2 \leq \cdots \to \infty\}$$

Each eigenvalue has finite multiplicity, and the normalized eigenfunctions $\{e_k\}_{k=0}^\infty$ form an orthonormal basis of $L^2(X, \mu; \bbC^n)$.
\end{corollary}

\begin{remark}[Domain and Self-Adjointness]
\label{rem:domain-self-adjoint}
The Laplacian $\Delta$ arising from the Dirichlet form (Theorem \ref{thm:beurling-deny}) is manifestly self-adjoint with domain $\Dom(\Delta) = H^{2,2}(X)$ (Corollary after Theorem \ref{thm:beurling-deny} in Section 3). By the spectral theorem for self-adjoint operators, $\Delta$ admits a unique spectral decomposition into eigenvalues and eigenfunctions. The resolvent $(I + \Delta)^{-1}$ is well-defined as a bounded operator, and compactness follows from the embedding $H^{1,2}(X) \hookrightarrow L^2(X)$ via Rellich--Kondrachov. No deficiency index computation is required; self-adjointness is guaranteed by the Beurling--Deny representation.
\end{remark}

\subsection{Eigenfunction Regularity}

\begin{theorem}[Hölder Regularity of Eigenfunctions]
\label{thm:holder-regularity}
For $Q < 4$, every eigenfunction $\phi_n$ of $\Delta$ satisfies:
$$\phi_n \in C^{0,\alpha}(X)$$
with Hölder exponent $\alpha = 1 - Q/4 > 0$.
\end{theorem}

\begin{proof}
This follows from the Sobolev-to-Hölder embedding theorem on Ahlfors-regular spaces with Poincaré inequalities \cite{heinonen2001analysis, sturm2006geometry}.

For an eigenfunction $\phi_n$ with eigenvalue $\lambda_n$, we have $\phi_n \in H^{1,2}(X)$ by definition. The iterated equation $\Delta \phi_n = \lambda_n \phi_n$ implies $\phi_n \in H^{2,2}(X)$.

By the Sobolev--Hölder embedding, $H^{1+\epsilon,2}(X) \hookrightarrow C^{0,\alpha}(X)$ with $\alpha + 2/Q = 1$. Setting $\alpha = 1 - Q/4$ requires $1 - Q/4 + Q/2 = 1$, which simplifies to $Q/4 > 0$—true for all $Q > 0$.

The critical threshold is $Q = 4$, where the embedding fails.
\end{proof}

\begin{remark}[Hölder Exponent Interpretation]
The Hölder exponent $\alpha = 1 - Q/4$ decreases as the dimension $Q$ increases. For $Q < 4$, we have $\alpha > 0$, ensuring continuity of eigenfunctions. This regularity is essential for analyzing zero sets and measure-theoretic properties.
\end{remark}

\subsection{Heat Kernel and Semigroup}

\begin{theorem}[Heat Semigroup Existence]
\label{thm:heat-semigroup}
For $t > 0$, the heat semigroup $e^{-t\Delta}$ is a strongly continuous semigroup of self-adjoint operators. It admits a kernel representation:
$$\langle e^{-t\Delta} f, g \rangle = \int_X \int_X K_t(x,y) f(y) g(x) \, d\mu(x) d\mu(y)$$

where the heat kernel $K_t: X \times X \to \bbR$ is measurable and satisfies:
\begin{enumerate}
\item $K_t(x,y) = K_t(y,x)$ (symmetry)
\item $\int_X K_t(x,y) \, d\mu(y) \leq 1$ for all $x, t$ (subunit property)
\item $\int_X K_t(x,z) K_s(z,y) \, d\mu(z) = K_{t+s}(x,y)$ (semigroup property)
\end{enumerate}
\end{theorem}

\begin{proof}
The heat semigroup is generated by $-\Delta$ via the spectral theorem:
$$e^{-t\Delta} = \sum_{k=0}^\infty e^{-t\lambda_k} |e_k \rangle \langle e_k|$$

Strong continuity follows from the spectral properties and Stone's theorem. The kernel existence is guaranteed by the Beurling--Deny representation and results of Davies \cite{davies1989heat}.
\end{proof}

\subsection{Trace and Weyl Asymptotics}

\begin{theorem}[Heat Kernel Trace Formula]
\label{thm:trace-formula-heat}
The trace of the heat semigroup admits the spectral representation:
$$\Tr(e^{-t\Delta}) = \sum_{k=0}^\infty e^{-t\lambda_k} = \int_X K_t(x, x) \, d\mu(x)$$

which converges absolutely for all $t > 0$.
\end{theorem}

\begin{theorem}[Weyl Asymptotics]
\label{thm:weyl-asymptotics}
The eigenvalue counting function $N(E) := \#\{k : \lambda_k \leq E\}$ satisfies:
$$N(E) \sim \frac{C_{\mathrm{Weyl}}}{(4\pi)^{Q/2} \Gamma(Q/2 + 1)} E^{Q/2}$$
as $E \to \infty$, where $C_{\mathrm{Weyl}}$ depends on the measure and geometry of $X$.
\end{theorem}

\begin{proof}[Proof Sketch]
Apply Tauberian theorems (Karamata, Hardy--Littlewood) to the trace formula. The key relation is:
$$\int_0^\infty e^{-tE} N(E) \, dE = \Tr(e^{-t\Delta}) = \sum_{k} e^{-t\lambda_k}$$

Taking Laplace transforms and using the small-$t$ asymptotics of the heat kernel, which are determined by the Hausdorff dimension $Q$, yields the power law with exponent $Q/2$.

For $Q$-regular spaces, this exponent is universal.
\end{proof}

\subsection{Spectral Gap and Poincaré Constant}

\begin{theorem}[Spectral Gap]
\label{thm:spectral-gap}
The spectral gap satisfies:
$$\lambda_1 - \lambda_0 \geq \frac{C_P^{-2}}{r_0^2}$$

where $C_P$ is the Poincaré constant and $r_0$ is a characteristic radius of $X$.
\end{theorem}

\begin{proof}
The Poincaré inequality directly implies a lower bound on the first nonzero eigenvalue via variational characterization:
$$\lambda_1 = \inf_{\phi \perp e_0} \frac{\cE[\phi, \phi]}{\norm{\phi}_{L^2}^2}$$

The Poincaré inequality gives:
$$\cE[\phi, \phi] \geq C_P^{-2} \norm{\phi - \langle \phi \rangle}_{L^2}^2 \geq C_P^{-2} (1 - o(1)) \norm{\phi}_{L^2}^2$$

for $\phi$ orthogonal to constants, yielding the bound.
\end{proof}

\subsection{Summary: Spectral Properties}

Key results established:
\begin{itemize}
\item \textbf{Discrete Spectrum} ($Q < 4$): Eigenvalues $0 \leq \lambda_0 < \lambda_1 < \lambda_2 < \cdots \to \infty$
\item \textbf{Hölder Regularity}: Eigenfunctions are $C^{0,\alpha}$ with $\alpha = 1 - Q/4$
\item \textbf{Heat Semigroup}: Admits kernel with trace formula
\item \textbf{Weyl Law}: Eigenvalue density grows like $E^{Q/2}$
\item \textbf{Spectral Gap}: $\lambda_1 - \lambda_0 > 0$ (from Poincaré inequality)
\end{itemize}

These form the foundation for the three-channel structure and spectral encoding of Riemann zeros discussed in subsequent sections.
