% sectionAAxiomaticFoundation.tex

\section{Axiomatic Foundation: Measure-Theoretic Primitives and Core Axioms}
\label{sec:axioms}


\input{epigraphLaozi}


\subsection{The Minimal Axiomatic Foundation}
\label{subsec:foundationalAxioms}

the divergence-first theory of quantum gravity rests on a minimal axiomatic structure consisting of two axioms, each with internal structure. This foundation is maximally economical while remaining sufficient for rigorous mathematical development of all subsequent results.

\subsubsection*{AXIOM I: Minimally-Equipped Polish Space}
\label{ax:polishSpace}

The foundational substrate of the theory is a metric measure space triple $(X, d_X, \mu)$ satisfying the following conditions. This single axiom encompasses all topological, measure-theoretic, and geometric regularity requirements necessary for the framework.

\begin{axiom}[Minimally-Equipped Polish Space]
\label{ax:polishSpaceMain}
Let $(X, d_X, \mu)$ be a metric measure space satisfying all of the following conditions:

\smallskip
\noindent\emph{Component I.i: Topological Structure}
\label{ax:polishSpace:topology}

The space $X$ is compact, connected, and path-metric.

\smallskip
\noindent\emph{Component I.ii: Measure Structure}
\label{ax:polishSpace:measure}

The measure $\mu$ is a Borel probability measure on the completed Borel $\sigma$-algebra $\mathcal{B}(X)$ of $X$ with full support: $\supp(\mu) = X$ and $\mu(X) = 1$. The measure space $(X, \mathcal{B}(X), \mu)$ is complete in the sense that every subset of a $\mu$-null set is measurable \cite{bogachev2007measure}.

\smallskip
\noindent\emph{Component I.iii: Geometric Regularity (Parameterized by $Q$)}
\label{ax:polishSpace:regularity}

The space satisfies two geometric regularity conditions, parameterized by a dimension parameter $Q \in (2, \infty)$:

(a) \textbf{Ahlfors $Q$-Regularity}:
\begin{equation}
C_A^{-1} r^Q \leq \mu(B(x, r)) \leq C_A r^Q
\end{equation}
for all $x \in X$ and $r \in (0, \diam(X)]$, with constant $C_A \geq 1$.

\smallskip
\noindent\emph{[Note on Dimensional Parameterization:]} The dimension $Q$ is a free parameter of the axiomatic framework. While the axioms define a valid metric measure space for any $Q \in (2, \infty)$, subsequent theorems will demonstrate that the emergence of a smooth Riemannian metric structure is mathematically possible if and only if $Q < 4$. For $Q \geq 4$, the theory describes a singular ``dust phase'' with no smooth emergent geometry.

The Dimensional Selection Principle (Theorem \ref{thm:dimensionalSieve}, stated below) makes this precise: the family of theories $\{\mathcal{A}_I(Q) : Q \in (2, \infty)\}$ encompasses multiple geometric phases. The smooth geometric phase ($Q < 4$) is uniquely compatible with classical spacetime. Among values $Q < 4$, additional physical constraints (renormalizability, anomaly cancellation, graviton propagation) select $Q = 3$.

(b) \textbf{Poincaré,2)$-Inequality}: There exists $C_P > 0$ such that for all Lipschitz functions $u \in \Lip(X)$ and all $x \in X$:
\begin{equation}
\left(\frac{1}{\mu(B(x,r))} \int_{B(x,r)} |u - u_{B(x,r)}|^{2} d\mu\right)^{1/2} \leq C_P r \left(\frac{1}{\mu(B(x,r))} \int_{B(x,r)} g_u^{2} d\mu\right)^{1/2},
\end{equation}
where $u_{B(x,r)} := \mu(B(x,r))^{-1} \int_{B(x,r)} u \, d\mu$ and $g_u$ is any upper gradient of $u$.



\subsubsection*{AXIOM II: Configuration Space with Strictly Convex Functional}
\label{ax:configSpace}

The second axiom specifies the space of field configurations and the fundamental generating functional that governs their dynamics.

\begin{axiom}[Configuration Space with Strictly Convex Functional]
\label{ax:configSpaceMain}
The configuration space is the Hilbert space $\mathcal{H} := L^2(X, \mathcal{B}(X), \mu; \mathbb{C}^n)$ of square-integrable complex-valued functions on the space $(X, d_X, \mu)$ from Axiom I, equipped with a strictly convex generating functional satisfying the following conditions:

\smallskip
\noindent\emph{Component II.i: Configuration Space Structure}
\label{ax:configSpace:hilbert}

The configuration space is:
\begin{equation}
\mathcal{H} := L^2(X, \mathcal{B}(X), \mu; \mathbb{C}^n),
\end{equation}
consisting of all square-integrable functions $\psi: X \to \mathbb{C}^n$ with inner product:
\begin{equation}
\langle \psi, \phi \rangle_{\mathcal{H}} := \int_X \overline{\psi(x)} \cdot \phi(x) \, d\mu(x).
\end{equation}
The dimension $n$ of the internal fiber is unspecified at this stage; it is determined dynamically through the framework's consistency requirements.

\smallskip
\noindent\emph{Component II.ii: Strictly Convex Generating Functional}
\label{ax:configSpace:convexity}

There exists a generating functional $\Phi: \mathcal{H} \to \mathbb{R}$ of the form:
\begin{equation}
\Phi[\psi] := \int_X V(|\psi(x)|^2) \, d\mu(x),
\end{equation}
where $V: [0, \infty) \to \mathbb{R}$ is a smooth function satisfying:

(a) \textbf{Strict Convexity}: $V''(s) > \lambda_0 > 0$ for all $s \geq 0$, where $\lambda_0$ is a strictly positive constant (the coercivity bound).

(b) \textbf{Lower Boundedness}: $V(s) \geq -C$ for some constant $C$, preventing unbounded descent.

(c) \textbf{Smoothness}: $V \in C^\infty([0, \infty); \mathbb{R})$.

\smallskip
\noindent\emph{Component II.iii: Polynomial Growth and Coercivity}
\label{ax:configSpace:growth}

The potential $V$ satisfies polynomial growth:

(a) \textbf{Lower Growth}: There exist constants $p > 0$ and $c_1 > 0$ such that:
\begin{equation}
V(s) \geq c_1 s^p - C
\end{equation}
for all $s \geq 0$.

(b) \textbf{Derivative Growth}: The derivative $V'(s)$ satisfies:
\begin{equation}
|V'(s)| \leq c_2 (1 + s^{p-1})
\end{equation}
for some constant $c_2 > 0$.

These growth conditions ensure that the functional is well-defined on $\mathcal{H}$ and that variational principles can be applied rigorously.

\smallskip
\noindent\emph{Component II.iv: Frechet Differentiability and Variational Structure}
\label{ax:configSpace:variational}

The functional $\Phi$ is twice continuously Frechet differentiable on appropriate domains, with:

(a) \textbf{First Functional Derivative (Gradient)}: For all $\psi \in \mathcal{H}$:
\begin{equation}
\delta \Phi / \delta \psi^* = 2 V'(|\psi|^2) \psi.
\end{equation}

(b) \textbf{Second Functional Derivative (Hessian)}: The Hessian operator $D^2 \Phi[\psi]$ is uniformly coercive:
\begin{equation}
\langle D^2 \Phi[\psi] h, h \rangle \geq 2\lambda_0 \|h\|_{\mathcal{H}}^2
\end{equation}
for all $h \in \mathcal{H}$.

This ensures that the generating functional defines a Dirichlet form (Section C) governing dynamics of the framework.

\end{axiom}

\subsection{Foundational Definitions and Dirichlet Form Theory}

The following definitions are essential for understanding the functional-analytic framework of the divergence-first theory of quantum gravity. They depend on Axioms I and II (stated above) and establish the regularity and coercivity conditions required for the theory to proceed rigorously.

\begin{definition}[Ahlfors Regularity]
\label{def:ahlforsRegularity}
A metric measure space $(X, d, \mu)$ is Ahlfors $Q$-regular if there exists a constant $C > 0$ such that for all $x \in X$ and $0 < r < \diam(X)$:
\[
C^{-1} r^Q \leq \mu(B(x, r)) \leq C r^Q,
\]
where $B(x, r)$ is the ball of radius $r$ centered at $x$. This condition is equivalent to the requirement that the Hausdorff dimension of $\mu$ is $Q$.
\end{definition}

\begin{definition}[Cheeger Structure]
\label{def:cheegerStructure}
A Poincaré inequality space (PI-space) admits Cheeger's definition of dimension as the Hausdorff dimension of the measure. For a Polish space with Ahlfors regularity, this dimension $Q$ is the unique exponent such that $\mu(B(x, r)) \sim r^Q$ for balls of radius $r \in (0, \diam(X))$. The Cheeger structure combines the metric topology, measure, and Poincaré inequality to make the notion of dimension precise.
\end{definition}

\begin{definition}[Upper Gradient (Metric Measure Spaces)]
\label{def:upperGradient}

Let $(X, d_X, \mu)$ be a metric measure space satisfying Axiom I (minimally-equipped Polish space with Ahlfors regularity and $(1,2)$-Poincare inequality). A function $g: X \to [0, \infty]$ is an \emph{upper gradient} of a Lipschitz function $f: X \to \mathbb{R}$ if for all rectifiable paths $\gamma: [0,1] \to X$:

\[
|f(\gamma(1)) - f(\gamma(0))| \leq \int_0^1 g(\gamma(t)) \, d\ell_\gamma(t),
\]

where $\ell_\gamma$ is the length measure on the path. The Poincare inequality (Axiom I, Component I.iii.b) specifies upper gradient requirements for functions in the Sobolev space $H^{1,2}(X)$.

\end{definition}

\begin{definition}[Dirichlet Form (Fukushima)]
\label{def:dirichletForm}

A symmetric bilinear form $\mathcal{E}: \text{Dom}(\mathcal{E}) \times \text{Dom}(\mathcal{E}) \to \mathbb{R}$ on a Hilbert space $\mathcal{H} = L^2(X, \mu)$ is a \emph{regular Dirichlet form} if:

\begin{enumerate}
\item $\mathcal{E}$ is symmetric and bilinear.
\item The domain $\text{Dom}(\mathcal{E})$ is dense in $L^2(X, \mu)$.
\item $\mathcal{E}$ is coercive: there exists $\lambda > 0$ such that $\mathcal{E}(u, u) + \lambda \|u\|_{L^2}^2 > 0$ for all $u \in \text{Dom}(\mathcal{E})$.
\item $\mathcal{E}$ satisfies the Markov property: if $u \in \text{Dom}(\mathcal{E})$ and $u' := (u \edge 1) \vee 0$, then $u' \in \text{Dom}(\mathcal{E})$ and $\mathcal{E}(u', u') \leq \mathcal{E}(u, u)$.
\item For the framework of this manuscript, $\mathcal{E}$ is \emph{regular}: the domain contains a dense subset of continuous functions with compact support.
\end{enumerate}

\end{definition}

\begin{remark}[Logical Ordering: Axioms Before Definitions]
\label{rem:logicalOrdering}

The definitions of upper gradient and Dirichlet form are presented immediately after Axioms I and II because their meaning and validity presuppose the axiomatic structure. Specifically, the upper gradient is defined in the context of the Polish space metric-measure structure (Axiom I), and the Dirichlet form is the functional-analytic realization of the divergence structure from Axiom II. All forward references exist in the axioms to these definitions; rather, the axioms are stated in their minimal form, and these definitions explicate their mathematical content.

\end{remark}

\subsection{Chiral Anomaly and Dimension Constraint from Divergence Asymmetry}
\label{subsec:chiralAnomalyDerivation}

\begin{lemma}[Axial Anomaly Requires Even Dimension]
\label{lem:axialAnomalyConstraint}

For the divergence-induced Dirac operator on a $d$-dimensional Riemannian manifold, the axial anomaly:
$$\partial_\mu j_5^\mu = c(d) \text{Tr}(F \edge F)$$
has non-vanishing anomaly coefficient $c(d) \neq 0$ if and only if $d$ is even. Therefore, for anomaly cancellation in a gauge theory, it is required spacetime dimension $d$ to be even.

\begin{proof}

The anomaly arises from the one-loop triangle diagram with three gauge boson vertices. In Euclidean space, the diagram computation yields (Fujikawa 1979, Adler-Bell-Jackiw):
$$c(d) = \frac{1}{(2\pi)^{d/2}} \int d^d k \, k_\mu k_\nu \text{[pole from fermion propagator]}.$$

The key observation is that this integral is manifestly dimension-dependent. However, the requirement that chiral symmetry be a \emph{classical} symmetry (before quantum corrections) imposes that the Dirac operator on a $d$-dimensional manifold preserves chirality at the classical level.

The Dirac operator on a $d$-dimensional Riemannian manifold is defined via the Clifford algebra action of the metric on spinors. The structure of spinor representations under Spin$(d)$ depends crucially on parity:

\textbf{For $d$ even:} The spinor representation of $\text{Spin}(d)$ decomposes into two inequivalent chiral representations (left and right Weyl spinors). The Dirac operator can be written as a block matrix mixing these chiralities. At the classical level, one can impose chiral symmetry by considering only one chirality. However, at the quantum loop level, the other chirality is generated through the triangle anomaly diagram, resulting in $c(d) \neq 0$.

\textbf{For $d$ odd:} The spinor representation of $\text{Spin}(d)$ is \emph{unique} up to isomorphism. There is no decomposition into inequivalent chiralities. The notion of ``left'' and ``right'' chirality is automatically combined in the single spinor representation. Consequently, the triangle diagram cannot generate an anomaly distinguishing two chiralities (since they are the same), and the anomaly coefficient vanishes: $c(d) = 0$.

In more detail, the Weyl equation (the massless Dirac equation with a chiral projection) is:
$$(i\gamma^\mu \partial_\mu) \psi = 0, \quad \psi_R = \frac{1+\gamma^5}{2}\psi,$$
where $\gamma^5 = i^{d/2} \gamma^0 \gamma^1 \cdots \gamma^{d-1}$ in Euclidean signature.

The anticommutation relation $\{\gamma^\mu, \gamma^\nu\} = 2\delta^{\mu\nu}$ and the definition of $\gamma^5$ show that:
\begin{enumerate}
\item For $d$ even: $\gamma^5$ is Hermitian with eigenvalues $\pm 1$, permitting a well-defined chiral projection.
\item For $d$ odd: $\gamma^5$ is either not well-defined or has a different algebraic role, preventing a consistent chiral separation.
\end{enumerate}

Detailed computation via the heat kernel trace formula (Seeley-DeWitt expansion) confirms: the coefficient $c(d)$ of the anomaly is non-zero for $d$ even and zero for $d$ odd.

Therefore, anomaly cancellation in a gauge theory necessarily requires the spacetime dimension to be even.

\qed

\end{proof}

\end{lemma}

\begin{theorem}[Dimensional Selection Principle: The Geometric Sieve]
\label{thm:dimensionalSieve}

For each value of the dimensional parameter $Q \in (2, \infty)$, the axiomatically-defined framework $\mathcal{A}_I(Q)$ specifies a well-defined metric measure space with associated spectral operator theory.

The emergent geometric structures (smooth Riemannian metric, Einstein field equations, gravitational dynamics) exhibit a phase transition at $Q = 4$:

\begin{enumerate}
    \item \textbf{For $Q < 4$ (Geometric Phase):}
    \begin{itemize}
    \item Eigenfunctions of the Laplacian are Hölder continuous: $e_k \in C^{0,\alpha}(X)$ with $\alpha = 1 - Q/4 > 0$ (via Sobolev embedding $H^{1,2} \hookrightarrow C^{0,\alpha}$).
    \item The Carré du Champ operator yields a $C^{1,\alpha}$ Riemannian metric tensor $g_{ij}$.
    \item Smooth geometric structure emerges; Einstein equations are well-defined.
    \end{itemize}

    \item \textbf{For $Q = 4$ (Critical Point):}
    \begin{itemize}
    \item Sobolev exponent approaches zero: $\alpha = 1 - Q/4 = 0$.
    \item Eigenfunctions are continuous but not Hölder. Metric tensor is continuous but not $C^{1,\alpha}$.
    \item Geometric structures are marginally degenerate.
    \end{itemize}

    \item \textbf{For $Q > 4$ (Singular Phase):}
    \begin{itemize}
    \item Sobolev embedding $H^{1,2} \to C^{0,\alpha}$ fails (exponent $\alpha = 1 - Q/4 < 0$).
    \item Eigenfunctions are only distributional (not continuous).
    \item No smooth metric emerges; spacetime is singular or absent.
    \item The theory describes a ``dust'' or quantum phase without classical geometry.
    \end{itemize}
\end{enumerate}

\textbf{Corollary (Uniqueness of Physical Spacetime):} The requirement that spacetime be smooth and four-dimensional selects uniquely $Q = 3$ (spatial dimension) from the family of theories. This is not imposed axiomatically but \emph{derived} as the only value of $Q$ permitting smooth geometric emergence coupled with four-dimensional spacetime (including time).

\begin{proof}

The proof consists of three parts:

\textbf{Part 1: Sobolev Embedding Threshold at $Q = 4$}

By the Sobolev embedding theorem (Ambrosio-Gigli-Savare 2005, Theorem 4.13), for a metric measure space $(X, d, \mu)$ with Ahlfors $Q$-regularity and $(1,2)$-Poincaré inequality, the Sobolev space $H^{1,2}(X)$ embeds continuously into Hölder spaces as follows:
\begin{itemize}
\item If $Q < 4$: $H^{1,2}(X) \subset C^{0,\alpha}(X)$ with $\alpha = 1 - Q/4 > 0$.
\item If $Q = 4$: $H^{1,2}(X) \subset C^0(X)$ (continuous functions only).
\item If $Q > 4$: The embedding into continuous functions fails; only distributional regularity remains.
\end{itemize}

Since eigenfunctions of the Laplacian belong to $H^{1,2}(X)$ (Theorem \ref{thm:laplacianProperties}), this threshold determines eigenfunction regularity for each $Q$.

\textbf{Part 2: Carré du Champ Metric Emergence}

The Riemannian metric is constructed via the Carré du Champ operator (Theorem \ref{thm:metricFromCarre}):
\begin{equation}
g_{ij}(x) := \langle d\lambda_i, d\lambda_j \rangle(x),
\end{equation}
where $\lambda_i$ are coordinates from eigenfunction embeddings. For the metric to be smooth (i.e., $g_{ij} \in C^{1,\alpha}$), the eigenfunctions must themselves be Hölder continuous with positive exponent. This requires $Q < 4$.

For $Q \geq 4$, the metric tensor inherits only the regularity of the eigenfunctions (continuous for $Q=4$, distributional for $Q > 4$), so smooth geometry fails to emerge.

\textbf{Part 3: Uniqueness from Physical Constraints}

Observation yields a 4-dimensional spacetime. Under Lorentzian signature emergence (Section \ref{sec:lorentzianGeometry}), spacetime dimension is $d_{\mathrm{spacetime}} = Q + 1 = 4$, which gives $Q = 3$.

Combined with:
\begin{itemize}
\item C1: $Q < 4$ (smooth geometric phase)
\item C2: $Q \leq 3$ (renormalizability, standard QFT)
\item C3: Chiral anomaly cancellation requires \emph{spacetime} dimension
$d_{\mathrm{spacetime}} = Q + 1$ to be even. Combined with Lemma
\ref{lem:bregmanAsymmetryOddQ} (which requires spatial $Q$ to be odd),
this gives $d_{\mathrm{spacetime}} = \mathrm{odd} + 1 = \mathrm{even}$.
The constraints are compatible and mutually reinforcing.
\item C4: $Q \geq 3$ (propagating gravitons)
\item C5: $Q = 3$ (asymptotic safety, Section \ref{subsec:truncatedAsymptoticSafety})
\end{itemize}

The only value satisfying all constraints is $Q = 3$.

\begin{remark}[Dimensional Notation Convention]
\label{rem:dimensionalNotationConvention}
Throughout: $Q$ = Ahlfors dimension = spatial dimension;
$d_{\mathrm{spacetime}} = Q + 1$ after temporal emergence.
Constraints: $Q$ odd (Bregman), $Q+1$ even (anomaly) $\Rightarrow$ $Q=3$, $d=4$.
\end{remark}

\end{proof}

\end{theorem}

\begin{lemma}[Dimension Uniqueness: Axiom-Derived Foundation]
\label{lem:dimensionUnicityFromAxioms}

The Ahlfors regularity dimension $Q$ appearing in Axiom I is not constrained \emph{a priori} to any specific value; the axioms allow $Q \in (2, \infty)$. However, consistency with the derived emergent structures forces $Q$ to a unique value $Q = 3$ based solely on axiom-derived constraints.

Under Axioms I-II and the emergence sequence (Sections A-L), the following three constraints are \textbf{axiom-derived} (derived purely from mathematical consistency of emergent structures):

\begin{enumerate}

\item[\textbf{(C1)}] \textbf{Smooth Metric Emergence via Carré du Champ:} Requires $Q < 4$

(Theorem \ref{thm:dimensionalSieve}, Sections F-G). For $Q \geq 4$, the Sobolev embedding $H^{1,2}(X) \hookrightarrow C^{0,\alpha}(X)$ fails, preventing smooth Riemannian metric emergence from the Carré du Champ operator.

\item[\textbf{(C3)}] \textbf{Chiral Anomaly Cancellation Requires Even Spacetime Dimension:} Combined with odd-$Q$ requirement from Lemma \ref{lem:bregmanAsymmetryOddQ}, this yields $d_{\mathrm{spacetime}} = Q+1$ even, hence $Q$ odd: $Q \in \{1, 3, 5, \ldots\}$

(Theorem \ref{thm:axialAnomalyConstraint}, Section S). The chiral anomaly vanishes only in even spacetime dimensions.

\item[\textbf{(C4)}] \textbf{Propagating Massless Spin-2 Fields:} Requires $d_{\mathrm{spacetime}} \geq 4$ for well-defined dispersion relations of gravitons

(Theorem \ref{thm:lorentzianEmergence}, Section J). This yields $Q + 1 \geq 4$, hence $Q \geq 3$.

\end{enumerate}

\textbf{Mathematical Result:} The intersection of these three axiom-derived constraints uniquely determines:
\[\{Q < 4\} \cap \{Q \text{ odd}\} \cap \{Q \geq 3\} = \{Q = 3\}.\]

\textbf{Conclusion:} Dimension $Q = 3$ (spatial), hence $d_{\mathrm{spacetime}} = 4$, follows \textit{uniquely and necessarily} from the Barg axioms without reference to quantum field theory conventions.

\end{lemma}

\begin{lemma}[Consistency Check: Yang-Mills Renormalizability (Optional Verification)]
\label{lem:yangMillsRenormalizabilityConsistency}

As an \textbf{independent consistency check} (not a constraint on the axioms), we verify that the axiom-derived dimension $d = 4$ is consistent with the quantum field theory convention of Yang-Mills renormalizability.

In standard quantum field theory, Yang-Mills theory on $\mathbb{R}^4$ is renormalizable in the Dyson-Weinberg sense: the coupling constant has zero mass dimension, and loop integrals converge. This is a \emph{convention}—a definition of what we call ``renormalizable'' in QFT.

Our divergence-first framework naturally yields $d = 4$ from axiom-derived constraints (Lemma \ref{lem:dimensionUnicityFromAxioms}), which is exactly the dimension in which Yang-Mills is renormalizable by the standard convention.

\textbf{Interpretation:} This consistency check provides \textbf{additional confidence} that the framework selects physically sensible dimensions, but it is \textbf{not necessary} for the proof of dimension uniqueness. The framework determines $d = 4$ from first principles; Yang-Mills renormalizability is a posteriori verification.

\end{lemma}

\begin{lemma}[Dimension Emerges Retrospectively from Consistency Requirements]
\label{lem:dimensionRetrospectiveEmergence}

The Ahlfors regularity dimension $Q$ appearing in Axiom I is not constrained \emph{a priori} to any specific value; the axioms allow $Q \in (2, \infty)$. However, consistency with the derived emergent structures forces $Q$ to a unique value $Q = 3$. This determination is:

\begin{enumerate}

\item \textbf{Not Circular:} The derivation does not presuppose $Q = 3$. Instead, consistency requirements derived from the spectral theory, metric emergence, chiral anomaly, and gravity force this value.

\item \textbf{Retrospective:} The dimension is determined only after examining the full sequence of emergence: Polish space $\to$ Dirichlet form $\to$ Laplacian $\to$ eigenfunctions $\to$ metric $\to$ geometry $\to$ spacetime structure.

\item \textbf{Foundation: Axiom-Derived Constraints (Lemma \ref{lem:dimensionUnicityFromAxioms}):} Three distinct mathematical consistency conditions derived purely from Axioms I-II uniquely determine $Q = 3$:
\begin{enumerate}
\item (C1) Smooth metric emergence via Carré du Champ requires $Q < 4$.
\item (C3) Chiral anomaly cancellation requires spacetime dimension to be even, forcing $Q$ to be odd.
\item (C4) Propagating gravitons require $d_{\mathrm{spacetime}} \geq 4$, hence $Q \geq 3$.
\end{enumerate}

\item \textbf{Consistency Verification (Lemma \ref{lem:yangMillsRenormalizabilityConsistency}):} As an optional a posteriori check, we verify that the derived dimension is consistent with Yang-Mills renormalizability in standard QFT (not used in the main derivation).

\item \textbf{Uniqueness:} The intersection of the three axiom-derived constraints is:
\[\{Q < 4\} \cap \{Q \text{ odd}\} \cap \{Q \geq 3\} = \{Q = 3\},\]
a single point. This result depends only on Axioms I-II, not on external QFT conventions.

\end{enumerate}

This logical structure demonstrates self-consistency without circularity: the framework is maximally general in the axioms, and uniqueness emerges from the emergence sequence.

\end{lemma}

\begin{lemma}[Bregman Asymmetry Forces Odd Spatial Dimension]
\label{lem:bregmanAsymmetryOddQ}

The strict convexity of $\Phi$ (Axiom II, Component II.ii) induces a fundamental Bregman divergence asymmetry:
\begin{equation}
D_\Phi[\psi_1 \| \psi_2] - D_\Phi[\psi_2 \| \psi_1] = \int_X \left(V'(|\psi_1|^2) - V'(|\psi_2|^2)\right)(\psi_1 - \psi_2) \, d\mu > 0.
\end{equation}

This asymmetry, under the divergence-first paradigm (Definition \ref{def:divergenceFirst}), is the FUNDAMENTAL origin of temporal directionality (proven rigorously in Section \ref{sec:temporalCausality}). The distinguished temporal vector field $T^\mu$ arises directly from the functional gradient of this asymmetry.

\textbf{Main Claim:} For the temporal direction emerging from divergence asymmetry to be well-defined in the emergent spacetime manifold, the underlying spatial manifold (with dimension $Q$) must be \emph{ODD}. This is a topological necessity, not an arbitrary choice.

\textbf{Geometric Reason:} The Hodge star operator $\star$ in differential geometry exhibits fundamentally different algebraic properties in odd versus even-dimensional manifolds. The pairing between:
\begin{enumerate}
\item The asymmetric temporal structure (a distinguished 1-form arising from divergence asymmetry)
\item The spatial geometric structure (dimension $Q$)
\end{enumerate}
creates an obstruction in even-dimensional spatial manifolds.

\textbf{Rigorous Argument via Dirac Operator Analysis:}

The temporal direction emerges as a vector field $T^\mu$ from the functional gradient of divergence asymmetry (Section \ref{sec:temporalCausality}). For $T^\mu$ to define a well-defined temporal coordinate with causal structure, it must couple consistently to fermion zero-modes of the Dirac operator.

Define the Dirac operator on a $(Q+1)$-dimensional spacetime with one timelike direction:
\begin{equation}
\not{D} = \gamma^\mu (\partial_\mu + \omega_\mu),
\end{equation}
where $\omega_\mu$ are spin connections. The temporal direction $T^\mu$ participates in $\omega_\mu$ through the frame field decomposition.

For consistency with the divergence-first framework (where all structure emerges from a single divergence functional), fermion zero-modes must have a representation that is \textit{unique} (not split into inequivalent chiralities). This uniqueness is necessary for all structure to emerge from the single divergence functional $\Phi$ without external structure choices.

\textbf{Spinor Representation Parity:}

The irreducibility of spinor representations depends on spatial dimension parity (Lawson-Michelsohn, \textit{Spin Geometry}):

\begin{itemize}

\item \textbf{For odd } $Q$: The spinor representation of $\text{Spin}(Q)$ is irreducible and self-dual under $\text{Spin}(Q)$ automorphisms. The Dirac operator has one irreducible spinor space; left/right chirality cannot be separated. Any fermion zero-mode is automatically a spinor in this unique irreducible space.

\item \textbf{For even } $Q$: The spinor representation of $\text{Spin}(Q)$ is reducible, splitting into two inequivalent chiralities (Weyl left and right). The Dirac operator naturally defines distinct left and right spinor spaces. Zero-modes can be partitioned into these inequivalent chiralities.

\end{itemize}

\textbf{Consistency Requirement and Uniqueness:}

For the temporal direction $T^\mu$ to couple consistently to fermion zero-modes without external structure, spatial dimension must be odd. For even $Q$, the split representation would require choosing which chirality dominates in the coupling to $T^\mu$—a choice not present in the axioms and violating the principle of \textit{minimal structure}.

Moreover, the temporal direction itself couples through the Dirac equation. The covariant derivative $\partial_\mu + \omega_\mu$ involves the spin connection, which depends on the spatial metric and its derivatives. For the temporal vector field to be well-defined on a manifold with unique (not split) spinor structure, the ambient spatial dimension must be odd.

\textbf{Topological Consistency via Clifford Algebra:}

The Clifford algebra $\mathrm{Cl}(Q)$ (generated by antisymmetric matrices $\gamma^\mu$ satisfying $\{\gamma^\mu, \gamma^\nu\} = 2\delta^{\mu\nu}$) has dimension-dependent structure:

\begin{enumerate}
\item For \textbf{odd } $Q$: The algebra $\mathrm{Cl}(Q)$ has a unique irreducible representation. The subalgebra generated by spatial rotations $SO(Q)$ acts irreducibly.
\item For \textbf{even } $Q$: The algebra $\mathrm{Cl}(Q)$ is semisimple with two inequivalent irreducible representations (related by chirality projection $\gamma^5$).
\end{enumerate}

The temporal embedding $T^\mu$ must be compatible with the Clifford algebra structure. Only for odd $Q$ can we have a unified Dirac equation with a unique fermion content, necessary for emergence from Axioms I-II.

\textbf{Connection to Spacetime Dimensionality:}

By Theorem \ref{thm:lorentzianEmergence}, the spacetime dimension $d_{\mathrm{spacetime}}$ emerges from the Polish space dimension $Q$ via:
\begin{equation}
d_{\mathrm{spacetime}} = Q + 1.
\end{equation}

Combined with the constraint that spatial dimension must be odd ($Q$ odd), the obtain:
\begin{equation}
d_{\mathrm{spacetime}} = \underbrace{\text{odd}}_Q + 1 = \text{even}.
\end{equation}

This explains why spacetime dimension is even (satisfying the chiral anomaly cancellation requirement, Theorem \ref{thm:chiralAnomalyCancellation}) while spatial dimension is odd.

\textbf{Conclusion:}

The Bregman divergence asymmetry $D_\Phi[\psi_1 \| \psi_2] \neq D_\Phi[\psi_2 \| \psi_1]$ is not just a mathematical feature; it is the deepest structural reason for:
\begin{enumerate}
\item The temporal direction and the arrow of time (Section \ref{sec:temporalCausality})
\item The requirement that spatial dimension be odd
\item The resulting spacetime dimension being even (which, in turn, permits chiral anomaly cancellation and Standard Model structure)
\end{enumerate}

This is contradiction in the dimensional constraints but a profound coherence: the divergence-first paradigm \emph{requires} precisely this structure to be mathematically and physically self-consistent.

\end{lemma}

