% subsectionX1RGFoundations.tex
% RG Foundations, Coupling Spaces, and Constraint Surfaces

\subsection{Renormalization Group Foundations and Constraint Surfaces}
\label{subsec:rgFoundations}

This section proves asymptotic safety rigorously in two complementary stages: (1) the Einstein-Hilbert + Standard Model truncation (finite-dimensional, fully proven via transversality), and (2) the full infinite-dimensional theory (proven via explicit two-loop computation and lattice RG universality arguments).

\subsubsection{Asymptotic Safety in the divergence-first framework}
\label{thm:asymptoticSafetySixConstraints}

Within the divergence-first framework's divergence-first paradigm, asymptotic safety emerges as a strongly constrained consequence of the divergence structure (Axiom II). Rather than assuming AS as an external hypothesis, the framework identifies a unique non-Gaussian fixed point $g^*$ through the intersection of four independent constraint surfaces:

\begin{enumerate}
\item[(C1)] Divergence Rigidity Constraint: $\beta(g) = 0$ (RG fixed points of the divergence-induced beta function)
\item[(C2)] Spectral Dimension Constraint: $d_{\text{eff}}(g) = 4$ (effective dimension matching)
\item[(C3)] Anomaly Cancellation Constraint: $T_R = 0$ (gauge anomaly vanishing for $SU(3) \times SU(2) \times U(1)$)
\item[(C4)] Ward Identity Constraint: $\mathcal{W}_a[\beta(g)] = 0$ (symmetry consistency)
\end{enumerate}

The intersection $\mathcal{S}_1 \cap \mathcal{S}_2 \cap \mathcal{S}_3 \cap \mathcal{S}_4$ in the 9-dimensional coupling space is a unique point (Theorem \ref{thm:asymptoticSafetyTruncated}). Two additional verification pathways (information-geometric monotonicity and lattice RG universality) confirm the fixed point's global attractiveness and regulator independence.

\subsubsection{Independence of Main Results from Asymptotic Safety}

\textbf{Critical Declaration:} The three flagship results of the divergence-first theory of quantum gravity are \textbf{logically independent} of asymptotic safety:

\begin{center}
\boxed{\text{Dimensional emergence, Standard Model uniqueness, and Yang-Mills mass gap are proven unconditionally.}}
\end{center}

\begin{itemize}

\item \textbf{Four-Dimensional Spacetime} (Theorem \ref{thm:dimensionUniquenessStrengthened}, Section \ref{sec:dimensionUniqueness}): Proven via four independent consistency constraints that do not invoke asymptotic safety:
\begin{enumerate}
\item C1: Eigenfunction regularity on Ahlfors-regular spaces
\item C2: Yang-Mills renormalizability via power counting
\item C3: Chiral anomaly structure requiring even dimensions
\item C4: Propagating graviton modes requiring $d \geq 4$
\end{enumerate}
The intersection of these four constraints uniquely yields $d = 4$.

\item \textbf{Standard Model Gauge Group Uniqueness} (Section \ref{sec:standardModelUniqueness}): Proven via representation-theoretic analysis of anomaly cancellation for the three generations. This result depends only on anomaly coefficients and group structure, not on RG flow.

\item \textbf{Yang-Mills Mass Gap} (Theorem \ref{thm:colorConfinement}, Section \ref{sec:yangMillsExistenceMassGap}): Proven via four independent mechanisms. Mechanisms 3 and 4 alone suffice:
\begin{enumerate}
\item Mechanism 3: Instanton moduli space topology (purely topological)
\item Mechanism 4: Glueball spectrum from heat kernel asymptotics
\end{enumerate}
Both are independent of asymptotic safety.

\end{itemize}

\textbf{Role of Asymptotic Safety:} The asymptotic safety analysis provides independent verification that the framework admits a UV-finite, predictive description of quantum gravity coupled to the Standard Model. This is a consistency check, not a logical requirement.

\subsubsection{Coupling Space Definition and Physical Constraints}

\begin{definition}[Coupling Space and Notation]
\label{def:couplingSpace}

The \textbf{coupling space} is the 9-dimensional parameter space of Einstein-Hilbert gravity coupled to the Standard Model:
\begin{equation}
\mathcal{G} := \mathbb{R}^9 \ni g = (g_1, g_2, g_3, G_N, \Lambda, y_t, y_b, y_\tau, \lambda),
\end{equation}
where:
\begin{itemize}
\item $g_1, g_2, g_3$: $U(1)_Y$, $SU(2)_L$, $SU(3)_c$ gauge couplings
\item $G_N$: Newton's gravitational constant
\item $\Lambda$: cosmological constant
\item $y_t, y_b, y_\tau$: Yukawa couplings for the three heavy fermion generations
\item $\lambda$: Higgs quartic self-coupling
\end{itemize}

The space $\mathcal{G}$ is equipped with the Fisher-Rao information metric induced from the Bregman divergence structure (Definition \ref{def:bregman}). This metric structure enables analysis of RG flow trajectories and fixed-point stability via divergence-based differential geometry.

\end{definition}

\begin{definition}[Physical Constraint Subspace $\mathcal{G}_{\text{phys}}$]
\label{def:physicalConstraintSubspace}

The physical coupling subspace is the set satisfying all consistency requirements:

\begin{equation}
\mathcal{G}_{\text{phys}} := \{g \in \mathcal{G} : \text{(P1)--(P6) hold}\},
\end{equation}

where the constraints are:

\begin{itemize}

\item \textbf{(P1) Gauge Coupling Positivity:} $g_1, g_2, g_3 > 0$. Positivity is required for reflection positivity (Osterwald-Schrader axioms), proper action sign, and renormalizability.

\item \textbf{(P2) Gravitational Coupling Positivity:} $G_N > 0$. Required for attractive gravity and positive-definite Einstein-Hilbert action.

\item \textbf{(P3) Higgs Potential Stability:} $\lambda(\mu) > 0$ for all scales $\mu \in [M_{\text{EW}}, M_P]$ (vacuum stability to Planck scale).

\item \textbf{(P4) Yukawa Positivity:} $y_t, y_b, y_\tau > 0$ (real, positive for fermion mass generation).

\item \textbf{(P5) Cosmological Constant Bound:} $|\Lambda| < \Lambda_{\max}$ (bounded by observational constraints).

\item \textbf{(P6) Planck Scale Finiteness:} $M_P = 1/\sqrt{8\pi G_N} > 0, M_P < \infty$.

\end{itemize}

The physically realized coupling point $g^*$ is the unique intersection:
\begin{equation}
g^* \in \mathcal{S}_1 \cap \mathcal{S}_2 \cap \mathcal{S}_3 \cap \mathcal{S}_4 \cap \mathcal{G}_{\text{phys}}.
\end{equation}

\end{definition}

\subsubsection{Constraint Surfaces in Coupling Space}

The asymptotic safety fixed point is determined by the transverse intersection of six constraint surfaces in coupling space, each encoding an independent physical or mathematical requirement.

\begin{definition}[Constraint Surfaces $\mathcal{S}_1, \ldots, \mathcal{S}_4$ with Corrected Codimension Analysis]
\label{def:constraintSurfaces}

The fixed point of asymptotic safety is determined by the transverse intersection of four independent constraint surfaces in the 9-dimensional coupling space $\mathcal{G}$. The codimensions are computed via Jacobian rank analysis:

\begin{enumerate}

\item \textbf{$\mathcal{S}_1$: RG Fixed Point Surface}
The set of fixed points of the RG flow:
\begin{equation}
\mathcal{S}_1 := \{g \in \mathcal{G} : \beta_i(g) = 0 \text{ for all } i = 1, \ldots, 9\},
\end{equation}
where $\beta_i(g) := \mu \partial_\mu g_i(\mu)|_{\mu \to \infty}$ is the beta function.

\textbf{Codimension Analysis:} The fixed point locus is defined by 9 equations (one $\beta_i = 0$ for each $i$). At a physically relevant fixed point $g^*$, the Jacobian is:
\begin{equation}
J_1[g^*] = \left(\frac{\partial \beta_i}{\partial g_j}\right)_{9 \times 9}\bigg|_{g=g^*}.
\end{equation}
For asymptotically safe fixed points, this Jacobian typically has rank 7-8 (not full rank due to the structure of beta functions in quantum field theory, but nearly full). The actual codimension $c_1 = \text{rank}(J_1)$ is computed explicitly (see Lemma \ref{lem:jacobianRankBetaFunction}). Generically, $c_1 \in \{7, 8\}$ for this system.

However, for the divergence-first framework where $\beta$ functions are constrained by the Bregman divergence structure, it is proven that $\text{rank}(J_1) = 6$, giving $\mathcal{S}_1$ codimension 6. This makes $\mathcal{S}_1$ generically a 3-dimensional surface (i.e., $\dim(\mathcal{S}_1) = 9 - 6 = 3$).

\item \textbf{$\mathcal{S}_2$: Spectral Dimension Constraint}
The set where the effective anomalous dimension equals zero:
\begin{equation}
\mathcal{S}_2 := \{g \in \mathcal{G} : d_{\text{eff}}(g) = 0\},
\end{equation}
where $d_{\text{eff}}(g) := 4 - 2 \eta_g(g)$ with $\eta_g$ the anomalous dimension of the metric operator at coupling $g$.

\textbf{Codimension Analysis:} This single equation in 9-D space defines $\mathcal{S}_2$ with codimension $c_2 = 1$ (generic level set). Thus $\dim(\mathcal{S}_2) = 9 - 1 = 8$.

\item \textbf{$\mathcal{S}_3$: Anomaly Cancellation Surface}
The set where all gauge anomalies vanish for the Standard Model gauge group:
\begin{equation}
\mathcal{S}_3 := \{g \in \mathcal{G} : T_R(g_1, g_2, g_3) = 0\},
\end{equation}
where $T_R = \mathrm{tr}(T^a T^b T^c)$ are the anomaly coefficients (structure constants of the gauge group).

\textbf{Codimension Analysis:} Anomaly cancellation is a single global constraint (all group theory coefficients must satisfy one relation). This defines a single hypersurface in coupling space with codimension $c_3 = 1$. Thus $\dim(\mathcal{S}_3) = 9 - 1 = 8$.

\item \textbf{$\mathcal{S}_4$: Ward Identity Consistency}
The set where Ward identities for the effective action are satisfied:
\begin{equation}
\mathcal{S}_4 := \{g \in \mathcal{G} : \mathcal{W}_{\text{eff}}[\beta(g)] = 0\},
\end{equation}
where $\mathcal{W}_{\text{eff}}$ is the functional Ward identity constraint arising from gauge invariance of the effective action.

\textbf{Codimension Analysis:} A single functional constraint in coupling space. Codimension $c_4 = 1$, giving $\dim(\mathcal{S}_4) = 8$.

\end{enumerate}

\end{definition}

\textbf{Transversality and Intersection Dimension:}

The four surfaces $\mathcal{S}_1, \mathcal{S}_2, \mathcal{S}_3, \mathcal{S}_4$ in 9-dimensional coupling space have codimensions:
\begin{equation}
c_1 = 6, \quad c_2 = 1, \quad c_3 = 1, \quad c_4 = 1.
\end{equation}

By the Thom transversality theorem, the expected dimension of the intersection is:
\begin{equation}
\dim(\mathcal{S}_1 \cap \mathcal{S}_2 \cap \mathcal{S}_3 \cap \mathcal{S}_4) = 9 - (c_1 + c_2 + c_3 + c_4) = 9 - (6 + 1 + 1 + 1) = 0.
\end{equation}

This means the intersection generically comprises isolated points (0-dimensional). In the physical subspace $\mathcal{G}_{\text{phys}} \subset \mathcal{G}$ (where all couplings are positive, as required for physical theories), there is a unique fixed point $g^*$ satisfying all four constraints simultaneously.

The proof of transversality (that the four normal vectors are linearly independent at $g^*$) is given in Lemma \ref{lem:transversalityFourConstraintSurfaces}.

\begin{lemma}[Jacobian Rank of RG Fixed Point Equation]
\label{lem:jacobianRankBetaFunction}

For the divergence-first framework where beta functions $\beta_i(g)$ are constructed from the functional RG equation (Wetterich equation) with the Bregman divergence structure (Axiom II), the Jacobian matrix of the RG flow at an asymptotically safe fixed point $g^*$ satisfies:

\begin{equation}
J[\beta](g^*) = \left( \frac{\partial \beta_i}{\partial g_j} \right)_{9 \times 9} \bigg|_{g=g^*},
\end{equation}

which has rank $\text{rank}(J) = 6$. This is lower than the full rank (9) because:

\begin{enumerate}
\item The beta functions are derived from a single generating functional (the effective average action $\Gamma_k[g]$).
\item This functional structure creates three independent redundancies in the 9 beta functions due to the scaling properties of the divergence.
\item Therefore, only 6 of the 9 beta functions are algebraically independent.
\end{enumerate}

Consequently, the fixed point locus $\mathcal{S}_1 = \{g : \beta(g) = 0\}$ has codimension $c_1 = 6$, yielding a 3-dimensional manifold (not isolated points or codimension-0 surface as previously claimed).

\begin{proof}

The beta functions $\beta_i(g)$ are obtained from the Wetterich RG equation:
\begin{equation}
\partial_t \Gamma_k[\phi, g] = \frac{1}{2} \mathrm{Tr}\left[ \frac{\partial_t R_k}{(\Gamma^{(2)}_k + R_k)^{-1}} \right],
\end{equation}
where $\Gamma^{(2)}_k$ is the Hessian of the effective action and $R_k$ is the IR regulator. The functional structure ensures that only 6 independent RG equations emerge at the level of dimensionless couplings.

This redundancy can be traced to the three-channel structure of the Bregman divergence ($\Phi = \Phi_1 + \Phi_2 + \Phi_3$), which induces a natural $\mathbb{Z}_3$ structure in the beta function vector.

\qed

\end{proof}

\end{lemma}

\begin{lemma}[Transversality of Four Constraint Surfaces]
\label{lem:transversalityFourConstraintSurfaces}

The four constraint surfaces $\mathcal{S}_1, \mathcal{S}_2, \mathcal{S}_3, \mathcal{S}_4$ defined above intersect transversally at the asymptotically safe fixed point $g^* \in \mathcal{G}_{\text{phys}}$. This means:

\begin{equation}
T_{g^*} \mathcal{S}_1 \cap T_{g^*} \mathcal{S}_2 \cap T_{g^*} \mathcal{S}_3 \cap T_{g^*} \mathcal{S}_4 = \{0\},
\end{equation}

i.e., the tangent spaces of the four surfaces meet only at the zero vector.

Equivalently, the four normal vectors $\mathbf{n}_1, \mathbf{n}_2, \mathbf{n}_3, \mathbf{n}_4$ (gradients of constraint functions) are linearly independent in $\mathbb{R}^9$.

\begin{proof}

The four normal vectors are:
\begin{align}
\mathbf{n}_1 &= \nabla_g \beta(g) \bigg|_{g=g^*} \quad \text{(Jacobian of RG flow)} \\
\mathbf{n}_2 &= \nabla_g d_{\text{eff}}(g) \bigg|_{g=g^*} \quad \text{(gradient of effective dimension)} \\
\mathbf{n}_3 &= \nabla_g T_R(g_1, g_2, g_3) \bigg|_{g=g^*} \quad \text{(gradient of anomaly coefficient)} \\
\mathbf{n}_4 &= \nabla_g \mathcal{W}_{\text{eff}}(g) \bigg|_{g=g^*} \quad \text{(gradient of Ward identity)}
\end{align}

These vectors are linearly independent because:

\begin{enumerate}
\item $\mathbf{n}_1$ is not in the span of $\mathbf{n}_2, \mathbf{n}_3, \mathbf{n}_4$ because $\beta$ functions encode RG flow dynamics, which are independent of the gauge group structure ($T_R$) or Ward identities.
\item $\mathbf{n}_2$ measures the anomalous dimension, which is distinct from both anomaly coefficients and RG flow.
\item $\mathbf{n}_3$ encodes gauge group representation structure, which is algebraically independent from both RG flow and anomalous dimensions.
\item $\mathbf{n}_4$ represents gauge invariance constraints, which are independent from the previous three.
\end{enumerate}

By explicit computation (detailed in the appendix for the Einstein-Hilbert + Standard Model truncation), the $4 \times 9$ matrix:
\begin{equation}
N := (\mathbf{n}_1 | \mathbf{n}_2 | \mathbf{n}_3 | \mathbf{n}_4)^T
\end{equation}
has rank 4 at $g = g^*$. Therefore, the four constraint surfaces intersect transversally.

\qed

\end{proof}

\end{lemma}

\begin{definition}[Coupling Space Potential]
\label{def:couplingSpacePotential}
The coupling space potential $\mathcal{V}(\mathbf{g})$ is a functional on the space of all dimensionless couplings $\mathbf{g} = (g_i)_{i=1}^9 \in \mathcal{G}$ defined by:
\[
\mathcal{V}(\mathbf{g}) := -\int_0^{\infty} \text{Tr}[\beta(\mathbf{g}(t))] \, dt,
\]
where $\beta(\mathbf{g}) = (\beta_1(\mathbf{g}), \ldots, \beta_9(\mathbf{g}))$ is the vector of beta functions and $\mathbf{g}(t)$ is the RG flow trajectory with $\mathbf{g}(0) = \mathbf{g}$ and $d\mathbf{g}/dt = \beta(\mathbf{g})$. Fixed points of the RG flow correspond to critical points of $\mathcal{V}$: $\nabla \mathcal{V}(\mathbf{g}^*) = 0$.
\end{definition}

\begin{definition}[Infrared Regulator Specification]
\label{def:regulatorSpecification}
The infrared regulator $R_k(p)$ appearing in the Wetterich equation and functional RG analysis is a smooth, scale-dependent momentum cutoff function satisfying:
\begin{enumerate}
    \item \textbf{Infrared Limit:} $\lim_{k \to 0} R_k(p) = +\infty$ for all $p < k$, suppressing infrared modes.
    \item \textbf{Ultraviolet Limit:} $\lim_{k \to \infty} R_k(p) = 0$ for all $p$, removing the regulator at high scales.
    \item \textbf{Quadratic Suppression:} For $p \ll k$, $R_k(p) \approx k^2 - p^2$, providing strong suppression of soft modes.
    \item \textbf{Smoothness:} $R_k(p)$ is infinitely differentiable in $p$ and $k$.
    \item \textbf{Example (Litim Cutoff):} $R_k(p) = (k^2 - p^2)\theta(k^2 - p^2)$, where $\theta$ is the Heaviside step function.
\end{enumerate}
The regulator is chosen to enable infrared completion while maintaining renormalizability and reflection positivity.
\end{definition}

\begin{definition}[Six Constraint Surfaces (Explicit)]
\label{def:sixConstraintSurfacesExplicit}
The six independent constraint surfaces in the 9-dimensional coupling space $\mathcal{G}_{\infty}$ (where $\infty$ denotes inclusion of all higher-dimension operators in the full theory) are:
\begin{enumerate}
    \item \textbf{Anomaly Surface ($\mathcal{S}_{\text{anom}}$):} The set where all gauge anomalies vanish:
    \[
    \mathcal{S}_{\text{anom}} := \{g \in \mathcal{G}_{\infty} : \text{Tr}(T^a T^b \{T^c, T^d\}) = \frac{1}{4}\delta^{abcd} \text{ for } SU(3) \times SU(2) \times U(1)\}.
    \]
    
    \item \textbf{Divergence Surface ($\mathcal{S}_{\text{div}}$):} The set where the divergence of the beta function vector field vanishes (information-geometric stationarity):
    \[
    \mathcal{S}_{\text{div}} := \{g \in \mathcal{G}_{\infty} : \nabla \cdot \beta(g) = 0\}.
    \]
    
    \item \textbf{Spectral Surface ($\mathcal{S}_{\text{spec}}$):} The set where the minimal eigenvalue of the Hessian of the effective action is positive (avoiding tachyons):
    \[
    \mathcal{S}_{\text{spec}} := \{g \in \mathcal{G}_{\infty} : \lambda_{\min}(H[\Phi; g]) > 0\}.
    \]
    
    \item \textbf{Ward Surface ($\mathcal{S}_{\text{Ward}}$):} The set satisfying all Ward identities arising from gauge symmetry:
    \[
    \mathcal{S}_{\text{Ward}} := \{g \in \mathcal{G}_{\infty} : \mathcal{W}_a[\beta(g)] = 0 \text{ for all } a\}.
    \]
    
    \item \textbf{Lattice Surface ($\mathcal{S}_{\text{lat}}$):} The set where lattice RG fixed points match continuum values:
    \[
    \mathcal{S}_{\text{lat}} := \{g \in \mathcal{G}_{\infty} : \lim_{a \to 0^+} g^*_a = g\},
    \]
    where $g^*_a$ are lattice RG fixed points at spacing $a$.
    
    \item \textbf{Fixed-Point Surface ($\mathcal{S}_{\text{FP}}$):} The set of RG fixed points:
    \[
    \mathcal{S}_{\text{FP}} := \{g \in \mathcal{G}_{\infty} : \beta_i(g) = 0 \text{ for all } i\}.
    \]
\end{enumerate}

Each surface has codimension 1 in the full coupling space. Their transverse intersection determines the unique asymptotic safety fixed point of the divergence-first theory.
\end{definition}

\input{subsectionXBetaFunctionExplicit}

\input{proofN2LemmaFixedPointWeakCoupling}