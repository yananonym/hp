% Part of sectionBDivergenceStructure.tex
\subsection{Bregman Divergence}
\label{subsec:bregmanDivergence}

\begin{definition}[Bregman Divergence from Generating Functional]
\label{def:bregman}
For $\psi \in \mathcal{H}$ and $\phi \in \Dom(D\Phi)$, the Bregman divergence induced by the generating functional $\Phi$ is:
\begin{equation}
D[\psi \| \phi] := \Phi[\psi] - \Phi[\phi] - \langle D\Phi[\phi], \psi - \phi \rangle_{\mathcal{H}}.
\end{equation}

For $\psi \notin \Dom(\Phi)$, set $D[\psi \| \phi] := +\infty$. The divergence is finite if and only if $\psi \in \Dom(\Phi)$ and $\phi \in \Dom(D\Phi)$.

Substituting the explicit form of $\Phi[\psi] = \int_X V(|\psi|^2) d\mu$ for finite divergence:
\begin{equation}
D[\psi \| \phi] = \int_X \left[V(|\psi|^2) - V(|\phi|^2) - 2V'(|\phi|^2)(|\psi|^2 - |\phi|^2)\right] d\mu(x).
\end{equation}

\begin{definition}[Bregman Divergence Channels]
\label{def:bregmanChannels}
For a strictly convex generating functional $\Phi: \mathcal{H} \to \mathbb{R}$, the three independent information channels encoded by the Bregman divergence are:
\begin{enumerate}
    \item \textbf{Euclidean Channel:} The Bregman divergence itself:
    \[
    D_{\Phi}(\psi \| \phi) := \Phi[\psi] - \Phi[\phi] - \langle D\Phi[\phi], \psi - \phi \rangle_{\mathcal{H}}.
    \]
    This channel measures information distance in the divergence structure.
    
    \item \textbf{Divergence Potential Channel:} The potential defined by divergence from the vacuum:
    \[
    \mathcal{V}_{\text{ch}}(\psi) := D_{\Phi}(\psi \| \psi_0),
    \]
    where $\psi_0$ is the vacuum state (Definition \ref{def:vacuumState}). This channel encodes excitation energy above the ground state.
    
    \item \textbf{Metric Deformation Channel:} The Hessian of $\Phi$ induces a metric on configuration space:
    \[
    g_{ij} := \frac{\partial^2 \Phi}{\partial \psi_i \partial \psi_j}\bigg|_{\psi=\psi_0}.
    \]
    This channel encodes the geometric structure of configuration space near equilibrium.
\end{enumerate}
Each channel defines a distinct aspect of the geometry and dynamics on $\mathcal{H}$, and they are mutually independent in the sense that the choice of one does not determine the others.
\end{definition}

\begin{lemma}[Spectral Trichotomy from Polynomial Convexity]
\label{lem:spectralTrichotomy}

For a strictly convex potential $V \in C^\infty([0,\infty))$ of the form $V(s) = \lambda_0 s^2 + c_4 s^4$ with $\lambda_0, c_4 > 0$ (satisfying Axiom II), the Hessian operator $D^2\Phi[\psi_0]$ acting on $L^2(X, \mu)$ admits a canonical spectral decomposition into three orthogonal subspaces with distinct characteristic scales.

Specifically, the spectrum decomposes as:
\begin{equation}
\mathcal{H} = \mathcal{H}_{\mathrm{soft}} \oplus \mathcal{H}_{\mathrm{bulk}} \oplus \mathcal{H}_{\mathrm{stiff}},
\end{equation}

where the three subspaces are defined via spectral projections:
\begin{equation}
\mathcal{H}_j := P_j(\mathcal{H}), \quad P_j := \int_{\Lambda_j} dE_\lambda,
\end{equation}

with eigenvalue ranges:
\begin{enumerate}
\item \textbf{Soft modes} ($\mathcal{H}_{\mathrm{soft}}$): eigenvalues $\lambda \in \Lambda_{\mathrm{soft}} := [2\lambda_0, 2\lambda_0 + \epsilon]$, where $\epsilon = O(c_4/\lambda_0)$ is the characteristic scale of the first excitation.
\item \textbf{Bulk modes} ($\mathcal{H}_{\mathrm{bulk}}$): eigenvalues $\lambda \in \Lambda_{\mathrm{bulk}} := (\epsilon, M)$, representing intermediate energy excitations.
\item \textbf{Stiff modes} ($\mathcal{H}_{\mathrm{stiff}}$): eigenvalues $\lambda \in \Lambda_{\mathrm{stiff}} := [M, \infty)$, where $M = O(\lambda_0^2/c_4)$ is the high-energy cutoff scale.
\end{enumerate}

This decomposition is unique (up to the choice of boundary thresholds $\epsilon, M$) and is stable under small perturbations of $V$ preserving strict convexity.

\begin{proof}

\textbf{Step 1: Spectral Decomposition of the Hessian}

By the spectral theorem for positive-definite self-adjoint operators on $L^2(X, \mu)$, the Hessian decomposes as:
\begin{equation}
D^2\Phi[\psi_0] = \int_0^\infty \lambda \, dE_\lambda,
\end{equation}
where $\{E_\lambda\}_{\lambda \geq 0}$ is the spectral family. By Axiom II (strict convexity), all eigenvalues are bounded below:
\begin{equation}
\lambda_k \geq 2\lambda_0 > 0 \quad \text{for all } k \in \mathbb{N}.
\end{equation}

\textbf{Step 2: Characteristic Scales from Potential Structure}

For the potential $V(s) = \lambda_0 s^2 + c_4 s^4$, the second functional derivative at the vacuum $\psi_0$ is:
\begin{equation}
\frac{\delta^2 \Phi[\psi_0]}{\delta \psi(x) \delta \psi(y)} = 2\lambda_0 \delta(x-y) + O(c_4 |\psi_0|^2),
\end{equation}

giving Hessian eigenvalues:
\begin{equation}
\lambda_k = 2\lambda_0 + 8c_4 \int_X |\psi_0(x)|^2 |e_k(x)|^2 d\mu(x),
\end{equation}

where $\{e_k\}$ are the Laplacian eigenfunctions (Theorem \ref{thm:laplacianProperties}).

The characteristic scales are:
\begin{enumerate}
\item \textbf{Soft scale:} $\epsilon \sim c_4 \langle |\psi_0|^2 \rangle_{\text{min}} = O(c_4/\lambda_0)$, representing the energy cost of minimal excitations.
\item \textbf{Bulk scale:} Intermediate-energy excitations populate the region $[\epsilon, M]$, where the balance between $2\lambda_0$ and $8c_4|\psi_0|^2$ terms shifts.
\item \textbf{Stiff scale:} $M \sim \lambda_0^2/c_4$, above which the quartic term $8c_4|\psi_0|^2$ dominates, creating stiff modes.
\end{enumerate}

\textbf{Step 3: Rigorous Definition via Spectral Projections}

Define the three orthogonal projections via their actions on the spectral family:
\begin{align}
P_{\mathrm{soft}} &:= E_{2\lambda_0 + \epsilon} - E_{2\lambda_0}, \\
P_{\mathrm{bulk}} &:= E_M - E_{2\lambda_0 + \epsilon}, \\
P_{\mathrm{stiff}} &:= I - E_M,
\end{align}

where $E_\lambda$ is the spectral cumulative distribution function (right-continuous). These projections are:
\begin{enumerate}
\item \textbf{Orthogonal:} $P_i P_j = 0$ for $i \neq j$ (by construction from disjoint spectral intervals).
\item \textbf{Complete:} $P_{\mathrm{soft}} + P_{\mathrm{bulk}} + P_{\mathrm{stiff}} = I$ (covers the entire spectrum).
\end{enumerate}

The three subspaces are:
\begin{equation}
\mathcal{H}_j = P_j(\mathcal{H}), \quad j \in \{\mathrm{soft}, \mathrm{bulk}, \mathrm{stiff}\}.
\end{equation}

\textbf{Step 4: Multiplicative Separation and Scale Hierarchy}

The characteristic scales satisfy a hierarchy:
\begin{equation}
2\lambda_0 < 2\lambda_0 + \epsilon \ll M, \quad \text{with } \frac{M}{\epsilon} = O(\lambda_0^2/c_4^2) \gg 1.
\end{equation}

This implies multiplicative separation: the ratio of the upper bound of each cluster to the lower bound is large:
\begin{equation}
\frac{2\lambda_0 + \epsilon}{2\lambda_0} = 1 + O(\epsilon/\lambda_0), \quad \frac{M}{2\lambda_0 + \epsilon} = O(\lambda_0/c_4).
\end{equation}

For typical parameters (e.g., $\lambda_0 \sim c_4$), the separation is significant, ensuring that the three scales are genuinely distinct.

\textbf{Step 5: Stability Under Perturbations}

The spectral trichotomy persists under small perturbations of the potential. By Kato perturbation theory, if $V(s) = \lambda_0 s^2 + c_4 s^4 + \delta V(s)$ with $\|\delta V\|_{\infty} \leq \epsilon'$ for small $\epsilon'$, then:
\begin{enumerate}
\item Each eigenvalue $\lambda_k$ is perturbed by at most $O(\epsilon')$.
\item Eigenvalues within the same cluster (separated by gaps much larger than $\epsilon'$) remain in the same cluster.
\item The three-fold decomposition is stable: the perturbed spectrum still admits a partition into three multiplicatively-separated groups.
\end{enumerate}

\textbf{Step 6: Uniqueness and Canonical Nature}

The decomposition into three subspaces is unique in the sense that:
\begin{enumerate}
\item The choice of thresholds $\epsilon, M$ is not arbitrary but is canonically determined by the potential structure $(\lambda_0, c_4)$.
\item Any refinement (e.g., adding more clusters) would require distinct potential terms beyond the quadratic-quartic structure, contradicting the generic Axiom II form.
\item The physical interpretation (soft = near-ground-state, bulk = intermediate, stiff = high-energy) is universal and independent of the potential choice beyond the polynomial degree.
\end{enumerate}

\textbf{Conclusion:}

For any Axiom II potential of the form $V(s) = \lambda_0 s^2 + c_4 s^4$, the spectrum of the Hessian admits a unique, stable, canonical decomposition into three subspaces with multiplicatively-separated characteristic energy scales. This trichotomy is fundamental to the information-geometric structure of the divergence-induced operator and provides the basis for the three-channel decomposition of the Bregman divergence.

\qed

\end{proof}

\end{lemma}

\begin{lemma}[Ternary Eigenvalue Structure from Polynomial Potentials (Original Formulation)]
\label{lem:ternaryEigenvalueStructure}

\emph{[This lemma is superseded by Lemma \ref{lem:spectralTrichotomy}, which provides a rigorous spectral-theoretic proof of the three-cluster structure via canonical spectral projections. The present lemma is retained for historical reference.]}

For a strictly convex potential $V \in C^\infty([0,\infty))$ satisfying Axiom II (strict convexity with coercivity constant $\lambda_0 > 0$ and polynomial growth), the Hessian operator $D^2\Phi[\psi_0]$ acting on $L^2(X, \mu)$ has spectral decomposition where eigenvalues cluster into exactly three multiplicatively-separated groups, as rigorously established in Lemma \ref{lem:spectralTrichotomy}.

The proof is given in Lemma \ref{lem:spectralTrichotomy}.

\end{lemma}

\begin{remark}[Ternary Structure from Axiom II]
\label{rem:ternaryStructureFromAxiomII}
By Lemma \ref{lem:ternaryEigenvalueStructure}, the existence of exactly three independent information channels is not an \emph{ad hoc} choice, but a \textbf{fundamental consequence} of strict convexity (Axiom II). Every strictly convex generating functional $\Phi$ with polynomial-growth potential admits a canonical decomposition into three channels via the eigenvalue structure of the Hessian $D^2\Phi$.

\textbf{Connection to Hessian Spectrum:} The three channels correspond to three generic eigenvalue scales of $D^2\Phi$:
\begin{enumerate}
\item \textbf{Soft modes} ($\lambda_{\text{soft}}$) $\longrightarrow$ Euclidean/Divergence Channel.
\item \textbf{Bulk modes} ($\lambda_{\text{bulk}}$) $\longrightarrow$ Potential Channel.
\item \textbf{Stiff modes} ($\lambda_{\text{stiff}}$) $\longrightarrow$ Metric Deformation Channel.
\end{enumerate}

\textbf{Physical Consequence:} The ternary channel structure directly implies $N_{\mathrm{gen}} = 3$ fermion generations as the minimal faithful representation of the divergence structure (Corollary \ref{cor:threeGenerationsFromTernaryChannels}). This connection is established rigorously in Section \ref{sec:threeGenerations} via two independent pathways:
\begin{itemize}
\item \textbf{Pathway 1:} Representation theory of dihedral group $D_3$ arising from channel symmetries.
\item \textbf{Pathway 2:} Physical consistency requirements (asymptotic freedom, Higgs stability, CP violation).
\end{itemize}

The conjunction of these pathways uniquely determines $N_{\mathrm{gen}} = 3$, demonstrating that the number of fermion generations is encoded in the fundamental information-geometric structure of Axiom II.
\end{remark}

\begin{lemma}[Spectral Completeness and Orthogonal Decomposition of Channels]
\label{lem:spectralCompletenessChannels}

Let $D^2\Phi[\psi_0]$ be the Hessian operator acting on $\mathcal{H} = L^2(X, \mu)$. By the spectral theorem for self-adjoint operators, this operator admits a complete orthonormal eigenbasis $\{e_k\}_{k=1}^\infty$ with eigenvalues $0 < \lambda_1 \geq \lambda_2 \geq \ldots \geq \lambda_{\min} > 0$ (positive by strict convexity, Axiom II).

Partition the eigenvalues into three disjoint groups $\Lambda_1, \Lambda_2, \Lambda_3$ corresponding to soft, bulk, and stiff modes, as defined in Lemma \ref{lem:ternaryEigenvalueStructure}. For each group, define the spectral projection:
\begin{equation}
P_j := \sum_{k: \lambda_k \in \Lambda_j} \langle \cdot, e_k \rangle e_k, \quad j = 1, 2, 3.
\end{equation}

The three Bregman channels are the restrictions of the functional $\Phi$ to the subspaces $\mathcal{H}_j := P_j(\mathcal{H})$. These satisfy:

\textbf{Part (a) - Orthogonal Decomposition:}

The three projections are orthogonal ($P_i P_j = 0$ for $i \neq j$) and complete ($P_1 + P_2 + P_3 = I$), so the functional space orthogonally decomposes:
\begin{equation}
\mathcal{H} = \mathcal{H}_1 \oplus \mathcal{H}_2 \oplus \mathcal{H}_3, \quad \mathcal{H}_j := P_j(\mathcal{H}).
\end{equation}

\textbf{Part (b) - Functional Independence:}

The three channel functionals:
\begin{equation}
\mathcal{D}_j[\psi] := \langle P_j \psi, D^2\Phi[\psi_0] P_j \psi \rangle, \quad j = 1, 2, 3,
\end{equation}
are functionally independent in the sense that their Jacobian matrix has full rank (rank 3) at a generic point in $\mathcal{H}$.

\textbf{Part (c) - Completeness Up to Higher Orders:}

Any Bregman divergence functional on $\mathcal{H}$ decomposes as:
\begin{equation}
D_{\Phi}[\psi \| \phi] = \sum_{j=1}^3 \langle P_j(\psi - \phi), D^2\Phi[\phi] P_j(\psi - \phi) \rangle + \mathcal{O}(\|\psi - \phi\|^3),
\end{equation}
where each term corresponds to one channel.

\begin{proof}

\textbf{Part (a): Orthogonal Decomposition}

By the spectral theorem for self-adjoint operators, $D^2\Phi[\psi_0]$ on $\mathcal{H} = L^2(X, \mu)$ admits orthogonal spectral decomposition:
\[
D^2\Phi[\psi_0] = \sum_{k=1}^\infty \lambda_k P_k,
\]
where $P_k$ are rank-one projections onto eigenvectors $e_k$ with eigenvalues $0 < \lambda_{\min} \leq \lambda_k \leq \lambda_{\max}$.

By Lemma \ref{lem:ternaryEigenvalueStructure}, these eigenvalues cluster into three scales. Partition the index set:
\[
\mathcal{I} = \mathcal{I}_1 \cup \mathcal{I}_2 \cup \mathcal{I}_3, \quad \text{(disjoint)}
\]
where $\mathcal{I}_j = \{k : \lambda_k \in \Lambda_j\}$. Define:
\[
P_j := \sum_{k \in \mathcal{I}_j} P_k.
\]

These satisfy:
\begin{enumerate}
\item \textbf{Orthogonality:} $P_i P_j = \delta_{ij} P_i$ (standard projection properties for spectral projections).
\item \textbf{Completeness:} $P_1 + P_2 + P_3 = I$ (partition of unity in eigenspace).
\end{enumerate}

Therefore, $\mathcal{H}$ orthogonally decomposes:
\[
\mathcal{H} = \mathcal{H}_1 \oplus \mathcal{H}_2 \oplus \mathcal{H}_3, \quad \mathcal{H}_j := P_j(\mathcal{H}).
\]

\textbf{Part (b): Functional Independence}

Consider the Jacobian of the channel functionals at a reference configuration $\psi_0$:
\[
\mathcal{J}_{ij} := \frac{\delta \mathcal{D}_i[\psi_0]}{\delta \psi_j}.
\]

Since $\mathcal{D}_j$ acts on the orthogonal subspace $\mathcal{H}_j$, and these subspaces do not overlap, the Jacobian matrix has block-diagonal structure with full rank in each block. Thus, the three functionals are linearly independent and hence functionally independent.

More rigorously: define the map $\Psi: \mathcal{H}^3 \to \mathbb{R}^3$ by $\Psi(\psi_1, \psi_2, \psi_3) = (\mathcal{D}_1[\psi_1], \mathcal{D}_2[\psi_2], \mathcal{D}_3[\psi_3])$. For generic $(\psi_1, \psi_2, \psi_3)$ with $\psi_j \in \mathcal{H}_j$, the differential $d\Psi$ is surjective onto $\mathbb{R}^3$, proving functional independence.

\textbf{Part (c): Completeness}

Expand the Bregman divergence in the eigenbasis:
\[
D_{\Phi}[\psi \| \phi] = \frac{1}{2} \sum_{k=1}^\infty \lambda_k |\langle e_k, \psi - \phi \rangle|^2 + \mathcal{O}(\|\psi - \phi\|^3).
\]

Partition this sum into three groups corresponding to $\Lambda_1, \Lambda_2, \Lambda_3$:
\[
D_{\Phi}[\psi \| \phi] = \sum_{j=1}^3 \frac{1}{2} \sum_{k \in \mathcal{I}_j} \lambda_k |\langle e_k, \psi - \phi \rangle|^2 + \mathcal{O}(\|\psi - \phi\|^3).
\]

Each term $\frac{1}{2} \sum_{k \in \mathcal{I}_j} \lambda_k |\langle e_k, \psi - \phi \rangle|^2$ is the divergence of the $j$-th channel, as it involves only eigenvectors in $\mathcal{H}_j$.

\end{proof}

\end{lemma}

\begin{lemma}[Independence and Completeness of Channels (Original Formulation)]
\label{lem:channelIndependenceCompleteness}

\emph{[This lemma is superseded by Lemma \ref{lem:spectralCompletenessChannels}, which provides a rigorous spectral-theoretic proof of the same claims. The present lemma is retained for historical reference.]}

The three channels (Euclidean, Potential, Metric) satisfy the properties stated in Lemma \ref{lem:spectralCompletenessChannels}: orthogonal decomposition, functional independence, and completeness up to higher-order terms. The proof is given in Lemma \ref{lem:spectralCompletenessChannels}.

\end{lemma}

