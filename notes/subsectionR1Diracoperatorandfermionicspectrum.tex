% Part of sectionRTheMoonSpinorFermionStructure.tex
\subsection{Dirac Operator and Fermionic Spectrum}
\label{subsec:diracOperatorAndFermionicSpectrum}

\begin{definition}[Dirac Operator on Lorentzian Manifold]
\label{def:diracOperator}
On the Lorentzian spacetime $(X, g)$ with signature $(-,+,+,+)$, the Dirac operator is:
\begin{equation}
\slashed{D} := \gamma^\mu (D_\mu + \omega_\mu),
\end{equation}
where:

\begin{enumerate}
\item $\gamma^\mu$ satisfy the Clifford algebra $\{\gamma^\mu, \gamma^\nu\} = 2g^{\mu\nu}$.
\item $D_\mu = \partial_\mu + iA_\mu$ is the gauge-covariant derivative.
\item $\omega_\mu = \frac{1}{4}\omega_\mu^{ab} \sigma_{ab}$ is the spin connection, with $\sigma_{ab} = \frac{i}{2}[\gamma_a, \gamma_b]$.
\end{enumerate}

The Dirac operator is formally self-adjoint with respect to the natural inner product on spinors.
\end{definition}

\begin{lemma}[Counting of Weyl Spinor Species per Generation]
\label{lem:WeylSpeciesCounting}

In the Standard Model (without right-handed neutrinos), Weyl spinor content per generation:

\textbf{Leptons:}
\begin{enumerate}
\item $\nu_e^L$ (left-handed electron neutrino)
\item $e^L$ (left-handed electron)
\item $e^R$ (right-handed electron)
\end{enumerate}

\textbf{Quarks:}
\begin{enumerate}
\item $u^L, d^L$ (left-handed, each in 3 colors = 6 species)
\item $u^R, d^R$ (right-handed, each in 3 colors = 6 species)
\end{enumerate}

\textbf{Total per generation:}
\begin{equation}
N_{\text{Weyl}} = 3 \text{ (leptons)} + 6 \text{ (quark colors)} + 6 \text{ (quark colors)} = 15.
\end{equation}

For representation theory: count all Weyl components = 15.

\end{lemma}

\begin{theorem}[Index of Dirac Operator]
\label{thm:diracSpectrum}
On a compact four-dimensional Riemannian manifold, the Dirac operator has:

\begin{enumerate}
\item Discrete spectrum with $\lambda \to \pm\infty$.
\item Spectral symmetry: $\lambda$ eigenvalue $\implies$ $-\lambda$ eigenvalue.
\item Index given by Atiyah-Singer formula:
\begin{equation}
\text{ind}(\slashed{D}) := \dim \ker(\slashed{D}_+) - \dim \ker(\slashed{D}_-) = \frac{1}{32\pi^2} \int_X \Tr(R \edge R) - \frac{1}{8\pi^2} \int_X \Tr(F \edge F),
\end{equation}
where $R$ is the Riemann curvature and $F$ is the gauge field strength.
\end{enumerate}

\begin{proof}
% proofThmDiracSpectrum.tex
% Proof content

\textbf{Part 1: Construction of Dirac Operator from Carre du Champ}

Let $(X, d, \mu)$ be the Polish measure space with emerged metric $d$ and emerged measure $\mu$ from Sections \ref{sec:metricEmergence}. The Carre du Champ operator $\Gamma$ defines:
\begin{equation}
\Gamma(f, g) := \frac{1}{2}(\mathcal{L}(fg) - f\mathcal{L}g - g\mathcal{L}f),
\end{equation}
where $\mathcal{L}$ is the generator of the Dirichlet form $\mathcal{E}$.

For the fermionic sector, the minimal upper gradient structure provides the spinor connection. Define the spinor bundle $S(X)$ with fiber $\mathbb{C}^4$ (four complex components). The covariant derivative on spinors is constructed from:

\begin{equation}
\nabla_\mu \psi := \partial_\mu \psi + \omega_\mu \psi,
\end{equation}

where $\omega_\mu$ is the spin connection derived from the Levi-Civita connection of the emerged metric $g_{\mu\nu}$ (satisfying Theorem \ref{thm:metricFromCarre}):

\begin{equation}
\omega_{\mu}^{ab} := \frac{1}{2} e^\nu_a (\nabla_\mu e^\mu_b - \nabla_b e_\mu^a) \, dx^\mu,
\end{equation}

where $e_\mu^a$ are the vielbein components relating the metric to Carre du Champ structure.

The Clifford algebra generators $\gamma^\mu$ (with Lorentzian signature convention $\{\gamma^\mu, \gamma^\nu\} = 2g^{\mu\nu}$) are represented as $4 \times 4$ complex matrices:

\begin{equation}
\gamma^0 = \begin{pmatrix} 0 & -i\mathbb{1}_2 \\ i\mathbb{1}_2 & 0 \end{pmatrix}, \quad \gamma^j = \begin{pmatrix} 0 & \sigma^j \\ -\sigma^j & 0 \end{pmatrix} \quad (j=1,2,3),
\end{equation}

where $\sigma^j$ are Pauli matrices.

The \textbf{Dirac operator on the emerged Lorentzian manifold} is defined as:

\begin{equation}
\label{eq:diracOperatorConstruction}
\slashed{D} := \gamma^\mu (D_\mu + \omega_\mu) = \gamma^\mu (\partial_\mu + iA_\mu + \frac{1}{4}\omega_\mu^{ab}\sigma_{ab}),
\end{equation}

where $A_\mu$ is the gauge field (electroweak and strong interactions, Theorems \ref{thm:su3StrongStructure}, \ref{thm:su2WeakStructure}), and $\sigma_{ab} = \frac{i}{2}[\gamma_a, \gamma_b]$ are the Lorentz generators on spinors.

This operator is formally self-adjoint:
\begin{equation}
\slashed{D}^\dagger = \gamma^0 \slashed{D} \gamma^0 = \slashed{D}.
\end{equation}

\textbf{Part 2: Spectral Properties and Weyl Asymptotics}

On a compact Riemannian manifold of dimension $d = 4$, the spectrum of $\slashed{D}$ is discrete. For the Lorentzian signature (via Wick rotation to Euclidean), define the spectral function:

\begin{equation}
N(\lambda) := \#\{\text{eigenvalues} \leq \lambda \text{ in absolute value}\}.
\end{equation}

By Weyl's asymptotic formula for Dirac operators on 4D Riemannian manifolds:

\begin{equation}
\label{eq:WeylDiracAsymptotics}
N(\lambda) = \frac{\text{vol}(X)}{(4\pi^2)} \lambda^4 + C_1 \lambda^3 + C_2 \lambda^2 + O(\lambda),
\end{equation}

where:
\begin{itemize}
\item $\text{vol}(X)$ is the four-dimensional volume computed from emerged metric $g$
\item $C_1$ involves topological terms (Pontryagin class, conformal structure)
\item $C_2$ involves curvature integrals
\end{itemize}

More precisely, the heat kernel expansion of $e^{-t\slashed{D}^2}$ yields:

\begin{equation}
\Tr e^{-t\slashed{D}^2} = \sum_{n=0}^\infty a_n t^{(n-4)/2},
\end{equation}

where the heat kernel coefficients $a_n$ are given by the Seeley-DeWitt expansion:

\begin{equation}
\label{eq:heatKernelCoefficients}
\begin{split}
a_0 &= \frac{\text{vol}(X)}{(4\pi)^{d/2}} \int_X \text{rank}(S) \, d\mu = \frac{\text{vol}(X)}{(4\pi)^2} \cdot 4, \\
a_1 &= \text{(involves Ricci scalar)} \\
a_2 &= \text{(involves Riemann curvature and gauge fields)}.
\end{split}
\end{equation}

The spectral counting function relates to the heat trace via:

\begin{equation}
N(\lambda) = \int_0^\infty \frac{d\tau}{\sqrt{4\pi\tau}} e^{\lambda^2 \tau} \Tr e^{-\tau \slashed{D}^2}.
\end{equation}

\textbf{Part 3: Spectrum Before Symmetry Breaking (Massless Fermions)}

Before electroweak symmetry breaking and Higgs mechanism, the fermion mass terms vanish. The Dirac operator is:

\begin{equation}
\slashed{D}_0 := \gamma^\mu (D_\mu + \omega_\mu),
\end{equation}

without explicit mass eigenvector terms. For a massless Weyl fermion (say, electron neutrino $\nu_e$), the equation of motion:

\begin{equation}
\slashed{D}_0 \nu_e = 0
\end{equation}

has continuous spectrum of approximately massless modes. The spectral gap is:

\begin{equation}
\Delta m_{\text{unbroken}} \sim (\text{loop corrections}) \sim \alpha(g) m_{\text{ref}} \sim 10^{-2} \text{ eV},
\end{equation}

where $\alpha(g)$ are perturbative corrections and $m_{\text{ref}}$ is a reference scale (e.g., QCD scale $\Lambda_{\text{QCD}} \sim 200$ MeV for quarks).

After symmetry breaking (Higgs mechanism, Section \ref{subsec:higgsMechanismAndElectroweakSymmetryBreaking}), fermions acquire masses from the Yukawa coupling:

\begin{equation}
\mathcal{L}_{\text{Yukawa}} = -y_f \bar{\psi}_f H \psi_f + \text{h.c.},
\end{equation}

which becomes (after $H \to \langle H \rangle = v/\sqrt{2}$):

\begin{equation}
m_f = \frac{y_f v}{\sqrt{2}},
\end{equation}

giving physical masses for all massive fermions (electron, muon, quarks).

\textbf{Part 4: Chirality Structure and Helicity Eigenstates}

Define the chirality operator:
\begin{equation}
\gamma_5 := i\gamma^0 \gamma^1 \gamma^2 \gamma^3, \quad \gamma_5^2 = \mathbb{1}, \quad \{\gamma_5, \gamma^\mu\} = 0.
\end{equation}

The chiral projectors are:
\begin{equation}
P_L := \frac{1 - \gamma_5}{2}, \quad P_R := \frac{1 + \gamma_5}{2},
\end{equation}

with $P_L + P_R = \mathbb{1}$ and $P_L P_R = 0$.

Left-handed and right-handed spinor eigenstates satisfy:
\begin{equation}
\gamma_5 \psi_L = -\psi_L, \quad \gamma_5 \psi_R = +\psi_R.
\end{equation}

The Dirac operator anticommutes with $\gamma_5$:
\begin{equation}
\{\slashed{D}, \gamma_5\} = 0 \implies \slashed{D} P_L = P_R \slashed{D}, \quad \slashed{D} P_R = P_L \slashed{D}.
\end{equation}

This means the spectrum is symmetric about zero: $\lambda$ eigenvalue $\iff$ $-\lambda$ eigenvalue.

For weak interactions (Section \ref{sec:weakForce}), only left-handed fermions couple to $SU(2)_L$ gauge bosons, establishing the chiral asymmetry essential to the Standard Model.

\textbf{Part 5: Eigenvalue Distribution and Density of States}

The density of eigenvalues $\rho(\lambda)$ is defined via:
\begin{equation}
\label{eq:spectralDensity}
\rho(\lambda) := \frac{dN}{d\lambda} = \frac{\text{vol}(X)}{(4\pi^2)} \cdot 4\lambda^3 + \text{lower order terms}.
\end{equation}

For large $|\lambda|$, the spectrum becomes increasingly dense, with density growing as $\lambda^3$ (the fourth power of the Weyl formula, differentiated).

At the Fermi energy $E_F$ (relevant for matter at finite temperature), the density of states determines thermodynamic properties:
\begin{equation}
C_V \sim \rho(E_F) T \quad \text{(low-temperature heat capacity)}.
\end{equation}

For the Standard Model at low energies, the relevant spectral region includes:
\begin{itemize}
\item Massless sector before symmetry breaking: $\lambda \sim 10^{-2}$ eV
\item Higgs and electroweak bosons: $\lambda \sim 100$ GeV
\item Top quark and beyond: $\lambda \sim 100$ GeV to TeV scale
\end{itemize}

\textbf{Part 6: Index Theorem and Topological Constraints}

The Atiyah-Singer index of the Dirac operator is:
\begin{equation}
\label{eq:indexFormula}
\text{ind}(\slashed{D}) := \dim \ker(\slashed{D}_+) - \dim \ker(\slashed{D}_-) = \int_X \hat{A}(R) \edge \text{ch}(F),
\end{equation}

where:
\begin{itemize}
\item $\hat{A}(R)$ is the A-roof genus (depends on Riemann curvature)
\item $\text{ch}(F)$ is the Chern character of gauge field strength $F$
\end{itemize}

For Yang-Mills instantons with topological charge:
\begin{equation}
Q_{\text{inst}} = \frac{1}{32\pi^2} \int_X \Tr(F \edge F),
\end{equation}

the index provides the number of zero modes:
\begin{equation}
n_- - n_+ = Q_{\text{inst}} \quad \text{(for SU(2) gauge theory, say)},
\end{equation}

where $n_\pm$ are the numbers of left/right-handed zero modes.

This topological constraint is crucial for:
\begin{enumerate}
\item Chirality of weak interactions (Theorem \ref{thm:su2WeakStructure})
\item Anomaly cancellation (Lemma \ref{lem:anomalyCoefficients})
\item Non-perturbative effects (instantons, sphalerons)
\end{enumerate}

\textbf{Part 7: Connection to Carre du Champ Structure}

The Carre du Champ $\Gamma(f, f)$ measures the ``squared gradient'' of a function:

\begin{equation}
\Gamma(f, f) = |\nabla f|^2.
\end{equation}

For spinor fields, this generalizes to the matrix-valued version. The Dirac operator satisfies the energy estimate:

\begin{equation}
\|\nabla \psi\|_{L^2} \leq C \|\slashed{D}\psi\|_{L^2} + C' \|\psi\|_{L^2},
\end{equation}

which follows from ellipticity of $\slashed{D}$. This establishes:
\begin{itemize}
\item Boundedness of the spectrum below (discrete spectrum below any $\lambda_0$)
\item Compactness of the resolvent $(z - \slashed{D})^{-1}$ for $z \not\in \sigma(\slashed{D})$
\item Completeness of the eigenbasis (spectral decomposition)
\end{itemize}

\textbf{Conclusion:} The Dirac spectrum on the emerged spacetime $(X, g, \nabla)$ is:

\begin{enumerate}
\item \textbf{Discrete}, with eigenvalues $\{\lambda_n\}_{n \in \mathbb{Z}}$ ordered $\cdots < \lambda_{-1} < 0 < \lambda_1 < \cdots$
\item \textbf{Symmetric about zero}, $\lambda_n = -\lambda_{-n}$
\item \textbf{Asymptotically dense}: $\rho(\lambda) \sim |\lambda|^3$ for large $|\lambda|$
\item \textbf{Conformally covariant}: Under conformal transformations $g \to e^{2\omega}g$, the spectrum scales as $e^{-\omega}\lambda_n$
\item \textbf{Topologically constrained}: Zero modes and index determined by global topology via Atiyah-Singer theorem
\end{enumerate}

Before symmetry breaking, massless fermions contribute continuous sectors; the Higgs mechanism generates gaps (masses), lifting the degeneracy and fixing the Standard Model fermion spectrum to the observed values (electrons, muons, taus, quarks, and  sterile neutrinos).
\end{proof}
\end{theorem}

