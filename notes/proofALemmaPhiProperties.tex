% proofLemPhiProperties.tex
% Proof content


\textbf{Proof of Lemma \ref{lem:phiProperties}}

The following derivation establishes strict convexity, weak lower semicontinuity, and Gateaux differentiability of $\Phi[\psi] := \int_X V(|\psi|^2) d\mu(x)$ under potential conditions (V1)-(V4).

\textit{\underline{Part (i): Strict Convexity}}

For $\psi_1, \psi_2 \in \mathcal{H}$ with $\psi_1 \neq \psi_2$ and $t \in (0,1)$, the must show:
\[
\Phi[t\psi_1 + (1-t)\psi_2] < t\Phi[\psi_1] + (1-t)\Phi[\psi_2].
\]

Let $\psi_t := t\psi_1 + (1-t)\psi_2$. By the triangle inequality:
\[
|\psi_t|^2 = |t\psi_1 + (1-t)\psi_2|^2 \leq (t|\psi_1| + (1-t)|\psi_2|)^2 = t|\psi_1|^2 + (1-t)|\psi_2|^2 + 2t(1-t)|\psi_1||\psi_2|.
\]

For simplicity, working point-wise: $|t\psi_1 + (1-t)\psi_2|^2 \leq t|\psi_1|^2 + (1-t)|\psi_2|^2$ (by convexity of $s \mapsto s^2$). Thus:
\begin{align}
\Phi[\psi_t] &= \int_X V(|\psi_t|^2) d\mu \\
&\leq \int_X V(t|\psi_1|^2 + (1-t)|\psi_2|^2) d\mu \quad \text{(by monotonicity of } V \text{)} \\
&< t\int_X V(|\psi_1|^2) d\mu + (1-t)\int_X V(|\psi_2|^2) d\mu \quad \text{(by strict convexity of } V \text{ from condition V2)} \\
&= t\Phi[\psi_1] + (1-t)\Phi[\psi_2].
\end{align}

Strict inequality holds at points where $|\psi_1| \neq |\psi_2|$ (which is a set of positive measure since $\psi_1 \neq \psi_2$), using the strict convexity of $V$ from Axiom (V2).

\textit{\underline{Part (ii): weak Lower Semicontinuity}}

Let $\psi_n \rightharpoonup \psi$ weakly in $\mathcal{H} = L^2(X, \mu; \mathbb{C}^n)$. Direct demonstration shows:
\[
\liminf_{n \to \infty} \Phi[\psi_n] \geq \Phi[\psi].
\]

Since $\psi_n \rightharpoonup \psi$ weakly in $L^2$, the sequence is bounded: $\sup_n \|\psi_n\|_{L^2} \leq M$. Thus $|\psi_n|^2$ is bounded in $L^1(X, \mu)$.

By weak compactness, $|\psi_n|^2$ has a weakly convergent subsequence $|\psi_{n_k}|^2 \rightharpoonup \xi$ in $L^1$. By uniqueness of weak limits and the fact that $|\psi_n|^2 \to |\psi|^2$ a.e. ( after extracting a subsequence), there is $\xi = |\psi|^2$.

Now, by the convexity of $V$ and Fatou's lemma:
\[
\liminf_{n \to \infty} \Phi[\psi_n] = \liminf_{n \to \infty} \int_X V(|\psi_n|^2) d\mu \geq \int_X \liminf_{n \to \infty} V(|\psi_n|^2) d\mu = \int_X V(|\psi|^2) d\mu = \Phi[\psi],
\]

where the inequality uses Fatou's lemma (valid since $V \geq 0$ by condition V1) and the lower semicontinuity of $V$ (which follows from continuity, inherited from the growth conditions).

\textit{\underline{Part (iii): Gateaux Differentiability on $\text{Dom}(D\Phi)$}}

For $\psi \in \text{Dom}(D\Phi)$ and test direction $h \in \mathcal{H}$, the compute:
\begin{align}
\frac{\Phi[\psi + th] - \Phi[\psi]}{t} &= \frac{1}{t}\int_X [V(|\psi + th|^2) - V(|\psi|^2)] d\mu \\
&= \int_X \frac{V(|\psi|^2 + t(2\text{Re}(\bar{\psi} \cdot h) + t|h|^2)) - V(|\psi|^2)}{t} d\mu.
\end{align}

By the mean value theorem, for each $x$ there exists $\theta_x \in (0,t)$ such that:
\[
\frac{V(|\psi|^2 + t(2\text{Re}(\bar{\psi} \cdot h) + t|h|^2)) - V(|\psi|^2)}{t} = V'(|\psi|^2 + \theta_x(\cdots)) \cdot (2\text{Re}(\bar{\psi} \cdot h) + O(t)).
\]

By condition (V3), $|V'(s)| \leq C_1(1 + s^{\alpha-1})$ for $\alpha \geq 2$. For $\psi \in \text{Dom}(D\Phi)$, there is $V'(|\psi|^2)\psi \in L^2$, which implies $V'(|\psi|^2) \in L^{\infty}$ or at worst in $L^p$ for suitable $p$ by Holder's inequality.

Thus:
\[
\left|\frac{V(|\psi + th|^2) - V(|\psi|^2)}{t} - V'(|\psi|^2) \cdot 2\text{Re}(\bar{\psi} \cdot h)\right| \to 0 \quad \text{as } t \to 0^+,
\]

by dominated convergence with integrand dominated by $C(1 + |\psi|^{2(\alpha-1)}) \cdot |h| \in L^1$ (using Sobolev embedding for $Q < 4$).

Therefore:
\[
D\Phi[\psi] \cdot h := \lim_{t \to 0^+} \frac{\Phi[\psi + th] - \Phi[\psi]}{t} = 2\int_X V'(|\psi|^2) \text{Re}(\bar{\psi} \cdot h) d\mu.
\]

This is continuous in $h$ (by Holder's inequality and $V'(|\psi|^2)\psi \in L^2$), so $\Phi$ is Gateaux differentiable with:
\[
D\Phi[\psi] = 2V'(|\psi|^2)\psi.
\]

\qed
