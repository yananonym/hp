% Part of sectionBDivergenceStructure.tex
\subsection{Exponential Asymmetry and Temporal Structure}
\label{subsec:exponentialAsymmetry}

\begin{definition}[Exponential Discrepancy Function]
\label{def:exponentialDiscrepancy}
For the probability functional $P(\lambda) = e^{-\lambda}$ of a Poisson process with rate parameter $\lambda > 0$, define the exponential discrepancy:
\begin{equation}
D_{\exp}(\lambda, a) := |e^{-a\lambda} - e^{-\lambda/a}|, \quad a > 1.
\end{equation}

This measures asymmetry under logarithmically symmetric perturbations: $\lambda \mapsto a\lambda$ versus $\lambda \mapsto \lambda/a$.
\end{definition}

\begin{theorem}[Critical Inflection Point and Information-Theoretic Balance]
\label{thm:criticalInflection}
The function $f(x) = e^{-1/x}$ for $x \in (0, \infty)$ has a unique inflection point at $x = 1/2$, corresponding to:
\begin{equation}
f(1/2) = e^{-2} \approx 0.1353.
\end{equation}

At this critical value, the second derivative vanishes and changes sign:

\begin{enumerate}[label=(\roman*)]
\item \textbf{Inflection Point:} The second derivative is:
\begin{equation}
f''(x) = e^{-1/x} \frac{1 - 2x}{x^4}.
\end{equation}
Setting $f''(x) = 0$ gives $x = 1/2$.

\item \textbf{Curvature Transition:}
\begin{align}
x < 1/2: & \quad f''(x) > 0 \quad \text{(concave up, accelerating)} \\
x > 1/2: & \quad f''(x) < 0 \quad \text{(concave down, decelerating)}
\end{align}

\item \textbf{Probabilistic Interpretation:} For a Poisson process with rate $\lambda$, the silence probability $P_0(\lambda) = e^{-\lambda}$ achieves equilibrium at $\lambda = 2$:
\begin{equation}
P_0(2) = e^{-2} \approx 0.1353.
\end{equation}
This represents the balance point between sparse (high silence) and dense (low silence) regimes.

\item \textbf{Logarithmic Symmetry Breaking:} For symmetric logarithmic perturbations around $\lambda = 2$ with factor $a = 3/2$:
\begin{align}
\lambda_- &= 4/3, \quad \lambda_+ = 3, \\
P_0(\lambda_-) &\approx 0.2636, \quad P_0(2) \approx 0.1353, \quad P_0(\lambda_+) \approx 0.0498.
\end{align}
The response is asymmetric: downward perturbations yield roughly $1.95\times$ increase, while upward perturbations yield roughly $0.37\times$ decrease. The asymmetry encodes temporal direction.
\end{enumerate}

\textbf{Proof:}

The second derivative is computed directly:
\begin{equation}
f'(x) = e^{-1/x} \cdot \frac{1}{x^2}, \quad f''(x) = \frac{d}{dx}\left[e^{-1/x} \cdot \frac{1}{x^2}\right] = e^{-1/x} \left[\frac{1}{x^4} - \frac{2}{x^3}\right] = e^{-1/x} \frac{1 - 2x}{x^4}.
\end{equation}

Sign analysis: $e^{-1/x} > 0$ and $x^4 > 0$ always, so $\text{sign}(f'') = \text{sign}(1 - 2x)$:
\begin{align}
1 - 2x > 0 & \iff x < 1/2, \\
1 - 2x < 0 & \iff x > 1/2.
\end{align}

At the critical point: $f(1/2) = \exp(-1/(1/2)) = e^{-2}$.

The logarithmic symmetry property: for logarithmically equidistant points, the numerical response is asymmetric due to the curvature change at the inflection point. This fundamental information-theoretic property underlies the arrow of time in the divergence structure.
\end{theorem}

\begin{corollary}[Bregman Divergence and the $e^{-2}$ Threshold]
\label{cor:bregmanExponentialUniversality}
The Bregman divergence structure shares the exponential asymmetry fundamental to the $e^{-2}$ critical value. For generating functional $\Phi[\psi] = \int_X V(|\psi|^2) d\mu$ with exponential-type growth near criticality, configurations separated by divergence $D[\psi \| \phi] \sim e^{-2}$ mark the transition between causally correlated and causally separated regimes.

The temporal asymmetry functional (Definition \ref{def:temporalFunctional}):
\begin{equation}
\mathcal{A}[\psi, \phi] := D[\psi \| \phi] - D[\phi \| \psi]
\end{equation}
exhibits maximal sensitivity to information redistribution at the critical threshold, establishing:
\begin{enumerate}[label=(\roman*)]
\item Temporal direction from divergence asymmetry
\item Causal cone structure in configuration space
\item Lorentzian signature emergence via Wick rotation
\end{enumerate}

The universality of $e^{-2}$ appearing in diverse contexts (analytic geometry, probability, quantum information theory, network theory) establishes it as a fundamental constant of information geometry.
\end{corollary}

