% proofThmAsymptoticFreedom.tex
% Proof content

\noindent\textbf{One-Loop Renormalization Group Analysis.}

In Yang-Mills theory with gauge group $G = SU(N_c)$ and $n_f$ quark flavors, the one-loop beta function for the gauge coupling $g_s$ is computed from loop diagrams in dimensional regularization (\cite{Weinberg1996theory}, Chapter 19; \cite{peskin1995introduction} 1995, Chapter 17):

\begin{equation}
\beta(g_s) = \frac{dg_s}{d\ln\mu} = -\frac{g_s^3}{16\pi^2}\left(\frac{11N_c}{3} - \frac{2n_f}{3}\right) + O(g_s^5).
\end{equation}

The coefficient $\frac{11N_c}{3}$ comes from the non-Abelian gluon self-interaction (cubic and quartic gluon vertices in the Yang-Mills action). The coefficient $\frac{2n_f}{3}$ (with opposite sign) comes from quark loop contributions.

\noindent\textbf{Asymptotic Freedom Condition.}

For the coupling to decrease at high energies (asymptotic freedom), it is required $\beta(g_s) < 0$:
\begin{equation}
\frac{11N_c}{3} - \frac{2n_f}{3} > 0 \quad \Rightarrow \quad n_f < \frac{11N_c}{2}.
\end{equation}

For $N_c = 3$ (the strong interaction gauge group $SU(3)$):
\begin{equation}
n_f < \frac{11 \cdot 3}{2} = 16.5.
\end{equation}

The Standard Model has $n_f = 6$ quark flavors (up, down, charm, strange, top, bottom), satisfying $6 < 16.5$. Thus the strong interaction is asymptotically free.

\noindent\textbf{Running of the Coupling.}

Integrating the one-loop equation:
\begin{equation}
\int_{g_0}^{g(\mu)} \frac{dg}{g^3 \beta(g)/g} = \int_{\mu_0}^\mu \frac{d\mu'}{\mu'},
\end{equation}
one obtains (\cite{Weinberg1996theory}, Eq. 19.4.5):
\begin{equation}
\alpha_s(\mu) := \frac{g_s^2(\mu)}{4\pi} = \frac{\alpha_s(\mu_0)}{1 + \frac{\alpha_s(\mu_0)}{3\pi}(11N_c/3 - 2n_f/3)\ln(\mu/\mu_0)}.
\end{equation}

As $\mu \to \infty$ (high energies), the denominator increases, so $\alpha_s(\mu) \to 0$. Thus $g_s(\mu) \to 0$, establishing asymptotic freedom.

\noindent\textbf{QCD Scale and Confinement.}

The running coupling has a Landau pole at a low-energy scale $\Lambda_{\text{QCD}}$:
\begin{equation}
\Lambda_{\text{QCD}} := \mu_0 \exp\left(-\frac{3\pi}{(11N_c - 2n_f)\alpha_s(\mu_0)}\right) \approx 200 \text{ MeV}.
\end{equation}

For $\mu > \Lambda_{\text{QCD}}$, the coupling remains weak ($g_s(\mu) \ll 1$), validating perturbation theory. For $\mu \sim \Lambda_{\text{QCD}}$ or lower, the coupling becomes $O(1)$ and perturbation theory breaks down.

In this nonperturbative regime, quarks and gluons cannot propagate as free states. Only color singlet bound states (hadrons) appear in the asymptotic spectrum. This is the origin of color confinement: the confinement scale $\Lambda_c \sim \Lambda_{\text{QCD}}$ arises dynamically from the asymptotic freedom beta function.
