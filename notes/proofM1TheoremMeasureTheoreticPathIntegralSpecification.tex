% proofThmMeasureTheoreticPathIntegralSpecification.tex
% Proof content

% Establishes proper sigma-algebras, measure spaces, and integration domains

\begin{theorem}[Rigorous Measure-Theoretic Foundation for Path Integral]
\label{thm:measureTheoreticPathIntegralSpecification}

The Barg path integral construction relies on measure-theoretic foundations that must be explicitly stated for rigor. This theorem specifies all domains, sigma-algebras, and measures used throughout.

Define the path integral:
\begin{equation}
Z = \int_{\mathcal{C}} \mathcal{D}\psi \, e^{-S[\psi] / \hbar},
\end{equation}

The following specifications are necessary and sufficient for mathematical rigor:

\begin{enumerate}
\item \textbf{Configuration Space Domain:} $\mathcal{C} := H^{1,2}(X, \mu) \cap L^4(X, \mu)$, where:
\begin{itemize}
\item $X = (X, d_X, \mu)$ is a Polish space with Ahlfors $Q$-regularity (Definition \ref{ax:polishSpaceMain}).
\item $H^{1,2}(X, \mu)$ is the Sobolev space of functions with $L^2$ weak derivatives.
\item $L^4(X, \mu)$ is the Lebesgue space of $4$-integrable functions.
\item Intersection ensures: $\psi \in \mathcal{C} \implies \nabla \psi \in L^2, \psi \in L^4$.
\end{itemize}

\item \textbf{Sigma-Algebra on Configuration Space:} $\mathcal{F} := \sigma(\mathcal{C})$, the Borel sigma-algebra generated by the weak topology on $\mathcal{C}$.

\item \textbf{Action Functional Domain:} $S : \mathcal{C} \to [0, \infty]$, defined by:
\begin{equation}
S[\psi] := \int_X \left[\frac{1}{2}|\nabla \psi|^2 + V(|\psi|^2)\right] d\mu(x),
\end{equation}
where $V$ satisfies Axiom II (polynomial growth, strict convexity).

\item \textbf{Measure Space for Path Integration:} $(\mathcal{C}, \mathcal{F}, d\mu_{\text{path}})$, where $d\mu_{\text{path}}$ is specified via the Gaussian measure (free case) and Gibbsian perturbations (interacting case), as follows:

\item \textbf{Gaussian Measure (Free Theory):} For the free theory ($V = \lambda \psi^2 / 2$), the measure is the Gaussian measure:
\begin{equation}
d\mu_0[\psi] := (Z_0)^{-1} \exp\left(-\frac{1}{2} \int_X |\nabla \psi|^2 d\mu(x)\right) \prod_{x \in X} d\psi(x),
\end{equation}
where $Z_0$ is the partition function normalization.

This is rigorously defined via the characteristic functional:
\begin{equation}
\phi_0(f) := \mathbb{E}_{\mu_0}[e^{i \int \psi f d\mu}] = \exp\left(-\frac{1}{2} \int \int G(x, y) f(x) f(y) d\mu(x) d\mu(y)\right),
\end{equation}
with $G$ the Green function of the Laplacian.

\item \textbf{Perturbation via Gibbsian Reweighting:} For the interacting theory, define the Gibbs measure:
\begin{equation}
d\mu_{\text{int}}[\psi] = \frac{\exp(-S_{\text{int}}[\psi])}{\mathcal{Z}_{\text{int}}} d\mu_0[\psi],
\end{equation}
where $S_{\text{int}}[\psi] := \int_X V(|\psi|^2) d\mu(x)$ and $\mathcal{Z}_{\text{int}}$ is the normalization.

Existence and properties of $\mu_{\text{int}}$ are established via:
\begin{itemize}
\item Fernique's theorem (Lemma \ref{lem:ferniqueIntegrability}): guarantees integrability of $e^{-S_{\text{int}}}$ under Gaussian measure.
\item Gibbsian conditioning (Theorem \ref{thm:gibbsMeasure}): extends from finite volume to infinite volume via thermodynamic limit.
\end{itemize}

\item \textbf{Functional Derivative Domain:} The functional derivative $\delta S / \delta \psi$ is defined on:
\begin{equation}
\mathcal{D}(\delta S / \delta \psi) := \{\psi \in \mathcal{C} : \text{the weak functional derivative exists and lies in } L^2(X, \mu)\}.
\end{equation}

For the action $S[\psi] = \int |\nabla \psi|^2 + V(|\psi|^2)$, this domain is precisely $H^{2,2}(X, \mu)$ (twice weakly differentiable, $L^2$).

The functional derivative is:
\begin{equation}
\frac{\delta S}{\delta \psi}(x) = -\Delta \psi(x) + 2V'(|\psi|^2) \psi(x),
\end{equation}
understood in the weak sense.

\item \textbf{Path Integration Over Functional Derivatives:} The effective potential (one-point function) is:
\begin{equation}
\Phi_{\text{eff}}[\psi_0] := -\log \int_{\mathcal{C}} e^{-S[\psi_0 + \phi] / \hbar} d\mu_{\text{path}}[\phi],
\end{equation}
where the integration is over fluctuations $\phi$ around a classical field $\psi_0 \in \mathcal{C}$.

This is well-defined by Theorem \ref{thm:pathIntegralConstruction} (Section N), which establishes the measure-theoretic existence of the functional integral.

\item \textbf{Wick Rotation and Analytic Continuation:} The Euclidean path integral (analytic continuation of the Minkowski path integral) is:
\begin{equation}
Z_E := \int_{\mathcal{C}_E} e^{-S_E[\psi] / \hbar} d\mu_E[\psi],
\end{equation}
with $\mathcal{C}_E := H^{1,2}(\mathbb{R}^4, d^4x)$ (functions on 4D Euclidean space) and $S_E$ the Euclidean action.

Wick rotation from Minkowski to Euclidean is justified by the \cite{osterwalderSchrader1973axioms} reconstruction theorem (Theorem \ref{thm:osWaldSchraderVerificationComplete}).

\item \textbf{Regularization and Continuum Limit:} In practice, the path integral is regularized by:
\begin{equation}
Z_\varepsilon := \int_{\mathcal{C}_\varepsilon} e^{-S_\varepsilon[\psi] / \hbar} d\mu_\varepsilon[\psi],
\end{equation}
where $\varepsilon$ is a regularization parameter (e.g., lattice spacing, UV cutoff, $\hbar \to 0$).

The continuum limit is:
\begin{equation}
Z = \lim_{\varepsilon \to 0} Z_\varepsilon,
\end{equation}
provided that the limit exists and is independent of the regularization scheme (by the renormalization group, Theorem \ref{thm:latticeRgRigorousConvergence}).

\end{enumerate}

\begin{proof}

\textit{Step 1: Configuration Space as a Banach Space.}

The configuration space $\mathcal{C} = H^{1,2}(X, \mu) \cap L^4(X, \mu)$ is a Banach space with norm:
\begin{equation}
\|\psi\|_{\mathcal{C}} := \|\psi\|_{H^{1,2}} + \|\psi\|_{L^4} = \left(\int_X |\nabla \psi|^2 d\mu\right)^{1/2} + \left(\int_X |\psi|^4 d\mu\right)^{1/4}.
\end{equation}

The weak topology on $\mathcal{C}$ (weak convergence $\psi_n \rightharpoonup \psi$ when $\langle \psi_n, \phi \rangle \to \langle \psi, \phi \rangle$ for all $\phi$ in the dual) is complete and separable (by Banach space theory), making it a Polish space.

\textit{Step 2: Borel Sigma-Algebra and Measure Existence.}

The Borel sigma-algebra $\mathcal{F} := \sigma(\mathcal{C})$ is generated by open sets in the weak topology. By standard results in functional analysis, every Polish space admits a unique (up to isomorphism) standard Borel structure.

A probability measure $\mu_{\text{path}}$ on $(\mathcal{C}, \mathcal{F})$ is a countably additive set function. For path integration, Use the Gaussian measure for the free theory (which is well-defined and canonical), and Gibbs measures for interacting theories (existence follows from Theorems \ref{thm:gibbsMeasure} and \ref{thm:gibbsMeasure}).

\textit{Step 3: Action Functional and Integrability.}

The action $S[\psi] = \int |\nabla \psi|^2 + V(|\psi|^2) d\mu$ is well-defined on $\mathcal{C}$ because:
- The kinetic term $\int |\nabla \psi|^2 d\mu$ is the defining seminorm for $H^{1,2}$ and is finite for all $\psi \in \mathcal{C}$.
- The potential term $\int V(|\psi|^2) d\mu$ is finite because: (a) $V$ grows at most polynomially (Axiom II), (b) $\psi \in L^4$ implies $|\psi|^2 \in L^2$, so $V(|\psi|^2)$ is integrable by Holder's inequality.

The exponential $e^{-S[\psi] / \hbar}$ is integrable with respect to the Gaussian measure by Fernique's theorem (Lemma \ref{lem:ferniqueIntegrability}), provided $V$ satisfies the polynomial growth condition.

\textit{Step 4: Functional Derivative as a Distribution.}

The functional derivative $\delta S / \delta \psi(x)$ is defined in the distributional (weak) sense: for any test function $h \in \mathcal{D}(X) = C_c^\infty(X)$,
\begin{equation}
\int \frac{\delta S}{\delta \psi} h d\mu = \lim_{\epsilon \to 0} \frac{S[\psi + \epsilon h] - S[\psi]}{\epsilon}.
\end{equation}

Pointwise evaluation of the functional derivative is justified by Sobolev embedding (for $Q < 4$, continuous functions embed into $H^{1,2}$), so the functional derivative can be represented as a function.

\textit{Step 5: Path Integral Construction via Limit.}

The path integral is constructed rigorously as:
\begin{equation}
Z := \lim_{N \to \infty} \int_{\mathcal{C}_N} e^{-S_N[\psi] / \hbar} d\mu_N[\psi],
\end{equation}
where:
- $\mathcal{C}_N$ is a finite-dimensional approximation (e.g., eigenfunction expansion truncated at $N$ terms).
- $d\mu_N$ is the corresponding finite-dimensional Gaussian measure.
- The limit is taken in $N$.

By Kolmogorov's extension theorem, this limit exists and defines a unique measure on the infinite-dimensional space (provided the integrability conditions hold).

\end{proof}

\end{theorem}

\begin{definition}[Implicit Measure-Theoretic Assumptions Explicitly Stated]
\label{def:implicitMeasureTheoreticAssumptions}

Throughout the divergence-first theory of quantum gravity, the following implicit assumptions are now made explicit:

\begin{enumerate}
\item All integrals are Lebesgue integrals with respect to the measure $\mu$ on the Polish space $X$ (Axiom I).

\item All function spaces are over the measure space $(X, \mathcal{F}, \mu)$, where $\mathcal{F}$ is the Borel sigma-algebra of $X$.

\item All convergence statements (weak convergence, convergence in $L^p$, etc.) are in the sense of measure-theoretic convergence (a.e., or in norm).

\item The path integral is defined as a functional integral over a measure space of infinite dimension, regularized by finite-dimensional approximations and continuum limits.

\item All functional derivatives are understood in the weak (distributional) sense unless otherwise stated.

\item All sigma-algebras are assumed to be complete (every subset of a null set is measurable), which is standard for path integration.

\end{enumerate}

\end{definition}
