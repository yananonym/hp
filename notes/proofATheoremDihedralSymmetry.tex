% proofThmDihedralSymmetry.tex
% Proof content

\noindent\textbf{Invariance of the Functional.}

The generating functional is $\Phi[\psi] = \int_X V(|\psi|^2) d\mu(x)$, where $V$ depends only on the magnitude $|\psi|^2 = \sum_{i=1}^N |\psi_i|^2$. For any unitary transformation $U \in U(N)$ with $U^\dagger U = I$:
\[
\Phi[U\psi] = \int_X V(|U\psi|^2) d\mu = \int_X V(\sum_i |U_{ij}\psi_j|^2) d\mu.
\]
Since $|U\psi|^2 = (U\psi)^\dagger(U\psi) = \psi^\dagger U^\dagger U \psi = \psi^\dagger \psi = |\psi|^2$, there is:
\[
\Phi[U\psi] = \int_X V(|\psi|^2) d\mu = \Phi[\psi].
\]
Thus the functional is invariant under the full unitary group $U(N)$. The dihedral group $D_N$ is a discrete subgroup of $U(N)$, so it preserves the functional.

\noindent\textbf{Dihedral Group Structure.}

The dihedral group $D_N$ of order $2N$ is generated by a rotation $R$ and a reflection $\sigma$ with the defining relations (Dummit-Foote 2004, Example 1.6.10):
\begin{equation}
R^N = e, \quad \sigma^2 = e, \quad \sigma R \sigma = R^{-1}.
\end{equation}

in the divergence-first framework, these correspond to:
\begin{itemize}
\item \textbf{Rotation $R$:} A cyclic permutation of the $N$ components:
\[
R(\psi_1, \ldots, \psi_N) := (\omega \psi_1, \omega^2 \psi_2, \ldots, \omega^N \psi_N),
\]
where $\omega = e^{2\pi i/N}$ is a primitive $N$-th root of unity. Then $R^N = I$ (identity), confirming the first relation.

\item \textbf{Reflection $\sigma$:} A charge conjugation-like involution:
\[
\sigma(\psi_1, \ldots, \psi_N) := (\bar{\psi}_N, \bar{\psi}_{N-1}, \ldots, \bar{\psi}_1),
\]
where bar denotes complex conjugation. Then $\sigma^2 = I$.

\item \textbf{Mixed Relation:} Direct calculation shows $\sigma R \sigma^{-1} = R^{-1}$ is satisfied, verifying the third dihedral relation.
\end{itemize}

\noindent\textbf{Physical Consequences.}

The dihedral symmetry $D_N$ acts on fermion multiplets via representation theory. By representation theory of $D_N$ (applied to the quark/lepton multiplets), the allowed quantum numbers must respect $D_N$-invariance. Each irreducible representation of $D_N$ constrains the structure of fermion generations (Theorem \ref{thm:threeGenerationsInfoGeometric}).

Specifically, when $D_N$ acts on the three-flavor multiplet $(\psi_u, \psi_c, \psi_t)$ of up-type quarks (or $(\psi_d, \psi_s, \psi_b)$ of down-type), the dihedral constraint forces a specific pattern of mixing angles and Yukawa couplings that matches the observed CKM matrix structure.
