% proofLemEigenfunctionMetricIndependent.tex
% Proof content


\begin{lemma}[Metric-Independent Eigenfunction Existence and Regularity]
\label{lem:eigenfunctionMetricIndependent}

Let $(X, \mu)$ be a Polish space satisfying Axiom \ref{ax:polishSpace}(a)--(c) with Ahlfors regularity dimension $Q < 4$. Equip $X$ with the canonical Dirichlet form:
\begin{equation}
\mathcal{E}(u, v) := \int_X \langle \nabla u(x), \nabla v(x) \rangle_{\text{upper gradient}} \, d\mu(x),
\end{equation}
where the upper gradient is defined using only measure-theoretic and topological structure (no metric needed a priori).

Then there exists a unique self-adjoint operator $\Delta: \Dom(\Delta) \to L^2(X, \mu)$ (the Laplacian) such that:
\begin{equation}
\mathcal{E}(u, v) = -\langle \Delta u, v \rangle_{L^2}.
\end{equation}

The Laplacian has orthonormal eigenfunctions $\{\phi_n\}_{n=1}^\infty$ with eigenvalues $0 \leq \lambda_1 \leq \lambda_2 \leq \cdots \to \infty$ satisfying:

\begin{enumerate}

\item \textbf{Existence (Spectral Theorem):} By the spectral theorem for self-adjoint operators (Reed-Simon Vol. I, Theorem VIII.7), there exists a spectral decomposition:
\begin{equation}
\Delta = \int_0^\infty \lambda \, dE_\lambda,
\end{equation}
where $E_\lambda$ is the spectral measure. The eigenfunctions $\phi_n$ are the vectors in the continuous spectrum, with $\Delta \phi_n = \lambda_n \phi_n$ and $\langle \phi_n, \phi_m \rangle = \delta_{nm}$.

\item \textbf{Regularity (Heat Kernel Bounds):} By heat kernel regularity theory (Theorem \ref{thm:heatKernelBounds}), the eigenfunctions satisfy Holder continuity:
\begin{equation}
\phi_n \in C^{0, \alpha}(X) \quad \text{with exponent} \quad \alpha = 1 - \frac{Q}{4} > 0 \quad \text{(since } Q < 4\text{)}.
\end{equation}

The Holder constant is bounded by:
\begin{equation}
[\phi_n]_{C^{0,\alpha}} \leq C \lambda_n^{(Q+2\alpha)/4},
\end{equation}
where $C$ depends on $Q$ and the Ahlfors regularity constant of $X$.

\item \textbf{Completeness (Sobolev Embedding):} For $Q < 4$, the compact Sobolev embedding:
\begin{equation}
H^{1,2}(X) \hookrightarrow\hookrightarrow L^2(X, \mu)
\end{equation}
implies that the eigenspaces $\text{span}\{\phi_n : \lambda_n \leq \Lambda\}$ grow discretely. The eigenbasis is complete in $L^2(X, \mu)$:
\begin{equation}
\overline{\text{span}\{\phi_n : n \in \mathbb{N}\}} = L^2(X, \mu).
\end{equation}

\item \textbf{Continuous Dependence on $Q$:} The regularity exponent $\alpha = 1 - Q/4$ decreases continuously from $\alpha = 1$ (as $Q \to 0^+$) to $\alpha = 0^+$ (as $Q \to 4^-$). The Holder constants $[\phi_n]_{C^{0,\alpha}}$ remain uniformly bounded in any interval $Q \in (Q_0, 4-\epsilon)$ with $Q_0 < 4$ and $\epsilon > 0$.

\end{enumerate}

\textbf{Logical Independence:} This construction of eigenfunctions is logically independent of the metric structure. The Dirichlet form is defined from the divergence operator (divergence structure axiom in Section A), and the spectral theorem applies immediately. The metric emerges only after eigenfunctions have been constructed (via the Carre du Champ operator), so the derivation follows a hierarchical logical order.

\end{lemma}

\begin{proof}

\textbf{Step 1: Laplacian as Self-Adjoint Operator}

By the Riesz representation theorem, the Dirichlet form $\mathcal{E}(u, v)$ defines a continuous sesquilinear form on $H^{1,2}(X) \times H^{1,2}(X)$. By Kato's theorem (Reed-Simon Vol. II, Theorem X.15), there exists a unique self-adjoint operator $\Delta$ such that:
\begin{equation}
\mathcal{E}(u, v) = \langle \Delta u, v \rangle_{L^2} + (1 + \|\Delta\|) \langle u, v \rangle_{L^2}
\end{equation}
with domain $\Dom(\Delta) = \{u \in H^{1,2} : \Delta u \in L^2\}$.

This domain is independent of any metric (choice, it) is determined purely by the measure $\mu$ and the divergence structure.

\textbf{Step 2: Spectral Decomposition}

By the spectral theorem for self-adjoint operators (Reed-Simon Vol. I, Theorem VIII.7), $\Delta$ admits a spectral decomposition:
\begin{equation}
\Delta = \int_0^\infty \lambda \, dE_\lambda,
\end{equation}
where $E_\lambda$ is the orthogonal spectral measure. The point spectrum $\sigma_p(\Delta)$ (eigenvalues) is discrete and accumulates at infinity.

For each eigenvalue $\lambda_n$, the eigenspace $E_{\lambda_n}$ is finite-dimensional, and it is possible to choose an orthonormal basis $\{\phi_{n,1}, \ldots, \phi_{n,k_n}\}$ of eigenfunctions. Enumerating all eigenfunctions in a single sequence $\{\phi_n : n \in \mathbb{N}\}$ and relabeling eigenvalues $\lambda_n$ (with multiplicity) gives:
\begin{equation}
\Delta \phi_n = \lambda_n \phi_n, \quad \langle \phi_n, \phi_m \rangle = \delta_{nm}.
\end{equation}

\textbf{Step 3: Heat Kernel Regularity}

Consider the heat equation $(\partial_t + \Delta) u = 0$ with initial condition $u(0, x) = \psi(x) \in L^2(X, \mu)$. The solution is:
\begin{equation}
u(t, x) = \int_X p_t(x, y) \psi(y) \, d\mu(y),
\end{equation}
where $p_t(x, y)$ is the heat kernel.

For Polish spaces with Ahlfors regularity dimension $Q$, the heat kernel satisfies the Gaussian bound (Davies, "Heat Kernels and Spectral Theory", Chapter 4):
\begin{equation}
p_t(x, y) \leq C t^{-Q/2} \exp\left(-c \frac{d(x,y)^2}{t}\right).
\end{equation}

In particular, for eigenfunctions ($u(t, \cdot) = e^{-\lambda_n t} \phi_n(\cdot)$):
\begin{equation}
e^{-\lambda_n t} \phi_n(x) = \int_X p_t(x, y) e^{-\lambda_n(t-s)} \phi_n(y) \, d\mu(y).
\end{equation}

By regularity of the heat kernel (Lemma \ref{lem:heatFlowRegularization}), all eigenfunctions satisfy:
\begin{equation}
\phi_n \in C^{0, \alpha}(X) \quad \text{for } \alpha = 1 - \frac{Q}{4}.
\end{equation}

For $Q < 4$, there is $\alpha > 0$, so all eigenfunctions are continuous.

\textbf{Step 4: Completeness via Sobolev Embedding}

For $Q < 4$, the Sobolev embedding theorem (Lemma \ref{lem:polishConsequences}) gives:
\begin{equation}
H^{1,2}(X) \hookrightarrow L^p(X) \quad \text{compactly for all } p < \frac{2Q}{Q-2}.
\end{equation}

By the \cite{biroli2000embedding} compactness theorem, the identity map $H^{1,2}(X) \to L^2(X)$ is compact. This implies the eigenvalues $\lambda_n \to \infty$ discretely (no accumulation below infinity).

For any $\psi \in L^2(X, \mu)$, expand in the eigenbasis:
\begin{equation}
\psi = \sum_{n=1}^\infty a_n \phi_n, \quad a_n = \langle \phi_n, \psi \rangle,
\end{equation}
and the partial sums $\sum_{n=1}^N a_n \phi_n \to \psi$ in $L^2$ by Parseval's identity. Thus:
\begin{equation}
\overline{\text{span}\{\phi_n : n \in \mathbb{N}\}} = L^2(X, \mu).
\end{equation}

\textbf{Step 5: Metric Independence}

The entire construction depends only on:
1. The measure $\mu$ (from Ahlfors regularity axiom).
2. The divergence operator (from divergence structure axiom).
3. The Dirichlet form $\mathcal{E}$ derived from divergence.
4. The spectral theorem (pure functional analysis).

At no point is an a priori Riemannian metric $g$ used. The metric emerges only \textit{after} eigenfunctions are constructed, via the Carre du Champ operator (Definition \ref{def:carreDuChamp}):
\begin{equation}
g_{ij}(x) := \frac{1}{2} \sum_n \frac{\partial_i \phi_n(x) \partial_j \phi_n(x)}{\lambda_n}.
\end{equation}

Thus, eigenfunction existence and regularity are \textit{derived consequences} of measure-theoretic and divergence structures, not presuppositions.

\textbf{Step 6: Continuous Dependence on $Q$}

The Holder exponent $\alpha = 1 - Q/4$ depends continuously on $Q$. For any fixed $Q_0 < 4$ and $\epsilon > 0$, the regularity bounds $[\phi_n]_{C^{0,\alpha}} \leq C \lambda_n^{(Q+2\alpha)/4}$ hold uniformly for $Q \in (Q_0, 4-\epsilon)$. This ensures stability of eigenfunction regularity under small perturbations of dimension.

\end{proof}
