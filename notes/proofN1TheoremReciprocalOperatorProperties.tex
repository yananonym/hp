% proofN3TheoremReciprocalOperatorProperties.tex
% Proof of Theorem: Fundamental Properties of the Reciprocal Transformation Operator
% Supporting material for enhanced Section N3

\subsection*{Proof of Theorem \ref{thm:reciprocalOperatorProperties}}

\textit{Fundamental Properties of $\mathcal{R}$: Isometry, Self-Adjointness, and Involution}

The following derivation establishes that the reciprocal transformation operator is an isometric, self-adjoint involution on the exponential-weight Hilbert space $\mathcal{H}_{\mathrm{exp}} = L^2((0,\infty), e^{-2/u}u^{-1/2}du)$.

\begin{proof}

\textbf{(RO1) Isometry: $\|\mathcal{R}f\|_{\mathcal{H}_{\mathrm{exp}}} = \|f\|_{\mathcal{H}_{\mathrm{exp}}}$}

The norm in $\mathcal{H}_{\mathrm{exp}}$ is:
\begin{equation}
\|f\|_{\mathcal{H}_{\mathrm{exp}}}^2 = \int_0^\infty |f(u)|^2 e^{-2/u} u^{-1/2} \, du.
\end{equation}

For $\mathcal{R}f$:
\begin{equation}
\|\mathcal{R}f\|_{\mathcal{H}_{\mathrm{exp}}}^2 = \int_0^\infty |(\mathcal{R}f)(u)|^2 e^{-2/u} u^{-1/2} \, du = \int_0^\infty |u^{-1/2} f(u^{-1})|^2 e^{-2/u} u^{-1/2} \, du.
\end{equation}

Simplifying:
\begin{equation}
\|\mathcal{R}f\|_{\mathcal{H}_{\mathrm{exp}}}^2 = \int_0^\infty u^{-1} |f(u^{-1})|^2 e^{-2/u} u^{-1/2} \, du = \int_0^\infty |f(u^{-1})|^2 e^{-2/u} u^{-3/2} \, du.
\end{equation}

Now perform the change of variables $v = 1/u$, so $u = 1/v$ and $du = -dv/v^2$:
\begin{align}
\|\mathcal{R}f\|_{\mathcal{H}_{\mathrm{exp}}}^2 
&= \int_\infty^0 |f(v)|^2 e^{-2v} (1/v)^{-3/2} \cdot (-dv/v^2) \\
&= \int_0^\infty |f(v)|^2 e^{-2v} v^{3/2} \cdot dv/v^2 \\
&= \int_0^\infty |f(v)|^2 e^{-2v} v^{-1/2} \, dv \\
&= \|f\|_{\mathcal{H}_{\mathrm{exp}}}^2.
\end{align}

Thus $\mathcal{R}$ is an isometry.

\textbf{(RO2) Self-Adjointness: $\langle \mathcal{R}f, g \rangle = \langle f, \mathcal{R}g \rangle$}

Compute:
\begin{align}
\langle \mathcal{R}f, g \rangle_{\mathcal{H}_{\mathrm{exp}}} 
&= \int_0^\infty (\mathcal{R}f)(u) \overline{g(u)} \, e^{-2/u} u^{-1/2} \, du \\
&= \int_0^\infty u^{-1/2} f(u^{-1}) \overline{g(u)} \, e^{-2/u} u^{-1/2} \, du \\
&= \int_0^\infty f(u^{-1}) \overline{g(u)} \, e^{-2/u} u^{-1} \, du.
\end{align}

Substitute $v = 1/u$:
\begin{align}
\langle \mathcal{R}f, g \rangle 
&= \int_0^\infty f(v) \overline{g(1/v)} \, e^{-2v} v \cdot dv/v^2 \\
&= \int_0^\infty f(v) \overline{g(1/v)} \, e^{-2v} v^{-1} \, dv.
\end{align}

Now compute the other direction:
\begin{align}
\langle f, \mathcal{R}g \rangle_{\mathcal{H}_{\mathrm{exp}}} 
&= \int_0^\infty f(u) \overline{(\mathcal{R}g)(u)} \, e^{-2/u} u^{-1/2} \, du \\
&= \int_0^\infty f(u) \overline{u^{-1/2} g(u^{-1})} \, e^{-2/u} u^{-1/2} \, du \\
&= \int_0^\infty f(u) \overline{g(u^{-1})} \, e^{-2/u} u^{-1} \, du.
\end{align}

This is identical to $\langle \mathcal{R}f, g \rangle$ after renaming $u \to v$. Thus $\mathcal{R}$ is self-adjoint.

\textbf{(RO3) Involution: $\mathcal{R}^2 = I$}

Apply $\mathcal{R}$ twice:
\begin{align}
(\mathcal{R}^2 f)(u) &= (\mathcal{R}(\mathcal{R}f))(u) \\
&= u^{-1/2} (\mathcal{R}f)(u^{-1}) \\
&= u^{-1/2} \cdot (u^{-1})^{-1/2} f((u^{-1})^{-1}) \\
&= u^{-1/2} \cdot u^{1/2} f(u) \\
&= f(u).
\end{align}

Thus $\mathcal{R}^2 = I$.

\textbf{(RO4) Spectrum: $\sigma(\mathcal{R}) = \{+1, -1\}$}

Since $\mathcal{R}$ is a self-adjoint involution, there is $\mathcal{R}^2 = I$, which means $(A - \lambda I)^2 = 0$ only if $\lambda^2 = 1$ for any eigenvalue $\lambda$. Therefore:
\begin{equation}
\sigma(\mathcal{R}) = \{+1, -1\}.
\end{equation}

The eigenspaces are:
\begin{align}
\mathcal{H}_{\mathrm{exp}}^+ &= \{f : \mathcal{R}f = f\} = \{f : f(u^{-1}) = u^{1/2} f(u)\}, \\
\mathcal{H}_{\mathrm{exp}}^- &= \{f : \mathcal{R}f = -f\} = \{f : f(u^{-1}) = -u^{1/2} f(u)\}.
\end{align}

Since $\mathcal{R}$ is self-adjoint with spectrum $\{+1, -1\}$, the space decomposes orthogonally:
\begin{equation}
\mathcal{H}_{\mathrm{exp}} = \mathcal{H}_{\mathrm{exp}}^+ \oplus \mathcal{H}_{\mathrm{exp}}^-.
\end{equation}

\textbf{Orthogonality of the Decomposition}

Let $f^+ \in \mathcal{H}_{\mathrm{exp}}^+$ and $f^- \in \mathcal{H}_{\mathrm{exp}}^-$. Then:
\begin{equation}
\langle f^+, f^- \rangle = \langle \mathcal{R}f^+, \mathcal{R}f^- \rangle = \langle f^+, -f^- \rangle = -\langle f^+, f^- \rangle.
\end{equation}

This implies $\langle f^+, f^- \rangle = 0$. Thus the decomposition is orthogonal.

\end{proof}

\subsection*{Geometric Interpretation}

The reciprocal transformation $\mathcal{R}f(u) = u^{-1/2}f(u^{-1})$ encodes the functional equation symmetry of the Riemann zeta function geometrically:

\begin{enumerate}

\item The transformation $u \to 1/u$ corresponds to the functional equation variable transformation $s \to 1-s$ in the zeta function.

\item The weight factor $u^{-1/2}$ corresponds to the gamma-factor prefactor in the functional equation.

\item The exponential-weight measure $d\mu(u) = e^{-2/u}u^{-1/2}du$ is invariant under $\mathcal{R}$, ensuring the transformation respects the inner product structure.

\item The spectrum $\{+1, -1\}$ of $\mathcal{R}$ means functions split into symmetric and antisymmetric components under the reciprocal transformation.

\item The auxiliary function $h(u)$ satisfies $h \in \mathcal{H}_{\mathrm{exp}}^+$ (symmetric eigenspace) due to its reciprocal symmetry $h(1/u) = u^{1/2}h(u)$.

\end{enumerate}

This geometric encoding of the functional equation as an operator-theoretic property provides a clean, coordinate-independent way to work with the reflection symmetry underlying the Riemann Hypothesis.

\subsection*{Commutation with Other Operators}

An important consequence: the reciprocal operator commutes with any operator that respects the reciprocal symmetry structure, such as the symmetrized HP operator of Definition \ref{def:symmetrizedHPOperator}. This ensures that eigenspaces of the HP operator can be chosen to lie in either $\mathcal{H}_{\mathrm{exp}}^+$ or $\mathcal{H}_{\mathrm{exp}}^-$, simplifying the spectral analysis.

