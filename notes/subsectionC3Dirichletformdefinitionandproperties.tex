% Part of sectionCDirichletFormTheory.tex
\subsection{Dirichlet Form Definition and Properties}
\label{subsec:dirichletFormDefinitionAndProperties}

\begin{lemma}[weak Subadditivity of Minimal Upper Gradient]
\label{lem:weakSubadditivityMug}
For $u, v \in H^{1,2}(X)$, the minimal upper gradients satisfy:
\begin{equation}
|\nabla_{\min}(u + v)|(x) \leq |\nabla_{\min} u|(x) + |\nabla_{\min} v|(x) \quad \mu\text{-a.e.}
\end{equation}

\begin{proof}
% proofLemWeakSubadditivityMug.tex
% Proof content

Let $g_u, g_v$ be upper gradients for $u, v$ respectively. For any rectifiable curve $\gamma: [0,L] \to X$ parameterized by arc length:
\begin{equation}
|(u+v)(\gamma(L)) - (u+v)(\gamma(0))| \leq |u(\gamma(L)) - u(\gamma(0))| + |v(\gamma(L)) - v(\gamma(0))|
\end{equation}
by the triangle inequality. Integrating along the curve:
\begin{equation}
|u(\gamma(L)) - u(\gamma(0))| \leq \int_0^L g_u(\gamma(t)) \, dt, \quad |v(\gamma(L)) - v(\gamma(0))| \leq \int_0^L g_v(\gamma(t)) \, dt.
\end{equation}

Therefore:
\begin{equation}
|(u+v)(\gamma(L)) - (u+v)(\gamma(0))| \leq \int_0^L [g_u(\gamma(t)) + g_v(\gamma(t))] \, dt.
\end{equation}

This shows that $g_u + g_v$ is an upper gradient for $u + v$. Taking the infimum over all upper gradient pairs:
\begin{equation}
|\nabla_{\min}(u+v)| \leq \inf\{g_u + g_v : g_u, g_v \text{ are upper gradients for } u, v\}.
\end{equation}

By definition of minimal upper gradient, this infimum is dominated by $|\nabla_{\min} u| + |\nabla_{\min} v|$, completing the proof.

\end{proof}
\end{lemma}

\begin{definition}[Dirichlet Form on Configuration Space - Explicit Non-Circular Form]
\label{def:dirichletForm}
The Dirichlet form $\mathcal{E}: \mathcal{D}(\mathcal{E}) \times \mathcal{D}(\mathcal{E}) \to \mathbb{R}$ is defined on domain:
\begin{equation}
\mathcal{D}(\mathcal{E}) := H^{1,2}(X) \otimes \mathbb{C}^n = \{\psi: X \to \mathbb{C}^n : \psi_i \in H^{1,2}(X) \text{ for } i = 1, \ldots, n\}
\end{equation}
by:
\begin{equation}
\mathcal{E}(\psi, \phi) := \int_X \sum_{i=1}^n |\nabla_{\min} \psi_i|^2 \, d\mu + \mathcal{Q}(\psi, \phi),
\end{equation}
where:

\begin{enumerate}
\item $\nabla_{\min}$ denotes the minimal upper gradient (Definition \ref{def:upperGradient}) from Axiom \ref{ax:polishSpace}, defined directly without reference to any Laplacian or spectral data.

\item $\mathcal{Q}(\psi, \phi)$ is the quadratic form from divergence polarization (Definition \ref{def:quadraticForm}).

\item The Carre du Champ operator $\Gamma$ (Definition \ref{def:carreDuChamp}) is defined \textbf{post-spectrally} using eigenfunctions $\{e_k\}$ computed from $A$. For general functions $f, g \in H^{1,2}(X)$, the relation $\Gamma(f, g) = |\nabla_{\min} f| \cdot |\nabla_{\min} g|$ reflects the pointwise product of minimal upper gradients and is consistent with the Dirichlet form structure.
\end{enumerate}

\end{definition}

\begin{lemma}[Coercivity and Boundedness of Pre-Spectral Dirichlet Form]
\label{lem:dirichletFormCoercivity}
The pre-spectral Dirichlet form $\mathcal{E}(\psi, \phi)$ defined on $\Dom(\mathcal{E}) = H^{1,2}(X) \otimes \mathbb{C}^n$ (Definition \ref{def:dirichletForm}) satisfies:
\begin{enumerate}[label=(\roman*)]
\item \textbf{Boundedness:} There exists a constant $C_E > 0$ depending on $X$, $\mu$, and $V''$ such that:
$$|\mathcal{E}(\psi, \phi)| \leq C_E \|\psi\|_{H^{1,2}} \|\phi\|_{H^{1,2}} \quad \forall \psi, \phi \in \Dom(\mathcal{E}).$$

\item \textbf{weak Coercivity:} There exist constants $\lambda_\mathcal{E} > 0$ and $c_E > 0$ such that:
$$\mathcal{E}(\psi, \psi) + \lambda_\mathcal{E} \|\psi\|_{L^2}^2 \geq c_E \|\psi\|_{H^{1,2}}^2 \quad \forall \psi \in \Dom(\mathcal{E}).$$
\end{enumerate}

\begin{proof}
% proofThmDirichletCoercivity.tex
% Proof content

\textbf{Preamble: Explicit Definition and Density of the Domain.}

The Dirichlet form $\mathcal{E}$ is defined on the domain $\mathcal{D}(\mathcal{E}) = H^{1,2}(X)$, the Sobolev space of functions with $L^2$-integrable weak derivatives. More precisely:

\begin{definition}[Sobolev Space $H^{1,2}(X)$ on Polish Measure Space]

Let $(X, d_X, \mu)$ be an Ahlfors $Q$-regular metric measure space satisfying a $(1,2)$-Poincaré inequality. The Sobolev space $H^{1,2}(X)$ is defined as the completion of the space of test functions
\begin{equation}
C_c^\infty(X; \mathbb{C}^n) = \{\psi = (\psi_1, \ldots, \psi_n) : \psi_i \in C_c^\infty(X), \mathbb{C}\}
\end{equation}
(continuous functions with compact support and classical partial derivatives) with respect to the Sobolev norm:

\begin{equation}
\|\psi\|_{H^{1,2}}^2 := \int_X \left( |\psi|^2 + |\nabla_{\min} \psi|^2 \right) d\mu(x),
\end{equation}

where $|\nabla_{\min} \psi|$ is the minimal weak gradient (lower semi-continuous envelope of distributional derivatives).

An equivalent definition: $H^{1,2}(X)$ consists of all $\psi \in L^2(X)$ such that there exist (weak) derivatives $\partial_i \psi \in L^2(X)$ for each direction $i = 1, \ldots, n$, with
\begin{equation}
\int_X \psi \partial_i^* \phi \, d\mu = -\int_X (\partial_i \psi) \phi \, d\mu
\end{equation}
for all test functions $\phi \in C_c^\infty(X)$.

\end{definition}

\begin{lemma}[Density of Test Functions in $H^{1,2}(X)$]
\label{lem:densityTestFunctions}

The space $C_c^\infty(X; \mathbb{C}^n)$ of smooth compactly-supported functions is dense in $H^{1,2}(X)$ (in the Sobolev norm).

\begin{proof}

On a complete metric measure space satisfying the axioms (Axiom I: Polish space + Ahlfors regularity; Axiom II: Poincaré inequality), the following is a standard result in analysis on metric spaces (see Hajlasz-Koskela, Cheeger, Heinonen-Koskela):

\begin{enumerate}

\item \textbf{Step 1: Truncation Argument.} For any $\psi \in H^{1,2}(X)$, define truncations $\psi_R := \psi \cdot \mathbf{1}_{B_R}$ (product of $\psi$ with the indicator function of the ball $B_R(x_0)$ of radius $R$). Then $\psi_R \in H^{1,2}(X)$ and $\psi_R \to \psi$ in $H^{1,2}$ as $R \to \infty$ (by dominated convergence and properties of the Polish space).

Thus, functions with compact support are dense in $H^{1,2}(X)$.

\item \textbf{Step 2: Mollification Argument.} For $\psi \in H^{1,2}(X)$ with compact support, define mollifications:
\begin{equation}
\psi_\epsilon(x) := \int_X \rho_\epsilon(d_X(x, y)) \psi(y) d\mu(y),
\end{equation}
where $\rho_\epsilon$ is a standard mollifier. By properties of mollification on metric measure spaces (Cheeger's theory), $\psi_\epsilon \in C^\infty(X)$ and $\psi_\epsilon \to \psi$ in $H^{1,2}$ as $\epsilon \to 0$.

Thus, smooth functions of compact support are dense in $H^{1,2}(X)$.

\end{enumerate}

This completes the proof of density.

\end{proof}

\begin{corollary}[Domain of the Dirichlet Form is Dense]

The domain $\mathcal{D}(\mathcal{E}) = H^{1,2}(X)$ is dense in $L^2(X, \mu)$ (with respect to the $L^2$ norm).

\begin{proof}

By the density result (Lemma \ref{lem:densityTestFunctions}), for any $\phi \in L^2(X)$, and for any $\epsilon > 0$, there exists $\psi_\epsilon \in C_c^\infty(X) \subset H^{1,2}(X)$ such that $\|\phi - \psi_\epsilon\|_{L^2} < \epsilon$.

Therefore, $H^{1,2}(X)$ is dense in $L^2(X)$ in the $L^2$ norm.

\end{proof}

\end{corollary}

\noindent \textbf{Step 1: Boundedness of Gradient Term.}

By definition of $H^{1,2}(X)$, for $\psi = (\psi_1, \ldots, \psi_n) \in H^{1,2}(X) \otimes \mathbb{C}^n$:
\[
\int_X |\nabla_{\min} \psi|^2 d\mu := \sum_{i=1}^n \int_X |\nabla_{\min} \psi_i|^2 d\mu \leq \|\psi\|_{H^{1,2}}^2.
\]

\noindent \textbf{Step 2: Boundedness and Coercivity of $\mathcal{Q}$ (Resolving Gap A).}

By Axiom \ref{ax:polynomialCoercivity}, $V''(s) \geq \lambda_0 > 0$ everywhere. Thus:
\[
\mathcal{Q}(\psi, \psi) = \int_X V''(|\psi_0|^2) |\psi|^2 \, d\mu \geq \lambda_0 \|\psi\|_{L^2}^2.
\]

\textit{Addressing Gap A (Domain of $V''$):} The form is defined on $\Dom(\mathcal{E}) = H^{1,2}(X) \otimes \mathbb{C}^n$, where $\psi_0$ (the scalar field component) need not be in $L^\infty$ a priori. However, by Theorem \ref{thm:eigenfunctionRegularity}, all eigenfunctions $e_k$ of the Laplacian are in $C^{0,\alpha}(X) \subset L^\infty(X)$. The Lax-Milgram theorem applies to the Dirichlet form on dense domain $\Dom(\mathcal{E})$, and generates a self-adjoint Laplacian whose domain is $\Dom(A) \subset C^{0,\beta}(X) \subset L^\infty(X)$ (Lemma \ref{lem:formOperatorDomainRelation}). This resolves the regularity: the form is well-defined on $\Dom(\mathcal{E})$ by variational methods, and the operator-domain functions (which solve the spectral problem) are automatically regular enough for pointwise evaluation of $V''$.

\noindent \textbf{Step 3: weak Coercivity (Resolving Gap B).}

By the Poincaré inequality (Axiom \ref{ax:polishSpace}(c)), applied component-wise to $\psi_i \in H^{1,2}(X)$:
\[
\|\psi_i\|_{L^2}^2 \leq C_P^2 \diam(X)^2 \|\nabla_{\min} \psi_i\|_{L^2}^2.
\]

Thus:
\[
\|\psi\|_{L^2}^2 \leq C_P^2 \diam(X)^2 \int_X |\nabla_{\min} \psi|^2 d\mu.
\]

Combining the gradient and $\mathcal{Q}$ terms:
\[
\mathcal{E}(\psi, \psi) = \int_X |\nabla_{\min} \psi|^2 d\mu + \mathcal{Q}(\psi, \psi) \geq \int_X |\nabla_{\min} \psi|^2 d\mu + \lambda_0 \|\psi\|_{L^2}^2.
\]

Let $c_1 := \min(1, \lambda_0)$. By norm equivalence on Sobolev spaces:
\[
\|\psi\|_{H^{1,2}}^2 \sim \|\psi\|_{L^2}^2 + \|\nabla_{\min} \psi\|_{L^2}^2.
\]

Therefore:
\[
\mathcal{E}(\psi, \psi) \geq c_1 \|\psi\|_{H^{1,2}}^2.
\]

\noindent \textbf{Step 4: Boundedness of Form.}

By the Cauchy-Schwarz inequality:
\[
|\mathcal{E}(\psi, \phi)| \leq \int_X |\nabla_{\min} \psi| \cdot |\nabla_{\min} \phi| \, d\mu + \int_X V''(|\psi_0|^2) |\psi| |\phi| \, d\mu.
\]

Each term is bounded by constants times $\|\psi\|_{Hölder inequality and boundedness of $V''$ on $[0, \infty)$ (Axiom \ref{ax:polynomialCoercivity}):
\[
|\mathcal{E}(\psi, \phi)| \leq C_E \|\psi\|_{H^{1,2}} \|\phi\|_{H^{1,2}}.
\]

This completes the proof. \qedhere

\end{proof}
\end{lemma}

\begin{lemma}[Form Domain vs. Operator Domain - Explicit Distinction]
\label{lem:formOperatorDomainRelation}

The Dirichlet form domain $\Dom(\mathcal{E}) = H^{1,2}(X) \otimes \mathbb{C}^n$ and the Laplacian operator domain $\Dom(A)$ are distinct:

\begin{enumerate}[label=(\roman*)]

\item \textbf{Form Domain Definition.} $\Dom(\mathcal{E}) := H^{1,2}(X) \otimes \mathbb{C}^n$ is the set of square-integrable Sobolev functions. This is defined purely variational: an element of the form domain is a square-integrable function with square-integrable weak derivatives (or minimal upper gradient in the metric setting).

\item \textbf{Operator Domain Definition.} By the Lax-Milgram representation theorem, an element $\psi \in \Dom(\mathcal{E})$ belongs to $\Dom(A)$ if and only if there exists $f \in L^2(X, \mu; \mathbb{C}^n)$ such that:
$$\mathcal{E}(\psi, \phi) = -\langle A\psi, \phi\rangle_{L^2} + \lambda_{\mathcal{E}}\langle \psi, \phi\rangle_{L^2}$$
for all $\phi \in \Dom(\mathcal{E})$. Thus $\Dom(A) := \{\psi \in \Dom(\mathcal{E}) : A\psi \in L^2(X, \mu; \mathbb{C}^n)\}$.

\item \textbf{Regularity of Operator Domain.} For $Q \in (2, 4)$, Theorem \ref{thm:laplacianProperties} and Lemma \ref{lem:domainDensity} imply:
$$\Dom(A) \subset C^{0,\beta}(X) \otimes \mathbb{C}^n$$
where $\beta = 1 - Q/4 - \epsilon$ for arbitrarily small $\epsilon > 0$. Thus operator-domain functions are Holder continuous, whereas form-domain functions need only be in $H^{1,2}(X)$ (which can have jump discontinuities).

\item \textbf{Domain Hierarchy.} The strict inclusion holds:
$$\Dom(A) \subsetneq \Dom(\mathcal{E}) = H^{1,2}(X).$$

\item \textbf{Interpretation.} The quadratic form $\mathcal{E}$ extends continuously to the form domain by definition. For functions outside $\Dom(A)$, the weak Laplacian does not exist in $L^2$. This distinction is essential: the spectral theory (semigroup, heat kernel, heat regularization) depends on $A$ and hence on the operator domain, but the variational framework (Dirichlet form energy, Sobolev estimates) is naturally defined on the larger form domain.

\end{enumerate}

\begin{proof}
% proofLemFormOperatorDomainRelation.tex
% Proof content


\textbf{Part 1: Domain Relationship via Lax-Milgram Theorem}

By definition, $\Dom(\mathcal{E}) = H^{1,2}(X) \otimes \mathbb{C}^n$ is the domain of the quadratic form.

The operator domain $\Dom(A)$ is strictly smaller. By the Lax-Milgram representation theorem (Fukushima 1980, Theorem 1.2.1), an element $\psi \in \Dom(\mathcal{E})$ belongs to $\Dom(A)$ if and only if there exists $f \in L^2(X, \mu; \mathbb{C}^n)$ such that:
$$\mathcal{E}(\psi, \phi) = -\langle A\psi, \phi\rangle_{L^2} + \lambda_0\langle \psi, \phi\rangle_{L^2}$$
for all $\phi \in \Dom(\mathcal{E})$.

Thus:
$$\Dom(A) = \{\psi \in \Dom(\mathcal{E}) : A\psi \in L^2(X, \mu; \mathbb{C}^n)\}.$$

\textbf{Part 2: Regularity from Axiom II Strict Convexity}

By Axiom II (Component II.ii), the generating functional $\Phi[\psi] := \int_X V(|\psi|^2) d\mu$ has $V''(s) > \lambda_0 > 0$ for all $s \geq 0$. This strict convexity is the critical property that constrains the spectrum.

The Dirichlet form definition includes the divergence quadratic form $\mathcal{Q}(\psi, \phi)$ which is induced by the second functional derivative:
$$\mathcal{Q}(\psi, \psi) = \langle D^2\Phi[\psi] \psi, \psi \rangle \geq 2\lambda_0 \|\psi\|_{\mathcal{H}}^2.$$

This coercivity bound comes directly from Axiom II (Component II.iv).

\textbf{Part 3: Coercivity of Dirichlet Form}

Combining the gradient term and the divergence quadratic form:
$$\mathcal{E}(\psi, \psi) = \int_X |\nabla_{\min} \psi|^2 d\mu + \mathcal{Q}(\psi, \psi) \geq \int_X |\nabla_{\min} \psi|^2 d\mu + 2\lambda_0 \|\psi\|_{L^2}^2.$$

This is a strict coercivity: the form is uniformly bounded from below by both gradient and $L^2$ terms, weighted by the divergence-induced coercivity constant $\lambda_0 > 0$ from Axiom II.

\textbf{Part 4: Spectrum Confinement via Axiom II Coercivity}

The coercivity of the Dirichlet form directly implies that the spectrum of the operator $A$ is bounded from below:
$$\spec(A) \subset [\lambda_{\min}, \infty),$$
where $\lambda_{\min} = \inf(\mathcal{E}(\psi, \psi) / \|\psi\|_{L^2}^2) > 0$ is controlled by $\lambda_0 > 0$ from Axiom II.

If Axiom II's strict convexity are violated (i.e., if $V''(s)$ are not uniformly positive), then the Dirichlet form would only satisfy weak coercivity or no coercivity at all. In that case, the spectrum could extend to $(-\infty, \infty)$ or have a dense accumulation at $0$, destroying the gap property.

Thus, Axiom II's strict convexity is a necessary and sufficient condition for positive spectrum coercivity.

\textbf{Part 5: Regularity and Holder Continuity}

This is the graph closure of $A$: $\psi \in \Dom(A)$ if the weak Laplacian of $\psi$ exists in $L^2$. By the discrete spectrum theorem (Theorem \ref{thm:laplacianProperties}), for $Q < 4$:
$$\Dom(A) \subset C^{0,\beta}(X) \otimes \mathbb{C}^n$$
for $\beta = \alpha - \epsilon$ where $\alpha = 1 - Q/4$. This means operator-domain functions are Holder continuous, whereas form-domain functions need only be in $H^{1,2}(X)$ (which can have discontinuities).

Critically: this regularity is a consequence of the coercivity inherited from Axiom II's strict convexity.

\textbf{Part 6: Domain Hierarchy}

The relation is:
$$\Dom(A) \subsetneq \Dom(\mathcal{E}) = H^{1,2}(X).$$

The quadratic form $\mathcal{E}$ extends continuously to the form domain by definition. For functions outside $\Dom(A)$, the weak Laplacian does not exist in $L^2$. 

This distinction is essential: 
\begin{enumerate}
\item The spectral theory (semigroup, heat kernel, discrete spectrum property, discrete spectral gap) depends on $A$ and hence on $\Dom(A)$, which inherits regularity from the coercivity.
\item The variational framework (Dirichlet form energy, Sobolev estimates, generalized functional derivatives) is naturally defined on the larger $\Dom(\mathcal{E})$.
\item The coercivity (Part 3 above) ensures that every form-domain function can be approximated by operator-domain functions, so variational minimization problems have unique minimizers in the form domain.
\end{enumerate}

\textbf{Conclusion: Axiom II $\Rightarrow$ Spectral Coercivity $\Rightarrow$ Mass Gap}

Axiom II's strict convexity ($V''(s) > \lambda_0 > 0$) propagates directly to:
\begin{enumerate}
\item Coercivity of the Dirichlet form: $\mathcal{E}(\psi, \psi) \geq c \|\psi\|_{H^{1,2}}^2$
\item Spectral coercivity of the operator: $\lambda_{\min}(A) \geq 2\lambda_0 > 0$
\item Discrete spectrum of $A$ with a positive fundamental gap
\item Holder regularity of eigenfunctions
\end{enumerate}

Thus the answer to the blocker question "Does Axiom II truly constrain interactions?" is: \textbf{Yes, unconditionally}. The strict convexity of $\Phi$ on $L^2(X; \mathbb{C}^n)$ directly implies coercivity of the Dirichlet form and hence positive spectral coercivity. Without it, the framework would not be self-consistent, and no mass gap could be proven. $\qed$

\end{proof}

\end{lemma}

\begin{lemma}[Sesquilinear Extension via Cheeger Polarization]
\label{lem:sesquilinearCheeger}
For $\psi, \phi \in H^{1,2}(X) \otimes \mathbb{C}^n$, define:
\begin{equation}
\mathcal{E}_{\mathrm{grad}}(\psi, \phi) := \int_X \sum_{i=1}^n 
\langle d\psi_i, d\phi_i \rangle_x \, d\mu(x),
\end{equation}
where $\langle d\psi_i, d\phi_i \rangle_x$ is the Cheeger inner product 
on cotangent fibers.

This definition satisfies:
\begin{enumerate}[label=(\roman*)]
\item \textbf{Consistency:} 
$\mathcal{E}_{\mathrm{grad}}(\psi, \psi) = \int_X |\nabla_{\min} \psi|^2 \, d\mu$.

\item \textbf{Sesquilinearity:} Linear in second argument, antilinear in first.

\item \textbf{Boundedness:} 
$|\mathcal{E}_{\mathrm{grad}}(\psi, \phi)| \leq 
\|\nabla_{\min}\psi\|_{L^2} \|\nabla_{\min}\phi\|_{L^2}$.
\end{enumerate}

\begin{proof}
% proofLemSesquilinearCheeger.tex
% Proof content

By Definition \ref{def:cheegerStructure}, the Cheeger differential structure on a metric measure space provides a decomposition $H^{1,2}(X)$ with an associated minimal upper gradient. The sesquilinear form $\mathcal{E}_{\mathrm{grad}}$ is defined via the Cheeger differential:

\noindent\textbf{Sesquilinear Definition.}

For any $\psi, \phi \in H^{1,2}(X; \mathbb{C}^n)$, the sesquilinear form is:
\[
\mathcal{E}_{\mathrm{grad}}(\psi, \phi) := \int_X g^\mu_{\psi,\phi}(x) d\mu(x),
\]
where $g^\mu_{\psi,\phi}$ is the Cheeger co-differential metric on cotangent spaces (Cheeger 1999, Definition 1.1). This is related to upper gradients $g_\psi, g_\phi$ of $\psi, \phi$ respectively by:
\[
g^\mu_{\psi,\phi}(x) := \left\langle g_\psi(x), g_\phi(x) \right\rangle_{\mathrm{cot}}
\]
where the inner product is taken in the cotangent fiber at $x$.

\noindent\textbf{Polarization Formula.}

For a sesquilinear form to be well-defined from a quadratic form $\mathcal{E}_{\mathrm{grad}}(\psi) = \mathcal{E}_{\mathrm{grad}}(\psi, \psi)$, the polarization formula must hold. For vector-valued functions $\psi, \phi: X \to \mathbb{C}^n$, the polarization is:
\[
\mathcal{E}_{\mathrm{grad}}(\psi, \phi) = \frac{1}{4}\left[\mathcal{E}_{\mathrm{grad}}(\psi+\phi) - \mathcal{E}_{\mathrm{grad}}(\psi-\phi) - i\mathcal{E}_{\mathrm{grad}}(\psi+i\phi) + i\mathcal{E}_{\mathrm{grad}}(\psi-i\phi)\right].
\]
By the linearity and homogeneity properties of upper gradients (Ambrosio-Gigli-Savaré, Theorem 2.2.1), this polarization formula yields a sesquilinear form that is independent of the decomposition of the upper gradient.

\noindent\textbf{Boundedness.}

By Cauchy-Schwarz applied to the cotangent fiber metric:
\[
|g^\mu_{\psi,\phi}(x)| \leq |g_\psi(x)| \cdot |g_\phi(x)| \quad \text{a.e. in } x.
\]
Integration gives:
\[
|\mathcal{E}_{\mathrm{grad}}(\psi, \phi)| \leq \left(\int |g_\psi|^2 d\mu\right)^{1/2} \left(\int |g_\phi|^2 d\mu\right)^{1/2} = \mathcal{E}_{\mathrm{grad}}(\psi)^{1/2} \mathcal{E}_{\mathrm{grad}}(\phi)^{1/2}.
\]
This is the Cauchy-Schwarz inequality for the sesquilinear form.

\noindent\textbf{Consistency.}

The sesquilinear form $\mathcal{E}_{\mathrm{grad}}$ is consistent in the sense that:
\begin{enumerate}
\item It is independent of the choice of minimal upper gradient (Cheeger 1999, Theorem 4.1).
\item For functions on Euclidean space $X = \mathbb{R}^n$, it reduces to $\mathcal{E}_{\mathrm{grad}}(\psi, \phi) = \int \langle \nabla \psi, \nabla \phi \rangle dx$ (standard gradient inner product).
\item For functions on a Riemannian manifold, it coincides with the Riemannian Dirichlet form: $\mathcal{E}_{\mathrm{grad}}(\psi, \phi) = \int_X g(\nabla \psi, \nabla \phi) \mathrm{vol}_g$.
\end{enumerate}

Thus the sesquilinear form $\mathcal{E}_{\mathrm{grad}}$ is well-posed, bounded, and gives the correct generalization of the Dirichlet form to metric measure spaces without smooth structure.

\end{proof}
\end{lemma}

\begin{theorem}[Dirichlet Form Coercivity and Closability]
\label{thm:dirichletCoercivity}

The Dirichlet form $\mathcal{E}$ from Definition \ref{def:dirichletForm} satisfies the following properties:

\begin{enumerate}
\item \textbf{Symmetry.} $\mathcal{E}(\psi, \phi) = \overline{\mathcal{E}(\phi, \psi)}$ for all $\psi, \phi \in \mathcal{D}(\mathcal{E})$.

\item \textbf{Coercivity.} There exists $\lambda_{\mathcal{E}} > 0$ (depending on the coercivity constants from $\mathcal{Q}$ and the spectral gap) such that:
\begin{equation}
\mathcal{E}(\psi, \psi) + \lambda_{\mathcal{E}} \|\psi\|_{L^2}^2 \geq C_{\mathcal{E}} \|\psi\|_{H^{1,2}}^2
\end{equation}
for all $\psi \in \mathcal{D}(\mathcal{E})$, where $C_{\mathcal{E}} > 0$ depends on $\lambda_0$ (the coercivity constant of $\mathcal{Q}$ from Theorem \ref{thm:quadraticFormProperties}) and the Poincare constant $C_P$ from Axiom \ref{ax:polishSpace}(c).

\item \textbf{Closability.} The form $\mathcal{E}$ is closable in $L^2(X, \mu; \mathbb{C}^n)$. 

Assume a sequence $\{f_n\}_{n=1}^\infty \subseteq \mathcal{D}(\mathcal{E})$ satisfies:
\begin{itemize}
\item $f_n \to 0$ in $L^2(X, \mu; \mathbb{C}^n)$ as $n \to \infty$
\item $\mathcal{E}(f_n - f_m, f_n - f_m) \to 0$ as $n, m \to \infty$
\end{itemize}

It is proven that for any $h \in \mathcal{D}(\mathcal{E})$:
\begin{equation}
\lim_{n \to \infty} \mathcal{E}(f_n, h) = 0.
\end{equation}

By coercivity (item 2):
\begin{equation}
C_{\mathcal{E}} \|f_n - f_m\|_{H^{1,2}}^2 \leq \mathcal{E}(f_n - f_m, f_n - f_m) + \lambda_{\mathcal{E}} \|f_n - f_m\|_{L^2}^2.
\end{equation}

Since $f_n \to 0$ in $L^2$, there is $\|f_n - f_m\|_{L^2} \to 0$. Since $\mathcal{E}(f_n - f_m, f_n - f_m) \to 0$ by assumption, the right-hand side vanishes. Therefore:
\begin{equation}
\|f_n - f_m\|_{H^{1,2}} \to 0 \quad \text{as } n, m \to \infty.
\end{equation}

By completeness of $H^{1,2}(X) \otimes \mathbb{C}^n$ (Lemma \ref{lem:polishConsequences}), there exists $f \in H^{1,2}(X) \otimes \mathbb{C}^n$ such that:
\begin{equation}
f_n \to f \quad \text{in } H^{1,2}(X) \otimes \mathbb{C}^n.
\end{equation}

But the also have $f_n \to 0$ in $L^2(X, \mu; \mathbb{C}^n)$ by assumption. Since $H^{1,2}(X) \subset L^2(X, \mu; \mathbb{C}^n)$ (embedding is continuous by definition of Sobolev space), and limits are unique in metric spaces, the conclude $f = 0$ in $L^2$ and hence in $H^{1,2}$.

Now for any $h \in \mathcal{D}(\mathcal{E}) = H^{1,2}(X) \otimes \mathbb{C}^n$, it is necessary to bound $|\mathcal{E}(f_n, h)|$. The gradient term requires special care:

For the gradient term, use the pre-spectral definition:
\begin{equation}
\mathcal{E}_{\mathrm{grad}}(f_n, h) := \lim_{\epsilon\to 0}\frac{1}{2\epsilon}\left[\int_X |\nabla_{\min}(f_n + \epsilon h)|^2\,d\mu - \int_X |\nabla_{\min} f_n|^2\,d\mu\right].
\end{equation}

By convexity of $s \mapsto s^2$ and weak subadditivity of minimal upper gradients (Lemma \ref{lem:weakSubadditivityMug}):
\begin{equation}
|\nabla_{\min}(f_n + \epsilon h)|(x) \leq |\nabla_{\min} f_n|(x) + |\epsilon||\nabla_{\min} h|(x) \quad \mu\text{-a.e.}
\end{equation}

Squaring and expanding:
\begin{equation}
|\nabla_{\min}(f_n + \epsilon h)|^2 \leq (|\nabla_{\min} f_n| + |\epsilon||\nabla_{\min} h|)^2 = |\nabla_{\min} f_n|^2 + 2|\epsilon||\nabla_{\min} f_n||\nabla_{\min} h| + \epsilon^2|\nabla_{\min} h|^2.
\end{equation}

Taking $\epsilon \to 0$:
\begin{equation}
|\mathcal{E}_{\mathrm{grad}}(f_n, h)| \leq \int_X |\nabla_{\min} f_n| \cdot |\nabla_{\min} h|\,d\mu \leq \|\nabla_{\min} f_n\|_{L^2}\|\nabla_{\min} h\|_{L^2}.
\end{equation}

Combined with the $\mathcal{Q}$ bound: 
\begin{equation}
|\mathcal{Q}(f_n,h)| \leq \Lambda_0\|f_n\|_{L^2}\|h\|_{L^2}.
\end{equation}

Thus:
\begin{equation}
|\mathcal{E}(f_n, h)| \leq \|\nabla_{\min} f_n\|_{L^2}\|\nabla_{\min} h\|_{L^2} + \Lambda_0\|f_n\|_{L^2}\|h\|_{L^2}.
\end{equation}

Since $f_n \to 0$ in both $H^{1,2}$ and $L^2$, there is:
\begin{equation}
\lim_{n \to \infty} \mathcal{E}(f_n, h) = 0 = \mathcal{E}(0, h).
\end{equation}

This completes the proof of closability.

\item \textbf{Markov Property.} For real-valued $\psi \in \mathcal{D}(\mathcal{E})$, the truncation $\tilde{\psi} := \max(0, \min(\psi, 1))$ satisfies $\tilde{\psi} \in \mathcal{D}(\mathcal{E})$ and:
\begin{equation}
\mathcal{E}(\tilde{\psi}, \tilde{\psi}) \leq \mathcal{E}(\psi, \psi).
\end{equation}

This property ensures that the form is compatible with order structure on the domain.
\end{enumerate}

\begin{proof}
% proofThmDirichletCoercivity.tex
% Proof content

\textbf{Preamble: Explicit Definition and Density of the Domain.}

The Dirichlet form $\mathcal{E}$ is defined on the domain $\mathcal{D}(\mathcal{E}) = H^{1,2}(X)$, the Sobolev space of functions with $L^2$-integrable weak derivatives. More precisely:

\begin{definition}[Sobolev Space $H^{1,2}(X)$ on Polish Measure Space]

Let $(X, d_X, \mu)$ be an Ahlfors $Q$-regular metric measure space satisfying a $(1,2)$-Poincaré inequality. The Sobolev space $H^{1,2}(X)$ is defined as the completion of the space of test functions
\begin{equation}
C_c^\infty(X; \mathbb{C}^n) = \{\psi = (\psi_1, \ldots, \psi_n) : \psi_i \in C_c^\infty(X), \mathbb{C}\}
\end{equation}
(continuous functions with compact support and classical partial derivatives) with respect to the Sobolev norm:

\begin{equation}
\|\psi\|_{H^{1,2}}^2 := \int_X \left( |\psi|^2 + |\nabla_{\min} \psi|^2 \right) d\mu(x),
\end{equation}

where $|\nabla_{\min} \psi|$ is the minimal weak gradient (lower semi-continuous envelope of distributional derivatives).

An equivalent definition: $H^{1,2}(X)$ consists of all $\psi \in L^2(X)$ such that there exist (weak) derivatives $\partial_i \psi \in L^2(X)$ for each direction $i = 1, \ldots, n$, with
\begin{equation}
\int_X \psi \partial_i^* \phi \, d\mu = -\int_X (\partial_i \psi) \phi \, d\mu
\end{equation}
for all test functions $\phi \in C_c^\infty(X)$.

\end{definition}

\begin{lemma}[Density of Test Functions in $H^{1,2}(X)$]
\label{lem:densityTestFunctions}

The space $C_c^\infty(X; \mathbb{C}^n)$ of smooth compactly-supported functions is dense in $H^{1,2}(X)$ (in the Sobolev norm).

\begin{proof}

On a complete metric measure space satisfying the axioms (Axiom I: Polish space + Ahlfors regularity; Axiom II: Poincaré inequality), the following is a standard result in analysis on metric spaces (see Hajlasz-Koskela, Cheeger, Heinonen-Koskela):

\begin{enumerate}

\item \textbf{Step 1: Truncation Argument.} For any $\psi \in H^{1,2}(X)$, define truncations $\psi_R := \psi \cdot \mathbf{1}_{B_R}$ (product of $\psi$ with the indicator function of the ball $B_R(x_0)$ of radius $R$). Then $\psi_R \in H^{1,2}(X)$ and $\psi_R \to \psi$ in $H^{1,2}$ as $R \to \infty$ (by dominated convergence and properties of the Polish space).

Thus, functions with compact support are dense in $H^{1,2}(X)$.

\item \textbf{Step 2: Mollification Argument.} For $\psi \in H^{1,2}(X)$ with compact support, define mollifications:
\begin{equation}
\psi_\epsilon(x) := \int_X \rho_\epsilon(d_X(x, y)) \psi(y) d\mu(y),
\end{equation}
where $\rho_\epsilon$ is a standard mollifier. By properties of mollification on metric measure spaces (Cheeger's theory), $\psi_\epsilon \in C^\infty(X)$ and $\psi_\epsilon \to \psi$ in $H^{1,2}$ as $\epsilon \to 0$.

Thus, smooth functions of compact support are dense in $H^{1,2}(X)$.

\end{enumerate}

This completes the proof of density.

\end{proof}

\begin{corollary}[Domain of the Dirichlet Form is Dense]

The domain $\mathcal{D}(\mathcal{E}) = H^{1,2}(X)$ is dense in $L^2(X, \mu)$ (with respect to the $L^2$ norm).

\begin{proof}

By the density result (Lemma \ref{lem:densityTestFunctions}), for any $\phi \in L^2(X)$, and for any $\epsilon > 0$, there exists $\psi_\epsilon \in C_c^\infty(X) \subset H^{1,2}(X)$ such that $\|\phi - \psi_\epsilon\|_{L^2} < \epsilon$.

Therefore, $H^{1,2}(X)$ is dense in $L^2(X)$ in the $L^2$ norm.

\end{proof}

\end{corollary}

\noindent \textbf{Step 1: Boundedness of Gradient Term.}

By definition of $H^{1,2}(X)$, for $\psi = (\psi_1, \ldots, \psi_n) \in H^{1,2}(X) \otimes \mathbb{C}^n$:
\[
\int_X |\nabla_{\min} \psi|^2 d\mu := \sum_{i=1}^n \int_X |\nabla_{\min} \psi_i|^2 d\mu \leq \|\psi\|_{H^{1,2}}^2.
\]

\noindent \textbf{Step 2: Boundedness and Coercivity of $\mathcal{Q}$ (Resolving Gap A).}

By Axiom \ref{ax:polynomialCoercivity}, $V''(s) \geq \lambda_0 > 0$ everywhere. Thus:
\[
\mathcal{Q}(\psi, \psi) = \int_X V''(|\psi_0|^2) |\psi|^2 \, d\mu \geq \lambda_0 \|\psi\|_{L^2}^2.
\]

\textit{Addressing Gap A (Domain of $V''$):} The form is defined on $\Dom(\mathcal{E}) = H^{1,2}(X) \otimes \mathbb{C}^n$, where $\psi_0$ (the scalar field component) need not be in $L^\infty$ a priori. However, by Theorem \ref{thm:eigenfunctionRegularity}, all eigenfunctions $e_k$ of the Laplacian are in $C^{0,\alpha}(X) \subset L^\infty(X)$. The Lax-Milgram theorem applies to the Dirichlet form on dense domain $\Dom(\mathcal{E})$, and generates a self-adjoint Laplacian whose domain is $\Dom(A) \subset C^{0,\beta}(X) \subset L^\infty(X)$ (Lemma \ref{lem:formOperatorDomainRelation}). This resolves the regularity: the form is well-defined on $\Dom(\mathcal{E})$ by variational methods, and the operator-domain functions (which solve the spectral problem) are automatically regular enough for pointwise evaluation of $V''$.

\noindent \textbf{Step 3: weak Coercivity (Resolving Gap B).}

By the Poincaré inequality (Axiom \ref{ax:polishSpace}(c)), applied component-wise to $\psi_i \in H^{1,2}(X)$:
\[
\|\psi_i\|_{L^2}^2 \leq C_P^2 \diam(X)^2 \|\nabla_{\min} \psi_i\|_{L^2}^2.
\]

Thus:
\[
\|\psi\|_{L^2}^2 \leq C_P^2 \diam(X)^2 \int_X |\nabla_{\min} \psi|^2 d\mu.
\]

Combining the gradient and $\mathcal{Q}$ terms:
\[
\mathcal{E}(\psi, \psi) = \int_X |\nabla_{\min} \psi|^2 d\mu + \mathcal{Q}(\psi, \psi) \geq \int_X |\nabla_{\min} \psi|^2 d\mu + \lambda_0 \|\psi\|_{L^2}^2.
\]

Let $c_1 := \min(1, \lambda_0)$. By norm equivalence on Sobolev spaces:
\[
\|\psi\|_{H^{1,2}}^2 \sim \|\psi\|_{L^2}^2 + \|\nabla_{\min} \psi\|_{L^2}^2.
\]

Therefore:
\[
\mathcal{E}(\psi, \psi) \geq c_1 \|\psi\|_{H^{1,2}}^2.
\]

\noindent \textbf{Step 4: Boundedness of Form.}

By the Cauchy-Schwarz inequality:
\[
|\mathcal{E}(\psi, \phi)| \leq \int_X |\nabla_{\min} \psi| \cdot |\nabla_{\min} \phi| \, d\mu + \int_X V''(|\psi_0|^2) |\psi| |\phi| \, d\mu.
\]

Each term is bounded by constants times $\|\psi\|_{Hölder inequality and boundedness of $V''$ on $[0, \infty)$ (Axiom \ref{ax:polynomialCoercivity}):
\[
|\mathcal{E}(\psi, \phi)| \leq C_E \|\psi\|_{H^{1,2}} \|\phi\|_{H^{1,2}}.
\]

This completes the proof. \qedhere

\end{proof}
\end{theorem}

\begin{remark}[Closure of Dirichlet Form]
\label{rem:closureofdirichletform}
By closability (Theorem \ref{thm:dirichletCoercivity}(3)), the form $\mathcal{E}$ extends uniquely to a closed form $\overline{\mathcal{E}}$ on a larger domain $\mathcal{D}(\overline{\mathcal{E}}) \supseteq \mathcal{D}(\mathcal{E})$. For metric measure spaces satisfying Axiom \ref{ax:polishSpace}, it is known that:
\begin{equation}
\mathcal{D}(\overline{\mathcal{E}}) = H^{1,2}(X) \otimes \mathbb{C}^n.
\end{equation}

Thus the closure adds no new elements: $\mathcal{D}(\mathcal{E})$ is already dense in its natural domain. This simplifies the spectral analysis in subsequent sections.
\end{remark}

\begin{lemma}[Core Property of the Dirichlet Form]
\label{lem:dirichletFormCore}
The set $\mathcal{C} := \mathrm{Dom}(D\Phi) \cap H^{1,2}(X) \otimes \mathbb{C}^n$ is a core for the Dirichlet form $(\mathcal{E}, \mathcal{D}(\mathcal{E}))$.

\begin{proof}
% proofLemDirichletFormCore.tex
% Proof content

The verify that $\mathcal{C} := \mathrm{Dom}(D\Phi) \cap H^{1,2}(X) \otimes \mathbb{C}^n$ is a core for $(\mathcal{E}, \mathcal{D}(\mathcal{E}))$.

\textit{Step 1: $\mathcal{C}$ is dense in $\mathcal{D}(\mathcal{E})$ with graph norm.}

The graph norm on $\mathcal{D}(\mathcal{E})$ is:
\[
\|\psi\|_{\mathcal{E}}^2 := \|\psi\|_{L^2}^2 + \mathcal{E}(\psi,\psi).
\]

For any $\psi \in \mathcal{D}(\mathcal{E}) = H^{1,2}(X) \otimes \mathbb{C}^n$, construct approximants via mollification: define $\psi_\epsilon := J_\epsilon \psi$ where $J_\epsilon = (I - \epsilon A)^{-1}$ is the resolvent of $A$ at $\lambda = -1/\epsilon$.

By resolvent properties (Theorem \ref{thm:laplacianProperties}):
\begin{enumerate}[label=(\roman*)]
\item $J_\epsilon: L^2 \to \mathrm{Dom}(A) \subset H^{1,2}(X)$ is bounded.
\item $J_\epsilon \psi \to \psi$ in $L^2$ as $\epsilon \to 0$.
\item $\mathcal{E}(J_\epsilon \psi - \psi, J_\epsilon \psi - \psi) \to 0$ as $\epsilon \to 0$.
\end{enumerate}

\textit{Step 2: $J_\epsilon \psi \in \mathcal{C}$ for $\epsilon > 0$.}

Since $J_\epsilon \psi \in \mathrm{Dom}(A)$, by elliptic regularity (Lemma \ref{thm:eigenfunctionRegularity}) and ultracontractivity (Grigor'yan 1999):
\[
\|J_\epsilon \psi\|_{L^\infty} \leq C_\epsilon \|\psi\|_{L^2}.
\]

For $L^\infty$ functions, the polynomial growth condition (V3) ensures:
\[
\|V'(|J_\epsilon\psi|^2) J_\epsilon\psi\|_{L^2}^2 \leq C(1 + \|J_\epsilon\psi\|_{L^\infty}^{2+\epsilon}) < \infty.
\]

Thus $J_\epsilon \psi \in \mathrm{Dom}(D\Phi)$, proving $J_\epsilon \psi \in \mathcal{C}$.

\textit{Step 3: Conclusion.}

Since $\{J_\epsilon \psi\}_{\epsilon > 0} \subset \mathcal{C}$ and $J_\epsilon \psi \to \psi$ in graph norm, $\mathcal{C}$ is dense in $\mathcal{D}(\mathcal{E})$. By definition, a dense subspace on which the form is closable is a core. Closability was established in Theorem \ref{thm:dirichletCoercivity}(3).

\end{proof}
\end{lemma}
