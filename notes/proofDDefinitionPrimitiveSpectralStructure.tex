% proofDefPrimitiveSpectralStructure.tex
% Definition of Primitive Spectral Structure (Blocker B1 Fix)

\begin{definition}[Primitive Spectral Structure]
\label{def:primitiveSpectralStructure}

Let $(X,\mu)$ be a $\sigma$-finite measure space. A primitive spectral structure is a densely-defined, closed, symmetric, positive quadratic form
\[
\mathcal{E}: D(\mathcal{E}) \subset L^2(X,\mu) \to \mathbb{R}
\]
satisfying:

\begin{enumerate}

\item \textbf{(Markov Property)} For every $u \in D(\mathcal{E})$, the truncation $u^\edge := \min(1, \max(u, 0))$ (the indicator of $\{u > 0\}$) satisfies $u^\edge \in D(\mathcal{E})$ and
\[
\mathcal{E}(u^\edge, u^\edge) \leq \mathcal{E}(u, u).
\]
This ensures that the form respects the probabilistic structure of harmonic functions.

\item \textbf{(Locality)} For every $u, v \in D(\mathcal{E})$, if $u$ is constant on a neighborhood of $\operatorname{supp}(v)$ (in the sense that $u$ is locally constant relative to the measure-theoretic support), then $\mathcal{E}(u, v) = 0$. This encodes the local character of the energy and is equivalent to saying that the form is generated by a local operator (e.g., a differential operator).

\item \textbf{(Poincaré Inequality)} There exists a constant $C_P > 0$ (the Poincaré constant) such that for every bounded measurable $U \subset X$ with $0 < \mu(U) < \infty$, and every $u \in D(\mathcal{E})$ supported in $U$,
\[
\int_U \left|u - u_U\right|^2 \, d\mu \leq C_P \mathcal{E}(u, u),
\]
where $u_U := \frac{1}{\mu(U)} \int_U u \, d\mu$ is the average of $u$ over $U$. The Poincaré inequality is equivalent to a spectral gap condition: the smallest positive eigenvalue of the associated generator is bounded below by $C_P^{-1}$.

\end{enumerate}

\end{definition}
