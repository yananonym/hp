% Part of Section 0: Introduction
\subsection{Inverting the Foundational Hierarchy}
\label{subsec:theInversion}

Contemporary physics constructs fundamental theory by placing spacetime as a given substrate within which quantum fields and forces operate. Whether spacetime is treated as a fixed background, quantized through canonical methods, or quantized via path integrals, the hierarchy remains unchanged: geometry precedes dynamics. All physical phenomena unfold within pre-assumed geometric structure.

This framework inverts the hierarchy entirely. Rather than treating spacetime as given and asking how quantum mechanics and interactions emerge within it, the approach takes as its primitive notion an asymmetric Bregman divergence. This divergence originates in information geometry and quantifies how probability distributions differ from one another. The foundational claim is unambiguous: spacetime geometry, quantum mechanics, gauge interactions, and gravitational dynamics emerge as necessary mathematical consequences of divergence structure. All additional physical input beyond the two axioms is required.

Divergence generates spacetime. The primitive object is pre-geometric. Geometry emerges from measure-theoretic foundations as a derived structure. The dimensionality of spacetime emerges from consistency conditions, not assumption. Its signature, whether Euclidean or Lorentzian, emerges from divergence asymmetry rather than assumption. The gauge symmetries governing interactions emerge from divergence structure, not assumption. The particle content of the universe emerges from topological structure, not assumption. All of these follow uniquely from requiring internal mathematical consistency. This inversion represents the defining innovation: information structure determines the physical arena.
