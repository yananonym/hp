% proofLemCompletionConsistency.tex
% Proof content


\textbf{Proof of Lemma \ref{lem:completionConsistency}}

The axiom $(X, \mathcal{B}(X), \mu)$ states that the measure is complete: every subset of a $\mu$-null set is measurable. the verify that this completion is transparent to the Sobolev and Dirichlet form constructions; they can be developed on the Borel $\sigma$-algebra $\mathcal{B}(X)$ without needing the completion.

\textit{\underline{Part (i): Borel Measurability of Minimal Upper Gradient}}

By \cite{cheeger1999differentiation}, Theorem 4.38), for any $u \in H^{1,2}(X)$, the minimal upper gradient $|\nabla_{\min} u|$ admits a \emph{Borel} measurable representative.

\textit{Proof:} The minimal upper gradient is defined as:
\begin{equation}
|\nabla_{\min} u|(x) := \inf \{g(x) : g \text{ is an upper gradient of } u\}.
\end{equation}

For each $n \in \mathbb{N}$, the set of upper gradients $g$ with $\|g\|_{L^2} \leq n$ is a closed convex subset of $L^2(X, \mu)$. The infimum over this set is a lower semicontinuous function of $x$, hence Borel measurable.

Consequently, $|\nabla_{\min} u|$ is the pointwise infimum of a countable family of Borel functions (taking $n = 1, 2, 3, \ldots$), and lower semicontinuous functions are Borel. Thus $|\nabla_{\min} u|$ is Borel measurable.

\textit{\underline{Part (ii): Dirichlet Form Definition on Borel $\sigma$-algebra}}

The Dirichlet form is defined as:
\begin{equation}
\mathcal{E}(u, v) := \int_X \langle du, dv \rangle d\mu
\end{equation}
for $u, v \in H^{1,2}(X)$, where $\langle du, dv \rangle$ is defined via the Cheeger differentiable structure (Lemma \ref{lem:cheegerStructure}).

By Sturm (2003, Theorem 4.5), this form is well-defined on $H^{1,2}(X)$ using only Borel measurable representatives. The completion $\bar{\mathcal{E}}$ of $\mathcal{E}$ (in the norm $\sqrt{\mathcal{E}(u,u) + \|u\|_{L^2}^2}$) extends the form to $H^{1,2}(X)$.

The completion of the measure does not enter: the integral $\int_X$ is over Borel sets and Borel measurable functions. The fact that $\mu$ is complete means that sets of $\mu$-measure zero have the property that subsets of zero-measure sets are measurable, but this is relevant only for conditional expectation and disintegration, not for the basic form definition.

\textit{\underline{Part (iii): Eigenfunction Constructions}}

The spectral theory of the semigroup $(e^{tA})_{t \geq 0}$ generated by the Dirichlet form requires only measure completion. Specifically:

\begin{enumerate}[label=(\roman*)]
\item \textbf{Eigenfunction Definition:} An eigenfunction $e_k$ satisfies:
\begin{equation}
\mathcal{E}(e_k, v) = \lambda_k \int_X e_k v \, d\mu \quad \text{for all } v \in H^{1,2}(X).
\end{equation}

By the spectral theorem for self-adjoint operators (via Stone-Weierstrass and the Riesz representation theorem), eigenfunctions exist and are unique up to $\mu$-null sets. Crucially, $e_k$ can be chosen to be \emph{Borel measurable}: by regularity of the heat kernel (Grigor'yan 1999), the eigenfunctions can be taken continuous on $X$, hence Borel.

\item \textbf{Orthogonality and Completeness:} The system $\{e_k\}$ is orthogonal and complete in $L^2(X, \mu)$:
\begin{equation}
\int_X e_j e_k d\mu = \delta_{jk}, \quad L^2(X,\mu) = \overline{\text{span}\{e_k : k \in \mathbb{N}\}}.
\end{equation}

These properties hold with respect to integration against Borel measurable sets and functions.

\item \textbf{Holder Regularity:} By Theorem \ref{thm:eigenfunctionRegularity}, for $Q < 4$, all eigenfunctions satisfy:
\begin{equation}
e_k \in C^{0,\alpha}(X), \quad \alpha = 1 - Q/4 > 0.
\end{equation}

Continuous functions are Borel measurable.
\end{enumerate}

\textit{\underline{Part (iv): Completion is Orthogonal to Core Spectral Constructions}}

The measure completion is relevant only when conditioning on sub-$\sigma$-algebras or when extending from the cylinder algebra to the full Borel $\sigma$-algebra in path integral constructions (as in Theorem \ref{thm:pathIntegralConstruction}).

In static measure-theoretic definitions (eigenfunction spaces, Sobolev norms, Dirichlet forms), the completion does not appear:
\begin{itemize}
\item The definition of $H^{1,2}(X)$ uses only measurability, not completeness.
\item Minimal upper gradients are Borel (as shown above).
\item Integration $\int_X f \, d\mu$ for Borel measurable $f$ is well-defined without completion.
\end{itemize}

The completion is introduced for convenience in probability theory (conditioning, independence), but all core constructions use Borel sets only.

\textit{\underline{Conclusion}}

Under Axiom \ref{ax:polishSpace}:
\begin{enumerate}[label=(\roman*)]
\item The minimal upper gradient $|\nabla_{\min} u|$ is Borel measurable, not merely measurable with respect to the completion.

\item The Dirichlet form $\mathcal{E}$ is well-defined on the Borel $\sigma$-algebra $\mathcal{B}(X)$ without invoking measure completion.

\item All eigenfunction constructions and their Holder regularity are derived using only Borel measurable objects.

\item The completion is transparent: it does not alter any Borel-level quantities and is relevant only for probability-theoretic extensions (as in path integral constructions).
\end{enumerate}

Therefore, the completion is fully consistent with all Sobolev and spectral constructions throughout the divergence-first theory of quantum gravity.

\qed
