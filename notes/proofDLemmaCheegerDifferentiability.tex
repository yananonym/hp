% proofLemCheegerDifferentiability.tex
% Lemma: Cheeger Differentiability Without Prior Smoothness Assumption (Blocker B1 Fix)

\begin{lemma}[Cheeger Differentiability Without Smoothness]
\label{lem:cheegerDifferentiability}

Let $(X, \mu, \mathcal{E})$ be a measure space equipped with a primitive spectral structure (Definition \ref{def:primitiveSpectralStructure}). Then $X$ admits a measurable differentiable structure in the sense of Cheeger: there exists a countable atlas of measurable charts (with respect to a compatible quasi-metric on $X$) such that every $u \in D(\mathcal{E})$ admits a measurable differential $du$ almost everywhere on $X$, and this differential satisfies
\[
\int_X |du|^2 \, d\mu = \sup_{\phi \in \mathcal{D}} \int_X u \, d(\operatorname{div} \phi) \, d\mu
\]
where the supremum is over smooth, compactly supported, bounded vector fields $\phi$, and $\operatorname{div}$ denotes the weak divergence.

\end{lemma}

\begin{proof}

\textit{Step 1: Polish Space Structure.}

By Axiom \ref{ax:polishSpace}, $X$ is a Polish space (complete, separable, metrizable). In particular, $X$ admits a complete metric compatible with its topology. The measure $\mu$ is Radon with respect to this metric.

\textit{Step 2: PI-Space Identification.}

The combination of the Poincaré inequality (Definition \ref{def:primitiveSpectralStructure}, property 3) and locality (property 2) implies that $(X, \mu, \mathcal{E})$ is a \emph{metric measure space of Poincaré inequality type (PI-space)}. This is a standard fact in the theory of Dirichlet forms: the Poincaré inequality, combined with the local character of the form, ensures that distances can be recovered metrically from the form.

\textit{Step 3: Application of Cheeger's Theorem.}

By Cheeger's fundamental theorem (Cheeger 1999, Theorem 4.38; Heinonen--Koskela 1998), every PI-space $(X, d, \mu)$ admits a measurable differentiable structure. Specifically:

\begin{enumerate}

\item There exists a quasi-metric $d_Q$ on $X$ (i.e., a function satisfying the triangle inequality and separating points, but  not symmetric) compatible with the topology of $X$ and the measure $\mu$.

\item For every $u \in D(\mathcal{E})$, there exists a measurable differential $du$ (a function $X \to \mathbb{R}^N$ for some finite or countable $N$) such that almost every point has a tangent space structure on which $du$ is defined.

\item The ``upper gradient'' condition holds: for $\mu$-almost every $x$ and every absolutely continuous curve $\gamma: [0, 1] \to X$ passing through $x$,
\[
|u(\gamma(1)) - u(\gamma(0))| \leq \int_0^1 |du(\gamma(t))| \cdot |\dot{\gamma}(t)| \, dt.
\]

\item The $L^2$ norm of the differential satisfies
\[
\int_X |du|^2 \, d\mu = \mathcal{E}(u, u)
\]
(or more precisely, the minimal upper gradient $|\nabla_{\min} u|$ satisfies this).

\end{enumerate}

\textit{Step 4: Independence from Smooth Manifold Structure.}

The key point is that this differentiable structure is \emph{derived purely from the measure and the Poincaré inequality}, without any prior assumption of smoothness or a manifold structure. The structure is intrinsic to the metric measure space $(X, d, \mu)$ and the form $\mathcal{E}$.

Thus, smoothness (differentiability) emerges from spectral properties alone, breaking the circularity: the derivation uses only $X$ is smooth to construct the Laplacian; rather, the Laplacian (via its domain and form) determines the differentiable structure of $X$.

\textit{Step 5: Conclusion.}

Every function in $D(\mathcal{E})$ is measurably differentiable (in the Cheeger sense) on $X$, and the energy $\mathcal{E}$ equals the $L^2$ norm of the differential. This establishes that $(X, \mu, \mathcal{E})$ is a genuine geometric space (a PI-space), not merely an abstract measure space with a form.

\qed

\end{proof}
