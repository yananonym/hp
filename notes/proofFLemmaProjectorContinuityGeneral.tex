% proofLemProjectorContinuityGeneral.tex
% Proof content


\textbf{Proof of Lemma \ref{lem:projectorContinuityGeneral}}

\begin{proof}
Let $A_\delta = A + \delta B$ be a perturbation with $\|B\|_{\text{op}} \leq M$ and $\delta \in [0, \delta_0]$ sufficiently small.

\textit{\underline{Step 1: Isolated Eigenvalue and Spectral Gap}}

By Lemma~\ref{lem:spectralGapComplete}, each eigenvalue $\lambda_N$ of $A$ is isolated with discrete spectrum $\sigma(A) = \{\lambda_0 < \lambda_1 < \lambda_2 < \cdots\}$ satisfying $\lambda_k \to \infty$.

Define the spectral gap at $\lambda_N$:
\[
\gamma_N := \min(|\lambda_N - \lambda_{N-1}|, |\lambda_{N+1} - \lambda_N|) > 0.
\]

By the spectral gap theorem (Theorem \ref{thm:spectralGapCompleteness}), there exist universal constants $c_0 > 0$ and $\alpha > 0$ such that:
\[
\gamma_N \geq c_0 \lambda_N^{1-\alpha} \quad \text{for all } N \geq 0.
\]

In particular, for the fundamental eigenvalue $\lambda_0$, the gap is $\gamma_0 = \lambda_1 - \lambda_0 \geq c_0 > 0$ (bounded away from zero).

\textit{\underline{Step 2: Resolvent Perturbation Bound}}

For self-adjoint operators $A_0$ and $A_\epsilon = A_0 + \delta A$ with $\|\delta A\|_{\text{op}} \leq C\epsilon$, the resolvent difference satisfies:
\begin{equation}
\|(A_\epsilon - z)^{-1} - (A_0 - z)^{-1}\|_{\text{op}} 
= \|(A_0 + \delta A - z)^{-1} (\delta A) (A_0 - z)^{-1}\|_{\text{op}}
\leq \frac{C\epsilon}{|\mathrm{Im}(z)|^2} \quad \text{for } z \in \mathbb{C} \setminus \mathbb{R}.
\end{equation}

\textit{\underline{Step 2b: Spectral Projector from Contour Integral}}

The $k$-th spectral projector is:
\begin{equation}
P_k(\epsilon) = \frac{1}{2\pi i} \oint_{\mathcal{C}_k} (A_\epsilon - z)^{-1} dz,
\end{equation}
where $\mathcal{C}_k$ is a contour enclosing $\lambda_k(\epsilon)$ but no other eigenvalues.

\textit{\underline{Step 3: Projector Difference Bound}}

By deformation of the contour:
\begin{equation}
P_k(\epsilon) - P_k(0) = \frac{1}{2\pi i} \oint_{\mathcal{C}_k} [(A_\epsilon - z)^{-1} - (A_0 - z)^{-1}] dz.
\end{equation}

Using Step 2 and choosing $\mathcal{C}_k$ at distance $\gamma/2$ from the spectrum:
\begin{equation}
\|P_k(\epsilon) - P_k(0)\|_{\text{op}} \leq \frac{1}{2\pi} \oint_{\mathcal{C}_k} \frac{C\epsilon}{(\gamma/2)^2} |dz| 
\leq \frac{C\epsilon}{\gamma^2} \cdot \text{circumference of } \mathcal{C}_k.
\end{equation}

For a circular contour of radius $r \sim \gamma$:
\begin{equation}
\|P_k(\epsilon) - P_k(0)\|_{\text{op}} \leq \frac{C\epsilon}{\gamma}.
\end{equation}

This bound is uniform for $\epsilon \in [0, \epsilon_0]$ where $\epsilon_0 := \gamma/4$ ensures that perturbations do not shift eigenvalues out of the contour.

\textit{\underline{Step 4: Uniform Stability}}

Since $\gamma = \lambda_1 - \lambda_0 > 0$ (spectral gap from Theorem \ref{thm:spectralEmbedding}), the bound holds uniformly. For small enough $\epsilon$ (depending on $\gamma$), the spectral projectors form a Lipschitz-continuous family.

\textit{\underline{Old Step 2: Resolvent Perturbation Theory}}

Consider a circle $\Gamma_N$ of radius $r_N = \gamma_N/4$ centered at $\lambda_N$. For $z \in \Gamma_N$, the resolvent $(A - z)^{-1}$ is well-defined with:
\[
\|(A - z)^{-1}\|_{\text{op}} \leq \frac{1}{\text{dist}(z, \sigma(A))} \leq \frac{4}{\gamma_N}.
\]

The perturbed resolvent satisfies:
\[
(A_\delta - z)^{-1} = (A - z)^{-1} - \delta (A - z)^{-1} B (A - z)^{-1} + \delta^2 (A - z)^{-1} B (A - z)^{-1} B (A - z)^{-1} + \cdots
\]

By the Neumann series expansion:
\[
(A_\delta - z)^{-1} = (A - z)^{-1} \sum_{n=0}^\infty (-\delta B (A - z)^{-1})^n,
\]

which converges in operator norm for:
\[
\delta < \frac{1}{\|B\|_{\text{op}} \|(A - z)^{-1}\|_{\text{op}}} \leq \frac{\gamma_N}{4M}.
\]

\textit{\underline{Step 3: Spectral Projector and Norm Bounds}}

The spectral projector for the eigenvalue $\lambda_N$ is:
\[
P_N(\delta) := -\frac{1}{2\pi i} \oint_{\Gamma_N} (A_\delta - z)^{-1} dz.
\]

For the unperturbed projector $P_N(0)$, there is $\|P_N(0)\|_{\text{op}} = 1$ (projection onto a one-dimensional eigenspace).

The difference is:
\[
P_N(\delta) - P_N(0) = -\frac{\delta}{2\pi i} \oint_{\Gamma_N} (A - z)^{-1} B (A - z)^{-1} dz + O(\delta^2).
\]

Taking operator norms and using the resolvent bound:
\[
\|P_N(\delta) - P_N(0)\|_{\text{op}} \leq \frac{\delta}{2\pi} \cdot \text{length}(\Gamma_N) \cdot \|(A - z)^{-1} B (A - z)^{-1}\|_{\text{op}}.
\]

Since $\text{length}(\Gamma_N) = 2\pi r_N = \pi \gamma_N / 2$ and $\|(A-z)^{-1}\|_{\text{op}} \leq 4/\gamma_N$:
\[
\|(A - z)^{-1} B (A - z)^{-1}\|_{\text{op}} \leq \|(A-z)^{-1}\|_{\text{op}}^2 \|B\|_{\text{op}} \leq \frac{16M}{\gamma_N^2}.
\]

Therefore:
\[
\|P_N(\delta) - P_N(0)\|_{\text{op}} \leq \frac{\delta \cdot \pi \gamma_N / 2}{2\pi} \cdot \frac{16M}{\gamma_N^2} = \frac{4M\delta}{\gamma_N}.
\]

\textbf{Explicit bound:}
\[
\boxed{\|P_N(\lambda + \delta\lambda) - P_N(\lambda)\|_{\text{op}} \leq C_N \delta\lambda}
\]
where $C_N = \frac{4M}{\gamma_N}$ depends on the spectral gap.

\textit{\underline{Step 4: Uniformity in $N$}}

it is necessary to verify that the constant $C_N$ is bounded uniformly in $N$. By the spectral gap estimate from Step 1:
\[
\gamma_N \geq c_0 \lambda_N^{1-\alpha}.
\]

Thus:
\[
C_N = \frac{4M}{\gamma_N} \leq \frac{4M}{c_0 \lambda_N^{1-\alpha}} = \frac{4M}{c_0} \lambda_N^{\alpha - 1}.
\]

For $\alpha < 1$ (which holds by the specific form of Weyl asymptotics), $C_N$ decreases as $N \to \infty$, so the bound is uniform.

For the fundamental eigenvalue $\lambda_0$, there is the explicit bound:
\[
C_0 = \frac{4M}{\gamma_0} \leq \frac{4M}{c_0} =: \bar{C},
\]
which is a universal constant independent of $N$.

\textit{\underline{Step 5: Eigenfunction Continuity}}

Since $\dim P_N(\delta) = \dim P_N(0) = 1$ (simple eigenvalue by Lemma~\ref{lem:genericSimplicity}), the eigenspace is one-dimensional. Write the normalized eigenfunction as:
\[
e_N^\delta = e^{i\theta(\delta)} e_N^0,
\]
where the phase $\theta(\delta)$ is chosen so that $\text{Re}\langle e_N^\delta, e_N^0 \rangle > 0$.

By analytic perturbation theory (\cite{kato1995perturbation}, Theorem II.4.10), the eigenfunction and eigenvalue have convergent power series:
\[
e_N^\delta = e_N^0 + \delta f_1 + \delta^2 f_2 + \cdots,
\]
\[
\lambda_N^\delta = \lambda_N^0 + \delta \lambda_N^{(1)} + \delta^2 \lambda_N^{(2)} + \cdots.
\]

The first-order corrections are:
\[
\lambda_N^{(1)} = \langle e_N^0 | B | e_N^0 \rangle \leq \|B\|_{\text{op}} \|e_N^0\|^2 = \|B\|_{\text{op}} \leq M,
\]
\[
\|f_1\|_{L^2} = O(M/\gamma_N).
\]

Thus:
\[
|\lambda_N^\delta - \lambda_N^0| \leq C' M \delta \quad \text{and} \quad \|e_N^\delta - e_N^0\|_{L^2} \leq C'' M \delta / \gamma_N.
\]

\textit{\underline{Step 6: Application to Truncated Effective Action}}

The truncated effective action is:
\[
\Gamma_N[\phi] := -\log \det(P_N(A) + E_N[\phi]),
\]
where $P_N(A)$ is the projector onto the truncated eigenspace and $E_N[\phi]$ encodes interactions.

By Lemma~\ref{lem:projectorContinuityGeneral}, $P_N(A)$ depends continuously on $A$ with operator norm derivative bounded by:
\[
\frac{dP_N}{dA} = O(1/\gamma_N).
\]

The functional derivative with respect to a field $\phi$ induces perturbations of $A$, so the effective action inherits continuity:
\[
\left|\frac{\delta \Gamma_N[\phi]}{\delta \phi(x)}\right| \leq C_N \int_X dy \, K(x, y) \left|\frac{\delta A[\phi]}{\delta \phi(y)}\right|,
\]
where $K(x, y)$ is the heat kernel of $A$ (bounded by $C/t^{Q/2}$ for small $t$).

\textit{\underline{Step 7: Convergence as $N \to \infty$}}

The full effective action $\Gamma[\phi]$ is obtained by taking $N \to \infty$ in the truncation:
\[
\Gamma[\phi] = \lim_{N \to \infty} \Gamma_N[\phi].
\]

Since $C_N \to 0$ as $N \to \infty$ (from Step 4), the functional derivatives of the truncated actions converge:
\[
\lim_{N \to \infty} \left|\frac{\delta \Gamma_N[\phi]}{\delta \phi(x)} - \frac{\delta \Gamma[\phi]}{\delta \phi(x)}\right| = 0.
\]

This establishes the well-definedness of the path integral regularization in the divergence-first framework and justifies the spectral truncation method used throughout the manuscript.

\qed
\end{proof}