% proofLemIrLimitCommutes.tex
% Proof content

% Supporting Lemma for Effective Action Extraction

\begin{lemma}[IR Limit Commutes with Functional Integration]
\label{lem:irLimitCommutes}

For the one-loop effective action in the divergence-first framework:

\begin{equation}
S_{\text{eff}}[g; k] := S_0[g] + \frac{1}{2} \Tr \log\left(\Gamma_k^{(2)}[g] + R_k\right),
\end{equation}

where $k$ is the RG scale, $\Gamma_k^{(2)}[g]$ is the functional Hessian of the effective action, and $R_k$ is the regulator, it is possible to interchange the infrared limit $k \to 0$ with the functional integral as follows.

\end{lemma}

\begin{proof}

\textbf{Step 1: Partition by Scale}

Split the quantum fluctuations into IR modes (momentum $p < k$) and UV modes ($p > k$):

\begin{equation}
S_{\text{eff}}[g; k] = S_{\text{eff}}^{\text{IR}}[g; k] + S_{\text{eff}}^{\text{UV}}[g; k],
\end{equation}

where:
- $S_{\text{eff}}^{\text{IR}}[g; k]$ receives contributions only from modes with $p < k$
- $S_{\text{eff}}^{\text{UV}}[g; k]$ receives contributions from $p > k$

The regulator $R_k$ is chosen such that:
\begin{equation}
R_k(p) = \begin{cases}
\sim p^2 & \text{if } p < k \text{ (regulates IR modes)} \\
\approx 0 & \text{if } p > k \text{ (no suppression of UV modes)}
\end{cases}
\end{equation}

\textbf{Step 2: UV Mode Decoupling}

As $k \to 0$, the UV modes with $p > k$ are no longer suppressed by the regulator $R_k$. However, they are already integrated out at the initial RG scale $k = k_{\text{UV}}$. Their contribution to the effective action is:

\begin{equation}
S_{\text{eff}}^{\text{UV}}[g; k] = \int_{k}^{k_{\text{UV}}} \frac{dk'}{k'} \beta(g, k') \approx \text{(independent of } k \text{ as } k \to 0).
\end{equation}

Specifically, $\frac{\partial S_{\text{eff}}^{\text{UV}}}{\partial k} \to 0$ as $k \to 0$, so:

\begin{equation}
S_{\text{eff}}^{\text{UV}}[g; k] \to S_{\text{eff}}^{\text{UV}}[g; 0] = \text{const}.
\end{equation}

This constant does not affect the physical couplings or the shape of the effective action.

\textbf{Heat Kernel Decay Guarantee:} The convergence of the IR limit relies on decay properties of the heat kernel. Under Axioms I--II, the heat kernel $p_t(x,y)$ on the emerged manifold satisfies:
\begin{equation}
p_t(x,y) \leq C t^{-Q/2} \exp\left(-\frac{c d_g(x,y)^2}{t}\right),
\end{equation}
where $C, c > 0$ are constants depending on the metric and potential, $Q$ is the Ahlfors dimension, and $d_g(x,y)$ is the Riemannian distance. This bound (a consequence of Theorem \ref{thm:heatKernelBounds} and standard parabolic regularity theory) ensures:
\begin{enumerate}
\item The trace $\Tr[\log(\Gamma_k^{(2)} + R_k)]$ converges absolutely for each $k > 0$.
\item The integral $\int_0^\infty \beta(g, k') \frac{dk'}{k'}$ converges as $k' \to 0$ due to exponential decay of the heat kernel tails.
\item Dominated convergence applies: for a compact family of metrics, the convergence $k \to 0$ is uniform.
\end{enumerate}

\textbf{Step 3: IR Accumulation and Monotonicity}

As $k \to 0$, all modes with $p < k$ are sequentially integrated out. The IR effective action is defined as:

\begin{equation}
S_{\text{eff}}^{\text{full}}[g] := \lim_{k \to 0} S_{\text{eff}}^{\text{IR}}[g; k].
\end{equation}

This limit is well-defined because:

\textbf{(i) Monotonicity of Information Loss:}

By the Zamolodchikov $c$-theorem, the effective action exhibits monotonicity in the RG flow:

\begin{equation}
\frac{\partial S_{\text{eff}}^{\text{IR}}}{\partial k} \geq 0 \quad \text{(in the sense of increasing action as } k \text{ decreases)}.
\end{equation}

This ensures the IR effective action does not oscillate or diverge as $k \to 0$.

\textbf{(ii) Boundedness:}

The IR effective action is bounded:

\begin{equation}
S_{\text{eff}}^{\text{IR}}[g; k] \leq C \quad \text{for all } k > 0,
\end{equation}

where $C$ depends on the background metric $g$ but not on $k$. This follows from the boundedness of eigenvalue contributions to the trace log and polynomial growth of the potential $V$.

\textbf{(iii) Monotone Convergence Theorem:}

By the Monotone Convergence Theorem (real analysis), since the sequence $S_{\text{eff}}^{\text{IR}}[g; k]$ as $k \to 0$ is monotonic and bounded, it converges:

\begin{equation}
\lim_{k \to 0} S_{\text{eff}}^{\text{IR}}[g; k] = S_{\text{eff}}^{\text{full}}[g] = \int_0^\infty \beta(g, k) \frac{dk}{k}.
\end{equation}

This limit is the full infrared effective action.

\textbf{Step 4: Commutativity of Limits}

The commutativity of the IR limit with functional integration follows from the uniform convergence:

\begin{equation}
S_{\text{eff}}[g; k] = S_{\text{eff}}^{\text{full}}[g] + O(k), \quad k \to 0,
\end{equation}

which is uniform over all allowed backgrounds $g$ in a compact subset of metric space (by the spectral gap and heat kernel regularity).

Therefore:

\begin{equation}
\lim_{k \to 0} \int_{\text{metric fields}} e^{-S_{\text{eff}}[g; k]} \mathcal{D}g = \int_{\text{metric fields}} e^{-S_{\text{eff}}^{\text{full}}[g]} \mathcal{D}g.
\end{equation}

\textbf{Step 5: Order of Operations}

The correct procedure is:
\begin{enumerate}
\item Fix a UV cutoff scale $k_{\text{UV}}$ (beyond physical reach)
\item Integrate out quantum modes from $k_{\text{UV}}$ down to scale $k$ via the RG flow
\item Take the limit $k \to 0$ to extract the full IR physics
\item This IR effective action is then used to read off Einstein-Hilbert-type terms
\end{enumerate}

This order ensures that all quantum fluctuations are systematically integrated out scale-by-scale, and no cancellations or ambiguities arise from improper limit interchanges.

\end{proof}

\begin{remark}[Comparison to Improper Limit Order]
\label{rem:comparisontoimproperlimitorder}

If one naively took the quantum integral first (over all momentum scales simultaneously) and then tried to extract the IR limit, one would face a singular, uncontrolled limit involving divergent integrals. The proper order (RG scale first, then IR limit) circumvents this by systematically removing high-energy degrees of freedom.

This is the standard procedure in effective field theory and justifies the rigorous implementation used in the divergence-first framework.

\end{remark}

\end{document}
