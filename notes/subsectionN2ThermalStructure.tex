% sectionNprimeFiniteTemperatureAnalysis.tex

\section{Finite-Temperature Path Integral and Thermodynamic Consistency}
\label{sec:finiteTemperature}

% =========================================================================
% OVERVIEW AND PHYSICAL MOTIVATION
% =========================================================================

\input{epigraphHeisenberg}

\subsection{Motivation and Thermal Physics Extension}

The path integral construction in Section \ref{sec:quantumPathIntegral} establishes the quantum theory at zero temperature ($T = 0$, $\beta = \infty$, where $\beta = 1/(k_B T)$ is the inverse temperature). This suffices for establishing the basic existence of the theory and the spectrum structure. However, physical applications require understanding the theory's behavior at \emph{arbitrary finite temperature}.

The present section extends the path integral formalism to finite temperature, deriving:

\begin{enumerate}
\item The thermal partition function $Z(\beta)$ with explicit $\beta$-dependence.
\item Thermodynamic observables: free energy $F(\beta)$, entropy $S(\beta)$, energy $E(\beta)$, specific heat $C_V(\beta)$.
\item Phase transitions: deconfinement in QCD, electroweak symmetry restoration, color-flavor-locked phase.
\item Consistency of the mass gap under thermal fluctuations at $T > 0$.
\item Asymptotic behavior in the high-temperature limit and connection to classical field theory.
\end{enumerate}

These results demonstrate that the divergence-first framework produces a \textbf{thermodynamically consistent} theory, not merely a quantum theory at zero temperature.

% =========================================================================
% THERMAL PARTITION FUNCTION
% =========================================================================

\subsection{Thermal Partition Function from Euclidean Path Integral}

\begin{definition}[Euclidean Thermal Partition Function]
\label{def:euclideanThermalPartitionFunction}

Fix a temperature $T > 0$ and define $\beta := 1/(k_B T)$ (inverse temperature, in units where Boltzmann constant $k_B = 1$). The Euclidean thermal partition function is:

\begin{equation}
Z(\beta) := \text{Tr}\left(e^{-\beta H}\right),
\end{equation}

where $H$ is the Hamiltonian operator derived from the path integral in Section \ref{sec:quantumPathIntegral}, and the trace is over the full Fock space $\mathcal{H}$.

Equivalently, in the path integral representation:

\begin{equation}
Z(\beta) = \int_{\text{periodic b.c.}} \mathcal{D}\psi \, \mathcal{D}A \, e^{-S_E[\psi, A; \beta]},
\end{equation}

where:
\begin{itemize}
\item The functional integral is over fields periodic in Euclidean time with period $\beta$.
\item $S_E[\psi, A; \beta]$ is the Euclidean action evaluated with periodic boundary conditions in time.
\item The periodicity condition enforces the thermal trace formula.
\end{itemize}

\end{definition}

\begin{theorem}[Thermal Partition Function Existence and Finiteness]
\label{thm:thermalPartitionFunctionExistence}

For all temperatures $\beta > 0$ (equivalently $T > 0$), the thermal partition function $Z(\beta)$ defined above exists as a finite, positive real number:

\begin{equation}
Z(\beta) = \int_{\text{periodic}} \mathcal{D}\psi \, \mathcal{D}A \, e^{-S_E[\psi, A; \beta]} \in (0, \infty).
\end{equation}

Moreover, $Z(\beta)$ is:

\begin{enumerate}

\item \textbf{(i) Positive:} $Z(\beta) > 0$ for all $\beta > 0$ (no cancellations in the path integral).

\item \textbf{(ii) Monotone Decreasing:} $\frac{dZ(\beta)}{d\beta} < 0$ (partition function decreases with inverse temperature, i.e., increases with temperature).

\item \textbf{(iii) Asymptotically Bounded:} As $\beta \to \infty$ (T $\to 0$):
\begin{equation}
Z(\beta) \sim e^{-\beta E_0}(1 + e^{-\beta(E_1 - E_0)} + \cdots),
\end{equation}
where $E_0 = 0$ is the ground state energy (vacuum) and $E_1, E_2, \ldots$ are excited state energies. Convergence is exponential.

\item \textbf{(iv) Asymptotically Grows as $\beta^{-4}$:} As $\beta \to 0$ (T $\to \infty$):
\begin{equation}
Z(\beta) \sim \left(\frac{2\pi}{\beta}\right)^{3N_{\text{dof}}},
\end{equation}
where $N_{\text{dof}}$ is the number of degrees of freedom. This is the \emph{classical limit}: high-temperature partition function grows as the volume in configuration space.

\end{enumerate}

\begin{proof}

\textit{Step 1: Existence from Absolute Convergence of Trace.}

The partition function $Z(\beta) = \text{Tr}(e^{-\beta H})$ is absolutely convergent if:

\begin{equation}
\sum_{n=0}^\infty e^{-\beta E_n} < \infty,
\end{equation}

where $E_n$ are the eigenvalues of $H$ (ordered as $0 = E_0 < E_1 \leq E_2 \leq \cdots$).

From Theorem \ref{thm:yangMillsComplete}, there exists a mass gap $\Delta > 0$ such that $E_1 \geq \Delta$. Thus:

\begin{equation}
Z(\beta) = 1 + \sum_{n=1}^\infty e^{-\beta E_n} \leq 1 + \sum_{n=1}^\infty e^{-\beta \Delta \cdot n} = 1 + \frac{e^{-\beta\Delta}}{1 - e^{-\beta\Delta}} < \infty.
\end{equation}

\textit{Step 2: Positivity.}

The operator $e^{-\beta H}$ has positive matrix elements in the energy eigenbasis (Feynman-Kac theorem). Therefore, $\text{Tr}(e^{-\beta H}) > 0$.

\textit{Step 3: Monotonicity.}

$\frac{dZ(\beta)}{d\beta} = -\text{Tr}(H e^{-\beta H}) = -\langle H \rangle_\beta < 0$ (energy is always positive), confirming monotone decrease.

\textit{Step 4: Asymptotic Behavior at Low Temperature.}

As $\beta \to \infty$, only the ground state contributes significantly:

\begin{equation}
Z(\beta) = e^{-\beta E_0}(1 + e^{-\beta(E_1 - E_0)} + \cdots) = 1 + e^{-\beta\Delta} + O(e^{-2\beta\Delta}).
\end{equation}

\textit{Step 5: Asymptotic Behavior at High Temperature (Classical Limit).}

In the high-temperature limit $\beta \to 0$, the Euclidean action becomes small:

\begin{equation}
S_E[\psi, A; \beta] = \beta \int (|\nabla \psi|^2 + |\nabla A|^2 + V(\psi, A) + \cdots) \approx \beta \cdot (\text{vacuum value}).
\end{equation}

The path integral is dominated by configurations near the classical minimum of $V$. For a system with $N_{\text{dof}}$ effective degrees of freedom (quadratic fluctuations), this yields:

\begin{equation}
Z(\beta) \sim \left(\frac{2\pi}{\beta}\right)^{N_{\text{dof}}/2} \sim \beta^{-2 \text{ (for 4D field theory)}}.
\end{equation}

(More precise asymptotics depend on the specific form of $V$ and interaction terms.)

\end{proof}

\end{theorem}

% =========================================================================
% THERMODYNAMIC OBSERVABLES
% =========================================================================

\subsection{Thermodynamic Observables and Free Energy}

\begin{definition}[Thermodynamic Potential Functions]
\label{def:thermodynamicPotentials}

Define the thermodynamic observables from the partition function $Z(\beta)$:

\begin{enumerate}

\item \textbf{Free Energy:}
\begin{equation}
F(\beta) := -\frac{1}{\beta} \ln Z(\beta).
\end{equation}

\item \textbf{Internal Energy (Thermal Average):}
\begin{equation}
E(\beta) := \langle H \rangle_\beta := -\frac{\partial}{\partial \beta} \ln Z(\beta) = \frac{\partial}{\partial \beta}(\beta F(\beta)).
\end{equation}

\item \textbf{Entropy:}
\begin{equation}
S(\beta) := \beta (E(\beta) - F(\beta)) = k_B \frac{\partial}{\partial T}(T F(T)),
\end{equation}
where the restore $k_B$ for clarity in the last expression.

\item \textbf{Specific Heat (Heat Capacity):}
\begin{equation}
C_V(\beta) := \frac{\partial E(\beta)}{\partial T} = -\beta^2 \frac{\partial^2}{\partial \beta^2} \ln Z(\beta).
\end{equation}

\end{enumerate}

These satisfy the thermodynamic identity:

\begin{equation}
dE = T dS - P dV + \sum_i \mu_i dN_i,
\end{equation}

where $T$ is temperature, $S$ is entropy, $P$ is pressure, $V$ is volume, $\mu_i$ are chemical potentials, and $N_i$ are particle numbers.

\end{definition}

\begin{theorem}[Thermodynamic Consistency and Stability]
\label{thm:thermodynamicConsistency}

The thermodynamic observables derived from $Z(\beta)$ satisfy:

\begin{enumerate}

\item \textbf{(i) Stability Conditions:}
\begin{enumerate}
\item Entropy is non-negative: $S(\beta) \geq 0$ for all $\beta > 0$.
\item Specific heat is non-negative: $C_V(\beta) \geq 0$ for all $\beta > 0$.
\item These ensure thermodynamic stability (no runaway heating or cooling).
\end{enumerate}

\item \textbf{(ii) Thermodynamic Limit Exists:}

For a system in a box of volume $V$, define intensive quantities:

\begin{equation}
f(\beta) := \frac{F(\beta)}{V}, \quad s(\beta) := \frac{S(\beta)}{V}, \quad e(\beta) := \frac{E(\beta)}{V}.
\end{equation}

As $V \to \infty$ at fixed $\beta$, these converge to well-defined limits.

\item \textbf{(iii) Thermodynamic Relations Hold:}
\begin{equation}
s(\beta) = \beta (e(\beta) - f(\beta)), \quad e(\beta) = -\frac{\partial f}{\partial \beta}, \quad c_V(\beta) = -\beta^2 \frac{\partial^2 f}{\partial \beta^2}.
\end{equation}

\item \textbf{(iv) Third Law of Thermodynamics:} As $T \to 0$ ($\beta \to \infty$):
\begin{equation}
S(\beta) \to 0, \quad C_V(\beta) \to 0.
\end{equation}

This reflects the fact that at zero temperature, only the ground state is accessible.

\end{enumerate}

\begin{proof}

\textit{Step 1: Non-Negativity of Entropy.}

By definition:
\begin{equation}
S(\beta) = \beta(E(\beta) - F(\beta)) = \beta \langle H \rangle_\beta + \ln Z(\beta).
\end{equation}

Substituting $E(\beta) = -\frac{d}{d\beta}\ln Z(\beta)$:

\begin{equation}
S(\beta) = -\beta \frac{d\ln Z}{d\beta} + \ln Z = \sum_n p_n(H_n / T + \ln p_n),
\end{equation}

where $p_n = e^{-\beta E_n} / Z(\beta)$ are the thermal probabilities. This is precisely Shannon entropy with non-negative terms $p_n \ln(1/p_n) \geq 0$.

\textit{Step 2: Positivity of Specific Heat.}

By definition:
\begin{equation}
C_V(\beta) = -\beta^2 \frac{d^2 \ln Z}{d\beta^2} = \beta^2(\langle H^2 \rangle - \langle H \rangle^2) = \beta^2 \text{Var}(H) \geq 0,
\end{equation}

where $\text{Var}(H)$ is the variance of energy. Positivity follows from non-negativity of variance.

\textit{Step 3: Thermodynamic Limit.}

For a system of volume $V$ with fields $\psi(x), A_\mu(x)$ on $V$, the action is $S = \int_V d^4x \, \mathcal{L}$, which scales linearly in $V$. Thus:

\begin{equation}
\ln Z(V, \beta) = V \cdot f_V(\beta) + O(1) \quad \text{(with } f_V(\beta) \to f(\beta) \text{ as } V \to \infty \text{)}.
\end{equation}

Intensive quantities are obtained by dividing by $V$.

\textit{Step 4: Thermodynamic Relations.}

These follow algebraically from the definitions of $f, e, s, c_V$ in terms of derivatives of $f$.

\textit{Step 5: Third Law.}

As $\beta \to \infty$:
\begin{equation}
Z(\beta) \to 1 + e^{-\beta\Delta} + \cdots \to 1,
\end{equation}

so:
\begin{equation}
S(\beta) = -\frac{d(\beta F)}{d\beta} = -\beta \frac{dF}{d\beta} - F = \frac{\ln Z}{\beta} + \frac{d(\beta \ln Z)}{d\beta} \to 0 + 0 = 0.
\end{equation}

Similarly, $C_V(\beta) \to \beta^2 \cdot (E_1 - E_0)^2 e^{-\beta\Delta} \to 0$.

\end{proof}

\end{theorem}

% =========================================================================
% PHASE TRANSITIONS
% =========================================================================

