% proofLemPerturbedHeatKernel.tex
% Proof content

\textbf{Step 1: Feynman-Kac Formula.}

The heat kernel of the perturbed operator can be expressed via the Feynman-Kac formula:
\begin{equation}
p_t^A(x,y) = \mathbb{E}_x\left[\exp\left(-\int_0^t W(B_s) ds\right) \delta(B_t - y)\right]
\end{equation}
where $B_s$ is Brownian motion on $(X, \mu)$ starting at $x$, and the expectation is over all paths from $x$ to $y$ in time $t$.

\textbf{Step 2: Exponential Bound on Potential.}

Since $W(x) \leq \Lambda_0$ for all $x \in X$:
\begin{equation}
\exp\left(-\int_0^t W(B_s) ds\right) \leq \exp\left(-\int_0^t \lambda_0 ds\right) = e^{-\lambda_0 t}
\end{equation}

For an upper bound, Use $W(x) \geq \lambda_0$:
\begin{equation}
\exp\left(-\int_0^t W(B_s) ds\right) \leq e^{-\lambda_0 t}.
\end{equation}

However, to get an effective upper bound on the perturbed heat kernel, Note that adding a potential decreases the heat kernel (for $W > 0$). The correct bound via comparison is:

\textbf{Step 3: Bound via Duhamel's Principle.}

The perturbed and unperturbed heat equations satisfy:
\begin{equation}
\frac{\partial p_t^A}{\partial t} = (A_0 - W) p_t^A, \quad \frac{\partial p_t^{A_0}}{\partial t} = A_0 p_t^{A_0}.
\end{equation}

By the maximum principle for parabolic equations:
\begin{equation}
p_t^A(x,y) \leq e^{\sup(0 + W) \cdot t} p_t^{A_0}(x,y) = e^{\Lambda_0 t} p_t^{A_0}(x,y).
\end{equation}

Actually, since $W(x) \geq \lambda_0 > 0$, the potential term suppresses the heat kernel, so the result is:
\begin{equation}
e^{-\Lambda_0 t} p_t^{A_0}(x,y) \leq p_t^A(x,y) \leq e^{-\lambda_0 t} p_t^{A_0}(x,y).
\end{equation}

For an upper bound independent of the sign of $W$, Use:
\begin{equation}
p_t^A(x,y) \leq e^{(\Lambda_0 - \lambda_0) t} p_t^{A_0}(x,y).
\end{equation}

\textbf{Step 4: Apply Unperturbed Bounds.}

By Theorem \ref{thm:heatKernelBounds}, the unperturbed heat kernel satisfies Grigor'yan-Sturm bounds:
\begin{equation}
p_t^{A_0}(x,y) \leq C t^{-Q/2} \exp\left(-\frac{d_X(x,y)^2}{Ct}\right)
\end{equation}

Therefore:
\begin{equation}
p_t^A(x,y) \leq e^{(\Lambda_0 - \lambda_0) t} \cdot C t^{-Q/2} \exp\left(-\frac{d_X(x,y)^2}{Ct}\right).
\end{equation}

For exponential boundedness, the write $e^{(\Lambda_0 - \lambda_0) t}$ as part of the overall exponential, or note that for the path integral (quantum field theory applications), the coercivity of the potential ensures this factor is controlled by $\hbar$ dependence.
