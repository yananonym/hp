\subsection{Black Hole Solutions in Emergent Geometry}
\label{subsec:blackHoleSolutions}

\begin{theorem}[Schwarzschild Solution from Einstein-Hilbert Action]
\label{thm:schwarzschildEmergent}

in the divergence-first framework, the Einstein-Hilbert effective action (Theorem \ref{thm:einsteinHilbertEmergence}) yields the Schwarzschild metric as a spherically symmetric solution:

\begin{equation}
ds^2 = -\left(1 - \frac{2M}{r}\right) dt^2 + \left(1 - \frac{2M}{r}\right)^{-1} dr^2 + r^2(d\theta^2 + \sin^2\theta \, d\phi^2),
\end{equation}

where $M$ is the mass parameter related to the source energy density.

\begin{proof}

\textbf{Part 1: Spherical Symmetry Ansatz}

Assume the metric is spherically symmetric: $g_{\mu\nu} = g_{\mu\nu}(r, t)$ invariant under rotations. In Schwarzschild-like coordinates:

\begin{equation}
ds^2 = -e^{2\Phi(r)} dt^2 + e^{2\Lambda(r)} dr^2 + r^2 d\Omega^2,
\end{equation}

where $\Phi(r), \Lambda(r)$ are functions to be determined.

The Einstein equations from the Lagrangian:

\begin{equation}
S = \int dx^4 \sqrt{g} \left[\frac{1}{16\pi G_N}R - \Lambda_0 + \mathcal{L}_m\right]
\end{equation}

in the vacuum ($\mathcal{L}_m = 0, \Lambda_0 = 0$) reduce to:

\begin{align}
R_{tt} &= 0, \\
R_{rr} &= 0, \\
R_{\theta\theta} &= 0.
\end{align}

\textbf{Part 2: Solving for the Metric Functions}

Standard calculation: From $R_{tt} = 0$: $\Phi'(r) + \Lambda'(r) = 0 \Rightarrow \Phi = -\Lambda$.

From $R_{rr} = 0$: $\Phi''(r) + (\Phi'(r))^2 = 0$.

Solution: $\Phi(r) = -\log(1 - 2M/r)$, yielding:

\begin{equation}
e^{2\Phi} = 1 - 2M/r, \quad e^{2\Lambda} = (1 - 2M/r)^{-1}.
\end{equation}

\textbf{Part 3: Integration into divergence-first framework}

The derived metric is identical to the Schwarzschild solution of GR. The mass parameter $M$ emerges from the boundary conditions at spatial infinity, where the ADM mass formula applies:

\begin{equation}
M = \frac{1}{2} \int_{\partial \Sigma} (g_{ij,i} - g_{ii,j}) dS^j,
\end{equation}

evaluated on spatial slices $\Sigma$.

in the divergence-first framework, this mass is determined by the total energy of the matter field configuration whose divergence structure gave rise to the geometry in the first place. Thus $M$ is not arbitrary but constrained by the energy of the source configuration.

\end{proof}

\end{theorem}

\begin{theorem}[Black Hole Thermodynamics]
\label{thm:blackHoleThermodynamics}

Black holes in the emergent geometry satisfy the laws of thermodynamics with entropy identified as geometric entropy of the event horizon.

\begin{enumerate}
\item[(i)] \textbf{Event Horizon:} A Schwarzschild black hole has event horizon at $r_s = 2M$ where the metric component $g_{tt} = 0$.

\item[(ii)] \textbf{Entropy (Bekenstein-Hawking):} The entropy is:

\begin{equation}
S_{\text{BH}} = \frac{A}{4\ell_P^2} = \frac{4\pi r_s^2}{4\ell_P^2} = \frac{\pi (2M)^2}{\ell_P^2} = \frac{4\pi M^2}{\ell_P^2},
\end{equation}

where $\ell_P = \sqrt{\hbar G_N}$ is the Planck length and $A = 4\pi r_s^2$ is the horizon area.

\item[(iii)] \textbf{Hawking Temperature:} The surface gravity at the horizon:

\begin{equation}
\kappa = \frac{1}{4M},
\end{equation}

determines the Hawking temperature via the Unruh effect:

\begin{equation}
T_H = \frac{\hbar \kappa}{2\pi} = \frac{\hbar}{8\pi M}.
\end{equation}

\item[(iv)] \textbf{Evaporation Rate:} The power radiated via Hawking radiation:

\begin{equation}
\frac{dE}{dt} = -\frac{\hbar c^6}{15360 \pi G_N^2 M^2} = -\frac{1}{15360\pi} \ell_P^2 \frac{dM}{dt}.
\end{equation}

Integrating: $M(t) = M_0 \left(1 - t/t_0\right)^{1/3}$ with evaporation time:

\begin{equation}
t_0 = \frac{5120 \pi G_N^2 M_0^3}{\hbar c^6} \approx 2.1 \times 10^{67} \left(\frac{M_0}{M_\odot}\right)^3 \text{ years}.
\end{equation}

\end{enumerate}

\begin{proof}

\textbf{Part 1: Entropy from Euclidean Path Integral}

In the Euclidean continuation, the Schwarzschild metric becomes the Euclidean Schwarzschild metric, which is regular at the horizon if the Euclidean time is periodic with period $\beta = 1/T_H$.

The partition function:

\begin{equation}
Z = e^{-\beta E} = \int \mathcal{D}g \, e^{-S_E[g]}
\end{equation}

where $S_E$ is the Euclidean action (Einstein-Hilbert + boundary terms).

The Gibbons-Hawking-York boundary term at the horizon:

\begin{equation}
S_{\text{boundary}} = \frac{1}{8\pi G_N} \int_{\partial M} K \, dA,
\end{equation}

where $K$ is the extrinsic curvature. For the Euclidean Schwarzschild metric, standard calculation yields:

\begin{equation}
S_E = -\frac{M}{2T_H},
\end{equation}

where $T_H = \hbar/(8\pi M)$.

The entropy from thermodynamic relation $S = \beta E - \log Z$:

\begin{equation}
S = \beta M - (-M/(2T_H)) = \frac{8\pi M^2}{\hbar}.
\end{equation}

Converting to geometric units where $\ell_P^2 = \hbar G_N$ and $\hbar = 1$:

\begin{equation}
S = 4\pi M^2 / \ell_P^2 = A / (4\ell_P^2).
\end{equation}

This is the standard Bekenstein-Hawking entropy.

\textbf{Part 2: Hawking Temperature from Euclidean Periodicity}

In the Euclidean Schwarzschild geometry, regularity at the horizon requires the Euclidean time coordinate to be periodic:

\begin{equation}
\tau \sim \tau + 4\pi M \Rightarrow T_H = \frac{1}{4\pi M}.
\end{equation}

In natural units ($\hbar = c = 1$), this matches the standard result.

\textbf{Part 3: Evaporation Rate (Hawking Radiation)}

The power radiated by a black hole at temperature $T_H$:

\begin{equation}
P = \sigma T_H^4 A = \sigma T_H^4 \cdot 4\pi r_s^2,
\end{equation}

where $\sigma$ is the Stefan-Boltzmann constant for the black body spectrum of Hawking radiation. Standard quantum field theory in curved spacetime yields:

\begin{equation}
P = \frac{\hbar c^6}{15360 \pi G_N^2 M^2}.
\end{equation}

Since $dE/dt = -P$ and $E = Mc^2$:

\begin{equation}
\frac{dM}{dt} = -\frac{\hbar c^4}{15360 \pi G_N^2 M^2}.
\end{equation}

Integrating yields the evaporation time $t_0$ as stated above.

\end{proof}

\end{theorem}

\begin{theorem}[Kerr Metric in Emergent Geometry]
\label{thm:kerrMetric}

The Einstein equations also admit the Kerr metric for rotating black holes:

\begin{equation}
ds^2 = -\frac{\Delta}{\rho^2}(dt - a\sin^2\theta \, d\phi)^2 + \frac{\rho^2}{\Delta} dr^2 + \rho^2 d\theta^2 + \frac{\sin^2\theta}{\rho^2}(a \, dt - (r^2 + a^2)d\phi)^2,
\end{equation}

where $\Delta = r^2 - 2Mr + a^2$, $\rho^2 = r^2 + a^2\cos^2\theta$, and $a$ is the rotation parameter. The thermodynamic properties (entropy, temperature, angular momentum) follow from generalization of the Schwarzschild analysis.

\begin{proof}

Standard derivation from Einstein equations in Boyer-Lindquist coordinates. See Weinberg (1972), Chapter 13 for full details.

\end{proof}

