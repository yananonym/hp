% Part of sectionQTheStarStrongInteractionsEmergence.tex
\subsection{Gauss Law Constraint and Gauge Fixing}
\label{subsec:gaussLawConstraintAndGaugeFixing}

\begin{theorem}[Gauss Law from Gauge Constraints]
\label{thm:gaussLawConstraint}
The Yang-Mills equations of motion for $SU(3)_c$ gauge theory possess constraints arising from gauge symmetry:

\begin{enumerate}
\item \textbf{Equations of Motion.} Variation of the Yang-Mills action gives:
\begin{equation}
\partial_\nu G^{\mu\nu,A} = j^\mu_A,
\end{equation}
where $j^\mu_A$ is the color current from matter (quarks and gluons).

\item \textbf{Gauss Law Constraint.} The temporal component ($\mu = 0$) yields:
\begin{equation}
\partial_i G^{0i,A} = \nabla_i E^i_A = j^0_A \equiv \rho^A,
\end{equation}
where $E^i_A := G^{0i,A}$ is the chromoelectric field and $\rho^A$ is the color charge density. In vacuum ($j^0_A = 0$), this reduces to:
\begin{equation}
\nabla \cdot E^A = 0 \quad \text{(color charge conservation)}.
\end{equation}

\item \textbf{Secondary Constraint.} The Gauss law itself is a constraint (not a dynamical equation) that relates the electric field to sources and must be satisfied at all times. It restricts the allowed initial conditions to lie on the physical constraint surface:
\begin{equation}
\Sigma_{\text{phys}} := \{\text{field configurations satisfying } \nabla \cdot E^A = \rho^A\}.
\end{equation}

\item \textbf{Gauge Redundancy.} The vector potential $G_\mu^A$ has $4 \times 8 = 32$ components, but only $2 \times 8 = 16$ physical degrees of freedom (two polarization states per gluon type). The four equations of motion (temporal and three spatial components) provide four constraints per color, but one is the Gauss law (not dynamical), leaving three constraints plus four solutions from integrating the spatial equations. The remaining redundancy corresponds to gauge freedom.

\item \textbf{Gauge Fixing.} To eliminate gauge redundancy in the path integral, impose a gauge condition such as:

\begin{enumerate}[label=(\roman*)]
\item \textbf{Lorenz Gauge:} $\partial_\mu G^\mu_A = 0$ (covariant gauge, simplifies calculations).
\item \textbf{Coulomb Gauge:} $\nabla \cdot G^A = 0$ (physical gauge, separates instantaneous and radiative parts).
\item \textbf{Axial Gauge:} $G^A_3 = 0$ (breaks manifest Lorentz covariance but useful for calculations).
\end{enumerate}

Each choice trades manifest covariance for explicit physical interpretation or calculational convenience.
\end{enumerate}

\begin{proof}
% proofThmGaussLawConstraint.tex
% Proof content

The Gauss law arises as follows. Variation of the action:
\begin{equation}
\delta S_{\text{YM}} = -\int G^{\mu\nu,A} \partial_\nu \delta G_{\mu}^A \sqrt{g} d^4x = \int \partial_\nu G^{\mu\nu,A} \delta G_{\mu}^A \sqrt{g} d^4x
\end{equation}
gives the Euler-Lagrange equations. The temporal equation ($\mu = 0$):
\begin{equation}
\partial_\nu G^{0\nu,A} = j^0_A
\end{equation}
is the Gauss law. Unlike spatial equations ($\mu = 1,2,3$), which are second-order and determine time evolution, the temporal equation is first-order (containing only $\partial_0 G^{00,A} = 0$, which is automatic by antisymmetry) and instead becomes a constraint.

In the Hamiltonian formulation, the Gauss law is the generator of gauge transformations: infinitesimal gauge transformations $\delta G_\mu^A = D_\mu \epsilon^A = (\partial_\mu + g_s f^{ABC} G_\mu^B)\epsilon^C$ leave the physical constraint surface (Gauss law surface) invariant.

For gauge fixing in the path integral, one imposes a condition like $\partial_\mu G^\mu_A = 0$ and introduces Faddeev-Popov ghosts to account for the Jacobian of the gauge-fixing map (see Theorem \ref{thm:faddeevPopov} below).

\end{proof}
\end{theorem}

\begin{theorem}[Faddeev-Popov Procedure for Path Integral]
\label{thm:faddeevPopov}
To compute the path integral with gauge-fixed action, use the Faddeev-Popov procedure:

\begin{enumerate}
\item \textbf{Gauge-Fixed Path Integral.} With Lorenz gauge $\partial_\mu G^\mu_A = 0$:
\begin{equation}
Z = \int \mathcal{D}[G] \delta(\partial_\mu G^\mu_A) \, \det\left(\frac{\delta(\partial_\mu G^\mu_A)}{\delta \epsilon_A}\right) \exp\left(-S_{YM}[G] - S_{matter}[q, G]\right)
\end{equation}

\item \textbf{Faddeev-Popov Determinant.} The functional derivative:
\begin{equation}
\frac{\delta(\partial_\mu G^\mu_A)}{\delta \epsilon_B} = -\partial_\mu D_\mu^{AB} = -(\square + \text{gluon self-interaction})^{AB}
\end{equation}
is encoded via ghost fields $c^A, \bar{c}^A$ (anticommuting Grassmann variables):
\begin{equation}
\det(\text{Faddeev-Popov}) \equiv \int \mathcal{D}[\bar{c}]\mathcal{D}[c] \, \exp\left(-S_{FP}[c, \bar{c}, G]\right),
\end{equation}
where the Faddeev-Popov ghost action is:
\begin{equation}
S_{FP}[c, \bar{c}, G] = \int \bar{c}^A \partial_\mu D_\mu^{AB}[G] c^B \sqrt{g} \, d^4x.
\end{equation}

\item \textbf{Gauge-Fixed Lagrangian.} The complete effective action becomes:
\begin{equation}
S_{eff}[G, q, c, \bar{c}] = S_{YM}[G] + S_{matter}[q, G] + S_{GF}[G] + S_{FP}[c, \bar{c}, G],
\end{equation}
where the gauge-fixing term is:
\begin{equation}
S_{GF}[G] = \frac{1}{2\xi} \int (\partial_\mu G^\mu_A)^2 \sqrt{g} \, d^4x.
\end{equation}
The gauge-fixing parameter $\xi$ controls the gauge (Lorenz gauge for $\xi = 1$, Landau gauge for $\xi = 0$).

\item \textbf{Path Integral.} The partition function becomes:
\begin{equation}
Z = \int \mathcal{D}[G] \mathcal{D}[q] \mathcal{D}[\bar{c}]\mathcal{D}[c] \, \exp(-S_{eff}/\hbar),
\end{equation}
which is manifestly gauge-invariant under residual (unbroken) global color symmetries and independent of the gauge-fixing choice (to leading order).

\item \textbf{Physical Observables.} Gauge-invariant operators such as:
\begin{equation}
\text{Tr}(P e^{i \int_C G}) = \text{Wilson loops}
\end{equation}
(holonomy along contour $C$) remain invariant. The constraint surface (Gauss law) is automatically preserved by the Faddeev-Popov procedure, ensuring physical consistency.
\end{enumerate}

\begin{proof}
% proofThmFaddeevPopov.tex
% Proof content

\textbf{Gauge Redundancy and the Need for Gauge Fixing}

The Yang-Mills path integral:
\begin{equation}
Z = \int \mathcal{D}A_\mu^A \, e^{iS_{\text{YM}}[A]/\hbar}
\end{equation}
contains an over-counting problem: two gauge potentials $A_\mu^A$ and $A_\mu'^A$ related by a gauge transformation,
\begin{equation}
A'_\mu(x) = \Omega(x) A_\mu(x) \Omega(x)^{-1} - i (\partial_\mu \Omega) \Omega^{-1},
\end{equation}
where $\Omega(x) \in G$ (the gauge group, e.g., $SU(3)$ for strong interactions), yield the same physical action $S_{\text{YM}}[A] = S_{\text{YM}}[A']$.

Thus the functional integral over-counts: each physical configuration is represented infinitely many times (once for each gauge transformation in the gauge group).

To correct this, the divide by the ``volume'' of the gauge group:
\begin{equation}
Z = \frac{1}{\text{Vol}(G)} \int \mathcal{D}A_\mu^A \, e^{iS_{\text{YM}}[A]/\hbar} \times (\text{Jacobian}).
\end{equation}

\textbf{Faddeev-Popov Determinant and Ghost Fields}

\textbf{Step 1: Gauge Fixing Condition}

The impose a gauge fixing condition $\mathcal{G}[A] = 0$ that picks out one representative from each gauge orbit. A common choice in covariant quantization is the \emph{Lorenz gauge}:
\begin{equation}
\mathcal{G}[A] := \partial^\mu A_\mu^A(x) = 0.
\end{equation}

The insert this constraint into the path integral via a functional delta function:
\begin{equation}
1 = \int \mathcal{D}\lambda^A(x) \, \prod_x \delta(\mathcal{G}[A](x)) \, e^{-i\int dx \lambda^A(x) \mathcal{G}[A](x)/\hbar},
\end{equation}
where $\lambda^A(x)$ are Lagrange multiplier fields (gauge fixing parameters).

\textbf{Step 2: Faddeev-Popov Determinant}

When the vary the gauge potential $A \to A + \delta A$, the gauge fixing condition varies as:
\begin{equation}
\delta \mathcal{G}[A] = \frac{\delta \mathcal{G}}{\delta A_\mu^A}(x) \delta A_\mu^A(x) =: M^{AB}(x, y) \delta A_B(y),
\end{equation}
where $M^{AB}$ is the Faddeev-Popov operator. For Lorenz gauge:
\begin{equation}
M^{AB}[A](x,y) = \delta^{AB} \square \delta(x-y) + f^{ABC} (\partial^\mu A_\mu^C(x)) \delta(x-y),
\end{equation}
where $f^{ABC}$ are the structure constants of the gauge group.

The Jacobian of the change of variables from the original $A$ to a decomposition involving the gauge orbit direction is:
\begin{equation}
\text{Jacobian} = \left|\det M[A]\right| = \frac{1}{\text{Vol}(G)}.
\end{equation}

Thus:
\begin{equation}
\prod_x \delta(\mathcal{G}[A](x)) = \frac{\det M[A]^{-1}}{\text{Vol}(G)},
\end{equation}

and the path integral becomes:
\begin{equation}
Z = \int \mathcal{D}A_\mu^A \int \mathcal{D}\lambda^A \, \det M[A] \exp\left(iS_{\text{YM}}[A]/\hbar - \frac{i}{\hbar}\int d^4x \lambda^A(x) \partial^\mu A_\mu^A(x)\right).
\end{equation}

\textbf{Step 3: Ghost Field Representation of the Determinant}

The Faddeev-Popov determinant can be represented using Grassmann (fermionic) variables via:
\begin{equation}
\det M[A] = \int \mathcal{D}c^A \mathcal{D}\overline{c}^A \exp\left(\frac{i}{\hbar}\int d^4x \, \overline{c}^A M^{AB}[A] c^B\right),
\end{equation}
where $c^A(x)$ and $\overline{c}^A(x)$ are anticommuting scalar (Grassmann) ghost fields with statistics opposite to the gauge field.

The full path integral in Lorenz gauge becomes:
\begin{equation}
\begin{aligned}
Z &= \int \mathcal{D}A_\mu^A \int \mathcal{D}c^A \mathcal{D}\overline{c}^A \int \mathcal{D}\lambda^A \\
&\times \exp\left(\frac{i}{\hbar}\left[S_{\text{YM}}[A] + S_{\text{ghost}}[c, \overline{c}, A] + S_{\text{gauge fixing}}[A, \lambda]\right]\right),
\end{aligned}
\end{equation}
where:
\begin{align}
S_{\text{ghost}} &:= \int d^4x \, \overline{c}^A D^\mu_{\mu} c^A, \\
S_{\text{gauge fixing}} &:= -\int d^4x \, \lambda^A \partial^\mu A_\mu^A.
\end{align}

Integrating over the Lagrange multipliers $\lambda^A$ enforces the Lorenz gauge condition, yielding the \emph{gaugefixed path integral}:
\begin{equation}
Z = \int \mathcal{D}A_\mu^A \int \mathcal{D}c^A \mathcal{D}\overline{c}^A \exp\left(\frac{i}{\hbar}\left[S_{\text{YM}}[A] + S_{\text{ghost}}[c, \overline{c}, A]\right]\right).
\end{equation}

\textbf{Interpretation in the divergence-first framework}

Within the divergence-first paradigm:

The Faddeev-Popov ghost fields arise naturally as \emph{derived objects} from the divergence structure. By Definition \ref{def:divergencePotential}, the divergence is the fundamental structure. The Yang-Mills gauge field is built on the emerged Lorentzian manifold; the redundancy in the path integral description reflects a measure-theoretic over-counting, which the Faddeev-Popov procedure corrects.

The ghost contribution to the effective action is:
\begin{equation}
S_{\text{eff, ghost}} = -\Tr \ln(D^\mu D_\mu) \bigg|_{\text{ghost}}
\end{equation}
where $D_\mu$ is the covariant derivative (which itself emerges from the divergence structure via Theorem \ref{thm:metricFromCarre}).

\textbf{Unitarity and Physical Observables}

For the theory to be unitary, all gauge-dependent degrees of freedom (the longitudinal component of $A$ and the ghosts) must decouple from physical observables. This is guaranteed by the BRST symmetry of the gaugefixed Lagrangian:

\begin{equation}
s A_\mu^A = D_\mu c^A, \quad s c^A = -\frac{1}{2}f^{ABC}c^B c^C, \quad s \overline{c}^A = \lambda^A,
\end{equation}

where $s$ is the nilpotent BRST operator ($s^2 = 0$). Physical states are BRST cohomology classes, ensuring independence from gauge-fixing scheme.

\textbf{Conclusion}

The Faddeev-Popov procedure is an essential component of covariant quantization. in the divergence-first framework, it arises as a correction to the path integral measure to account for gauge redundancy in the emerged gauge theories.

\end{proof}
\end{theorem}

\begin{remark}[Gauss Law Implementation in the Framework]
\label{rem:gausslawimplementationintheframework}
The Gauss law constraint $\nabla \cdot E^A = \rho^A$ emerges naturally from:
\begin{enumerate}
\item The divergence structure $D[\psi \| \phi]$ (Definition \ref{def:bregman}) is gauge-invariant by design (depends only on magnitudes).
\item The coupling between chromoelectric field and quark currents arises from the minimal coupling Hamiltonian.
\item In the continuum limit of the lattice theory (Section \ref{subsec:su3TrialityEmergence}), the Gauss law is the leading-order constraint determining the lattice dynamics.
\end{enumerate}

This confirms that the divergence-first framework naturally produces gauge-constrained QCD dynamics.
\end{remark}

