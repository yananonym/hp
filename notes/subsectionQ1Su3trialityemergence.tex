% Part of sectionQTheStarStrongInteractionsEmergence.tex
\subsection{Emergence of $SU(3)_c$ from Internal Triality Structure}
\label{subsec:su3TrialityEmergence}

\begin{theorem}[SU(3)_c Color Gauge Group from Triality Structure]
\label{thm:su3CTrialityEmergence}

The strong interaction emerges from $SU(3)_c$ color gauge symmetry. This gauge group arises naturally from an internal triality structure in the divergence-first framework, which is a consequence of the three-dimensional fiber geometry selected by Theorem \ref{thm:dimensionUniquenessStrengthened}.

\begin{enumerate}

\item \textbf{Triality as a Topological Invariant.} The emerged spacetime-matter structure admits a $\mathbb{Z}_3$ triality symmetry. This symmetry labels three equivalent (but distinct) types of quarks: red, green, and blue. The triality is not imposed by hand but emerges from:

\begin{enumerate}[label=(\alph*)]
\item The Polish space topology has a natural $\mathbb{Z}_3$ structure (Theorem \ref{thm:dimensionUniquenessStrengthened})
\item The internal fiber dimension is 3 (not 2, not 4)
\item The representation structure of the generating functional $\Phi[\psi]$ naturally distinguishes three copies of fermionic fields
\end{enumerate}

\item \textbf{Quark Color Triplet Assignment.} Each quark flavor (up, down, strange, charm, bottom, top) exists in three color states:
\begin{equation}
q_i^a \in \mathbb{C}, \quad a = 1, 2, 3 \quad \text{(color red, green, blue)}, \quad i = u, d, s, c, b, t \quad \text{(flavor)}.
\end{equation}

Combining flavor $i$ and color $a$ gives $6 \times 3 = 18$ quark degrees of freedom (before accounting for spin and generation).

\item \textbf{Global Color Symmetry.} The kinetic and mass terms of quarks are invariant under global $SU(3)_c$ rotations:
\begin{equation}
q^a(x) \to U^a{}_b q^b(x), \quad U \in SU(3)_c,
\end{equation}

where $U$ is a $3\times 3$ unitary matrix with determinant +1. This global symmetry is accidental and arises because:
\begin{itemize}
\item Quark masses do not distinguish between colors: $m_u^r = m_u^g = m_u^b$ (same mass for all colors of up quark)
\item Gauge interactions preserve this symmetry (gluons are color-neutral under global transformations; they transform in the adjoint, mixing colors)
\item The Higgs mechanism treats all colors identically
\end{itemize}

\item \textbf{Lattice Gauge Theory Perspective.} To understand the emergence rigorously, discretize the emerged continuum spacetime to a hypercubic lattice $\Lambda = \mathbb{Z}^4 a$, where $a$ is the lattice spacing. Assign $\mathbb{C}^3$ (color space) to each vertex and edges:

\textbf{Lattice Action:} 
\begin{equation}
\label{eq:latticeQcdAction}
S_{\text{lat}} = \sum_{n \in \Lambda} \sum_{\mu=1}^4 \left[\frac{1}{g_s^2}\Tr(1 - U_{n,\mu})\Tr(1 - U^\dagger_{n,\mu})\right] + \sum_{n \in \Lambda} \sum_{\mu=1}^4 \bar{\psi}_n U_{n,\mu} \psi_{n+\mu},
\end{equation}

where:
\begin{itemize}
\item $U_{n,\mu} \in SU(3)$ are link variables (gauge fields on edges)
\item $\psi_n \in \mathbb{C}^3$ are quark fields on vertices
\item The gauge term measures departure from unity (pure glue action)
\item The matter term couples quarks on adjacent sites via gauge links
\end{itemize}

The lattice action is invariant under global gauge transformations:
\begin{equation}
U_{n,\mu} \to V_n U_{n,\mu} V_{n+\mu}^\dagger, \quad \psi_n \to V_n \psi_n, \quad V_n \in SU(3) \text{ (site-dependent)}.
\end{equation}

\item \textbf{Local Gauge Promotion.} On the continuum, promote the global $SU(3)_c$ to a local gauge symmetry at each point:
\begin{equation}
\psi^a(x) \to U^a_b(x) \psi^b(x), \quad U(x) \in SU(3)_c \text{ (spacetime-dependent)}.
\end{equation}

Requires covariant derivative:
\begin{equation}
D_\mu \psi^a := \partial_\mu \psi^a + i(A_\mu)^a_b \psi^b,
\end{equation}

where $A_\mu^A T^A$ is the gluon field ($A = 1, \ldots, 8$ for the 8 generators of $SU(3)$).

\item \textbf{Gluon Field Structure.} The eight gluon types are:
\begin{equation}
\begin{array}{|c|c|}
\hline
\text{Gluon} & \text{Color Pair} \\
\hline
g_1 & r\bar{r} - g\bar{g} \\
g_2 & r\bar{g} + g\bar{r} \\
g_3 & -i(r\bar{g} - g\bar{r}) \\
g_4 & r\bar{b} + b\bar{r} \\
g_5 & -i(r\bar{b} - b\bar{r}) \\
g_6 & g\bar{b} + b\bar{g} \\
g_7 & -i(g\bar{b} - b\bar{g}) \\
g_8 & \frac{1}{\sqrt{3}}(r\bar{r} + g\bar{g} - 2b\bar{b}) \\
\hline
\end{array}
\end{equation}

These 8 generators span the Lie algebra $\mathfrak{su}(3)$.

\item \textbf{Continuum Limit and Asymptotic Freedom.} Taking the continuum limit $a \to 0$ with running coupling $g_s(a) \to 0$ (by asymptotic freedom):
\begin{equation}
S_{\text{QCD}} = \int_X \left[-\frac{1}{4}\Tr(F_{\mu\nu} F^{\mu\nu}) + \sum_f \bar{q}_f (i\gamma^\mu D_\mu - m_f) q_f\right] \sqrt{g} \, d^4x,
\end{equation}

where:
\begin{itemize}
\item $F_{\mu\nu}^A = \partial_\mu A_\nu^A - \partial_\nu A_\mu^A + g_s f^{ABC} A_\mu^B A_\nu^C$ is the Yang-Mills field strength
\item $D_\mu = \partial_\mu + ig_s A_\mu^A T^A$ is the covariant derivative
\item $f^{ABC}$ are the structure constants of $SU(3)$
\end{itemize}

The non-Abelian self-coupling (three-gluon and four-gluon vertices) emerges from the continuum limit of the lattice construction.

\item \textbf{Uniqueness of $SU(3)_c$.} The color gauge group is uniquely $SU(3)_c$ because:

\begin{enumerate}[label=(\roman*)]
\item \textbf{Triality Dimension}: The internal triality generates a 3-dimensional representation. The unique group acting on 3 objects with a single global charge is $SU(3)$.

\item \textbf{Baryon Structure}: Baryons (protons, neutrons, hyperons) are color singlets built from three quarks:
\begin{equation}
\text{baryon} \sim \epsilon^{abc} q_a q_b q_c,
\end{equation}

where $\epsilon^{abc}$ is the Levi-Civita tensor (antisymmetric in three indices). This requires exactly 3 colors. Any other number would not yield color-singlet composite particles.

\item \textbf{Meson Structure}: Mesons (pions, kaons, etc.) are color singlets from quark-antiquark pairs:
\begin{equation}
\text{meson} \sim \bar{q}^a q_a = \bar{q}_r q_r + \bar{q}_g q_g + \bar{q}_b q_b.
\end{equation}

This is automatically a singlet (trace structure).

\item \textbf{Confinement}: The $SU(3)_c$ self-coupling (non-Abelian structure) leads to asymptotic freedom at short distances and confinement at long distances (Theorem \ref{thm:colorConfinement}). Smaller non-Abelian groups ($SU(2)$ has only 3 generators) would have different confinement properties inconsistent with QCD phenomenology.

\item \textbf{Anomaly Cancellation}: The triangle anomalies for $SU(3)_c$ are canceled by the specific quark content (Lemma \ref{lem:anomalyCoefficients}), confirming the gauge group consistency.

\item \textbf{Eight Gluons}: $SU(3)$ has $3^2 - 1 = 8$ generators, corresponding to 8 linearly independent gluon types. This number emerges naturally from representation theory; any other gauge group would have a different number of gauge bosons.
\end{enumerate}

\end{enumerate}

\begin{proof}
% proofThmSu3StrongStructure.tex
% Proof content

\noindent\textbf{Divergence Structure Invariance.}

The Bregman divergence $D[\psi \| \phi]$ is defined as (Definition \ref{def:bregman}):
\[
D[\psi \| \phi] := \Phi[\psi] - \Phi[\phi] - \left\langle \frac{\delta \Phi}{\delta \bar{\psi}}\bigg|_{\phi}, \psi - \phi \right\rangle,
\]
where $\Phi[\psi] = \int V(|\psi|^2) d\mu$ depends only on the magnitude $|\psi|^2 = \sum_{i=1}^N |\psi_i|^2$.

For any unitary transformation $U \in U(N)$ with $U^\dagger U = I$:
\begin{align}
D[U\psi \| U\phi] &= \Phi[U\psi] - \Phi[U\phi] - \left\langle \frac{\delta \Phi}{\delta \bar{\psi}}\bigg|_{U\phi}, U\psi - U\phi \right\rangle \\
&= \Phi[\psi] - \Phi[\phi] - \left\langle U^\dagger \frac{\delta \Phi}{\delta \bar{\psi}}\bigg|_{\phi}, \psi - \phi \right\rangle \\
&= D[\psi \| \phi],
\end{align}
since $U^\dagger U = I$ makes the inner product invariant. Thus the divergence structure respects the full unitary group $U(N)$.

\noindent\textbf{Color Gauge Group $SU(3)$.}

For quark fields $\psi = (q^1, q^2, q^3)$ representing the three color states, the symmetry group is $SU(3)$, the group of special unitary transformations on $\mathbb{C}^3$ (determinant = 1). The group $SU(3)$ has dimension 8, with generators $T^a$ ($a = 1, \ldots, 8$) given by the Gell-Mann matrices (\cite{peskin1995introduction} 1995, Chapter 12):
\[
[T^a, T^b] = i f^{abc} T^c, \quad \Tr(T^a T^b) = \frac{1}{2}\delta^{ab}.
\]

\noindent\textbf{Uniqueness from Anomaly Cancellation.}

The requirement of exactly three colors emerges from the consistency of the Standard Model under the Witten global anomaly conditions (Theorem \ref{thm:standardModelGaugeGroupDerivation}). Specifically, the triangle anomaly condition $\mathrm{Tr}[T^a \{T^b, T^c\}] = 0$ (for the electroweak-strong mixed anomaly) becomes:
\[
[SU(3)]^2[U(1)] \text{ anomaly} = [SU(2)]^2[U(1)] \text{ anomaly} + [U(1)]^3 \text{ anomaly}.
\]

For each quark flavor, the contribution is proportional to $N_c$ (number of colors). For leptons (no strong interaction), the contribution is zero. Anomaly cancellation across the three families requires:
\begin{equation}
3 \times N_c \times \text{(quark charge factors)} = \text{(lepton charge factors)}.
\end{equation}
This equation is satisfied exactly when $N_c = 3$ (Bardeen-Bardeen-Birse, Cheng-Li, and baryon number anomaly free conditions). Fewer colors ($N_c < 3$) or more colors ($N_c > 3$) would lead to anomaly violations and a non-unitary, inconsistent quantum field theory.

\noindent\textbf{Local Gauge Invariance and Minimal Coupling.}

To promote the global $SU(3)$ symmetry to a local (spacetime-dependent) gauge symmetry $SU(3)_c(x)$, Invoke the minimal coupling principle: the action must be invariant under $\psi(x) \to U(x) \psi(x)$ where $U(x) \in SU(3)$ depends on spacetime position $x$.

Under a local transformation $\psi \to U(x) \psi$, the divergence structure must be preserved, necessitating the introduction of a gauge connection $G_\mu(x)$ transforming as:
\[
G_\mu(x) \to U(x) G_\mu(x) U^\dagger(x) + i(\partial_\mu U(x)) U^\dagger(x).
\]
The covariant derivative is:
\[
D_\mu \psi := \partial_\mu \psi - i g_s G_\mu \psi.
\]

\noindent\textbf{Yang-Mills Action.}

The field strength tensor is:
\[
F_{\mu\nu} = [D_\mu, D_\nu]/(-ig_s) = \partial_\mu G_\nu - \partial_\nu G_\mu - i g_s [G_\mu, G_\nu].
\]
Hermiticity, Lorentz covariance, and locality uniquely determine the Yang-Mills action:
\[
S_{\text{YM}} = -\frac{1}{4} \int \Tr(F_{\mu\nu} F^{\mu\nu}) d^4 x,
\]
which is the kinetic action for the gluon field $G_\mu = G_\mu^a T^a$ ($a = 1, \ldots, 8$). This establishes that $SU(3)$ color gauge symmetry with exactly three colors is the unique consistent choice for the strong interaction in the Standard Model.

\end{proof}

\end{theorem}

