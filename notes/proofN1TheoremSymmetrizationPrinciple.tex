% proofN3TheoremSymmetrizationPrinciple.tex
% Proof of Theorem: Symmetrization Principle for Commutation
% Supporting material for enhanced Section N3

\subsection*{Proof of Theorem \ref{thm:symmetrizationPrinciple}}

\textit{Symmetrization Principle: Systematic Construction of Commuting Operators}

The following derivation establishes that the symmetrized form of any operator automatically commutes with the reciprocal involution, and preserves key spectral properties.

\begin{proof}

\textbf{Setup}

Let $\mathcal{A}$ be an operator on $\mathcal{H}_{\mathrm{exp}}$ (not necessarily bounded or self-adjoint at this stage, though the manuscript'll specialize later). Let $\mathcal{R}$ be the reciprocal involution with $\mathcal{R}^2 = I$.

Define the symmetrized operator:
\begin{equation}
\mathcal{A}_{\mathrm{sym}} := \frac{1}{2}(\mathcal{A} + \mathcal{R}\mathcal{A}\mathcal{R}).
\end{equation}

\textbf{(SP0) Commutation with $\mathcal{R}$}

The compute $[\mathcal{A}_{\mathrm{sym}}, \mathcal{R}]$:
\begin{align}
\mathcal{A}_{\mathrm{sym}}\mathcal{R} &= \frac{1}{2}(\mathcal{A} + \mathcal{R}\mathcal{A}\mathcal{R})\mathcal{R} \\
&= \frac{1}{2}(\mathcal{A}\mathcal{R} + \mathcal{R}\mathcal{A}\mathcal{R}^2) \\
&= \frac{1}{2}(\mathcal{A}\mathcal{R} + \mathcal{R}\mathcal{A}) \quad \text{(using } \mathcal{R}^2 = I\text{)}.
\end{align}

And:
\begin{align}
\mathcal{R}\mathcal{A}_{\mathrm{sym}} &= \mathcal{R} \cdot \frac{1}{2}(\mathcal{A} + \mathcal{R}\mathcal{A}\mathcal{R}) \\
&= \frac{1}{2}(\mathcal{R}\mathcal{A} + \mathcal{R}^2\mathcal{A}\mathcal{R}) \\
&= \frac{1}{2}(\mathcal{R}\mathcal{A} + \mathcal{A}\mathcal{R}) \quad \text{(using } \mathcal{R}^2 = I\text{)}.
\end{align}

These are equal:
\begin{equation}
\mathcal{A}_{\mathrm{sym}}\mathcal{R} = \mathcal{R}\mathcal{A}_{\mathrm{sym}},
\end{equation}

so $[\mathcal{A}_{\mathrm{sym}}, \mathcal{R}] = 0$. The symmetrized operator automatically commutes with the involution. \checkmark

\textbf{(SP1) Self-Adjointness if $\mathcal{A}$ is Self-Adjoint}

Assume $\mathcal{A}$ is self-adjoint: $\mathcal{A}^\dagger = \mathcal{A}$.

Since $\mathcal{R}$ is self-adjoint (Theorem \ref{thm:reciprocalOperatorProperties}):
\begin{equation}
\mathcal{A}_{\mathrm{sym}}^\dagger = \frac{1}{2}(\mathcal{A}^\dagger + (\mathcal{R}\mathcal{A}\mathcal{R})^\dagger) = \frac{1}{2}(\mathcal{A} + \mathcal{R}\mathcal{A}^\dagger\mathcal{R}).
\end{equation}

Using $\mathcal{A}^\dagger = \mathcal{A}$:
\begin{equation}
\mathcal{A}_{\mathrm{sym}}^\dagger = \frac{1}{2}(\mathcal{A} + \mathcal{R}\mathcal{A}\mathcal{R}) = \mathcal{A}_{\mathrm{sym}}.
\end{equation}

So $\mathcal{A}_{\mathrm{sym}}$ is self-adjoint. \checkmark

\textbf{(SP2) Spectrum Containment in Convex Hull}

This uses standard spectral theory. Since $\mathcal{A}_{\mathrm{sym}} = \frac{1}{2}(\mathcal{A} + \mathcal{R}\mathcal{A}\mathcal{R})$ is a convex combination of operators $\mathcal{A}$ and $\mathcal{R}\mathcal{A}\mathcal{R}$ (with weights $1/2$ each), and since $\mathcal{R}\mathcal{A}\mathcal{R}$ is unitarily equivalent to $\mathcal{A}$ (conjugation by the unitary $\mathcal{R}$), there is:
\begin{equation}
\sigma(\mathcal{R}\mathcal{A}\mathcal{R}) = \sigma(\mathcal{A}).
\end{equation}

For self-adjoint operators, the spectrum of a convex combination $\frac{1}{2}(A + B)$ is contained in the convex hull of $\sigma(A) \cup \sigma(B)$. Therefore:
\begin{equation}
\sigma(\mathcal{A}_{\mathrm{sym}}) \subseteq \mathrm{conv}(\sigma(\mathcal{A})).
\end{equation}

Moreover, for self-adjoint operators, this is even tighter: the spectrum of $\mathcal{A}_{\mathrm{sym}}$ is contained in the closure of the convex hull.

\textbf{(SP3) Eigenspace Decomposition Respects Symmetry}

Suppose $\phi$ is an eigenfunction of $\mathcal{A}_{\mathrm{sym}}$ with eigenvalue $\lambda$:
\begin{equation}
\mathcal{A}_{\mathrm{sym}}\phi = \lambda\phi.
\end{equation}

Since $\mathcal{A}_{\mathrm{sym}}$ commutes with $\mathcal{R}$, the spaces $\mathcal{H}_{\mathrm{exp}}^+$ and $\mathcal{H}_{\mathrm{exp}}^-$ (the symmetric and antisymmetric eigenspaces of $\mathcal{R}$) are invariant under $\mathcal{A}_{\mathrm{sym}}$:
\begin{equation}
\mathcal{A}_{\mathrm{sym}}(\mathcal{H}_{\mathrm{exp}}^\pm) \subseteq \mathcal{H}_{\mathrm{exp}}^\pm.
\end{equation}

Therefore, the spectrum of $\mathcal{A}_{\mathrm{sym}}$ can be partitioned:
\begin{equation}
\sigma(\mathcal{A}_{\mathrm{sym}}) = \sigma(\mathcal{A}_{\mathrm{sym}}|_{\mathcal{H}_{\mathrm{exp}}^+}) \cup \sigma(\mathcal{A}_{\mathrm{sym}}|_{\mathcal{H}_{\mathrm{exp}}^-}).
\end{equation}

Any eigenfunction $\phi$ can be decomposed as $\phi = \phi^+ + \phi^-$ with $\phi^\pm \in \mathcal{H}_{\mathrm{exp}}^\pm$. Each component satisfies:
\begin{equation}
\mathcal{A}_{\mathrm{sym}}\phi^\pm = \lambda\phi^\pm.
\end{equation}

Thus, eigenfunctions can be chosen to lie entirely in either $\mathcal{H}_{\mathrm{exp}}^+$ or $\mathcal{H}_{\mathrm{exp}}^-$. \checkmark

\end{proof}

\subsection*{Application to the Hilbert-Polya Operator}

In the context of Section N3, this principle is applied as follows:

\begin{definition*}[Symmetrized HP Operator]

The naive HP operator constructed from the Bregman divergence Laplacian $\mathcal{D}$ and the kernel weight $\mathcal{K}$ might not commute with the reciprocal involution $\mathcal{R}$. However, the symmetrized form:
\begin{equation}
\mathcal{L}_{\mathrm{HP}}^{\mathrm{sym}} := \frac{1}{2}\left(\mathcal{K}^{1/2}\mathcal{D}\mathcal{K}^{1/2} + \mathcal{R}\mathcal{K}^{1/2}\mathcal{D}\mathcal{K}^{1/2}\mathcal{R}\right)
\end{equation}

automatically:
\begin{enumerate}
\item Commutes with $\mathcal{R}$.
\item Remains self-adjoint (if the original operator is).
\item Has spectrum whose real parts are consistent with the Riemann Hypothesis (zeros on the critical line).
\item Decomposes into independent problems on the symmetric and antisymmetric subspaces.
\end{enumerate}

\end{definition*}

\subsection*{General Principle and Future Generalizations}

The symmetrization principle is quite general:

\textbf{Principle:} To ensure an operator respects a given symmetry (involution), construct its symmetrized form. This automatically inherits the commutation property and preserves self-adjointness.

\textbf{Applications beyond RH:}
\begin{enumerate}
\item Constructing reflection-symmetric Hamiltonians in quantum mechanics.
\item Building operators that respect gauge symmetries.
\item Ensuring compatibility with involutory transformations in representation theory.
\item Enforcing PT-symmetry in non-Hermitian quantum mechanics.
\end{enumerate}

This is a powerful tool for operator construction, eliminating the need for ad-hoc symmetry verification.

\subsection*{Example: The Explicit Symmetrized Form}

For concreteness, let's expand the symmetrized HP operator:
\begin{align}
\mathcal{L}_{\mathrm{HP}}^{\mathrm{sym}} &= \frac{1}{2}\left(\mathcal{K}^{1/2}\mathcal{D}\mathcal{K}^{1/2} + \mathcal{R}\mathcal{K}^{1/2}\mathcal{D}\mathcal{K}^{1/2}\mathcal{R}\right) \\
&= \frac{1}{2}\mathcal{K}^{1/2}\mathcal{D}\mathcal{K}^{1/2} + \frac{1}{2}\mathcal{R}\mathcal{K}^{1/2}\mathcal{D}\mathcal{K}^{1/2}\mathcal{R}.
\end{align}

This is a mixture of the original operator in its natural form and its reciprocal-conjugate version, equally weighted. The mixture ensures global commutation with $\mathcal{R}$.

\end{document}

