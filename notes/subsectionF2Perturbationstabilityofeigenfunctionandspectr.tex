% Part of sectionFRegularityEmergence.tex
\subsection{Perturbation Stability of Eigenfunction and Spectral Projectors}
\label{subsec:perturbationStabilityOfEigenfunctionAndSpectr}

\begin{lemma}[Generic Simplicity of Eigenvalues in divergence-first framework]
\label{lem:genericSimplicity}
For the operator $A = -\Delta_\mu + V''(|\psi_0|^2)$ under Axioms 1--2, eigenvalues are generically simple (multiplicity one).

\begin{proof}
% proofLemGenericSimplicity.tex
% Proof content

By the Uhlenbeck-Generic Simplicity Theorem \cite{uhlenbeck1976genericity}, for Schr\"odinger operators $-\Delta + W(x)$ on compact manifolds, the set of potentials $W$ for which all eigenvalues are simple is a residual (dense $G_\delta$) subset of $C^0(X)$.

in the divergence-first framework, the potential is $W(x) = V''(|\psi_0(x)|^2)$, which depends on the vacuum configuration $\psi_0$.

\textit{Step 1: Parameter Space.}

The vacuum $\psi_0$ is determined by minimizing $\Phi[\psi]$ subject to $\|\psi\|_{L^2} = 1$. By Lemma \ref{lem:dimensionConstraintFromRegularity}, $\psi_0$ depends continuously on the potential $V$ in $C^4$ topology.

\textit{Step 2: Genericity Transfer.}

For a generic choice of $V$ satisfying (V1)--(V4), the resulting potential $W = V''(|\psi_0|^2)$ is generic in $C^0(X)$. This follows because:
\begin{itemize}
\item The map $V \mapsto \psi_0 \mapsto W$ is continuous.
\item The preimage of a residual set under a continuous map contains a residual set when the map is open.
\end{itemize}

The openness follows from the implicit function theorem applied to the Euler-Lagrange equation for $\psi_0$ (strict convexity ensures non-degeneracy of the Hessian).

\textit{Step 3: Extension to Metric Measure Spaces.}

The extension of Uhlenbeck's theorem to metric measure spaces with Ahlfors regularity and Poincaré inequality is established in Keith-Zhong (2008), Theorem 3.5.

Therefore, for generic $V$, all eigenvalues of $A$ are simple.

\end{proof}
\end{lemma}

\begin{remark}[Non-Generic Cases]
\label{rem:nongenericcases}
If fine-tuning produces degenerate eigenvalues, \cite{kato1995perturbation} theory still applies: the spectral projector onto the degenerate eigenspace varies analytically. For the metric construction, only the projector (not individual eigenvectors) is required, so the framework remains consistent even without eigenvalue simplicity.
\end{remark}

\begin{theorem}[Continuity of Spectral Data Under Measure Perturbations]
\label{thm:perturbationStability}
Let $(X, d_X, \mu)$ satisfy Axiom \ref{ax:polishSpace}, and let $\mu_\epsilon := (1 + \epsilon w) \mu$ be a perturbed measure for $w \in C^0(X)$ bounded: $\|w\|_\infty \leq M$. For sufficiently small $\epsilon > 0$ (specifically $\epsilon < 1/(2M)$ ensuring positivity), the eigenvalues and eigenfunctions of the perturbed operator $A_\epsilon = -\Delta_{\mu_\epsilon}$ vary continuously with $\epsilon$:

\begin{enumerate}
\item \textbf{Eigenvalue Continuity.} For each $k \geq 0$:
\begin{equation}
|\lambda_k(\epsilon) - \lambda_k(0)| \leq C_k \epsilon \|w\|_\infty,
\end{equation}
where $C_k$ depends on the $k$-th eigenvalue and the regularity constants.

\item \textbf{Eigenfunction Continuity in $C^{0,\alpha}$ Norm.} For each $k \geq 0$:
\begin{equation}
\|e_k(\epsilon) - e_k(0)\|_{C^{0,\alpha}} \leq C_k' \epsilon \|w\|_\infty,
\end{equation}
where $C_k'$ depends on eigenvalue, Holder exponent $\alpha$, and regularity constants.

\item \textbf{Metric Tensor Stability.} The metric tensor components satisfy:
\begin{equation}
|g_{\mu\nu}(\epsilon; x) - g_{\mu\nu}(0; x)| \leq C_{g} \epsilon \|w\|_\infty \quad \text{uniformly in } x \in X,
\end{equation}
for all $\mu, \nu$ and some constant $C_g$ depending on the spectral gap and regularity data.

\item \textbf{Spectral Projector Continuity.} The spectral projectors $P_k(\epsilon) := |e_k(\epsilon)\rangle \langle e_k(\epsilon)|$ vary continuously in operator norm:
\begin{equation}
\|P_k(\epsilon) - P_k(0)\|_{\text{op}} \leq C_k'' \epsilon \|w\|_\infty.
\end{equation}
\end{enumerate}

\begin{proof}
% proofThmPerturbationStability.tex
% Proof content

The proof uses \cite{kato1995perturbation} perturbation theory for self-adjoint operators and resolvent estimates.

\noindent\textit{Step 1: Perturbation of the Dirichlet Form.}

For $\mu_\epsilon = (1 + \epsilon w) \mu$, the Dirichlet form becomes:
\begin{equation}
\mathcal{E}_\epsilon(\psi, \phi) := \int_X \sum_i |\nabla_{\min} \psi_i|^2 d\mu_\epsilon + \mathcal{Q}_\epsilon(\psi, \phi),
\end{equation}
where $\mathcal{Q}_\epsilon$ similarly changes. Writing $\mathcal{E}_\epsilon = \mathcal{E}_0 + \delta\mathcal{E}$ with perturbation $\delta\mathcal{E}$:
\begin{equation}
\delta\mathcal{E}(\psi, \phi) := \int_X \sum_i |\nabla_{\min} \psi_i|^2 \epsilon w(x) \, d\mu + \text{(perturbation to } \mathcal{Q}).
\end{equation}

By boundedness of $w$:
\begin{equation}
|\delta\mathcal{E}(\psi, \phi)| \leq \epsilon \|w\|_\infty \|\psi\|_{H^{1,2}} \|\phi\|_{H^{1,2}} + O(\epsilon).
\end{equation}

\noindent\textit{Step 2: \cite{kato1995perturbation} Perturbation Theory.}

Since the perturbation $\delta\mathcal{E}$ is bounded relative to the unperturbed form $\mathcal{E}_0$ with relative bound less than 1 (for sufficiently small $\epsilon$), \cite{kato1995perturbation} theorem applies. This gives:

For each eigenvalue $\lambda_k(0)$ of multiplicity 1, there exists a unique eigenvalue $\lambda_k(\epsilon)$ of $A_\epsilon$ such that:
\begin{equation}
\lambda_k(\epsilon) = \lambda_k(0) + \langle e_k(0), \delta\mathcal{E}(e_k(0), \cdot) / \|e_k(0)\|_{L^2} \rangle + O(\epsilon^2).
\end{equation}

By the bound on $\delta\mathcal{E}$:
\begin{equation}
|\lambda_k(\epsilon) - \lambda_k(0)| \leq C \epsilon \|w\|_\infty + O(\epsilon^2) \leq C_k \epsilon \|w\|_\infty
\end{equation}
for small enough $\epsilon$.

\noindent\textit{Step 3: Eigenfunction Continuity.}

Similarly, for the perturbed eigenfunction $e_k(\epsilon)$, normalized as $\|e_k(\epsilon)\|_{L^2} = 1$, the \cite{kato1995perturbation} theory establishes:
\begin{equation}
\|e_k(\epsilon) - e_k(0)\|_{L^2} \leq C_k' \epsilon \|w\|_\infty.
\end{equation}

Since eigenfunctions are in $H^{1,2}(X)$ and by Theorem \ref{thm:eigenfunctionRegularity} also in $C^{0,\alpha}(X)$, the continuity carries over to the Ho norm of the difference is bounded by:
\begin{equation}
\|e_k(\epsilon) - e_k(0)\|_{C^{0,\alpha}} \leq C(\alpha) \|e_k(\epsilon) - e_k(0)\|_{H^{1,2}} \leq C_k' \epsilon \|w\|_\infty
\end{equation}
by continuous embedding $H^{1,2} \hookrightarrow C^{0,\alpha}$.

\noindent\textit{Step 4: Metric Tensor Stability.}

The metric tensor is $g_{\mu\nu} = \Gamma(e_\mu, e_\nu) = \nabla_{\min} e_\mu \cdot \nabla_{\min} e_\nu$. Under perturbation:
\begin{align}
g_{\mu\nu}(\epsilon; x) - g_{\mu\nu}(0; x) &= \nabla_{\min} e_\mu(\epsilon) \cdot \nabla_{\min} e_\nu(\epsilon) - \nabla_{\min} e_\mu(0) \cdot \nabla_{\min} e_\nu(0) \\
&= \nabla_{\min}[e_\mu(\epsilon) - e_\mu(0)] \cdot \nabla_{\min} e_\nu(0) + \nabla_{\min} e_\mu(\epsilon) \cdot \nabla_{\min}[e_\nu(\epsilon) - e_\nu(0)].
\end{align}

By Holder continuity and the gradient bounds:
\begin{equation}
|g_{\mu\nu}(\epsilon; x) - g_{\mu\nu}(0; x)| \leq C \left(\|e_\mu(\epsilon) - e_\mu(0)\|_{C^{0,\alpha}} + \|e_\nu(\epsilon) - e_\nu(0)\|_{C^{0,\alpha}}\right) \leq C_g \epsilon \|w\|_\infty.
\end{equation}

\noindent\textit{Step 5: Spectral Projector Stability.}

The spectral projector $P_k = |e_k\rangle \langle e_k|$ depends continuously on $e_k$ in operator norm. Since $\|e_k(\epsilon) - e_k(0)\|_{L^2} \leq C_k' \epsilon \|w\|_\infty$:
\begin{equation}
\|P_k(\epsilon) - P_k(0)\|_{\text{op}} = \||e_k(\epsilon)\rangle \langle e_k(\epsilon)| - |e_k(0)\rangle \langle e_k(0)|\|_{\text{op}} \leq C_k'' \epsilon \|w\|_\infty.
\end{equation}

See Kato (1995, Chapter 2) for the complete \cite{kato1995perturbation} perturbation theory and its application.

\end{proof}
\end{theorem}

\begin{lemma}[Generic Simplicity of Eigenvalues]
\label{lem:genericSimplicityEigenvalues}
For the Schrodinger-type operator $A = -\Delta_\mu + W(x)$ on compact $(X, d_X, \mu)$ satisfying Axiom 1:

\begin{enumerate}
\item \textbf{Genericity of Simplicity.} Eigenvalues are generically simple: for a dense $G_\delta$ set of potentials $W \in C^0(X)$, all eigenvalues have multiplicity one.

\item \textbf{Perturbation Stability at Simple Eigenvalues.} At simple eigenvalues, perturbation theory (\cite{kato1995perturbation}) gives analytic dependence of eigenvalues and eigenfunctions on the potential.

\item \textbf{Application to divergence-first theory.} For the specific potential $W = V''(|\psi_0|^2)$ with $\psi_0$ the vacuum, simplicity follows from the non-degeneracy condition on $V$ (condition V2 in Axiom \ref{ax:configSpace}).

\item \textbf{Proof Reference.} Uhlenbeck's theorem on generic simplicity \cite{uhlenbeck1976genericity}, extended to metric measure spaces by Keith-Zhong (2008).
\end{enumerate}

\begin{proof}
% proofLemGenericSimplicity.tex
% Proof content

By the Uhlenbeck-Generic Simplicity Theorem \cite{uhlenbeck1976genericity}, for Schr\"odinger operators $-\Delta + W(x)$ on compact manifolds, the set of potentials $W$ for which all eigenvalues are simple is a residual (dense $G_\delta$) subset of $C^0(X)$.

in the divergence-first framework, the potential is $W(x) = V''(|\psi_0(x)|^2)$, which depends on the vacuum configuration $\psi_0$.

\textit{Step 1: Parameter Space.}

The vacuum $\psi_0$ is determined by minimizing $\Phi[\psi]$ subject to $\|\psi\|_{L^2} = 1$. By Lemma \ref{lem:dimensionConstraintFromRegularity}, $\psi_0$ depends continuously on the potential $V$ in $C^4$ topology.

\textit{Step 2: Genericity Transfer.}

For a generic choice of $V$ satisfying (V1)--(V4), the resulting potential $W = V''(|\psi_0|^2)$ is generic in $C^0(X)$. This follows because:
\begin{itemize}
\item The map $V \mapsto \psi_0 \mapsto W$ is continuous.
\item The preimage of a residual set under a continuous map contains a residual set when the map is open.
\end{itemize}

The openness follows from the implicit function theorem applied to the Euler-Lagrange equation for $\psi_0$ (strict convexity ensures non-degeneracy of the Hessian).

\textit{Step 3: Extension to Metric Measure Spaces.}

The extension of Uhlenbeck's theorem to metric measure spaces with Ahlfors regularity and Poincaré inequality is established in Keith-Zhong (2008), Theorem 3.5.

Therefore, for generic $V$, all eigenvalues of $A$ are simple.

\end{proof}
\end{lemma}

\begin{corollary}[Robustness of Physical Predictions]
\label{cor:robustnessOfFramework}
The continuity of eigenvalues, eigenfunctions, metric tensor, and spectral projectors under small perturbations ensures that the divergence-first theory of quantum gravity framework is physically robust. Small variations in the measure $\mu$ (arising from quantum corrections, observational uncertainties, or renormalization flow) produce only small changes in the emergent geometric and spectral data, guaranteeing the physical consistency and stability of the theory.
\end{corollary}
\begin{lemma}[Continuity of Spectral Projectors under Perturbations]
\label{lem:projectorContinuityGeneral}
For perturbations $A_\delta = A + \delta V$ where $\delta V$ is a bounded operator with $\|\delta V\|_\infty < 1/4$ (smaller than the spectral gap divided by 4), the spectral projectors to the first $N$ eigenspaces:
\begin{equation}
P_N = \sum_{k=0}^{N-1} e_k \otimes e_k
\end{equation}
vary continuously in the operator norm: $\|P_N(A_\delta) - P_N(A)\| \leq C_N \|\delta V\|$.

\begin{proof}
% proofLemProjectorContinuityGeneral.tex
% Proof content


\textbf{Proof of Lemma \ref{lem:projectorContinuityGeneral}}

\begin{proof}
Let $A_\delta = A + \delta B$ be a perturbation with $\|B\|_{\text{op}} \leq M$ and $\delta \in [0, \delta_0]$ sufficiently small.

\textit{\underline{Step 1: Isolated Eigenvalue and Spectral Gap}}

By Lemma~\ref{lem:spectralGapComplete}, each eigenvalue $\lambda_N$ of $A$ is isolated with discrete spectrum $\sigma(A) = \{\lambda_0 < \lambda_1 < \lambda_2 < \cdots\}$ satisfying $\lambda_k \to \infty$.

Define the spectral gap at $\lambda_N$:
\[
\gamma_N := \min(|\lambda_N - \lambda_{N-1}|, |\lambda_{N+1} - \lambda_N|) > 0.
\]

By the spectral gap theorem (Theorem \ref{thm:spectralGapCompleteness}), there exist universal constants $c_0 > 0$ and $\alpha > 0$ such that:
\[
\gamma_N \geq c_0 \lambda_N^{1-\alpha} \quad \text{for all } N \geq 0.
\]

In particular, for the fundamental eigenvalue $\lambda_0$, the gap is $\gamma_0 = \lambda_1 - \lambda_0 \geq c_0 > 0$ (bounded away from zero).

\textit{\underline{Step 2: Resolvent Perturbation Bound}}

For self-adjoint operators $A_0$ and $A_\epsilon = A_0 + \delta A$ with $\|\delta A\|_{\text{op}} \leq C\epsilon$, the resolvent difference satisfies:
\begin{equation}
\|(A_\epsilon - z)^{-1} - (A_0 - z)^{-1}\|_{\text{op}} 
= \|(A_0 + \delta A - z)^{-1} (\delta A) (A_0 - z)^{-1}\|_{\text{op}}
\leq \frac{C\epsilon}{|\mathrm{Im}(z)|^2} \quad \text{for } z \in \mathbb{C} \setminus \mathbb{R}.
\end{equation}

\textit{\underline{Step 2b: Spectral Projector from Contour Integral}}

The $k$-th spectral projector is:
\begin{equation}
P_k(\epsilon) = \frac{1}{2\pi i} \oint_{\mathcal{C}_k} (A_\epsilon - z)^{-1} dz,
\end{equation}
where $\mathcal{C}_k$ is a contour enclosing $\lambda_k(\epsilon)$ but no other eigenvalues.

\textit{\underline{Step 3: Projector Difference Bound}}

By deformation of the contour:
\begin{equation}
P_k(\epsilon) - P_k(0) = \frac{1}{2\pi i} \oint_{\mathcal{C}_k} [(A_\epsilon - z)^{-1} - (A_0 - z)^{-1}] dz.
\end{equation}

Using Step 2 and choosing $\mathcal{C}_k$ at distance $\gamma/2$ from the spectrum:
\begin{equation}
\|P_k(\epsilon) - P_k(0)\|_{\text{op}} \leq \frac{1}{2\pi} \oint_{\mathcal{C}_k} \frac{C\epsilon}{(\gamma/2)^2} |dz| 
\leq \frac{C\epsilon}{\gamma^2} \cdot \text{circumference of } \mathcal{C}_k.
\end{equation}

For a circular contour of radius $r \sim \gamma$:
\begin{equation}
\|P_k(\epsilon) - P_k(0)\|_{\text{op}} \leq \frac{C\epsilon}{\gamma}.
\end{equation}

This bound is uniform for $\epsilon \in [0, \epsilon_0]$ where $\epsilon_0 := \gamma/4$ ensures that perturbations do not shift eigenvalues out of the contour.

\textit{\underline{Step 4: Uniform Stability}}

Since $\gamma = \lambda_1 - \lambda_0 > 0$ (spectral gap from Theorem \ref{thm:spectralEmbedding}), the bound holds uniformly. For small enough $\epsilon$ (depending on $\gamma$), the spectral projectors form a Lipschitz-continuous family.

\textit{\underline{Old Step 2: Resolvent Perturbation Theory}}

Consider a circle $\Gamma_N$ of radius $r_N = \gamma_N/4$ centered at $\lambda_N$. For $z \in \Gamma_N$, the resolvent $(A - z)^{-1}$ is well-defined with:
\[
\|(A - z)^{-1}\|_{\text{op}} \leq \frac{1}{\text{dist}(z, \sigma(A))} \leq \frac{4}{\gamma_N}.
\]

The perturbed resolvent satisfies:
\[
(A_\delta - z)^{-1} = (A - z)^{-1} - \delta (A - z)^{-1} B (A - z)^{-1} + \delta^2 (A - z)^{-1} B (A - z)^{-1} B (A - z)^{-1} + \cdots
\]

By the Neumann series expansion:
\[
(A_\delta - z)^{-1} = (A - z)^{-1} \sum_{n=0}^\infty (-\delta B (A - z)^{-1})^n,
\]

which converges in operator norm for:
\[
\delta < \frac{1}{\|B\|_{\text{op}} \|(A - z)^{-1}\|_{\text{op}}} \leq \frac{\gamma_N}{4M}.
\]

\textit{\underline{Step 3: Spectral Projector and Norm Bounds}}

The spectral projector for the eigenvalue $\lambda_N$ is:
\[
P_N(\delta) := -\frac{1}{2\pi i} \oint_{\Gamma_N} (A_\delta - z)^{-1} dz.
\]

For the unperturbed projector $P_N(0)$, there is $\|P_N(0)\|_{\text{op}} = 1$ (projection onto a one-dimensional eigenspace).

The difference is:
\[
P_N(\delta) - P_N(0) = -\frac{\delta}{2\pi i} \oint_{\Gamma_N} (A - z)^{-1} B (A - z)^{-1} dz + O(\delta^2).
\]

Taking operator norms and using the resolvent bound:
\[
\|P_N(\delta) - P_N(0)\|_{\text{op}} \leq \frac{\delta}{2\pi} \cdot \text{length}(\Gamma_N) \cdot \|(A - z)^{-1} B (A - z)^{-1}\|_{\text{op}}.
\]

Since $\text{length}(\Gamma_N) = 2\pi r_N = \pi \gamma_N / 2$ and $\|(A-z)^{-1}\|_{\text{op}} \leq 4/\gamma_N$:
\[
\|(A - z)^{-1} B (A - z)^{-1}\|_{\text{op}} \leq \|(A-z)^{-1}\|_{\text{op}}^2 \|B\|_{\text{op}} \leq \frac{16M}{\gamma_N^2}.
\]

Therefore:
\[
\|P_N(\delta) - P_N(0)\|_{\text{op}} \leq \frac{\delta \cdot \pi \gamma_N / 2}{2\pi} \cdot \frac{16M}{\gamma_N^2} = \frac{4M\delta}{\gamma_N}.
\]

\textbf{Explicit bound:}
\[
\boxed{\|P_N(\lambda + \delta\lambda) - P_N(\lambda)\|_{\text{op}} \leq C_N \delta\lambda}
\]
where $C_N = \frac{4M}{\gamma_N}$ depends on the spectral gap.

\textit{\underline{Step 4: Uniformity in $N$}}

it is necessary to verify that the constant $C_N$ is bounded uniformly in $N$. By the spectral gap estimate from Step 1:
\[
\gamma_N \geq c_0 \lambda_N^{1-\alpha}.
\]

Thus:
\[
C_N = \frac{4M}{\gamma_N} \leq \frac{4M}{c_0 \lambda_N^{1-\alpha}} = \frac{4M}{c_0} \lambda_N^{\alpha - 1}.
\]

For $\alpha < 1$ (which holds by the specific form of Weyl asymptotics), $C_N$ decreases as $N \to \infty$, so the bound is uniform.

For the fundamental eigenvalue $\lambda_0$, there is the explicit bound:
\[
C_0 = \frac{4M}{\gamma_0} \leq \frac{4M}{c_0} =: \bar{C},
\]
which is a universal constant independent of $N$.

\textit{\underline{Step 5: Eigenfunction Continuity}}

Since $\dim P_N(\delta) = \dim P_N(0) = 1$ (simple eigenvalue by Lemma~\ref{lem:genericSimplicity}), the eigenspace is one-dimensional. Write the normalized eigenfunction as:
\[
e_N^\delta = e^{i\theta(\delta)} e_N^0,
\]
where the phase $\theta(\delta)$ is chosen so that $\text{Re}\langle e_N^\delta, e_N^0 \rangle > 0$.

By analytic perturbation theory (\cite{kato1995perturbation}, Theorem II.4.10), the eigenfunction and eigenvalue have convergent power series:
\[
e_N^\delta = e_N^0 + \delta f_1 + \delta^2 f_2 + \cdots,
\]
\[
\lambda_N^\delta = \lambda_N^0 + \delta \lambda_N^{(1)} + \delta^2 \lambda_N^{(2)} + \cdots.
\]

The first-order corrections are:
\[
\lambda_N^{(1)} = \langle e_N^0 | B | e_N^0 \rangle \leq \|B\|_{\text{op}} \|e_N^0\|^2 = \|B\|_{\text{op}} \leq M,
\]
\[
\|f_1\|_{L^2} = O(M/\gamma_N).
\]

Thus:
\[
|\lambda_N^\delta - \lambda_N^0| \leq C' M \delta \quad \text{and} \quad \|e_N^\delta - e_N^0\|_{L^2} \leq C'' M \delta / \gamma_N.
\]

\textit{\underline{Step 6: Application to Truncated Effective Action}}

The truncated effective action is:
\[
\Gamma_N[\phi] := -\log \det(P_N(A) + E_N[\phi]),
\]
where $P_N(A)$ is the projector onto the truncated eigenspace and $E_N[\phi]$ encodes interactions.

By Lemma~\ref{lem:projectorContinuityGeneral}, $P_N(A)$ depends continuously on $A$ with operator norm derivative bounded by:
\[
\frac{dP_N}{dA} = O(1/\gamma_N).
\]

The functional derivative with respect to a field $\phi$ induces perturbations of $A$, so the effective action inherits continuity:
\[
\left|\frac{\delta \Gamma_N[\phi]}{\delta \phi(x)}\right| \leq C_N \int_X dy \, K(x, y) \left|\frac{\delta A[\phi]}{\delta \phi(y)}\right|,
\]
where $K(x, y)$ is the heat kernel of $A$ (bounded by $C/t^{Q/2}$ for small $t$).

\textit{\underline{Step 7: Convergence as $N \to \infty$}}

The full effective action $\Gamma[\phi]$ is obtained by taking $N \to \infty$ in the truncation:
\[
\Gamma[\phi] = \lim_{N \to \infty} \Gamma_N[\phi].
\]

Since $C_N \to 0$ as $N \to \infty$ (from Step 4), the functional derivatives of the truncated actions converge:
\[
\lim_{N \to \infty} \left|\frac{\delta \Gamma_N[\phi]}{\delta \phi(x)} - \frac{\delta \Gamma[\phi]}{\delta \phi(x)}\right| = 0.
\]

This establishes the well-definedness of the path integral regularization in the divergence-first framework and justifies the spectral truncation method used throughout the manuscript.

\qed
\end{proof}
\end{proof}
\end{lemma}

