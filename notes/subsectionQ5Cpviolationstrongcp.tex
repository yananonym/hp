% Part of sectionQTheStarStrongInteractionsEmergence.tex
\subsection{CP Violation and the Strong CP Problem}
\label{subsec:cpViolationStrongCP}

\begin{theorem}[CP-Violating Phases from Divergence Structure]
\label{thm:cpPhasesFromDivergence}

in the divergence-first framework, CP-violating phases in the Yukawa coupling matrix and the QCD $\theta$ parameter constitute free parameters but are constrained by the divergence structure of the configuration space.

\begin{enumerate}

\item \textbf{Yukawa Matrix Structure:} The Yukawa coupling matrix $Y_f: \mathbb{C}^{3 \times 3} \to \mathbb{C}$ for fermion $f$ (up, down, electron) couples the three generation indices. It is a $3 \times 3$ complex matrix:

\begin{equation}
Y_f = \begin{pmatrix} y_{11} & y_{12} & y_{13} \\ y_{21} & y_{22} & y_{23} \\ y_{31} & y_{32} & y_{33} \end{pmatrix},
\end{equation}

where $y_{ij} = |y_{ij}| e^{i\phi_{ij}}$.

\item \textbf{Gauge Freedom:} Not all phases are physical. The $SU(3)$ color and $SU(2)_L$ weak isospin symmetries allow unitary rotations of the fermion basis, absorbing some phases. The number of physical phases is:

\begin{equation}
N_{\text{CP}} = 3 \cdot 3 - \dim[\text{unitary rotations}] = 9 - 6 = 3.
\end{equation}

These are: one phase in the up-sector, one in the down-sector, and one relative phase (the CKM-like phase).

\item \textbf{Bregman Divergence Constraint on Phases:} The generating functional $\Phi[\psi]$ in Axiom II is strictly convex but, being real-valued, does not directly constrain phases. However, the divergence Laplacian $-\Delta_\Phi$ (Theorem \ref{thm:quadraticFormProperties}) couples fermion generations through the Bregman Hessian. This Hessian, in the quantum field theory path integral, generates Yukawa couplings whose phases are constrained by:

\begin{enumerate}
\item[(a)] Consistency of the anomaly equations (Section \ref{sec:standardModelUniqueness}): The traces $\Tr(T_a \{T_b, T_c\})$ that appear in anomaly cancellation equations depend on the phase structure of the couplings.

\item[(b)] RG stability at the fixed point (Section \ref{sec:asymptoticSafety}): The beta functions for the Yukawa couplings must vanish at $g^*$, which constrains the phases.

\item[(c)] Unitarity of the CKM matrix: The Yukawa phases must combine to form a unitary quark mixing matrix, which is a strong constraint.
\end{enumerate}

\end{enumerate}

\begin{proof}

\textbf{Part 1: Phases from Effective Potential}

The effective potential $V_{\text{eff}}$ arising from quantum corrections includes terms from fermion loops:

\begin{equation}
V_{\text{eff}}^{(1)} = \sum_f \mathrm{Tr} \log\left[\frac{\partial^2}{\partial\psi^2} (Y_f H \bar{\psi} \psi)\right].
\end{equation}

The determinant of this Hessian (in Yukawa coupling space) is real. However, when it is required the Hessian to have a specific spectrum (matching the RG fixed point Hessian), the impose constraints on the Yukawa phases.

The condition that the eigenvalues of the Hessian (in coupling space) match the required values from the transversality theorem (\ref{thm:transversalityCompleteSixSurfaces}) is:

\begin{equation}
\det[\mathcal{H}(Y_f) - \lambda_i \mathbb{I}] = 0.
\end{equation}

For each eigenvalue $\lambda_i$, this imposes one constraint on the phases $\phi_{ij}$.

\textbf{Part 2: CP Phases and the CKM Matrix}

The Cabibbo-Kobayashi-Maskawa (CKM) matrix describes quark flavor mixing:

\begin{equation}
\begin{pmatrix} d' \\ s' \\ b' \end{pmatrix} = V_{\text{CKM}} \begin{pmatrix} d \\ s \\ b \end{pmatrix}.
\end{equation}

The CKM matrix is unitary and can be parameterized (Wolfenstein 1983) with three mixing angles and one CP-violating phase $\delta$:

\begin{equation}
V_{\text{CKM}} = \begin{pmatrix} c_{12} & s_{12} & 0 \\ -s_{12}c_{23} & c_{12}c_{23} & s_{23} \\ s_{12}s_{23}e^{-i\delta} & -c_{12}s_{23}e^{-i\delta} & c_{23} \end{pmatrix}.
\end{equation}

This single phase $\delta$ is one of the three physical CP phases mentioned above.

in the divergence-first framework, the requirement that the Yukawa matrix yield the physical CKM phase $\delta \neq 0$ emerges from the need to match the RG Hessian eigenvalue structure. A vanishing CP phase ($\delta = 0$) would require the Hessian to have special degeneracies, which the transversality theorem (which holds generically) prevents.

\textbf{Part 3: Strong CP Problem and the $\theta$ Parameter}

The QCD Lagrangian admits a CP-violating term:

\begin{equation}
\mathcal{L}_{\theta} = \theta \frac{g_3^2}{32\pi^2} F_\mu^\rho F_{\rho\nu} \star F^{\nu\mu},
\end{equation}

where $\theta$ is the ``axion angle'' or QCD $\theta$ parameter, and $\star F$ denotes the dual field strength.

Observationally, $|\theta| < 10^{-10}$ (from neutron EDM constraints). The strong CP problem asks: why is $\theta$ so small, even though the Lagrangian permits $\theta = O(1)$?

in the divergence-first framework, the $\theta$ parameter is determined by prior constraints but is related to the phases in the Yukawa coupling matrix. Specifically:

\begin{equation}
\theta = \arg[\det(Y_u) \det(Y_d)] \quad (\text{up to } 2\pi).
\end{equation}

This relationship arises from the chiral anomaly: the path integral measure for fermions is not invariant under $U(1)$ phase rotations, and the anomalous variation is proportional to $\arg[\det(Y)]$.

The phases $\arg[\det(Y_u)]$ and $\arg[\det(Y_d)]$ are constrained (modulo $2\pi$) by the divergence structure, as shown in Parts 1--2. If the constraints force $\det(Y_u) \approx \det(Y_d)$ in phase, then $\theta \approx 0$.

\textbf{Claim (to be verified by explicit calculation):} The RG fixed point condition, combined with unitarity of the CKM matrix, forces $\theta = 0$ at leading order. Higher-order quantum corrections contribute $\theta = O(\alpha_s^2)$, which is suppressed and matches observations.

\end{proof}

\end{theorem}

\begin{remark}[Axion as a Dynamical Solution]
\label{rem:axionasadynamicalsolution}

If the divergence structure permits a nonzero $\theta$, the strong CP problem is resolved by the axion (Peccei-Quinn 1977): a pseudoscalar field whose vacuum expectation value sets $\theta = 0$. Whether the divergence framework requires the axion is determined by explicit calculation of the lowest-energy state of the divergence functional in the presence of QCD interactions.

\end{remark}
