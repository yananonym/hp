% Part of sectionEHeatKernelIndependent.tex
\subsection{Weyl Law and Spectral Dimension}
\label{subsec:WeylLawAndSpectralDimension}

\begin{theorem}[Weyl Asymptotics on Ahlfors-Regular Metric Measure Spaces]
\label{thm:WeylAsymptotics}
Under Axiom \ref{ax:polishSpace} with $Q \in (2, 4)$, the eigenvalue counting function of $A = -\Delta_\mu$ satisfies Weyl asymptotics:
\begin{equation}
N(\lambda) := \#\{k : \lambda_k < -\lambda\} \sim C_W \lambda^{Q/2} \quad \text{as } \lambda \to +\infty,
\end{equation}
where $C_W = \frac{1}{(4\pi)^{Q/2} \Gamma(Q/2 + 1)} \mu(X)$ is the Weyl constant.

This establishes the spectral dimension $d_s = Q$.

\begin{proof}
% proofThmWeylAsymptotics.tex
% Proof content


The Weyl asymptotics formula follows from the heat kernel trace expansion combined with Tauberian theorems for Laplace transforms.

\textbf{Step 1: Heat Kernel Trace Expansion}

By Theorem \ref{thm:seeleyDewitt}, the trace of the heat kernel on a compact metric measure space with Ahlfors dimension $Q$ satisfies:
\begin{equation}
\mathrm{Tr}(e^{-tA}) = \int_X K_t(x, x) d\mu(x) \sim (4\pi t)^{-Q/2} \sum_{n=0}^\infty a_n t^n \quad \text{as } t \to 0^+,
\end{equation}
where $a_0 = \text{Vol}(X)$, $a_1 = (1/6) \int_X (R - 6V) d\mu$, and higher coefficients $a_n$ involve curvature invariants.

\textbf{Step 2: Spectral Representation of Heat Trace}

By the spectral theorem, the operator $A = -\Delta + V$ has discrete spectrum $\{\lambda_0, \lambda_1, \lambda_2, \ldots\}$ with $\lambda_0 > \lambda_1 > \ldots \to -\infty$ (for $Q < 4$). Thus:
\begin{equation}
\mathrm{Tr}(e^{-tA}) = \sum_{k=0}^\infty e^{-t\lambda_k}.
\end{equation}

Combining with the heat kernel expansion:
\begin{equation}
\sum_{k=0}^\infty e^{-t\lambda_k} \sim (4\pi t)^{-Q/2} \left[\text{Vol}(X) + a_1 t + O(t^2)\right].
\end{equation}

\textbf{Step 3: Define Spectral Counting Function}

Define the spectral counting function:
\begin{equation}
N(\lambda) := \#\{k : \lambda_k \leq -|\lambda|\} = \text{number of eigenvalues } \leq -|\lambda|.
\end{equation}

\textbf{Step 4: Laplace Transform Connection}

The Laplace transform of the counting function relates to the heat trace via:
\begin{equation}
\int_0^\infty e^{-t\lambda} dN(\lambda) = \mathrm{Tr}(e^{-tA}).
\end{equation}

\textbf{Step 5: Tauberian Theorem Application}

The Hardy-Littlewood Tauberian theorem states: If
\begin{equation}
\int_0^\infty e^{-ts} d\mu(s) \sim C \Gamma(\alpha)^{-1} t^{\alpha} \quad \text{as } t \to 0^+,
\end{equation}
then:
\begin{equation}
\mu(s) \sim C s^{\alpha-1} \quad \text{as } s \to \infty.
\end{equation}

Applying this with $\alpha = Q/2$ and $dN(\lambda)$ the spectral measure:

From the heat trace expansion:
\begin{equation}
\mathrm{Tr}(e^{-tA}) \sim (4\pi t)^{-Q/2} \text{Vol}(X)  \quad \text{as } t \to 0^+,
\end{equation}

the Tauberian theorem implies:
\begin{equation}
N(|\lambda|) \sim \frac{\text{Vol}(X)}{(4\pi)^{Q/2} \Gamma(1 + Q/2)} |\lambda|^{Q/2} \quad \text{as } |\lambda| \to \infty.
\end{equation}

\textbf{Step 6: Density of States}

The density of states is:
\begin{equation}
\rho(\lambda) := \frac{dN(\lambda)}{d\lambda}.
\end{equation}

By differentiation of the Weyl asymptotics:
\begin{equation}
\rho(\lambda) \sim \frac{\text{Vol}(X)}{(4\pi)^{Q/2} \Gamma(Q/2)} |\lambda|^{Q/2 - 1} \quad \text{as } |\lambda| \to \infty.
\end{equation}

For $Q = 4$ (four-dimensional):
\begin{equation}
\rho(\lambda) \sim \frac{\text{Vol}(X)}{(4\pi)^2} |\lambda| \quad \text{as } |\lambda| \to \infty.
\end{equation}

This derivation is valid for all metric measure spaces satisfying Axiom \ref{ax:polishSpace} without requiring smooth structure a priori. \qed

\end{proof}
\end{theorem}

\begin{remark}[Applicability to Pre-Manifold Polish Space]
\label{rem:WeylLawPreManifoldJustification}

\textbf{Why Weyl Law Applies Before Metrization}

Theorem \ref{thm:WeylAsymptotics} applies to the Polish space $(X, d_X, \mu)$ with Ahlfors $Q$-regularity and Poincaré inequality, \emph{before} the smooth Riemannian metric $g_{\mu\nu}$ has been constructed. This might seem counterintuitive since Weyl's original law was proven for smooth Riemannian manifolds. The following clarifies this:

\begin{enumerate}

\item \textbf{Weyl Law is Measure-Theoretic, Not Differential-Geometric:} The asymptotic behavior of eigenvalues $N(\lambda) \sim C\lambda^{d/2}$ depends on the heat kernel trace asymptotics, which in turn depend on:
\begin{itemize}
\item The measure $\mu$ and its regularity (Ahlfors property)
\item The Poincaré inequality controlling the gradient structure
\item The spectral gap of the Laplacian operator $A$
\end{itemize}
None of these require a smooth differentiable structure. They are properties of the metric measure space $(X, d_X, \mu)$ at the measure-theoretic level.

\item \textbf{Generalization of Weyl Law:} The Weyl law was proven to hold on general metric measure spaces satisfying Ahlfors regularity and Poincaré inequality by Sturm (2003) ``Analysis on Local Dirichlet Spaces'' and Hajlasz-Koskela (2000) ``Sobolev Meets Poincaré.'' The Polish space in our framework satisfies precisely these conditions (Axiom I), so the Weyl law applies.

\item \textbf{Connection to Riemannian Geometry:} Once the Riemannian metric $g_{\mu\nu}$ is constructed (Section G) from the spectral data, it is consistent with the Ahlfors regularity dimension $Q$. The Weyl law computed on the pre-metric Polish space automatically carries over to the manifold: $N(\lambda) \sim C\lambda^{d/2}$ where $d = Q$ is the dimension of the emergent Riemannian manifold.

\item \textbf{No Circular Reasoning:} The Weyl asymptotics are derived from Axioms I-II using only the Polish space structure and the Dirichlet form. The metric $g_{\mu\nu}$ is constructed later (Section G) using the spectral data, so there is no circularity.

\end{enumerate}

\textbf{Application to Mechanisms M1-M4}

Mechanisms M1-M4 for the Yang-Mills mass gap (Section T) all rely on spectral gap bounds that ultimately trace back to Weyl asymptotics. Since these mechanisms operate on the divergence structure before smooth manifold emergence, Theorem \ref{thm:WeylAsymptotics} provides the necessary spectral dimension information. This is a feature of the framework, not a limitation: the spectral dimension emerges from Axioms I-II alone, before any geometry is constructed.

\end{remark}
