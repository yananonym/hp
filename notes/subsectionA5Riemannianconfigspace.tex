% Part of sectionAAxiomaticFoundation.tex
\subsection{Riemannian Structure on Configuration Space (Subsection A.5)}
\label{subsec:riemannianConfigSpace}

\begin{definition}[Riemannian Metric on Configuration Space from Hessian]
\label{def:riemannianMetricConfig}

The Hessian operator $D^2\Phi[\psi]: \mathcal{H} \times \mathcal{H} \to \mathbb{R}$ from Axiom II defines a Riemannian metric on the Hilbert manifold $\mathcal{H}$ by:
\begin{equation}
g_\psi(h, k) := \langle D^2\Phi[\psi] h, k \rangle_{\mathcal{H}},
\end{equation}
where $h, k \in T_\psi \mathcal{H}$ are tangent vectors. By strict convexity (Axiom II.iv), this metric is positive definite.

\end{definition}

\begin{theorem}[Riemannian Hilbert Manifold Structure from Axioms]
\label{thm:riemannianManifoldStructure}

Under Axiom II, the configuration space $(\mathcal{H}, g)$ becomes a Riemannian Hilbert manifold. This enables the application of Riemannian geometry, including:
\begin{enumerate}
\item Exponential map and geodesics
\item Curvature tensor and Ricci curvature
\item Hodge decomposition of differential forms
\item Floer homology (defined via gradient flows with respect to $g$)
\end{enumerate}

This Riemannian structure is entirely determined by the axioms and is independent of any physical input.

\end{theorem}

