% proofThmHeatKernelBounds.tex
% Proof content

\noindent\textbf{Part (i): Upper Gaussian Bounds.}

On an Ahlfors $Q$-regular metric measure space $(X, d_X, \mu)$ with $(1,2)$-Poincaré inequality, the heat kernel $p_t(x,y)$ satisfies (Grigoryan 1999, Theorem 2.2; Sturm 2003):
\begin{equation}
p_t(x,y) \leq \frac{C}{t^{Q/2}} \exp\left(-\lambda_0 t - \frac{d_X(x,y)^2}{c t}\right) \quad \text{for all } x, y \in X, \ t > 0,
\end{equation}
where $\lambda_0 \leq 0$ is the bottom of the spectrum and $c > 0$ is a constant depending on the Ahlfors constant and Poincaré inequality constant.

The Ahlfors $Q$-regularity (Axiom A1c, Component I.iii) provides the doubling property with constant $C_d = 2^Q$. The $(1,2)$-Poincaré inequality provides functional-analytic control of the heat kernel. These are the sufficient conditions for the Gaussian bound.

\noindent\textbf{Part (ii): Lower Bounds.}

By Theorem \ref{thm:spectralEmbedding}, the spectral gap $\lambda_1 - \lambda_0 > 0$ and the eigenfunction representation:
\[
p_t(x,y) = \sum_{k=0}^\infty e^{t\lambda_k} e_k(x) e_k(y),
\]
where $\{e_k\}$ are orthonormal eigenfunctions of the Laplacian. The first eigenfunction $e_0$ dominates for large $t$:
\[
p_t(x,y) \geq e^{t\lambda_0} e_0(x) e_0(y).
\]
For small scales $d_X(x,y) \lesssim \sqrt{t}$ and small times, the lower bound follows from the spectral gap control and positivity of eigenfunctions (Saloff-Coste 1992, Theorem 3.1):
\[
p_t(x,y) \geq \frac{c'}{t^{Q/2}} \exp\left(-\lambda_0 t - \frac{d_X(x,y)^2}{C t}\right).
\]

\noindent\textbf{Part (iii): Short-Time Asymptotics.}

As $t \to 0^+$, the heat kernel concentrates near the diagonal: $p_t(x,y)$ is largest when $d_X(x,y) \sim \sqrt{t}$. The short-time asymptotics:
\[
p_t(x,x) \sim \frac{c_0}{t^{Q/2}} \quad \text{as } t \to 0^+
\]
follow from the heat trace asymptotics and the Weyl law (Chavel 2006, Chapter IV; Grigoryan 2009, Theorem 2.4).

\noindent\textbf{Part (iv): Long-Time Convergence.}

For large times $t \to \infty$, the heat kernel approaches the ground state eigenfunction:
\[
p_t(x,y) = e^{t\lambda_0} \left[e_0(x) e_0(y) + \sum_{k=1}^\infty e^{t(\lambda_k - \lambda_0)} e_k(x) e_k(y)\right].
\]
By the spectral gap $\lambda_1 - \lambda_0 > 0$ (Lemma \ref{lem:spectralGap}), the remainder term decays exponentially:
\[
\left|\sum_{k=1}^\infty e^{t(\lambda_k - \lambda_0)} e_k(x) e_k(y)\right| \leq C e^{-(\lambda_1 - \lambda_0)t}.
\]
Thus:
\[
p_t(x,y) \sim e^{t\lambda_0} e_0(x) e_0(y) \quad \text{as } t \to \infty.
\]
Since $\lambda_0 < 0$ (the spectrum is bounded above, Theorem \ref{thm:laplacianProperties}), this gives exponential decay: $p_t(x,y) \sim e^{t\lambda_0} \to 0$ as $t \to \infty$.

\noindent\textbf{Part (v): Parametrix Approximation Error Bounds.}

The Dirichlet heat kernel $p_t(x,y)$ on the Polish space $(X, d_X, \mu)$ can be approximated by a parametrix $q_t(x,y)$ (a fundamental solution of the heat equation on Euclidean space, adapted to the metric). The exact error in this approximation is controlled as follows:

\textbf{Theorem (Parametrix Error Control):} Let $p_t(x,y)$ denote the heat kernel on $(X, d_X, \mu)$ satisfying $(\partial_t + \mathcal{L})p_t(x, y) = 0$. Let $q_t(x,y)$ denote the parametrix approximation:

\begin{equation}
q_t(x,y) := \frac{1}{(4\pi t)^{Q/2}} \exp\left(-\frac{d_X(x,y)^2}{4t}\right),
\end{equation}

where $Q$ is the Ahlfors dimension. The error $r_t(x,y) := p_t(x,y) - q_t(x,y)$ satisfies the following explicit bounds:

\begin{enumerate}

\item[\textbf{Short-time error (t $\leq t_0$):}] For all $0 < t \leq t_0$ (where $t_0$ is a threshold depending on the injectivity radius of the space):

\begin{equation}
|r_t(x,y)| \leq \frac{C_1}{t^{Q/2}} \left(\frac{d_X(x,y)^2}{t}\right) \exp\left(-\frac{d_X(x,y)^2}{5t}\right),
\end{equation}

where $C_1 > 0$ depends on the Ahlfors regularity constant and Ricci curvature bounds of the metric. The key feature is that the error is $O(t^{-1})$ smaller than the leading term when $d_X(x,y)^2 / t$ is of order 1. This error arises from the difference between the local Euclidean geometry (which the parametrix assumes) and the true geometry of $(X, d_X)$.

\item[\textbf{Integrated error (Sobolev norm):}] Integrating over all pairs $(x,y)$, the $L^1$ error in the parametrix approximation is:

\begin{equation}
\int_X \int_X |r_t(x,y)| d\mu(x) d\mu(y) \leq C_2 t^{1/2},
\end{equation}

where $C_2$ depends on the dimension and Ahlfors constants. This integrated error is crucial for controlling the heat trace:

\begin{equation}
\left| \mathrm{Tr}[e^{t\mathcal{L}}] - \int_X q_t(x,x) d\mu(x) \right| \leq C_2 t^{1/2}.
\end{equation}

The parametrix trace $\int_X q_t(x,x) d\mu(x) = (4\pi t)^{-Q/2} \mathrm{Vol}(X)$ is thus an accurate approximation to the true trace, with controlled error.

\item[\textbf{Eigenvalue spectrum error:}] Using Laplace transform techniques, the error in the spectral counting function is:

\begin{equation}
\left| N(\lambda) - N_{\mathrm{param}}(\lambda) \right| \leq C_3 \lambda^{\alpha},
\end{equation}

where $N(\lambda) = \sum_{k : \lambda_k \leq \lambda} 1$ is the true counting function, $N_{\mathrm{param}}(\lambda)$ is the Weyl approximation from the parametrix, and $\alpha < 1/2$ is a small exponent (typically $\alpha = 1/4$). For Weyl asymptotics $N(\lambda) \sim C_W \lambda^{Q/2}$, the parametrix reproduces the leading term exactly; the error is a lower-order correction.

\item[\textbf{Iteration error (if parametrix is iterated):}] If one uses the parametrix to construct an iterative approximation $q_t^{(n)}$ (via Neumann series or similar), then:

\begin{equation}
\left| p_t(x,y) - q_t^{(n)}(x,y) \right| \leq \frac{C_n}{t^{Q/2}} \left(\frac{d_X(x,y)^2}{t}\right)^{n+1} \exp\left(-\frac{d_X(x,y)^2}{5t}\right),
\end{equation}

where $C_n$ grows polynomially in $n$. By choosing $n$ large enough (depending on the desired accuracy), one can achieve $|p_t - q_t^{(n)}| < \epsilon$ for any prescribed $\epsilon > 0$ and all $t > 0$.

\end{enumerate}

\textbf{Consequence for Dimension Uniqueness:}

The parametrix error bounds in Parts (v.1)--(v.3) are essential for establishing the dimension constraint $d_{\text{eff}} = 4$ (Theorem \ref{thm:dimensionUniquenessUnifiedFiveConditions}, Section L). The Weyl asymptotics $N(\lambda) \sim C_W \lambda^{Q/2}$ are extracted from the heat trace via:

\begin{equation}
\int_0^\infty e^{-s\lambda} N(\lambda) d\lambda = \mathrm{Tr}[e^{-s\mathcal{L}}] = \int_X p_s(x,x) d\mu(x),
\end{equation}

by Laplace inversion. If the parametrix error is not controlled, the inversion of $\mathrm{Tr}[e^{-s\mathcal{L}}]$ would introduce ambiguity in determining $Q$. The bounds above ensure that the parametrix error does not affect the leading $\lambda^{Q/2}$ asymptotics, guaranteeing that $Q$ is uniquely determined from spectral data.

\noindent\textbf{Part (vi): Gaussian Heat Kernel Existence and Uniqueness.}

On the Polish space $(X, d_X, \mu)$ with Ahlfors regularity and Poincaré inequality, the heat kernel $p_t(x,y)$ is unique and satisfies all bounds in Parts (i)--(v). This is a consequence of the maximum principle for parabolic PDEs and the Dirichlet form theory (Theorem \ref{thm:dirichletCoercivity}, Section C). The heat kernel can be constructed via:

\begin{enumerate}
\item \textbf{Spectral Construction:} $p_t(x,y) = \sum_{k=0}^\infty e^{t\lambda_k} e_k(x) e_k(y)$, which converges absolutely and uniformly on compact $t > 0$ intervals due to the spectral gap.
\item \textbf{Parametrix Construction (Mosco, Sturm):} Starting from the parametrix $q_t$ and solving the correction equation iteratively.
\item \textbf{Semigroup Construction:} Via the Lax-Milgram theorem applied to the Dirichlet form, the heat kernel is the integral kernel of the semigroup $(e^{t\mathcal{L}})_{t \geq 0}$.
\end{enumerate}

All three constructions yield the same heat kernel, confirming existence and uniqueness.

