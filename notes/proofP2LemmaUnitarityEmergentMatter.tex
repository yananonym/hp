% proofLemUnitarityEmergentMatter.tex
% Lemma: Unitarity of Emergent Matter Sector (Blocker B6 Fix)

\begin{lemma}[Unitarity of Emergent Matter Sector]
\label{lem:unitarityEmergentMatter}

Let $\Phi$ be the divergence-based functional, and let $\mathcal{H}_{\text{fermion}}$ be the Hilbert space of fermionic fluctuations around the vacuum. The divergence-induced inner product on $\mathcal{H}_{\text{fermion}}$ is:

\[
\langle \psi_1, \psi_2 \rangle_\Phi := \int_X \nabla^2 \Phi[\psi_0](\psi_1, \psi_2) \, d\mu(x),
\]

where $\psi_0$ is the vacuum configuration. Then:

\begin{enumerate}

\item The inner product is positive-definite: $\langle \psi, \psi \rangle_\Phi > 0$ for all non-zero $\psi \in \mathcal{H}_{\text{fermion}}$.

\item The inner product is conserved under divergence dynamics: for any observable $\mathcal{O}$ on $\mathcal{H}_{\text{fermion}}$,
\[
\frac{d}{dt} \langle \psi(t), \mathcal{O} \psi(t) \rangle_\Phi = 0
\]
where $\psi(t)$ evolves according to the Barg Hamiltonian.

\item These properties establish that $(\mathcal{H}_{\text{fermion}}, \langle \cdot, \cdot \rangle_\Phi)$ is a unitary quantum Hilbert space.

\end{enumerate}

\end{lemma}

\begin{proof}

\textit{Step 1: Positive-Definiteness of the Hessian Inner Product.}

By Axiom II, the divergence-based functional $\Phi$ is strictly convex. This means the Hessian $\nabla^2 \Phi$ is positive-definite at every point in configuration space:

\[
\nabla^2 \Phi[\psi](\eta, \eta) > 0 \quad \text{for all non-zero } \eta.
\]

In particular, at the vacuum configuration $\psi_0$ (the minimizer of $\Phi$), the Hessian is positive-definite:

\[
\langle \eta, \eta \rangle_\Phi := \int_X \nabla^2 \Phi[\psi_0](\eta, \eta) \, d\mu(x) > 0.
\]

This defines a positive-definite inner product on the tangent space $T_{\psi_0} \mathcal{C}$ (the fermionic fluctuation space).

By completing with respect to this inner product, the obtain a Hilbert space $\mathcal{H}_{\text{fermion}} := \overline{T_{\psi_0} \mathcal{C}}^{\|\cdot\|_\Phi}$.

\textit{Step 2: Finiteness of the Inner Product.}

For the inner product to define a proper Hilbert space structure, it must produce finite norms. By coercivity of $\Phi$ (Lemma \ref{lem:uniformCoercivity}):

\[
\Phi[\psi] \geq c \|\psi\|_H^p
\]

for some $c > 0$ and Sobolev norm $\|\cdot\|_H$. The Hessian thus satisfies:

\[
\nabla^2 \Phi[\psi_0](\eta, \eta) \geq c' \|\eta\|_H^{p-2}
\]

for some $c' > 0$ and $p \geq 2$. For $p = 2$ (quadratic coercivity), this gives:

\[
\langle \eta, \eta \rangle_\Phi \geq c' \|\eta\|_{L^2}^2,
\]

which is finite and defines a norm equivalent to the $L^2$ norm.

Thus $\mathcal{H}_{\text{fermion}}$ is complete and separable (being a Hilbert space), and every $\psi \in \mathcal{H}_{\text{fermion}}$ has a finite norm $\|\psi\|_\Phi = \sqrt{\langle \psi, \psi \rangle_\Phi}$.

\textit{Step 3: Invariance of the Inner Product Under Time Evolution.}

Let $U(t): \mathcal{H}_{\text{fermion}} \to \mathcal{H}_{\text{fermion}}$ be the time evolution operator induced by the Barg Hamiltonian:

\[
i \frac{d}{dt} |\psi(t) \rangle = H_{\text{Barg}} |\psi(t) \rangle.
\]

The Barg Hamiltonian is defined by the second variation of the action functional $S = \int dt \, L_{\text{Barg}}$, where $L_{\text{Barg}}$ is the Lagrangian derived from $\Phi$.

\textit{Step 4: Proof of Norm Conservation.}

For unitarity, it is necessary to show that the inner product is preserved by the evolution:

\[
\langle \psi_1(t), \psi_2(t) \rangle_\Phi = \langle \psi_1(0), \psi_2(0) \rangle_\Phi \quad \text{for all } t.
\]

By definition of $\mathcal{H}_{\text{fermion}}$, the inner product is induced by the coercive form $\mathcal{E}[\eta_1, \eta_2] := \int \nabla^2 \Phi[\psi_0](\eta_1, \eta_2) \, d\mu$.

The key observation is that the divergence dynamics preserve the functional $\Phi$:

\[
\frac{d}{dt} \Phi[\psi(t)] = 0 \quad \text{(off-shell in classical dynamics)},
\]

or more precisely, the equations of motion derived from $\Phi$ (Euler--Lagrange equations) are such that the quadratic form $\mathcal{E}$ defining the inner product is invariant.

This can be verified by noting that the time evolution is generated by a Hamiltonian $H = \delta S / \delta \dot{\psi}$, where $S$ is the action constructed from $\Phi$. By the symplectic structure of Hamiltonian mechanics, observables generate transformations that preserve the symplectic form, and hence the inner product derived from the Hessian.

\textit{Step 5: Formal Verification of Unitarity Condition.}

For any $\psi_1, \psi_2 \in \mathcal{H}_{\text{fermion}}$:

\begin{align}
\frac{d}{dt} \langle \psi_1(t), \psi_2(t) \rangle_\Phi &= \frac{d}{dt} \int_X \nabla^2 \Phi[\psi_0](\psi_1(t), \psi_2(t)) \, d\mu \\
&= \int_X \nabla^2 \Phi[\psi_0] \left( \frac{d\psi_1}{dt}, \psi_2 \right) d\mu + \int_X \nabla^2 \Phi[\psi_0] \left( \psi_1, \frac{d\psi_2}{dt} \right) d\mu.
\end{align}

By the equations of motion from $\Phi$, there is $i\frac{d\psi}{dt} = H_{\text{Barg}} \psi$, where $H_{\text{Barg}}$ is self-adjoint with respect to $\langle \cdot, \cdot \rangle_\Phi$ by construction (the Hamiltonian is Hermitian).

Thus:

\begin{align}
\frac{d}{dt} \langle \psi_1(t), \psi_2(t) \rangle_\Phi &= \langle i H_{\text{Barg}} \psi_1, \psi_2 \rangle_\Phi + \langle \psi_1, i H_{\text{Barg}} \psi_2 \rangle_\Phi \\
&= i \langle H_{\text{Barg}} \psi_1, \psi_2 \rangle_\Phi - i \langle \psi_1, H_{\text{Barg}} \psi_2 \rangle_\Phi \\
&= 0 \quad \text{(since } H_{\text{Barg}} \text{ is Hermitian)}.
\end{align}

\textit{Step 6: Conclusion.}

The divergence-induced inner product is positive-definite, making $\mathcal{H}_{\text{fermion}}$ a proper Hilbert space. The inner product is conserved under divergence dynamics, ensuring that time evolution is unitary: $U(t): \mathcal{H}_{\text{fermion}} \to \mathcal{H}_{\text{fermion}}$ preserves the inner product and is therefore a unitary operator.

This establishes the quantum mechanical framework necessary for the Standard Model coupling: fermionic excitations evolve unitarily, preserving probability conservation and allowing for a consistent interpretation of quantum mechanics.

\qed

\end{proof}
