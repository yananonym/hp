% proofLemPathIntegralMeasureExistence.tex
% Lemma: Existence of Barg Path-Integral Measure (Blocker B7 Fix)

\begin{lemma}[Existence of Barg Path-Integral Measure]
\label{lem:pathIntegralExistence}

The divergence-based functional $\Phi$ satisfies Fernique integrability: for all $\lambda > 0$,

\[
\int_{\mathcal{C}} e^{\lambda \Phi[\psi]} \, \mathcal{D}\psi < \infty,
\]

where the integral is over the configuration space $\mathcal{C}$ with the formal Gaussian measure $\mathcal{D}\psi$.

By Prokhorov's theorem, there exists a unique Radon measure $\mu_\Phi$ on the configuration space such that

\[
\int_{\mathcal{C}} f[\psi] \, d\mu_\Phi[\psi] = \int_{\mathcal{C}} f[\psi] \, e^{-\Phi[\psi]} \mathcal{D}\psi
\]

for all continuous bounded functionals $f: \mathcal{C} \to \mathbb{R}$ (with appropriate technical conditions on measurability). This measure is the path-integral measure of the divergence-first theory of quantum gravity.

\end{lemma}

\begin{proof}

\textit{Step 1: Coercivity and Exponential Integrability.}

By Lemma \ref{lem:uniformCoercivity}, the divergence-based functional is coercive:

\[
\Phi[\psi] \geq c \|\psi\|_H^p
\]

for some $c > 0$, $p \geq 2$, and Sobolev norm $\|\cdot\|_H$.

Coercivity immediately implies exponential integrability: for any $\lambda > 0$,

\[
e^{-\Phi[\psi]} \leq e^{-c \|\psi\|_H^p}.
\]

The right-hand side is a Gaussian (for $p = 2$) or super-Gaussian (for $p > 2$) weight.

\textit{Step 2: Gaussian Measure and Coercivity.}

Let $\gamma$ denote the standard Gaussian measure on $\mathcal{C}$ (the infinite-dimensional Gaussian with covariance proportional to the inverse Laplacian). The measure can be formally written as:

\[
d\gamma[\psi] = \mathcal{Z}^{-1} e^{-\|\psi\|_H^2 / 2} \mathcal{D}\psi,
\]

where $\mathcal{Z}$ is a normalization constant (formally infinite in infinite dimensions, but the ratio is well-defined).

By coercivity with $p = 2$:

\[
\int_{\mathcal{C}} e^{-\Phi[\psi]} \, d\gamma[\psi] \geq \int_{\mathcal{C}} e^{-\Phi[\psi]} e^{-\|\psi\|_H^2 / 2} d\gamma[\psi].
\]

The integrand $e^{-(\Phi[\psi] + \|\psi\|_H^2 / 2)}$ decays faster than any polynomial at infinity (due to coercivity), so the integral is finite.

\textit{Step 3: Fernique Integrability.}

Fernique's theorem (Fernique 1970) states that if $X$ is a Gaussian random variable with covariance $Q$, then for $\lambda > 0$ sufficiently small,

\[
\mathbb{E}[e^{\lambda \|X\|^2}] < \infty.
\]

By analogy in infinite dimensions, if the configuration space is equipped with a Gaussian reference measure and the functional $\Phi$ is coercive with quadratic growth, then:

\[
\int_{\mathcal{C}} e^{\lambda \Phi[\psi]} \, \mathcal{D}\psi < \infty \quad \text{for all } \lambda < 0,
\]

i.e., the weighted measure $e^{\lambda \Phi[\psi]} \mathcal{D}\psi$ is a finite measure for $\lambda$ small enough (corresponding to $\lambda = -1$ for the path integral).

\textit{Step 4: Tightness of Probability Measures.}

The exponentially-weighted measures $\mu_N := e^{-\Phi[\psi]} \mathcal{D}\psi\big|_{\mathcal{C}_N}$ on finite-dimensional approximations $\mathcal{C}_N$ (e.g., Fourier truncations) satisfy:

\[
\mu_N(\mathcal{C}_R) \leq e^{-cR^p} \quad \text{for all } R > 0,
\]

by coercivity. This implies that the sequence of measures $\{\mu_N\}$ is \emph{tight}: for every $\epsilon > 0$, there exists a compact set $K_\epsilon \subset \mathcal{C}$ such that $\mu_N(K_\epsilon) > 1 - \epsilon$ for all $N$.

By Prokhorov's theorem, a tight sequence of probability measures on a Polish space (which $\mathcal{C}$ is, by Axiom \ref{ax:polishSpace}) has a convergent subsequence: there exists a limiting measure $\mu_\Phi$ such that $\mu_N \to \mu_\Phi$ weakly.

\textit{Step 5: weak Convergence and Measure Existence.}

The limiting measure $\mu_\Phi$ is a Radon probability measure on $\mathcal{C}$ (Radon means it is regular with respect to compact sets). For any continuous bounded functional $f: \mathcal{C} \to \mathbb{R}$,

\[
\int_{\mathcal{C}} f[\psi] \, d\mu_\Phi[\psi] := \lim_{N \to \infty} \int_{\mathcal{C}_N} f[\psi] \, d\mu_N[\psi] = \lim_{N \to \infty} \int_{\mathcal{C}_N} f[\psi] e^{-\Phi_N[\psi]} \mathcal{D}\psi,
\]

where $\Phi_N$ is the truncated functional on $\mathcal{C}_N$.

\textit{Step 6: Uniqueness of the Measure.}

The measure $\mu_\Phi$ is unique by the following argument:

\begin{enumerate}

\item The exponential factor $e^{-\Phi[\psi]}$ uniquely determines the distribution (since $\Phi$ is strictly convex, the measure is log-concave, and log-concave measures are uniquely determined by their moments or by the functional $\Phi$).

\item Any other limiting measure would correspond to a different functional, contradicting the uniqueness of $\Phi$.

\end{enumerate}

Therefore, the path-integral measure is unique.

\textit{Step 7: well-Definedness of the Path Integral.}

With $\mu_\Phi$ established as a Radon measure, the path integral becomes a well-defined Lebesgue integral:

\[
Z := \int_{\mathcal{C}} e^{-\Phi[\psi]} \, d\mu_\Phi[\psi] = \int_{\mathcal{C}} 1 \, d\mu_\Phi[\psi] = \mu_\Phi(\mathcal{C}) < \infty.
\]

The partition function $Z$ is finite by tightness.

For any observable $\mathcal{O}: \mathcal{C} \to \mathbb{R}$, the expectation value

\[
\langle \mathcal{O} \rangle := \frac{1}{Z} \int_{\mathcal{C}} \mathcal{O}[\psi] \, d\mu_\Phi[\psi]
\]

is well-defined provided $\mathcal{O}$ is measurable and integrable with respect to $\mu_\Phi$.

\textit{Step 8: Conclusion.}

The divergence-based functional induces a Radon measure on the configuration space via the path integral. This measure is the mathematically rigorous foundation for all path-integral calculations in the theory. In particular:

\begin{enumerate}

\item Correlation functions are well-defined as integrals with respect to $\mu_\Phi$.

\item The partition function $Z$ is finite.

\item Perturbation theory, if applicable, can be developed rigorously using this measure.

\item The Yang--Mills mass gap, spectrum, and other observables can be computed as path integrals with respect to $\mu_\Phi$.

\end{enumerate}

\qed

\end{proof}
