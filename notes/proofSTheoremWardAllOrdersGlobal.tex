% proofThmWardAllOrdersGlobal.tex
% Proof content


The following derivation establishes that the Ward identities hold to all orders in the renormalized perturbation theory via BRST cohomology.

\textbf{Step 1: One-Loop Anomaly Cancellation.}

By Lemma \ref{lem:anomalyCoefficients}, the six independent triangle anomalies of the Standard Model $SU(3)_c \times SU(2)_L \times U(1)_Y$ all vanish at one-loop order. These anomalies are computed from the one-loop triangle diagram with three external gauge boson legs.

\textbf{Step 2: BRST Symmetry and Quantum Consistency.}

The Ward identities are embodied in the BRST symmetry (Becchi-Rouet-Stora-Tyutin), which is a nilpotent fermionic gauge transformation that leaves the generating functional invariant. Define the BRST operator $\mathcal{Q}$ acting on fields and ghosts:
\begin{align}
\mathcal{Q} A_\mu^a &= \partial_\mu c^a + [c, A_\mu]^a, \\
\mathcal{Q} c^a &= -\frac{1}{2} [c, c]^a, \\
\mathcal{Q} \bar{c}^a &= B^a.
\end{align}

The key properties are:
\begin{enumerate}
\item $\mathcal{Q}^2 = 0$ (nilpotency).
\item The path integral measure is BRST invariant.
\item Physical states lie in the BRST cohomology $H^0(\mathcal{Q})$, i.e., $\mathcal{Q}|\psi\rangle = 0$ modulo exact terms.
\end{enumerate}

\textbf{Step 3: \cite{adlerBardeen1969} Non-Renormalization Theorem.}

The \cite{adlerBardeen1969} theorem (1969) states: If the chiral anomaly vanishes at one-loop, then it vanishes at all loop orders. The proof uses the fact that the divergence of the axial current receives contributions only from triangle diagrams at one-loop, and higher-loop corrections can be absorbed into field redefinitions and remain BRST-exact.

More formally, consider the generating functional of Green's functions:
$$W[J, \eta, \bar{\eta}] := \int \mathcal{D}[\text{fields}] \exp(i S[\text{fields}] + J \cdot \text{fields} + \ldots).$$

The Ward identities state:
$$\mathcal{Q} W = 0 \quad \text{(BRST-invariance of the path integral)}.$$

At each loop order $n$, the one-particle-irreducible (1PI) effective action $\Gamma^{(n)}$ receives loop contributions from $n$-loop Feynman diagrams. The BRST cohomology argument shows:

\begin{equation}
[\Gamma^{(n)}]_{\text{anomalous}} = \mathcal{Q} \Sigma^{(n)} + \text{(exact terms in } \mathcal{Q}\text{-cohomology)},
\end{equation}

where the anomalous part (non-BRST-invariant contribution) is cohomologically trivial if the one-loop anomaly vanishes.

\textbf{Step 4: All-Orders Preservation of Gauge Invariance.}

Since the one-loop anomaly coefficients all vanish (Lemma \ref{lem:anomalyCoefficients}), by \cite{adlerBardeen1969}:
\begin{enumerate}
\item The divergence of the axial current satisfies: $\partial_\mu j^\mu_5 = 0$ to all orders.
\item The three-point correlators $\langle T[j_\mu(x) j_\nu(y) j_\rho(z)]\rangle$ remain finite and consistent with the gauge structure.
\item Ghost contributions and counter-term structures maintain BRST cohomology without anomalies.
\end{enumerate}

\textbf{Step 5: Perturbative Consistency to All Orders.}

By induction on loop order:
\begin{enumerate}
\item Assume Ward identities hold to order $\hbar^{n-1}$.
\item At order $\hbar^n$, new divergences arise from $n$-loop diagrams. These are renormalized via counter-terms.
\item Counter-terms are chosen such that the resulting $n$-loop effective action remains BRST-invariant.
\item This is always possible (Kugo-Ojima theorem, Kugo 1987) if the one-loop anomaly vanishes.
\end{enumerate}

Therefore, the Standard Model Ward identities are preserved to all orders in perturbation theory, ensuring that:
\begin{itemize}
\item Gauge invariance is maintained at the quantum level.
\item Unitarity of the $S$-matrix is guaranteed.
\item Renormalization can be performed consistently order-by-order.
\end{itemize}

\qed
