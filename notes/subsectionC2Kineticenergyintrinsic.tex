% Part of sectionCDirichletFormTheory.tex
\subsection{Kinetic Energy as Intrinsic Geometric Property of Polish Spaces}
\label{subsec:kineticEnergyIntrinsic}

\begin{theorem}[Kinetic Energy Structure from Metric Measure Space - Intrinsic to Axiom I]
\label{thm:kineticEnergyIntrinsicToMetric}

The kinetic energy structure of the Dirichlet form is \textbf{not} a hidden axiom. It emerges as a canonical geometric invariant from the Polish space metric-measure structure (Axiom \ref{ax:polishSpace}) alone.

\begin{enumerate}[label=(\roman*)]

\item \textbf{Minimal Upper Gradient is Intrinsic:} On any metric measure space $(X, d, \mu)$ satisfying Ahlfors $Q$-regularity and Poincaré inequality (Axiom I), the minimal upper gradient $\nabla_{\min}$ is uniquely defined without external axiomatization. It depends only on the metric structure $(X, d, \mu)$.

\item \textbf{Kinetic Energy Form:} The quadratic form:
\begin{equation}
\mathcal{E}_{\text{kin}}[\psi] := \int_X |\nabla_{\min} \psi|^2 \, d\mu
\end{equation}
is a \textbf{canonical geometric invariant} arising directly from the metric structure. This form does \textbf{NOT} require external axiomatization (no ``Axiom III'').

\item \textbf{Foundation in Metric Theory:} By Cheeger's 1999 metric Sobolev space theory and subsequent developments (Hajłasz, Koskela, Shanmugalingam), the Sobolev space $H^{1,2}(X)$ is well-defined on metric measure spaces via:
\begin{equation}
\|f\|_{H^{1,2}}^2 := \|f\|_{L^2}^2 + \int_X |\nabla_{\min} f|^2 \, d\mu.
\end{equation}
This construction is intrinsic to metric geometry and requires no external structure.

\item \textbf{Coercivity for $Q < 4$:} For Ahlfors $Q$-regular spaces with Poincaré inequality, the kinetic energy form is coercive when $Q < 4$. Specifically, there exists $\lambda_0 > 0$ such that:
\begin{equation}
\mathcal{E}_{\text{kin}}[\psi] \geq \lambda_0 \|\psi\|_{H^{1,2}}^2.
\end{equation}
This bound depends only on the Ahlfors dimension $Q$ and the Poincaré constant (both intrinsic to the Polish space).

\item \textbf{Logical Acyclicity:} The emergence of kinetic structure is acyclic:
\begin{center}
Polish space $(X, d, \mu)$ [Axiom I]
$\to$ Minimal upper gradient $\nabla_{\min}$
$\to$ Kinetic form $\mathcal{E}_{\text{kin}}[\psi]$
$\to$ Sobolev space $H^{1,2}(X)$
$\to$ [No circularity]
\end{center}

The Bregman divergence quadratic form (Axiom II) is added \textit{independently} to yield the complete Dirichlet form. There is no hidden axiomatization and no circular dependency.

\end{enumerate}

\begin{proof}
% proofCTheoremKineticEnergyIntrinsic.tex
% Proof: Kinetic Energy Structure from Metric Measure Space - Intrinsic to Axiom I

This proof establishes that the kinetic energy structure of the Dirichlet form emerges as a canonical geometric invariant from the Polish space metric-measure structure (Axiom I) and requires only external axiomatization.

\noindent\textbf{Part (i): Minimal Upper Gradient is Intrinsic}

The minimal upper gradient on a metric measure space $(X, d, \mu)$ is defined purely in terms of the metric and measure structure:

\begin{definition}[Minimal Upper Gradient - Intrinsic Definition]
A function $g: X \to [0, \infty]$ is an \textit{upper gradient} of $f: X \to \mathbb{R}$ if for every rectifiable curve $\gamma$ connecting points $x$ and $y$:
\begin{equation}
|f(x) - f(y)| \leq \int_\gamma g \, ds,
\end{equation}
where the integral is over arc length on the rectifiable curve.

The \textit{minimal upper gradient} $\nabla_{\min} f$ is the pointwise infimum of all upper gradients of $f$. This is \textbf{uniquely defined} (up to $\mu$-null sets) for any function in $L^1(X)$ and depends only on $(X, d, \mu)$.
\end{definition}

\noindent\textbf{Key Property:} The minimal upper gradient is \textbf{intrinsic} to the metric measure space. It requires no external structure, no smoothness assumption, and no axiomatization beyond the metric $d$ and measure $\mu$.

\vspace{0.3cm}

\noindent\textbf{Part (ii): Kinetic Energy Form is a Canonical Invariant}

On an Ahlfors $Q$-regular metric measure space $(X, d, \mu)$ with Poincaré inequality, the kinetic energy form:
\begin{equation}
\mathcal{E}_{\text{kin}}[\psi] := \int_X |\nabla_{\min} \psi|^2 \, d\mu
\end{equation}
is a \textbf{canonical geometric invariant}. This means:

\begin{enumerate}
\item The form is \textbf{canonically defined}: Every such metric measure space has a unique minimal upper gradient, hence a unique kinetic energy form.
\item The form is \textbf{geometrically natural}: It arises from the metric structure without additional choices.
\item The form is \textbf{not externally imposed}: It is not an independent axiom but a consequence of the metric structure.
\end{enumerate}

\noindent\textbf{Part (iii): Foundation in Metric Sobolev Space Theory}

Cheeger's 1999 seminal work (``Differentiability of Lipschitz functions on metric measure spaces,'' Geom. Funct. Anal.) established the theory of Sobolev spaces on metric measure spaces. The subsequent development by Hajłasz, Koskela, Shanmugalingam, and others provides:

\begin{theorem}[Metric Sobolev Spaces - Cheeger and Successors]
On a metric measure space $(X, d, \mu)$ satisfying Ahlfors $Q$-regularity and Poincaré inequality, the Sobolev space:
\begin{equation}
H^{1,2}(X) := \left\{f \in L^2(X) : \exists g \in L^2(X), \, |f(x) - f(y)| \leq \int_\gamma g \, ds \text{ for all curves } \gamma\right\}
\end{equation}
is well-defined with norm:
\begin{equation}
\|f\|_{H^{1,2}}^2 := \|f\|_{L^2}^2 + \int_X |\nabla_{\min} f|^2 \, d\mu.
\end{equation}

This definition is:
\begin{enumerate}
\item \textbf{Intrinsic:} Depends only on $(X, d, \mu)$, no external structure.
\item \textbf{Canonical:} The minimal upper gradient is uniquely determined.
\item \textbf{Complete:} $H^{1,2}(X)$ is a Banach space under the norm above.
\item \textbf{Geometric:} Recovers the classical Sobolev space on smooth Riemannian manifolds.
\end{theorem}

\noindent\textbf{Proof Sketch:} The completeness of $H^{1,2}(X)$ follows from:
\begin{enumerate}
\item Cauchy sequences in the norm converge in both $L^2$ and in gradient (minimal upper gradient of limit is the limit of gradients, a.e.).
\item The Poincaré inequality (from Axiom I) ensures that the form is continuous and coercive.
\item Cheeger's theory shows that the minimal upper gradient is the correct generalization of weak gradient to metric spaces.
\end{enumerate}

\vspace{0.3cm}

\noindent\textbf{Part (iv): Coercivity for $Q < 4$}

For an Ahlfors $Q$-regular metric measure space with Poincaré inequality (Axiom I), the kinetic energy form satisfies:

\begin{lemma}[Coercivity of Kinetic Energy Form]
There exists $\lambda_0 > 0$ (depending on $Q$ and the Poincaré constant) such that:
\begin{equation}
\int_X |\nabla_{\min} \psi|^2 \, d\mu \geq \lambda_0 \|\psi\|_{H^{1,2}}^2
\end{equation}
for all $\psi \in H^{1,2}(X)$, provided $Q < 4$.
\end{lemma}

\noindent\textbf{Proof of Coercivity:}

For $\psi \in H^{1,2}(X)$, there is:
\begin{equation}
\|\psi\|_{H^{1,2}}^2 = \|\psi\|_{L^2}^2 + \int_X |\nabla_{\min} \psi|^2 \, d\mu.
\end{equation}

By the Poincaré inequality (Axiom I(c)), there exists $C_P > 0$ such that:
\begin{equation}
\|\psi\|_{L^2}^2 \leq C_P \int_X |\nabla_{\min} \psi|^2 \, d\mu.
\end{equation}

Therefore:
\begin{equation}
\|\psi\|_{H^{1,2}}^2 = \|\psi\|_{L^2}^2 + \int_X |\nabla_{\min} \psi|^2 \, d\mu \leq (C_P + 1) \int_X |\nabla_{\min} \psi|^2 \, d\mu.
\end{equation}

This gives:
\begin{equation}
\int_X |\nabla_{\min} \psi|^2 \, d\mu \geq \frac{1}{C_P + 1} \|\psi\|_{H^{1,2}}^2.
\end{equation}

Setting $\lambda_0 := \frac{1}{C_P + 1}$ establishes coercivity.

\noindent\textbf{Remark on $Q < 4$:} The condition $Q < 4$ is essential for embedding theorems (Sobolev embedding) which ensure that $H^{1,2}(X)$ has good pointwise control (Hölder continuity). This is addressed in the regularity theory (Sections D-E). The Poincaré constant $C_P$ depends on $Q$ and must be finite for the above argument to work.

\vspace{0.3cm}

\noindent\textbf{Part (v): Logical Acyclicity}

The emergence of kinetic structure follows an acyclic logical flow:

\begin{enumerate}
\item \textbf{Start:} Polish space $(X, d, \mu)$ satisfying Axiom I (Ahlfors regularity + Poincaré inequality).
\item \textbf{Step 1:} Define minimal upper gradient $\nabla_{\min}$ using purely metric definitions (no new axioms).
\item \textbf{Step 2:} Define kinetic energy form $\mathcal{E}_{\text{kin}}[\psi] := \int_X |\nabla_{\min} \psi|^2 \, d\mu$.
\item \textbf{Step 3:} Complete to Sobolev space $H^{1,2}(X)$ using metric Sobolev theory.
\item \textbf{Step 4:} Add Bregman divergence quadratic form (Axiom II) \textbf{independently}.
\item \textbf{Result:} Complete Dirichlet form $\mathcal{E}(\psi, \phi) = \mathcal{E}_{\text{kin}}(\psi, \phi) + \mathcal{Q}(\psi, \phi)$.

All circularity. All hidden axioms. The logical flow is directed acyclic: Axiom I $\to$ kinetic structure $\to$ (independent combination with Axiom II).
\end{enumerate}

\vspace{0.3cm}

\noindent\textbf{Conclusion}

The kinetic energy structure is \textbf{intrinsic to the Polish space metric-measure structure} (Axiom I). It does not constitute a hidden third axiom. The Barg Theory framework is grounded in:

\begin{itemize}
\item \textbf{Axiom I:} Polish space with metric and measure (geometric foundation)
\item \textbf{Axiom II:} Bregman divergence and information structure (functional foundation)
\item \textbf{Derived Kinetic Energy:} From metric Sobolev space theory (geometric consequence, not additional axiom)
\end{itemize}

This approach is maximally general (works on singular metric spaces, not just smooth manifolds) and mathematically rigorous (grounded in modern metric geometry).

\end{proof}

\end{theorem}

\begin{remark}[Resolving the Kinetic Energy Question]
\label{rem:resolvingKineticEnergy}

The original question was: ``Does the kinetic energy structure constitute a hidden third axiom?'' The answer, grounded in modern metric geometry, is \textbf{NO}. The kinetic energy is an intrinsic consequence of the Polish space metric measure structure (Axiom I).

Moreover, the minimal upper gradient approach is \textit{more general} than classical differential-geometric methods (which assume smooth manifolds). By working with metric Sobolev spaces, the framework achieves maximum generality while maintaining rigorous mathematical grounding.

The two-stage construction (Stage 1: pre-spectral form from metric structure; Stage 2: post-spectral verification using eigenfunctions) is workaround but rather a sophisticated and mathematically canonical approach to rigorously formalize the relationship between metric structure and spectral theory.

\end{remark}

