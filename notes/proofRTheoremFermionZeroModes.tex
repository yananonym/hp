% proofThmFermionZeroModes.tex
% Proof content


\textbf{Proof of Theorem \ref{thm:fermionZeroModes}}

The analyze the zero modes of the Dirac operator in the divergence-first framework using the Atiyah-Singer index theorem adapted to pre-metric geometry.

\textit{\underline{Part (i): Dirac Operator in Pre-Metric Framework}}

The Dirac operator $\slashed{D}$ is defined from the CarrÃ'À Ã¢‚¬„¢ÃƒÆ’€šÃ€šÃ‚© du Champ $\Gamma$ (Definition \ref{def:carreDuChamp}) and the emerged spinor structure (Theorem \ref{thm:spinorDoubleCover}). On spinor-valued functions $\psi: X \to S$ (where $S$ is the spinor bundle):
\[
\slashed{D} = \sum_{i=1}^Q \gamma^i \nabla_i,
\]
where $\gamma^i$ are the Clifford algebra generators satisfying $\{\gamma^i, \gamma^j\} = 2g^{ij}$, and $\nabla_i$ is the covariant derivative induced by the emerged Levi-Civita connection.

The operator $\slashed{D}$ is self-adjoint with respect to the $L^2(X, \mu)$ inner product, where $\mu$ is the effective measure from Axiom III.

By Theorem \ref{thm:resolventCompactness}, $\slashed{D}$ has discrete spectrum (counting multiplicity) and accumulates only at infinity. The zero mode space is:
\[
\ker(\slashed{D}) := \{\psi \in \mathcal{H} : \slashed{D}\psi = 0\}.
\]

\textit{\underline{Part (ii): Index of the Dirac Operator}}

The Fredholm index is:
\[
\text{ind}(\slashed{D}) := \dim(\ker(\slashed{D})) - \dim(\ker(\slashed{D}^\dagger)).
\]

Since $\slashed{D}$ is self-adjoint, $\slashed{D}^\dagger = \slashed{D}$ and $\ker(\slashed{D}^\dagger) = \ker(\slashed{D})$, so formally $\text{ind}(\slashed{D}) = 0$. However, when $\slashed{D}$ acts between different spinor chirality sectors:
\[
\slashed{D}_+: \Omega^{\text{even}} \to \Omega^{\text{odd}}, \quad \slashed{D}_-: \Omega^{\text{odd}} \to \Omega^{\text{even}},
\]
the index is well-defined:
\[
\text{ind}(\slashed{D}_+) = \dim(\ker(\slashed{D}_+)) - \dim(\ker(\slashed{D}_-)).
\]

\textit{\underline{Part (iii): Atiyah-Singer Index Theorem}}

By the Atiyah-Singer index theorem (Atiyah-Singer 1968), the index is given by:
\[
\text{ind}(\slashed{D}) = \int_X \hat{A}(M) \edge \text{ch}(E),
\]
where:
\begin{enumerate}[label=(\alph*)]
\item $\hat{A}(M)$ is the A-genus of the manifold $X$ (with the emerged metric structure)
\item $\text{ch}(E)$ is the Chern character of the spinor bundle $E = S$
\item The integration is over the top-dimensional cohomology
\end{enumerate}

For a $Q$-dimensional manifold, the A-genus and Chern character have explicit expansions in terms of the Riemann curvature tensor $R_{ijkl}$ and the gauge field strength $F = dA$.

\textit{\underline{Part (iv): Explicit Computation for Yang-Mills Background}}

In the presence of a non-abelian gauge field $A$ (Section \ref{sec:strongInteractionsEmergence}), the index of the coupled Dirac operator $\slashed{D}_A = \slashed{D} + \slashed{A}$ (where $\slashed{A}$ denotes coupling to the gauge field via $\gamma^i A_i$) is given by:
\[
\text{ind}(\slashed{D}_A) = \int_X \hat{A}(R) \edge \text{ch}(F/(2\pi i)),
\]

where $F = dA + A \edge A$ is the Yang-Mills curvature 2-form.

For an instanton background with self-dual Yang-Mills field ($F = \star F$), the instanton number is:
\[
Q_{\text{inst}} = \frac{1}{8\pi^2} \int_X \text{Tr}(F \edge F).
\]

By the index theorem for instantons (Atiyah-Drinfeld-Hitchin-Manin 1978):
\[
\text{ind}(\slashed{D}_A) = \int_X \hat{A}(R) \edge \text{ch}(\text{adj}) + Q_{\text{inst}} \cdot \text{rank}(G),
\]
where $\text{adj}$ denotes the adjoint representation of the gauge group $G$.

For $SU(3)$ color gauge theory (Section \ref{sec:strongInteractionsEmergence}), there is $\text{rank}(SU(3)) = 8$.

\textit{\underline{Part (v): Zero Mode Counting and Chirality}}

When the gauge field is non-trivial (instanton number $Q_{\text{inst}} > 0$), the index is positive, meaning $\dim(\ker(\slashed{D}_+)) > \dim(\ker(\slashed{D}_-))$.

The excess zero modes with positive chirality (left-handed) equal:
\[
n_- - n_+ = Q_{\text{inst}},
\]
where $n_\pm$ denotes the number of zero modes with chirality $\pm 1$.

In an instanton background with minimal topological charge $Q_{\text{inst}} = 1$, there is exactly one additional left-handed zero mode (zero mode of $\slashed{D}_-$ not cancelled by a right-handed partner).

By the chirality of fermion couplings (Theorem \ref{thm:spinorDoubleCover}, Section \ref{sec:spinorFermionStructure}), only left-handed fermions couple to weak interactions via $W^\pm$ bosons (Theorem \ref{thm:su2WeakStructure}). Thus the instanton background induces an imbalance in the fermionic content.

\textit{\underline{Part (vi): Topological Invariants and Spectral Data}}

The topological invariants $Q_{\text{inst}}$ can be expressed in terms of spectral data of the Dirac operator. Define the eta-function:
\[
\eta(s) := \sum_{\lambda_k > 0} \text{sign}(\lambda_k) |\lambda_k|^{-s},
\]
where the sum is over non-zero eigenvalues $\lambda_k$ of $\slashed{D}$ (with sign-weighted multiplicity for chiral modes).

By the Atiyah-Patodi-Singer index theorem (Atiyah-Patodi-Singer 1975), the index is related to the eta-invariant $\eta(0)$:
\[
\text{ind}(\slashed{D}) = \frac{\eta(0) + \dim(\ker(\slashed{D}))}{2}.
\]

The eta-invariant can be computed from the heat kernel asymptotics via:
\[
\eta(0) = \text{finite part of } \int_0^\infty t^{-1/2} \text{Tr}(\text{sign}(\slashed{D}) e^{-t\slashed{D}^2}) dt.
\]

This connects the index (topological) to spectral properties (analytic), demonstrating the deep relationship in the divergence-first framework.

\textit{\underline{Part (vii): Consistency with Matter Field Dynamics}}

The presence of zero modes is crucial for anomaly cancellation and the structure of the Standard Model:

\begin{enumerate}[label=(\alph*)]
\item Each generation of quarks and leptons corresponds to different topological sectors of the path integral
\item Zero modes in instanton backgrounds allow tunneling between these sectors
\item Anomaly cancellation (Theorem \ref{thm:standardModelGaugeGroupDerivation}) imposes constraints on the fermion content
\item The three-generation structure (Section \ref{sec:threeGenerations}) arises from the requirement that all anomalies cancel simultaneously
\end{enumerate}

The zero mode analysis via index theory is thus essential for understanding the low-energy effective field theory emerging from the divergence-first framework.

\qed
