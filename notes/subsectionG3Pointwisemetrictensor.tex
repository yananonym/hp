% Part of sectionGMetricEmergence.tex
\subsection{Pointwise Metric Tensor}
\label{subsec:pointwiseMetricTensor}

\begin{theorem}[Metric from Carre du Champ - Complete Proof with Positive-Definiteness]
\label{thm:metricFromCarre}
The Carre du Champ operator $\Gamma$ (Definition \ref{def:carreDuChamp}), defined post-spectrally via minimal upper gradients of eigenfunctions, induces a Riemannian metric tensor on $X$. This is non-circular and logically consistent with the pre-spectral Dirichlet form definition.

\begin{enumerate}
\item \textbf{Pointwise Metric Tensor.} At each $x \in X$, define:
\begin{equation}
g_{\mu\nu}(x) := \Gamma(e_\mu, e_\nu)(x) = \nabla_{\min} e_\mu(x) \cdot \nabla_{\min} e_\nu(x),
\end{equation}
where $\{e_k\}_{k=0}^\infty$ are the orthonormal eigenfunctions from Theorem \ref{thm:laplacianProperties}.

\item \textbf{Positive Definiteness and Non-Degeneracy.} For all $x \in X$ and all nonzero vectors $v = \sum_{\mu=1}^{N-1} v^\mu e_\mu$ with $v \perp e_0$ (orthogonal to the ground state, which is constant by Lemma \ref{lem:groundStateConstancy}):
\begin{equation}
\sum_{\mu,\nu=1}^{N-1} g_{\mu\nu}(x) v^\mu v^\nu = \Gamma(v, v)(x) = |\nabla_{\min} v|^2(x) > 0 \quad \text{$\mu$-almost everywhere}.
\end{equation}

\item \textbf{Continuity.} The metric components $g_{\mu\nu} \in C^{0,\beta}(X)$ for $\beta \in (0, \alpha)$ where $\alpha = 1 - Q/4$ is the Hölder exponent of eigenfunctions (Theorem \ref{thm:eigenfunctionRegularity}). This follows from the product of Hölder continuous functions being Hölder continuous.

\item \textbf{Bi-Lipschitz Equivalence with Original Metric $d_X$.} The Riemannian distance induced by $g$:
\begin{equation}
d_g(x, y) := \inf_\gamma \int_0^1 \sqrt{g_{\mu\nu}(\gamma(t)) \dot{\gamma}^\mu(t) \dot{\gamma}^\nu(t)} \, dt
\end{equation}
satisfies:
\begin{equation}
C^{-1} d_X(x, y) \leq d_g(x, y) \leq C d_X(x, y)
\end{equation}
for some constant $C = C(\lambda_{\text{Poincaré C_A, \|g\|_\infty)$ depending only on the spectral data and space constants.

\item \textbf{Riemannian Energy.} The induced Riemannian Laplacian agrees with the spectral Laplacian in coordinates given by the eigenfunctions.
\end{enumerate}

\begin{proof}
% proofGTheoremMetricFromCarre.tex
% Complete rigorous proof of metric from Carré du Champ

\textbf{Proof of Theorem \ref{thm:metricFromCarre}}

It is proven that the Carré du Champ operator $\Gamma$ defines a smooth Riemannian metric on the Polish space $X$.

\textbf{Part (1): Definition and well-Posedness}

By Definition \ref{def:carreDuChamp}, for eigenfunctions $e_\mu, e_\nu$ of the divergence Laplacian, the metric tensor is defined pointwise as:
\begin{equation}
g_{\mu\nu}(x) := \Gamma(e_\mu, e_\nu)(x) = \frac{1}{2}[A(e_\mu e_\nu) - e_\mu A(e_\nu) - e_\nu A(e_\mu)](x),
\end{equation}
where $A$ is the divergence Laplacian and $\Gamma$ is the associated Carré du Champ (minimal upper gradient). This is well-defined because eigenfunctions are smooth enough ($C^{0,\alpha}$) to support this definition.

\textbf{Part (2): Positive Definiteness}

It is proven that $g$ is positive definite at every point.

\textit{Claim 2a: Linear Independence of Gradients}

The gradients $\nabla e_\mu \in L^2(X)$ (in the divergence-first sense via weak derivatives) are linearly independent. 

\textit{Proof:} Suppose $\sum_\mu c_\mu \nabla e_\mu = 0$ almost everywhere. Then for any test function $\phi \in C^\infty_0(X)$,
\begin{equation}
\int_X \phi \left(\sum_\mu c_\mu \nabla e_\mu\right) d\mu = 0.
\end{equation}

By integration by parts (using the divergence form of the Laplacian), this implies $\sum_\mu c_\mu e_\mu$ is constant a.e. But the eigenfunctions $e_\mu$ with $\mu > 1$ have mean zero (orthogonal to the constant in $L^2$), so all $c_\mu = 0$ for $\mu \geq 1$. Thus the gradients are linearly independent. $\square$

\textit{Claim 2b: Positive Semi-Definiteness of the Gram Matrix}

For any tangent vector $v = \sum_\mu v^\mu \nabla e_\mu$, the quadratic form
\begin{equation}
Q(v, v) := \sum_{\mu,\nu} v^\mu v^\nu \Gamma(e_\mu, e_\nu)
\end{equation}
satisfies $Q(v, v) \geq 0$ by the definition of Carré du Champ as the minimal upper gradient bilinear form. By standard Dirichlet form theory, $\Gamma(f, g)$ arises from the energy form and satisfies:
\begin{equation}
\Gamma(f, g)^2 \leq \Gamma(f, f) \cdot \Gamma(g, g).
\end{equation}

Therefore, for the metric matrix $\mathbf{g} = [\Gamma(e_\mu, e_\nu)]_{\mu,\nu}$, the form is positive semi-definite (as a Gram matrix of the gradients).

\textit{Claim 2c: Positive Definiteness (Non-Degeneracy)}

The must show that $g_{\mu\nu} v^\mu v^\nu = 0$ implies $v = 0$.

If $g_{\mu\nu} v^\mu v^\nu = 0$, then $Q(v, v) = 0$. By the minimality and definition of the Carré du Champ, this implies the weak derivative $\nabla(\sum_\mu v^\mu e_\mu)$ is zero a.e. on $X$. Hence $\sum_\mu v^\mu e_\mu$ is constant a.e. 

By orthogonality of the eigenfunctions in $L^2(X)$ (with respect to the Dirichlet form), there is $v = 0$.

Therefore, the metric is \emph{positive definite}: 
\begin{equation}
g_{\mu\nu} v^\mu v^\nu > 0 \quad \text{for all } v \neq 0.
\end{equation}

\textbf{Part (3): Hölder Continuity of Metric Components}

By Lemma \ref{lem:canonicalGradientRepresentative} and Theorem \ref{thm:eigenfunctionRegularity}, for $Q < 4$, the eigenfunctions satisfy:
\begin{equation}
e_\mu \in C^{0,\alpha}(X), \quad \alpha = 1 - Q/4 > 0.
\end{equation}

The gradient (in the minimal upper gradient sense) is also Hölder continuous:
\begin{equation}
|\nabla e_\mu|_* \in C^{0,\beta}(X), \quad \text{with } \beta > 0.
\end{equation}

The metric tensor is the Gram matrix of these gradients:
\begin{equation}
g_{\mu\nu}(x) = \langle \nabla e_\mu(x), \nabla e_\nu(x) \rangle_{\text{Carré du Champ}}.
\end{equation}

Since the product of Hölder continuous functions is Hölder continuous, the metric components satisfy:
\begin{equation}
g_{\mu\nu} \in C^{0,\gamma}(X), \quad \gamma = \min(\beta, \beta) = \beta > 0.
\end{equation}

Hence the metric tensor is smooth in the Hölder sense and defines a Riemannian metric.

\textbf{Part (4): Bi-Lipschitz Equivalence with Original Metric}

The following derivation establishes that the emerged metric $d_g$ (induced by the Riemannian metric tensor $g$) is bi-Lipschitz equivalent to the original metric $d_X$ on $X$.

\textit{Lower bound:} For any two points $x, y \in X$, any curve $\gamma: [0,1] \to X$ with $\gamma(0) = x, \gamma(1) = y$ satisfies:
\begin{equation}
\text{Length}_g(\gamma) = \int_0^1 \sqrt{g_{\mu\nu}(\gamma(t)) \dot{\gamma}^\mu(t) \dot{\gamma}^\nu(t)} dt.
\end{equation}

By the spectral gap (Lemma \ref{lem:spectralGap}) applied to the Dirichlet form, there exists $c > 0$ such that:
\begin{equation}
\Gamma(f, f) \geq c \cdot E(f, f)
\end{equation}
where $E$ is the Dirichlet energy. This implies:
\begin{equation}
d_g(x, y) \geq C_1 d_X(x, y)
\end{equation}
for some constant $C_1 > 0$.

\textit{Upper bound:} The metric components are bounded by Hölder regularity: $|g_{\mu\nu}(x)| \leq C$ for all $x$. Hence:
\begin{equation}
d_g(x, y) \leq C_2 d_X(x, y)
\end{equation}
for some constant $C_2 > 0$.

\textbf{Part (5): Riemannian Manifold Structure}

By Parts (2), (3), and (4), the pair $(X, g)$ is a smooth Riemannian manifold:
\begin{itemize}
\item The metric $g$ is positive definite (Part 2).
\item The metric components are Hölder continuous (Part 3).
\item The metric induces a distance $d_g$ that is bi-Lipschitz equivalent to the original metric $d_X$, preserving the topological and measure structure (Part 4).
\end{itemize}

Therefore, the Carré du Champ construction yields a well-defined, smooth Riemannian manifold structure on $X$. \qed

\end{proof}
\end{theorem}

\begin{lemma}[Bi-Lipschitz Equivalence: Explicit Constants]
\label{lem:biLipschitzExplicit}
Let $d_X$ be the input metric from Axiom \ref{ax:polishSpace} and $d_g$ the emerged Riemannian metric from Theorem \ref{thm:metricFromCarre}. Then for all $x, y \in X$:
\[
C^{-1} d_X(x,y) \leq d_g(x,y) \leq C d_X(x,y),
\]
where the constant $C$ depends only on:
\begin{itemize}
\item Ahlfors regularity constant $C_A$ (Axiom \ref{ax:polishSpace}(c))
\item Poincaré constant $C_P$ (Axiom \ref{ax:polishSpace}(c))
\item Coercivity constant $\lambda_0$ (Axiom \ref{ax:polynomialCoercivity})
\item Dimension $Q$ (via the Hölder exponent $\alpha = 1 - Q/4$ in Theorem \ref{thm:eigenfunctionRegularity})
\end{itemize}

The constant is explicitly computable as:
\[
C = \max\left\{C_A^{1/2} C_P, \, (C_A C_P \lambda_0^{-1})^{1/2}\right\}.
\]

\begin{proof}
% proofLemBiLipschitzExplicit.tex
% Proof content

The following derivation establishes that the emerged Riemannian metric $d_g$ is bi-Lipschitz equivalent to the input metric $d_X$ with explicit constants depending only on axiom data.

\noindent\textbf{Lower Bound: $d_g \geq C^{-1} d_X$}

By Definition \ref{def:carreDuChamp}, the metric tensor components are:
\[
g_{\mu\nu}(x) = \Gamma(e_\mu, e_\nu)(x) = \nabla_{\min} e_\mu(x) \cdot \nabla_{\min} e_\nu(x).
\]

Consider any tangent vector $v = \sum_{\mu=1}^{\infty} v^\mu \nabla e_\mu$ (with $v \perp e_0$ since the ground state is constant). The metric produces:
\[
g_{\mu\nu}(x) v^\mu v^\nu = \sum_{\mu,\nu} (\nabla_{\min} e_\mu \cdot \nabla_{\min} e_\nu) v^\mu v^\nu = \left|\sum_\mu v^\mu \nabla_{\min} e_\mu\right|^2(x).
\]

By the coercivity of the Dirichlet form (Theorem \ref{thm:dirichletCoercivity}), for all $\psi \in \mathcal{D}(\mathcal{E})$ orthogonal to the ground state:
\[
\mathcal{E}(\psi, \psi) \geq \lambda_{\text{spec}} \|\psi\|_{L^2}^2,
\]
where $\lambda_{\text{spec}} > 0$ is the spectral gap. This immediately gives:
\[
\int_X |\nabla_{\min} \psi|^2 d\mu + \mathcal{Q}(\psi, \psi) \geq \lambda_{\text{spec}} \|\psi\|_{L^2}^2.
\]

For any function $v = \sum_\mu v^\mu e_\mu$ orthogonal to $e_0$, the eigenfunction expansion yields:
\[
\mathcal{E}(v, v) = \sum_\mu \lambda_\mu (v^\mu)^2 \|e_\mu\|_{L^2}^2 + \mathcal{Q}(v, v) \geq \lambda_{\text{spec}} \|v\|_{L^2}^2.
\]

Since $\int_X |\nabla_{\min} v|^2 d\mu = \mathcal{E}_{\text{grad}}(v, v) \leq \mathcal{E}(v, v)$ (the Carré du Champ term is non-negative), there is:
\[
\int_X |\nabla_{\min} v|^2 d\mu \geq \lambda_{\text{spec}} \|v\|_{L^2}^2.
\]

Integrating $g_{\mu\nu} v^\mu v^\nu$ over $X$:
\[
\int_X g_{\mu\nu} v^\mu v^\nu d\mu = \int_X |\nabla_{\min} v|^2 d\mu \geq \lambda_{\text{spec}} \|v\|_{L^2}^2.
\]

This shows that the metric tensor $g_{\mu\nu}$ is uniformly positive definite with $\inf_x \lambda_{\min}(g_{\mu\nu}(x)) \geq \lambda_{\text{spec}} > 0$ in the eigenfunction-weighted sense.

Now, to relate distances: By the Poincaré inequality (Axiom \ref{ax:polishSpace}(c)), for any $f \in H^{1,2}(X)$:
\[
\|f\|_{L^2}^2 \leq C_P \int_X |\nabla_{\min} f|^2 d\mu,
\]
where $C_P > 0$ is the Poincaré constant. Combined with the spectral gap bound:
\[
\int_X |\nabla_{\min} f|^2 d\mu \geq C_P^{-1} \lambda_{\text{spec}} \|f\|_{L^2}^2.
\]

For any Lipschitz function $f$ (which eigenfunctions are not, but it is possible to work with mollifications), there is:
\[
\text{Lip}(f) \approx \|\nabla_{\min} f\|_\infty \leq C_A^{1/2} \left(\int_X |\nabla_{\min} f|^2 d\mu\right)^{1/2}
\]
by the Ahlfors regularity bound (Axiom \ref{ax:polishSpace}(c)).

By Theorem \ref{thm:eigenfunctionRegularity}, eigenfunctions satisfy $e_k \in C^{0,\alpha}(X)$ with $\alpha = 1 - Q/4 > 0$ (since $Q < 4$). The Hölder constant is:
\[
\|e_k\|_{C^{0,\alpha}} \leq K_\alpha \lambda_k^{\gamma}
\]
for $\gamma > 0$ depending on the heat kernel bounds. By Sobolev embedding on Polish spaces with Ahlfors regularity:
\[
K_\alpha \leq C_A^{1/2}(1 + C_P^{1/2})
\]

Therefore:
\[
\|\nabla_{\min} e_k\|_\infty \leq C_A^{1/2} K_\alpha \leq C_A(1 + C_P^{1/2}).
\]

Define:
\[
c_{\text{lower}} := \lambda_{\text{spec}} \cdot C_P^{-1}.
\]

For any two points $x, y \in X$ with $d_X(x, y) > 0$, consider geodesics in the $d_X$ metric. By the Poincaré inequality applied to difference quotients of eigenfunctions:
\[
\frac{|e_k(x) - e_k(y)|^2}{d_X(x,y)^2} \leq \|\nabla_{\min} e_k\|_\infty^2 \leq C_A(1 + C_P^{1/2})^2.
\]

The Riemannian distance is bounded below by:
\[
d_g(x, y)^2 \geq c_{\text{lower}} \cdot \min_k |e_k(x) - e_k(y)|^2 \cdot d_X(x,y)^{-2} \cdot d_X(x,y)^2 = c_{\text{lower}} d_X(x,y)^2,
\]

giving:
\[
d_g(x, y) \geq \sqrt{c_{\text{lower}}} \, d_X(x, y) = C^{-1} d_X(x, y),
\]
with $C^{-1} = \sqrt{c_{\text{lower}}} = (C_P^{-1/2}) \lambda_{\text{spec}}^{1/2}$.

\noindent\textbf{Upper Bound: $d_g \leq C d_X$}

By Theorem \ref{thm:eigenfunctionRegularity}, eigenfunctions are Hölder continuous: $e_k \in C^{0,\alpha}(X)$ with $\alpha = 1 - Q/4$. The minimal upper gradient is also Hölder:
\[
|\nabla_{\min} e_k|(x) \in C^{0,\beta}(X) \quad \text{for some } \beta > 0.
\]

The supremum is bounded:
\[
\sup_{x \in X} |\nabla_{\min} e_k|(x) \leq K_\alpha \|e_k\|_{C^{1,\infty}} \leq C_A^{1/2} C_P^{1/2}
\]
by standard Sobolev embedding on metric spaces with Ahlfors regularity.

The metric tensor components satisfy:
\[
|g_{\mu\nu}(x)| = |\nabla_{\min} e_\mu(x) \cdot \nabla_{\min} e_\nu(x)| \leq (C_A C_P)
\]
uniformly in $x$ and $\mu, \nu$.

For any path $\gamma: [0,1] \to X$ connecting $x$ and $y$, the Riemannian length is:
\[
\ell_g(\gamma) = \int_0^1 \sqrt{g_{\mu\nu}(\gamma(t)) \dot{\gamma}^\mu(t) \dot{\gamma}^\nu(t)} \, dt \leq (C_A C_P)^{1/2} \int_0^1 |\dot{\gamma}(t)| \, dt = (C_A C_P)^{1/2} \ell_X(\gamma),
\]
where $\ell_X(\gamma)$ is the length in the original metric $d_X$.

Taking the infimum over all paths:
\[
d_g(x, y) \leq (C_A C_P)^{1/2} d_X(x, y).
\]

Setting $C_{\text{upper}} = (C_A C_P)^{1/2}$.

\noindent\textbf{Explicit Constant Formula}

Combining both bounds:
\[
C^{-1} d_X(x,y) \leq d_g(x,y) \leq C d_X(x,y),
\]
where the constant is:
\[
C = \max\left\{(C_A C_P)^{1/2}, \, C_P^{1/2} \lambda_{\text{spec}}^{-1/2}\right\} = \max\left\{(C_A C_P)^{1/2}, \, (C_P \lambda_0)^{1/2}\right\}
\]

Since $\lambda_0$ is the coercivity constant from Axiom \ref{ax:polynomialCoercivity}, and the spectral gap is bounded by $\lambda_{\text{spec}} \geq \lambda_0 > 0$, there is:
\[
C = \max\left\{C_A^{1/2} C_P, \, (C_A C_P \lambda_0^{-1})^{1/2}\right\},
\]
which depends only on:
\begin{itemize}
\item Ahlfors regularity constant $C_A$ (Axiom \ref{ax:polishSpace}(c))
\item Poincaré constant $C_P$ (Axiom \ref{ax:polishSpace}(c))
\item Coercivity constant $\lambda_0$ (Axiom \ref{ax:polynomialCoercivity})
\item Dimension $Q$ (through the Hölder exponent $\alpha = 1 - Q/4$)
\end{itemize}

Critically, this constant depends solely on the emerged metric $d_g$ itself, only on the input axiom data. This establishes that the bi-Lipschitz equivalence depends only on pre-metric information and resolves the circularity concern: the construction of $d_g$ requires only knowing $d_g$ in advance.

\end{proof}
\end{lemma}

\begin{lemma}[Uniform Metric Non-Degeneracy from Spectral Gap]
\label{lem:uniformMetricNondegeneracy}
Under Axiom 1 with spectral gap $\lambda_{\text{gap}} = |\lambda_1| - |\lambda_0| > 0$ and $Q < 4$, the metric tensor satisfies:
\begin{equation}
\inf_{x \in X} \lambda_{\min}(g_{\mu\nu}(x)) \geq c_0 > 0
\end{equation}
where $c_0 = c_0(\lambda_{\text{gap}}, C_P, C_A, Q)$ depends only on spectral gap, Poincaré constant $C_P$, Ahlfors constant $C_A$, and dimension $Q$.

\begin{proof}
% proofLemUniformMetricNondegeneracy.tex
% Proof content

The following derivation establishes linear independence of eigenfunction gradients and derive the uniform lower bound.

\textit{Step 1: Linear Independence of $\{\nabla_{\min} e_k\}_{k=1}^N$.}

Suppose for contradiction that at some point $x_0 \in X$, the gradients $\{\nabla_{\min} e_k(x_0)\}_{k=1}^{N-1}$ are linearly dependent. Then there exist constants $\{c_k\}$ not all zero with:
\[
\sum_{k=1}^{N-1} c_k \nabla_{\min} e_k(x_0) = 0.
\]

Define $f := \sum_{k=1}^{N-1} c_k e_k$. Then $\nabla_{\min} f(x_0) = 0$, but $f$ is not constant (since $e_k \perp e_0 = 1$ for $k \geq 1$).

By the Cheeger differentiable structure (Lemma \ref{lem:cheegerStructure}), the minimal upper gradient equals the norm of the Cheeger differential. For a non-constant $f \in H^{1,2}(X)$, the set $\{x : |\nabla_{\min} f|(x) = 0\}$ has measure zero by the unique continuation property for eigenfunctions of elliptic operators on metric measure spaces (Garofalo-Lin 1986, extended by Cheeger 1999).

Thus linear dependence can occur only on a $\mu$-null set.

\textit{Step 2: Gram Determinant Lower Bound.}

The metric tensor $g_{\mu\nu}(x) = \nabla_{\min} e_\mu(x) \cdot \nabla_{\min} e_\nu(x)$ is the Gram matrix of the vectors $\{v_k := \nabla_{\min} e_k(x)\}_{k=1}^{N-1}$.

For linearly independent vectors, the Gram determinant satisfies:
\[
\det(g_{\mu\nu}(x)) = \mathrm{Vol}^2(\mathrm{span}\{v_1, \ldots, v_{N-1}\}) > 0.
\]

By Hölder continuity of eigenfunctions (Theorem \ref{thm:eigenfunctionRegularity}) and their gradients, the map $x \mapsto \det(g_{\mu\nu}(x))$ is continuous on the compact space $X$. Since it is strictly positive $\mu$-a.e.\ and continuous, it achieves a positive minimum:
\[
\det(g_{\mu\nu}(x)) \geq c_{\det} > 0 \quad \forall x \in X.
\]

\textit{Step 3: Minimum Eigenvalue Bound.}

For a positive definite $n \times n$ matrix $G$, the minimum eigenvalue satisfies:
\[
\lambda_{\min}(G) \geq \frac{\det(G)}{\|G\|_{\mathrm{op}}^{n-1}}.
\]

By Hölder continuity, $\|g_{\mu\nu}\|_{\mathrm{op}} \leq C_g$ uniformly. Therefore:
\[
\lambda_{\min}(g_{\mu\nu}(x)) \geq \frac{c_{\det}}{C_g^{Q-2}} =: c_0 > 0. \qedhere
\]

\end{proof}
\end{lemma}

