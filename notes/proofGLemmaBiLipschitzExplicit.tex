% proofLemBiLipschitzExplicit.tex
% Proof content

The following derivation establishes that the emerged Riemannian metric $d_g$ is bi-Lipschitz equivalent to the input metric $d_X$ with explicit constants depending only on axiom data.

\noindent\textbf{Lower Bound: $d_g \geq C^{-1} d_X$}

By Definition \ref{def:carreDuChamp}, the metric tensor components are:
\[
g_{\mu\nu}(x) = \Gamma(e_\mu, e_\nu)(x) = \nabla_{\min} e_\mu(x) \cdot \nabla_{\min} e_\nu(x).
\]

Consider any tangent vector $v = \sum_{\mu=1}^{\infty} v^\mu \nabla e_\mu$ (with $v \perp e_0$ since the ground state is constant). The metric produces:
\[
g_{\mu\nu}(x) v^\mu v^\nu = \sum_{\mu,\nu} (\nabla_{\min} e_\mu \cdot \nabla_{\min} e_\nu) v^\mu v^\nu = \left|\sum_\mu v^\mu \nabla_{\min} e_\mu\right|^2(x).
\]

By the coercivity of the Dirichlet form (Theorem \ref{thm:dirichletCoercivity}), for all $\psi \in \mathcal{D}(\mathcal{E})$ orthogonal to the ground state:
\[
\mathcal{E}(\psi, \psi) \geq \lambda_{\text{spec}} \|\psi\|_{L^2}^2,
\]
where $\lambda_{\text{spec}} > 0$ is the spectral gap. This immediately gives:
\[
\int_X |\nabla_{\min} \psi|^2 d\mu + \mathcal{Q}(\psi, \psi) \geq \lambda_{\text{spec}} \|\psi\|_{L^2}^2.
\]

For any function $v = \sum_\mu v^\mu e_\mu$ orthogonal to $e_0$, the eigenfunction expansion yields:
\[
\mathcal{E}(v, v) = \sum_\mu \lambda_\mu (v^\mu)^2 \|e_\mu\|_{L^2}^2 + \mathcal{Q}(v, v) \geq \lambda_{\text{spec}} \|v\|_{L^2}^2.
\]

Since $\int_X |\nabla_{\min} v|^2 d\mu = \mathcal{E}_{\text{grad}}(v, v) \leq \mathcal{E}(v, v)$ (the Carré du Champ term is non-negative), there is:
\[
\int_X |\nabla_{\min} v|^2 d\mu \geq \lambda_{\text{spec}} \|v\|_{L^2}^2.
\]

Integrating $g_{\mu\nu} v^\mu v^\nu$ over $X$:
\[
\int_X g_{\mu\nu} v^\mu v^\nu d\mu = \int_X |\nabla_{\min} v|^2 d\mu \geq \lambda_{\text{spec}} \|v\|_{L^2}^2.
\]

This shows that the metric tensor $g_{\mu\nu}$ is uniformly positive definite with $\inf_x \lambda_{\min}(g_{\mu\nu}(x)) \geq \lambda_{\text{spec}} > 0$ in the eigenfunction-weighted sense.

Now, to relate distances: By the Poincaré inequality (Axiom \ref{ax:polishSpace}(c)), for any $f \in H^{1,2}(X)$:
\[
\|f\|_{L^2}^2 \leq C_P \int_X |\nabla_{\min} f|^2 d\mu,
\]
where $C_P > 0$ is the Poincaré constant. Combined with the spectral gap bound:
\[
\int_X |\nabla_{\min} f|^2 d\mu \geq C_P^{-1} \lambda_{\text{spec}} \|f\|_{L^2}^2.
\]

For any Lipschitz function $f$ (which eigenfunctions are not, but it is possible to work with mollifications), there is:
\[
\text{Lip}(f) \approx \|\nabla_{\min} f\|_\infty \leq C_A^{1/2} \left(\int_X |\nabla_{\min} f|^2 d\mu\right)^{1/2}
\]
by the Ahlfors regularity bound (Axiom \ref{ax:polishSpace}(c)).

By Theorem \ref{thm:eigenfunctionRegularity}, eigenfunctions satisfy $e_k \in C^{0,\alpha}(X)$ with $\alpha = 1 - Q/4 > 0$ (since $Q < 4$). The Hölder constant is:
\[
\|e_k\|_{C^{0,\alpha}} \leq K_\alpha \lambda_k^{\gamma}
\]
for $\gamma > 0$ depending on the heat kernel bounds. By Sobolev embedding on Polish spaces with Ahlfors regularity:
\[
K_\alpha \leq C_A^{1/2}(1 + C_P^{1/2})
\]

Therefore:
\[
\|\nabla_{\min} e_k\|_\infty \leq C_A^{1/2} K_\alpha \leq C_A(1 + C_P^{1/2}).
\]

Define:
\[
c_{\text{lower}} := \lambda_{\text{spec}} \cdot C_P^{-1}.
\]

For any two points $x, y \in X$ with $d_X(x, y) > 0$, consider geodesics in the $d_X$ metric. By the Poincaré inequality applied to difference quotients of eigenfunctions:
\[
\frac{|e_k(x) - e_k(y)|^2}{d_X(x,y)^2} \leq \|\nabla_{\min} e_k\|_\infty^2 \leq C_A(1 + C_P^{1/2})^2.
\]

The Riemannian distance is bounded below by:
\[
d_g(x, y)^2 \geq c_{\text{lower}} \cdot \min_k |e_k(x) - e_k(y)|^2 \cdot d_X(x,y)^{-2} \cdot d_X(x,y)^2 = c_{\text{lower}} d_X(x,y)^2,
\]

giving:
\[
d_g(x, y) \geq \sqrt{c_{\text{lower}}} \, d_X(x, y) = C^{-1} d_X(x, y),
\]
with $C^{-1} = \sqrt{c_{\text{lower}}} = (C_P^{-1/2}) \lambda_{\text{spec}}^{1/2}$.

\noindent\textbf{Upper Bound: $d_g \leq C d_X$}

By Theorem \ref{thm:eigenfunctionRegularity}, eigenfunctions are Hölder continuous: $e_k \in C^{0,\alpha}(X)$ with $\alpha = 1 - Q/4$. The minimal upper gradient is also Hölder:
\[
|\nabla_{\min} e_k|(x) \in C^{0,\beta}(X) \quad \text{for some } \beta > 0.
\]

The supremum is bounded:
\[
\sup_{x \in X} |\nabla_{\min} e_k|(x) \leq K_\alpha \|e_k\|_{C^{1,\infty}} \leq C_A^{1/2} C_P^{1/2}
\]
by standard Sobolev embedding on metric spaces with Ahlfors regularity.

The metric tensor components satisfy:
\[
|g_{\mu\nu}(x)| = |\nabla_{\min} e_\mu(x) \cdot \nabla_{\min} e_\nu(x)| \leq (C_A C_P)
\]
uniformly in $x$ and $\mu, \nu$.

For any path $\gamma: [0,1] \to X$ connecting $x$ and $y$, the Riemannian length is:
\[
\ell_g(\gamma) = \int_0^1 \sqrt{g_{\mu\nu}(\gamma(t)) \dot{\gamma}^\mu(t) \dot{\gamma}^\nu(t)} \, dt \leq (C_A C_P)^{1/2} \int_0^1 |\dot{\gamma}(t)| \, dt = (C_A C_P)^{1/2} \ell_X(\gamma),
\]
where $\ell_X(\gamma)$ is the length in the original metric $d_X$.

Taking the infimum over all paths:
\[
d_g(x, y) \leq (C_A C_P)^{1/2} d_X(x, y).
\]

Setting $C_{\text{upper}} = (C_A C_P)^{1/2}$.

\noindent\textbf{Explicit Constant Formula}

Combining both bounds:
\[
C^{-1} d_X(x,y) \leq d_g(x,y) \leq C d_X(x,y),
\]
where the constant is:
\[
C = \max\left\{(C_A C_P)^{1/2}, \, C_P^{1/2} \lambda_{\text{spec}}^{-1/2}\right\} = \max\left\{(C_A C_P)^{1/2}, \, (C_P \lambda_0)^{1/2}\right\}
\]

Since $\lambda_0$ is the coercivity constant from Axiom \ref{ax:polynomialCoercivity}, and the spectral gap is bounded by $\lambda_{\text{spec}} \geq \lambda_0 > 0$, there is:
\[
C = \max\left\{C_A^{1/2} C_P, \, (C_A C_P \lambda_0^{-1})^{1/2}\right\},
\]
which depends only on:
\begin{itemize}
\item Ahlfors regularity constant $C_A$ (Axiom \ref{ax:polishSpace}(c))
\item Poincaré constant $C_P$ (Axiom \ref{ax:polishSpace}(c))
\item Coercivity constant $\lambda_0$ (Axiom \ref{ax:polynomialCoercivity})
\item Dimension $Q$ (through the Hölder exponent $\alpha = 1 - Q/4$)
\end{itemize}

Critically, this constant depends solely on the emerged metric $d_g$ itself, only on the input axiom data. This establishes that the bi-Lipschitz equivalence depends only on pre-metric information and resolves the circularity concern: the construction of $d_g$ requires only knowing $d_g$ in advance.
