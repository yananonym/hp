% proofLemEigenfunctionRegularityBootstrap.tex
% Proof content

\textbf{Step 0 (Base case):} From Theorem \ref{thm:eigenfunctionRegularity}, eigenfunctions satisfy $e_k \in C^{0,\alpha}$ with $\alpha = 1 - Q/4$ using only the pre-metric structure (minimal upper gradient, Axiom I). This is derived from Sobolev embedding on metric measure spaces \cite{ambrosio2005gradient}: for $Q < 4$, the embedding $H^{1,2}(X) \hookrightarrow C^{0,\alpha}(X)$ holds with $\alpha = (1 - Q/4)$.

\textbf{Step 1 (First bootstrap):} With $e_k \in C^{0,\alpha}$, the metric tensor:
\[
g_{\mu\nu} = \Gamma(e_\mu, e_\nu) = \nabla_{\min} e_\mu \cdot \nabla_{\min} e_\nu
\]
is in $C^{0, \alpha'}$ for $\alpha' = \alpha - \epsilon$ (products of Hölder continuous functions remain Hölder continuous with exponent equal to the minimum of the component exponents). More precisely, for $\alpha, \alpha \in (0,1)$, if $f, g \in C^{0,\alpha}$, then $fg \in C^{0,\min(\alpha, \alpha)} = C^{0,\alpha}$. Since $e_\mu \in C^{0,\alpha}$ and weak derivatives $\nabla_{\min} e_\mu$ are bounded, their pointwise product is controlled by the Hölder norm.

\textbf{Step 2 (Elliptic regularity via Schauder theory):} With metric $g \in C^{0,\alpha'}$ and potential $W \in C^{0,\gamma}$ (where $\gamma$ is the Hölder regularity of $V''(|\psi_0|^2)$), the eigenvalue equation:
\[
\Delta_g e_k = \lambda_k e_k + W e_k
\]
has a right-hand side in $C^{0, \min(\alpha', \gamma)}$ (since $W e_k$ is a product of Hölder continuous functions). By Schauder elliptic regularity theory (\cite{gilbarg1983elliptic}, Section 6.3), this implies:
\[
e_k \in C^{2, \min(\alpha', \gamma)} =: C^{2, \delta},
\]
where $\delta = \min(\alpha - \epsilon, \gamma)$. Here there is used the fact that for an elliptic operator with $C^{0,\beta}$ coefficients, a solution to $L u = f$ with $f \in C^{0,\beta}$ satisfies $u \in C^{2,\beta}$ (interior regularity).

\textbf{Step 3 (Metric upgrade):} With $e_k \in C^{2,\delta}$, the Carré du Champ:
\[
\Gamma(e_\mu, e_\nu) = \sum_i \partial_i e_\mu \partial_i e_\nu
\]
is now a product of $C^{1,\delta}$ functions, hence lies in $C^{1,\delta}$. Therefore the metric $g_{\mu\nu} \in C^{1,\delta}$, enabling definition of the Levi-Civita connection and Riemann curvature tensor with $C^{0,\delta}$ components.

\textbf{Step 4 (Iteration and convergence):} Repeating Steps 2--3 with updated regularity gives $e_k \in C^{2,\delta_n}$ with $\delta_n$ increasing toward a limit $\delta_\infty$. The limit is determined by two factors:
\begin{itemize}
\item \textbf{Upper limit from potential regularity:} If $V \in C^{k+2}(\mathbb{R})$ (Axiom II allows $V$ analytic or smooth), then $W = V''(|\psi_0|^2) \in C^k$. Each Schauder iteration can improve regularity by 2 derivatives if sufficient regularity is present. For $V$ analytic, iterations yield $e_k \in C^\infty$.
\item \textbf{Lower limit from metric measure structure:} The Weyl asymptotic law $N(\lambda) \sim \lambda^{Q/2}$ implies a minimal regularity cap. For the heat kernel, results of Grigor'yan guarantee $C^{1,1}$ regularity of $p_t(x,y)$ under Axiom assumptions.
\end{itemize}

\textbf{Conclusion:} For Axiom II with analytic potential $V$, the bootstrap process produces $e_k \in C^\infty$ and $g \in C^\infty$. For smooth potential $V$, the bootstrap achieves at minimum $e_k \in C^{2,\delta}$ and $g \in C^{1,\delta}$ for $\delta = \min(\alpha - \epsilon, \gamma)$, which suffices for construction of:
\begin{itemize}
\item Levi-Civita connection $\nabla$ with continuous second derivatives
\item Riemann curvature tensor $R_{\mu\nu\rho\sigma} \in C^{0,\delta}$
\item Spinor bundles via Clifford algebra and spin connections
\item Dirac operator $\slashed{D}$ with well-defined pointwise action
\end{itemize}

This completes the regularity bootstrap to a level sufficient for all geometric and quantum field theoretic constructions. $\qed$
