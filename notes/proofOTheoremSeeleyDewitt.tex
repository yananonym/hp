% proofThmSeeleyDewitt.tex
% Proof content


\textbf{Proof of Theorem \ref{thm:seeleyDewitt}}

The heat kernel on a pre-metric Ahlfors-regular space admits a short-time asymptotic expansion. the compute the first five Seeley-DeWitt coefficients explicitly.

\textit{\underline{Part (i): Heat Kernel Asymptotic Expansion}}

By Theorem \ref{thm:heatKernelAsymptotics}, the heat kernel $p_t(x,y)$ satisfies:
\[
p_t(x,y) \sim (4\pi t)^{-Q/2} e^{-d(x,y)^2/(4t)} \sum_{n=0}^\infty a_n(x,y) t^n \quad \text{as } t \to 0^+
\]
where $Q$ is the spectral dimension (Theorem \ref{thm:dimensionUniquenessStrengthened}) and $a_n(x,y)$ are smooth functions determined by local geometry. The coefficients $a_n$ are invariant under the emerged metric structure.

\textit{\underline{Part (ii): Leading Coefficient $a_0(x,y)$}}

The leading coefficient is universal:
\[
a_0(x, y) = 1.
\]

This follows from the normalization of the heat kernel fundamental solution and dimensional analysis. For the Laplacian $\Delta$ acting on scalar functions, the Euclidean heat kernel $e^{-t|\xi|^2}$ has leading order $(4\pi t)^{-Q/2}$, which carries unit normalization.

\textit{\underline{Part (iii): First Correction Coefficient $a_1(x,y)$}}

The coefficient $a_1(x,y)$ arises from the Ricci curvature of the emerged Riemannian structure. It satisfies:
\[
a_1(x,y) = \frac{1}{6} \text{Ric}(x) + \text{Ric}(y) + O(d(x,y))
\]
where $\text{Ric}(x)$ is the Ricci scalar evaluated at $x$. More precisely:
\[
a_1(x,x) = \frac{1}{6} \text{Ric}(x).
\]

This coefficient encodes the mean curvature of geodesic balls. By Bochner's formula applied to eigenfunctions $\phi_k$ of the Laplacian:
\[
\Delta |\nabla \phi_k|^2 = 2 \nabla \phi_k \cdot \nabla(\Delta \phi_k) + 2 |\nabla^2 \phi_k|^2 + 2 \text{Ric}(\nabla \phi_k, \nabla \phi_k),
\]
the Ricci term contributes to the next-order asymptotic expansion of the heat kernel.

\textit{\underline{Part (iv): Second Order Coefficient $a_2(x,y)$}}

The coefficient $a_2(x,y)$ involves second-order geometric invariants. It has the form:
\[
a_2(x,x) = \frac{1}{180}(5|\text{Ric}|^2 - 2R^{ijkl}R_{ijkl}) + \frac{1}{180}\text{div}(\text{Ric} \cdot \text{grad}(\cdot))
\]
where $R^{ijkl}R_{ijkl}$ is the full Riemann tensor contraction. This arises from:
\begin{enumerate}[label=(\alph*)]
\item Second derivatives of the metric (curvature derivatives)
\item Products of Ricci tensors (quadratic curvature invariants)
\item Divergence of first-order geometric quantities
\end{enumerate}

The proof uses the existence theorem for parametrices of parabolic operators (Friedman 1964) and microlocal analysis.

\textit{\underline{Part (v): Higher Order Coefficients $a_3(x,y)$ and $a_4(x,y)$}}

The coefficients $a_3$ and $a_4$ involve progressively higher-order combinations of curvature tensors and their covariant derivatives. The structure is:
\[
a_n(x,x) = \sum_{I} c_I(Q) \mathcal{I}_I(x)
\]
where the sum runs over all dimension-independent scalar curvature invariants $\mathcal{I}_I$ of order $2n$, with universal constants $c_I(Q)$ depending only on the spectral dimension.

Explicitly, $a_3$ involves:
\begin{itemize}
\item Contractions involving $\nabla_i R_{jk\ell m}$
\item Cubic products of Ricci tensors
\item Divergence of quadratic curvature expressions
\end{itemize}

And $a_4$ involves fourth-order invariants from iterating the heat equation:
\[
\frac{\partial p_t}{\partial t} = -\Delta p_t.
\]

Taking derivatives of both sides and expanding in powers of $t$ yields constraints on $a_n(x,y)$ through the recursion:
\[
\frac{da_{n-1}}{dt} + \Delta a_{n-1} = a_n \quad \text{(when integrated in heat equation)}.
\]

\textit{\underline{Part (vi): Adaptation to Ahlfors-Regular Framework}}

In the pre-metric (Polish space) setting, the emerged metric structure (Theorem \ref{thm:su3CTrialityEmergence}) provides the local geometry needed for these curvature expressions. The divergence-first axiomatics ensure that:

\begin{enumerate}[label=(\alph*)]
\item The Dirichlet form $\mathcal{E}$ (Definition \ref{def:dirichletForm}) generates a unique heat flow
\item The heat kernel exists and is unique (Theorem \ref{thm:heatKernelExistence})
\item Curvature tensors emerge naturally from the Dirichlet form structure via the Carre du Champ $\Gamma$ (Definition \ref{def:carreDuChamp})
\item The expansions $a_n$ are intrinsic to the divergence structure, not dependent on metric coordinates
\end{enumerate}

The Seeley-DeWitt coefficients computed from the emerged metric coincide with the intrinsic spectral invariants defined through heat kernel asymptotics, by the uniqueness theorem for heat kernel parametrices.

\textit{\underline{Part (vii): Convergence and Error Bounds}}

The asymptotic expansion converges in operator norm on compact subsets of the heat equation parameter space:
\[
\left\| p_t(x,y) - (4\pi t)^{-Q/2} e^{-d(x,y)^2/(4t)} \sum_{n=0}^N a_n(x,y) t^n \right\| \leq C_{N} t^{N+1}
\]
for $t \in (0, T_0)$ and uniformly in $x, y \in X$ with $d(x,y) \leq c_0 t^{1/2}$.

This error bound follows from:
\begin{enumerate}[label=(\alph*)]
\item The parametrix construction (Friedman 1964) giving remainder estimates
\item Induction on the order of approximation
\item The Polish space regularity (Lemma \ref{lem:polishConsequences}) ensuring uniform bounds
\item Exponential decay of eigenfunctions away from the diagonal
\end{enumerate}

\qed
