% Part of sectionUTheWorldRecapitulation.tex
\subsection{Scope and Remaining Directions}
\label{subsec:scopeAndDirections}

From two minimal axioms, the framework successfully derives four-dimensional Lorentzian spacetime with correct signature and dimensionality, quantum mechanics through path integral formulation, general relativity as one-loop effective action, the Standard Model gauge group from anomaly cancellation, exactly three fermion generations from topological consistency, the Yang-Mills mass gap and existence, and ultraviolet finiteness through asymptotic safety.

Several directions deserve further investigation and development. The explicit construction of black hole solutions within the emergent geometry requires detailed analysis of divergence functional behavior in strong-field regimes. The semiclassical approximation is established in principle but computationally involved. Higher-loop quantum corrections beyond one-loop deserve systematic analysis to verify ultraviolet finiteness persists at two-loop and three-loop orders. Neutrino masses and mixing angles should emerge from instanton physics in the electroweak sector, though explicit computation of instanton moduli spaces remains incomplete. The roles of complex phases in Yukawa couplings and their relationship to CP violation and the strong CP problem require deeper investigation of the divergence structure's complex geometry. Baryogenesis, the matter-antimatter asymmetry, must be explained through divergence structure properties in the early universe. Cosmological initial conditions remain to be analyzed, including whether the divergence structure predicts specific initial conditions or low entropy initial states. Experimental predictions require identification of energy scales where the ultraviolet fixed point produces deviations from Standard Model predictions.

