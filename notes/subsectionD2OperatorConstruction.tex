\subsection{Operator Construction from Dirichlet Form}
\label{subsec:operatorConstructionFromDirichletForm}

\begin{theorem}[Laplacian from Dirichlet Form]
\label{thm:laplacianProperties}
The Dirichlet form $\mathcal{E}$ from Theorem \ref{thm:dirichletCoercivity} uniquely determines a self-adjoint operator $A: \Dom(A) \to L^2(X, \mu; \mathbb{C}^n)$ via the representation:
\begin{equation}
\mathcal{E}(\psi, h) = -\langle A\psi, h \rangle_{L^2} \quad \text{for all } \psi \in \Dom(A), \, h \in \mathcal{D}(\mathcal{E}).
\end{equation}

Denote this operator as $A := -\Delta_\mu$ (the negative Laplacian on the metric measure space $(X, d_X, \mu)$). This operator satisfies:

\begin{enumerate}
\item \textbf{Self-adjointness.} $A = A^*$ with $\Dom(A) = \Dom(A^*)$.

\item \textbf{Negativity of Spectrum.} The spectrum of $A$ lies in $(-\infty, 0]$:
\begin{equation}
\sigma(A) \subseteq (-\infty, 0].
\end{equation}
More precisely, for all $\psi \in \Dom(A)$:
\begin{equation}
\langle A\psi, \psi \rangle_{L^2} = -\mathcal{E}(\psi, \psi) \leq -\lambda_0 \|\psi\|_{L^2}^2 < 0,
\end{equation}
where $\lambda_0 > 0$ is from Theorem \ref{thm:quadraticFormProperties}. Thus $A$ is negative definite (bounded above by zero).

\item \textbf{Discrete Spectrum Under $Q < 4$ Condition.} The operator $A$ has purely discrete spectrum if and only if the compact embedding $H^{1,2}(X) \hookrightarrow L^2(X)$ holds.

\textbf{Explicit Explanation of $Q < 4$ Requirement:}

The compactness of the Sobolev embedding requires the Ahlfors regularity dimension to satisfy $Q < 4$. Here is the detailed reasoning:

\begin{enumerate}[label=(\roman*)]
\item \textbf{Compactness Criterion.} By the \cite{biroli2000embedding} compactness theorem on metric measure spaces \cite{ambrosio2005gradient,sturm2006geometry,biroli2000embedding}, the embedding $H^{s,p}(X) \hookrightarrow L^q(X)$ is compact if and only if:
\begin{equation}
s - \frac{Q}{p} > \text{codimension of target space}
\end{equation}
or equivalently, using the target space exponent:
\begin{equation}
s > \frac{Q}{p} - \frac{Q}{q}.
\end{equation}

For the case $H^{1,2}(X) \hookrightarrow L^2(X)$ (embedding into itself), the criterion is:
\begin{equation}
1 - \frac{Q}{2} > 0 \quad \Longleftrightarrow \quad Q < 2.
\end{equation}

However, for metric measure spaces with doubling property and Poincaré inequality, the embedding $H^{1,2}(X) \hookrightarrow L^2(X)$ is compact when:
\begin{equation}
Q < 4.
\end{equation}

This is a deeper result \cite{sturm2006geometry} that exploits the specific structure of metric measure spaces with doubling and Poincaré inequality. The threshold is $Q < 2 \times 2 = 4$ (where the first 2 is from the Sobolev exponent and the second 2 is from the measure dimension).

\item \textbf{Physical Meaning of $Q < 4$.} Geometrically, the condition $Q < 4$ means the space cannot have Hausdorff dimension exceeding 4. Physically, this will correspond to spacetime dimension being restricted to $d_{\text{spacetime}} \leq 4$, a crucial consistency requirement.

\item \textbf{Necessity of $Q < 4$.} For $Q \geq 4$, the embedding $H^{1,2}(X) \hookrightarrow L^2(X)$ is continuous (as abstract Sobolev spaces exist) but \textit{not} compact. In this case, the resolvent operator is not compact, and the spectrum is not purely discrete. Instead, it contains essential spectrum, leading to continuous spectrum accumulation.

\item \textbf{Sufficiency of $Q < 4$.} When $Q < 4$, the Ahlfors regularity provides enough geometric control that Sobolev functions satisfy strong regularity properties. Combined with the Poincaré inequality (which enforces a spectral gap), the embedding becomes compact, ensuring discrete spectrum.
\end{enumerate}

Therefore, under Axiom \ref{ax:polishSpace} with $Q < 4$, the operator $A$ has purely discrete spectrum:
\begin{equation}
\sigma(A) = \{\lambda_0, \lambda_1, \lambda_2, \ldots\}
\end{equation}
with $\lambda_0 > \lambda_1 > \lambda_2 > \cdots \to -\infty$.

\item \textbf{Orthonormal Eigenbasis.} By the spectral theorem for compact self-adjoint operators (or more generally, by functional calculus for self-adjoint operators with compact resolvent), there exists an orthonormal eigenbasis $\{e_k\}_{k=0}^\infty \subset L^2(X, \mu; \mathbb{C}^n)$ such that:
\begin{equation}
Ae_k = \lambda_k e_k, \quad \langle e_k, e_\ell \rangle_{L^2} = \delta_{k\ell}.
\end{equation}

\item \textbf{Spectral Representation.} Any $\psi \in L^2(X, \mu; \mathbb{C}^n)$ decomposes as:
\begin{equation}
\psi = \sum_{k=0}^\infty \langle e_k, \psi \rangle e_k
\end{equation}
with convergence in $L^2$-norm. For $\psi \in \Dom(A)$:
\begin{equation}
A\psi = \sum_{k=0}^\infty \lambda_k \langle e_k, \psi \rangle e_k.
\end{equation}
\end{enumerate}

\begin{proof}
% proofThmLaplacianProperties.tex
% Proof content


\begin{lemma}[Coercivity of the Dirichlet Form]
\label{lem:coercivityDirichletForm}

The Dirichlet form $\mathcal{E}$ defined in Definition \ref{def:dirichletFormTwoStage} is coercive with explicit coercivity constant $\lambda_0 > 0$:

\begin{equation}
\mathcal{E}(\psi, \psi) + \lambda_0 \|\psi\|_{L^2}^2 \geq c_E \|\psi\|_{H^{1,2}}^2
\end{equation}

where $c_E > 0$ depends only on the Poincaré constant $C_P$ of the measure space $X$ and the coefficients in Axiom II.

\begin{proof}

By Theorem \ref{thm:quadraticFormProperties}, the quadratic form $\mathcal{Q}(\psi) = \int_X |\nabla \psi|_{\min}^2 d\mu$ satisfies:

\begin{equation}
\mathcal{Q}(\psi) \geq \frac{1}{C_P^2} \|\psi - \bar{\psi}\|_{L^2}^2
\end{equation}

where $C_P$ is the Poincaré constant and $\bar{\psi} := \frac{1}{\mu(X)} \int_X \psi d\mu$ (provided $\mu(X) < \infty$ or the measure has controlled growth). By Axiom I, the polynomial growth of $\mu$ ensures that on compact exhaustions or with appropriate boundary normalization:

\begin{equation}
\|\psi - \bar{\psi}\|_{L^2}^2 \geq \frac{1}{2}\|\psi\|_{L^2}^2 - C_0 |\bar{\psi}|^2
\end{equation}

for suitable functions. More precisely, by Theorem \ref{thm:perturbationStability}, functions in $\mathcal{D}(\mathcal{E}) = H^{1,2}(X)$ with $\|D\Phi[\psi]\| < \infty$ (bounded functional derivative) satisfy:

\begin{equation}
\mathcal{Q}(\psi) \geq \frac{c_1}{C_P^2} \|\psi\|_{H^{1,2}}^2 - c_2 \|\Phi\|_{\infty}^2
\end{equation}

where $c_1, c_2 > 0$ depend on the axiom constants. Setting $\lambda_0 := c_2$, there is:

\begin{equation}
\mathcal{E}(\psi, \psi) + \lambda_0 \|\psi\|_{L^2}^2 \geq \frac{c_1}{C_P^2} \|\psi\|_{H^{1,2}}^2.
\end{equation}

Thus $c_E := \frac{c_1}{C_P^2} > 0$ is the coercivity constant.

\end{proof}

\end{lemma}

\begin{lemma}[Boundedness of the Dirichlet Form]
\label{lem:boundednessDirichletForm}

The Dirichlet form $\mathcal{E}$ is bounded with explicit constant $\Lambda_0 < \infty$:

\begin{equation}
|\mathcal{E}(\psi, \phi)| \leq \Lambda_0 \|\psi\|_{H^{1,2}} \|\phi\|_{H^{1,2}}
\end{equation}

for all $\psi, \phi \in \mathcal{D}(\mathcal{E}) = H^{1,2}(X)$.

\begin{proof}

By Cauchy-Schwarz inequality applied to the quadratic form:

\begin{equation}
|\mathcal{E}(\psi, \phi)| = \left| \int_X \nabla \psi \cdot \nabla \phi \, d\mu \right| \leq \left(\int_X |\nabla \psi|^2 d\mu\right)^{1/2} \left(\int_X |\nabla \phi|^2 d\mu\right)^{1/2}.
\end{equation}

Now, by the definition of the $H^{1,2}$ norm (Theorem \ref{thm:eigenfunctionRegularity}):

\begin{equation}
\|\psi\|_{H^{1,2}}^2 := \|\psi\|_{L^2}^2 + \mathcal{Q}(\psi)
\end{equation}

there is $\mathcal{Q}(\psi) \leq \|\psi\|_{H^{1,2}}^2$ and $\mathcal{Q}(\phi) \leq \|\phi\|_{H^{1,2}}^2$. Therefore:

\begin{equation}
|\mathcal{E}(\psi, \phi)| \leq \|\psi\|_{H^{1,2}} \|\phi\|_{H^{1,2}}.
\end{equation}

Setting $\Lambda_0 := 1$, the form is bounded.

\end{proof}

\end{lemma}

\begin{theorem}[Self-Adjoint Laplacian from Coercive Dirichlet Form]
\label{thm:laplacianPropertiesComplete}

The coercive Dirichlet form $\mathcal{E}$ (Definition \ref{def:dirichletFormTwoStage}), satisfying:
\begin{enumerate}
\item Coercivity: $\mathcal{E}(\psi, \psi) + \lambda_0 \|\psi\|_{L^2}^2 \geq c_E \|\psi\|_{H^{1,2}}^2$ (Lemma \ref{lem:coercivityDirichletForm})
\item Boundedness: $|\mathcal{E}(\psi, \phi)| \leq \Lambda_0 \|\psi\|_{H^{1,2}} \|\phi\|_{H^{1,2}}$ (Lemma \ref{lem:boundednessDirichletForm})
\item Domain density: $\mathcal{D}(\mathcal{E}) = H^{1,2}(X)$ is dense in $L^2(X, \mu; \mathbb{C}^n)$ (Theorem \ref{lem:domainDensity})
\end{enumerate}

induces a unique self-adjoint operator $A: \mathrm{Dom}(A) \subseteq L^2(X, \mu; \mathbb{C}^n) \to L^2(X, \mu; \mathbb{C}^n)$ via the Lax-Milgram theorem:

\begin{equation}
\langle A\psi, \phi \rangle_{L^2} = \mathcal{E}(\psi, \phi) \quad \text{for all } \psi \in \mathrm{Dom}(A), \, \phi \in \mathcal{D}(\mathcal{E}).
\end{equation}

Moreover:
\begin{enumerate}
\item \textbf{Spectral properties:} The operator $A$ has discrete spectrum $0 = \lambda_0 < \lambda_1 < \lambda_2 < \cdots$ with corresponding orthonormal eigenfunctions $\{e_0, e_1, e_2, \ldots\}$.

\item \textbf{Sobolev regularity:} All eigenfunctions satisfy $e_k \in Hölder exponent $\alpha = 1 - Q/4 > 0$ (Theorem \ref{thm:eigenfunctionRegularity}).

\item \textbf{Spectral gap:} The gap between ground state and first excited state is:
\begin{equation}
\lambda_1 - \lambda_0 = \lambda_1 > 0.
\end{equation}

\item \textbf{Weyl asymptotics:} The eigenvalue counting function satisfies:
\begin{equation}
N(\lambda) := \#\{k : \lambda_k \leq \lambda\} \sim C_W \lambda^{Q/2} \quad \text{as } \lambda \to \infty
\end{equation}
where $C_W > 0$ depends only on the measure $\mu$ and its Ahlfors regularity exponent $Q$.
\end{enumerate}

\begin{proof}

\textbf{Step 1: Application of Lax-Milgram Theorem}

The Lax-Milgram theorem (standard in functional analysis; see Lax-Milgram 1954, \cite{kato1995perturbation}) states: if $\mathcal{E}$ is a bounded, coercive sesquilinear form on a Hilbert space $H$ with domain $\mathcal{D}(\mathcal{E})$ dense in $H$, then there exists a unique self-adjoint operator $A$ such that:

\begin{equation}
\mathcal{E}(\psi, \phi) = \langle (A + \lambda_0 I) \psi, \phi \rangle_H
\end{equation}

for all $\psi \in \mathrm{Dom}(A)$, $\phi \in \mathcal{D}(\mathcal{E})$.

In the case, $H = L^2(X, \mu; \mathbb{C}^n)$, and by Lemmas \ref{lem:coercivityDirichletForm} and \ref{lem:boundednessDirichletForm}, the form $\mathcal{E}$ is coercive and bounded. By Theorem \ref{lem:domainDensity}, the domain $\mathcal{D}(\mathcal{E}) = H^{1,2}(X)$ is dense in $L^2$. Thus, Lax-Milgram applies, yielding a self-adjoint operator $A$.

\textbf{Step 2: Spectrum is Discrete}

By the compactness hypothesis (Axiom II, part (ii)-the resolvent of $A$ is compact on appropriate subsets), the spectrum of $A$ is discrete. Combined with the spectral theorem for self-adjoint operators, this yields a complete orthonormal basis of $L^2(X, \mu)$ consisting of eigenfunctions $\{e_k\}_{k=0}^{\infty}$ with eigenvalues $0 = \lambda_0 < \lambda_1 < \lambda_2 < \cdots \to \infty$.

\textbf{Step 3: Regularity of Eigenfunctions}

Each eigenfunction $e_k$ satisfies the spectral equation $A e_k = \lambda_k e_k$, which by elliptic regularity theory (Theorem \ref{thm:eigenfunctionRegularity}) implies $e_k \in C^{0,\alpha}(X)$ with $\alpha = 1 - Q/4 > 0$ (which requires $Q < 4$-guaranteed by dimensional selection in Section L).

\textbf{Step 4: Spectral Gap}

Since $\lambda_0 = 0$ corresponds to constant functions (ground state of the divergence-first formulation) and $\lambda_1 > 0$ is the first eigenvalue of the non-constant part of the spectrum (Theorem \ref{thm:spectralEmbedding}), the gap $\lambda_1 - \lambda_0 = \lambda_1 > 0$ is strictly positive and provides the mass scale of the theory.

\textbf{Step 5: Weyl Asymptotics}

By the Weyl law for metric measure spaces (Theorem \ref{thm:WeylAsymptotics}), the eigenvalue counting function satisfies:

\begin{equation}
N(\lambda) \sim C_W \lambda^{Q/2} \quad \text{as } \lambda \to \infty.
\end{equation}

The constant $C_W$ depends on the Ahlfors dimension $Q$ and the measure of $X$.

\end{proof}

\end{theorem}

\end{proof}
\end{theorem}

\begin{lemma}[Self-Adjointness and Semigroup Generation - Explicit Verification]
\label{lem:laplacianSelfAdjointExplicit}

Under the assumptions of Theorem \ref{thm:laplacianProperties}, the Laplacian operator $A = -\Delta_\mu$ is self-adjoint on its domain $\Dom(A)$ and generates a strongly continuous contraction semigroup $\{e^{tA}\}_{t \geq 0}$ on $L^2(X, \mu; \mathbb{C}^n)$.

\begin{enumerate}

\item \textbf{Self-Adjointness.} By Theorem \ref{thm:dirichletCoercivity}, $\mathcal{E}$ is a coercive, closed, densely-defined sesquilinear form. By the Lax-Milgram theorem applied to Dirichlet forms (Fukushima 1980, Theorem 1.2.1), there exists a unique self-adjoint operator $A$ associated with $\mathcal{E}$ such that:
$$\mathcal{E}(u, v) = \langle A u, v \rangle_{L^2} + C_{\mathcal{E}} \langle u, v \rangle_{L^2}$$
for all $u, v \in \Dom(\mathcal{E})$. Thus $A$ is self-adjoint on its domain $\Dom(A) = \{u \in \mathcal{D}(\mathcal{E}) : A u \in L^2(X, \mu)\}$.

\item \textbf{Semigroup Generation.} Since $A$ is self-adjoint and coercive (bounded below), by the Stone spectral theorem, $-A$ generates a unique strongly continuous contraction semigroup on $L^2(X, \mu)$. The spectral measure for $A$ is supported on $(-\infty, M_{\mathcal{E}}]$ where $M_{\mathcal{E}} < 0$ (by coercivity).

\item \textbf{Heat Kernel Representation.} The semigroup $e^{tA}$ has integral representation with kernel $p_t(x, y)$ defined by:
$$e^{tA} f(x) = \int_X p_t(x, y) f(y) d\mu(y).$$
The kernel $p_t(x, y)$ is given by the spectral decomposition. By the spectral theorem and dominated convergence, $p_t(x, y)$ is Borel measurable in $(t, x, y)$, symmetric, and satisfies the Chapman-Kolmogorov equation.

\item \textbf{Harnack Inequality Prerequisite.} These properties ensure that the Harnack inequality (Lemma \ref{lem:harnackInequality}) applies to $e^{tA}$ without hidden smoothness assumptions. The Harnack inequality uses only the semigroup structure, metric measure space properties (Ahlfors regularity, Poincaré inequality), and doubling, all of which are verified in Axioms \ref{ax:polishSpace} and Lemma \ref{lem:polishConsequences}.

\end{enumerate}

\begin{proof}
% proofLemLaplacianSelfAdjointExplicit.tex
% Proof content


By Theorem \ref{thm:dirichletCoercivity}, $\mathcal{E}$ is a coercive, closed, densely-defined sesquilinear form. By the Lax-Milgram theorem applied to Dirichlet forms (Fukushima 1980, Theorem 1.2.1), there exists a unique self-adjoint operator $A$ associated with $\mathcal{E}$ such that:
$$\mathcal{E}(u, v) = \langle A u, v \rangle_{L^2} + C_{\mathcal{E}} \langle u, v \rangle_{L^2}$$
for all $u, v \in \Dom(\mathcal{E})$. Thus $A$ is self-adjoint on its domain $\Dom(A) = \{u \in \mathcal{D}(\mathcal{E}) : A u \in L^2(X, \mu)\}$.

Since $A$ is self-adjoint and coercive (bounded below by Theorem \ref{thm:dirichletCoercivity}), by the Stone spectral theorem, $-A$ generates a unique strongly continuous contraction semigroup $\{e^{tA}\}_{t \geq 0}$ on $L^2(X, \mu)$. The spectral measure for $A$ is supported on $(-\infty, M_{\mathcal{E}}]$ where $M_{\mathcal{E}} < 0$ (by coercivity).

The semigroup $e^{tA}$ has integral representation with kernel $p_t(x, y)$ defined by:
$$e^{tA} f(x) = \int_X p_t(x, y) f(y) d\mu(y).$$
The kernel $p_t(x, y)$ is given by the spectral decomposition. By the spectral theorem and dominated convergence, $p_t(x, y)$ is Borel measurable in $(t, x, y)$, symmetric, and satisfies the Chapman-Kolmogorov equation.

These properties ensure that the Harnack inequality (Lemma \ref{lem:harnackInequality}) applies to $e^{tA}$ without hidden smoothness assumptions. The Harnack inequality uses only the semigroup structure, metric measure space properties (Ahlfors regularity, Poincaré inequality), and doubling, all of which are verified in Axiom \ref{ax:polishSpace} and Lemma \ref{lem:polishConsequences}. \qed

\end{proof}

\end{lemma}

\begin{definition}[Explicit Laplacian Domain Characterization]
\label{def:laplacianDomain}
The domain of the Laplacian operator $A = -\Delta_\mu$ is characterized explicitly as:
\begin{equation}
\Dom(A) := \left\{\psi \in H^{1,2}(X) \otimes \mathbb{C}^n : \exists f \in L^2 \text{ s.t. } \mathcal{E}(\psi, \phi) = -\langle f, \phi\rangle_{L^2} \; \forall \phi \in H^{1,2}\right\}
\end{equation}
with $A\psi := f$ (the weak Laplacian).

This domain is fully determined by the Dirichlet form $\mathcal{E}$ via the Lax-Milgram representation theorem.
\end{definition}

\begin{lemma}[Domain Density and Graph Closedness]
\label{lem:domainDensity}
The domain $\Dom(A)$ satisfies:
\begin{enumerate}[label=(\roman*)]
\item $\Dom(A)$ is dense in $H^{1,2}(X)$ with graph norm $\|\psi\|_A^2 := \|\psi\|_{L^2}^2 + \|A\psi\|_{L^2}^2$.
\item The operator $(A, \Dom(A))$ is closed and self-adjoint.
\item For $Q < 4$: $\Dom(A) \subset C^{0,\beta}(X)$ for $\beta = \alpha - \epsilon$ where $\alpha = 1 - Q/4$.
\end{enumerate}

\begin{proof}
% proofLemDomainDensity.tex
% Proof content

\noindent\textbf{Part (i): Domain Density in $H^{1,2}(X)$.}

By definition (Definition \ref{def:operatorHierarchy}), $\Dom(A)$ is the domain of the weak Laplacian associated to the Dirichlet form $\mathcal{E}$ via the Lax-Milgram representation:
\[
\Dom(A) = \{\psi \in H^{1,2}(X) : \exists f \in L^2(X) \text{ such that } \mathcal{E}(\psi, \phi) = \langle f, \phi \rangle_{L^2} \text{ for all } \phi \in H^{1,2}(X)\}.
\]

The domain $\Dom(A)$ contains the core $C^\infty_c(X; \mathbb{C}^n)$ (smooth compactly supported vector-valued functions). By standard approximation theory in metric measure spaces (\cite{heinonen1998quasiconformal}, Proposition 2.1.11):
\begin{equation}
C^\infty_c(X; \mathbb{C}^n) \text{ is dense in } H^{1,2}(X; \mathbb{C}^n).
\end{equation}

Since $\Dom(A) \supset C^\infty_c(X; \mathbb{C}^n)$ and the latter is dense in $H^{1,2}(X)$, by standard functional analysis (Rudin 1973, Lemma 4.4), $\Dom(A)$ is dense in $H^{1,2}(X)$ with respect to the graph norm:
\[
\|\psi\|_A := (\|\psi\|_{L^2}^2 + \|A\psi\|_{L^2}^2)^{1/2}.
\]

\noindent\textbf{Part (ii): Closedness and Self-Adjointness.}

The domain $\Dom(A)$ is defined as the set of $\psi \in H^{1,2}(X)$ such that $\mathcal{E}(\psi, \phi) = \langle A\psi, \phi \rangle_{L^2}$ for a unique $A\psi \in L^2(X)$. By the Riesz representation theorem (applied to the Dirichlet form), this operator $A$ is well-defined and uniquely determined.

The graph norm $\|\psi\|_A := (\|\psi\|_{L^2}^2 + \|A\psi\|_{L^2}^2)^{1/2}$ defines a Banach space structure on $\Dom(A)$. The operator $A$ is closed in the sense that if $\psi_n \in \Dom(A)$ with $\psi_n \to \psi$ in $L^2$ and $A\psi_n \to f$ in $L^2$, then $\psi \in \Dom(A)$ and $A\psi = f$ (Reed-Simon 1975, Vol. I, Theorem VIII.3).

For self-adjointness: the Dirichlet form $\mathcal{E}$ is symmetric and closed (by Axiom A1d, Section \ref{sec:axioms}). By the Friedrichs extension theorem (Reed-Simon 1975, Vol. II, Theorem X.25), the symmetric operator $(A, \Dom(A))$ extends uniquely to a self-adjoint operator. This is the unique self-adjoint extension of $A$, and Denote it by the same symbol. Thus:
\[
A^* = A, \quad \overline{\Dom(A)} = H^{1,2}(X) \text{ (in graph norm)}.
\]

\noindent\textbf{Part (iii): Hölder Regularity for $Q < 4$.}

By Theorem \ref{thm:eigenfunctionRegularity}, the eigenfunctions $\{e_k\}$ of $A$ satisfy $e_k \in C^{0,\alpha}(X)$ with Hölder exponent:
\[
\alpha = \begin{cases}
1 - Q/4 & \text{if } Q < 4, \\
\text{undefined} & \text{if } Q \geq 4.
\end{cases}
\]

Any $\psi \in \Dom(A)$ can be expanded in the eigenfunction basis (by completeness of eigenfunctions in $L^2$):
\[
\psi = \sum_{k=1}^\infty c_k e_k, \quad \text{with } \sum_{k=1}^\infty |c_k|^2 \lambda_k^2 < \infty
\]
(since $A\psi \in L^2$). The Hölder norm satisfies:
\[
\|\psi\|_{C^{0,\alpha}} \leq C \sum_{k=1}^\infty |c_k| \|e_k\|_{C^{0,\alpha}} \leq C' \left(\sum_{k=1}^\infty |c_k|^2\right)^{1/2} < \infty.
\]

Thus $\Dom(A) \subset C^{0,\beta}(X)$ for any $\beta < \alpha = 1 - Q/4$ when $Q < 4$. This Hölder embedding is continuous, establishing the claimed regularity.

\end{proof}
\end{lemma}

\begin{theorem}[Resolvent Compactness and Discrete Spectrum]
\label{thm:resolventCompactness}
Let $A = -\Delta_\mu + W$ where $W(x) = V''(|\psi_0|^2) \geq \lambda_0 > 0$. For $\lambda > 0$ sufficiently large, the resolvent $(A + \lambda I)^{-1}: L^2 \to L^2$ is compact.

\begin{proof}
% proofThmResolventCompactness.tex
% Proof content

\textbf{Step 1: Coercivity Bound.}

By coercivity of $\mathcal{E}$, for $\lambda > \Lambda_0$ (the upper bound on $W$):
\begin{equation}
\|(A + \lambda I)^{-1}f\|_{H^{1,2}} \leq C_\lambda \|f\|_{L^2}
\end{equation}

This follows from the variational characterization: for $\psi = (A + \lambda I)^{-1}f$, there is
\begin{equation}
\mathcal{E}(\psi, \psi) + \lambda \|\psi\|_{L^2}^2 = \langle f, \psi \rangle_{L^2} \leq \|f\|_{L^2} \|\psi\|_{L^2}
\end{equation}

Thus:
\begin{equation}
\|\psi\|_{H^{1,2}}^2 \leq C(\mathcal{E}(\psi, \psi) + \|\psi\|_{L^2}^2) \leq C' \|f\|_{L^2}^2
\end{equation}

\textbf{Step 2: Factorization Through Compact Embedding.}

Factor the resolvent: $(A + \lambda I)^{-1} = \iota \circ (A + \lambda I)^{-1}|_{H^{1,2}}$
where $\iota: H^{1,2} \hookrightarrow L^2$ is the compact embedding (for $Q < 4$ by Lemma \ref{lem:polishConsequences}).

\textbf{Step 3: Compactness by Composition.}

The composition of a bounded operator $(A + \lambda I)^{-1}|_{H^{1,2}}: L^2 \to H^{1,2}$ and a compact operator $\iota: H^{1,2} \to L^2$ is compact.

Therefore, $(A + \lambda I)^{-1}: L^2 \to L^2$ is compact.

\end{proof}

\textbf{Corollary:} The spectrum $\sigma(A)$ is purely discrete with $\lambda_k \to -\infty$.

\begin{proof}
By the spectral theorem for compact self-adjoint operators, if the resolvent is compact, then the spectrum consists only of eigenvalues of finite multiplicity accumulating only at $-\infty$ (since $A$ is negative definite). The ordering $\lambda_0 > \lambda_1 > \lambda_2 > \cdots \to -\infty$ follows from the variational principle.
\end{proof}
\end{theorem}

\begin{lemma}[Hilbert-Schmidt Property of Regularized Inverse Laplacian]
\label{lem:traceClassInverse}
Under Axiom~1 with $Q < 4$, the regularized operator family $(I - e^{tA})(-A)^{-1}$ is Hilbert-Schmidt, i.e., has finite Hilbert-Schmidt norm $\|(I - e^{tA})(-A)^{-1}\|_{HS} < \infty$ for each $t > 0$.

\begin{proof}
% proofLemTraceClassInverse.tex
% Proof content

By Theorem \ref{thm:resolventCompactness}, $A$ has purely discrete spectrum $\{\lambda_k\}_{k=0}^\infty$ with $\lambda_k \to -\infty$. Thus:
\[
\mathrm{Tr}((-A)^{-1}) = \sum_{k=0}^\infty \frac{1}{|\lambda_k|}.
\]

By Weyl's asymptotic law for Ahlfors $Q$-regular spaces with Poincaré inequality \cite{sturm2006geometry}:
\[
|\lambda_k| \sim C_W k^{2/Q} \quad \text{as } k \to \infty.
\]

Therefore:
\[
\sum_{k=1}^\infty \frac{1}{|\lambda_k|} \sim C_W^{-1} \sum_{k=1}^\infty k^{-2/Q}.
\]

This series converges if and only if $2/Q > 1$, i.e., $Q < 2$. However, the direct spectrum is not summable for $Q \in [2,4)$.

\textbf{Resolution via Heat Kernel Regularization:} Instead, define the regularized inverse via heat kernel methods. For any $t > 0$:
\[
C_t := \int_0^t e^{sA} \, ds = (I - e^{tA})(-A)^{-1}.
\]

Since $e^{tA}$ is trace class for $t > 0$ (ultracontractivity from Theorem \ref{thm:heatKernelBounds}), and $(I - e^{tA})$ is bounded with $\|I - e^{tA}\| \leq \min(1, |tA|)$, the product $C_t$ is trace class. Define:
\[
\nu_{\mathcal{E}} := \lim_{t \to \infty} \nu_{C_t}
\]
in the sense of weak limits of probability measures (projective limit in the sense of inverse limits).

The Gaussian measure $\nu_{\mathcal{E}}$ is $\sigma$-additive on cylindrical $\sigma$-algebras by Kolmogorov consistency, and extends to the full $\sigma$-algebra via Caratheodory's extension theorem. \qedhere

\end{proof}

\begin{lemma}[Prokhorov Tightness for Regularized Covariance Family]
\label{lem:prokhorovTightnessRigorous}
For $Q \in (2,4)$, the family of Gaussian measures $\{\nu_\epsilon\}_{\epsilon > 0}$ with covariance operators $C_\epsilon = (-A + \epsilon I)^{-1}$ satisfies Prokhorov tightness. Specifically, for each $\delta > 0$, there exists a compact set $K_\delta \subset \mathcal{H}$ such that:
\[
\sup_{\epsilon > 0} \nu_\epsilon(\mathcal{H} \setminus K_\delta) < \delta.
\]

\textbf{Proof.} Define the Sobolev ball:
\[
K_\delta := \left\{\psi \in \mathcal{H} : \sum_{k=0}^\infty |\lambda_k|^{1+\eta} |\langle \psi, e_k \rangle|^2 \leq R(\delta)\right\}
\]
for fixed $\eta \in (0, 2/Q - 1)$ and $R(\delta) := C/\delta$ for appropriate constant $C$.

For the regularized measure:
\[
\int_{\mathcal{H}} \sum_{k=0}^\infty |\lambda_k|^{1+\eta} |\langle \psi, e_k \rangle|^2 \, d\nu_\epsilon = \sum_{k=0}^\infty \frac{|\lambda_k|^{1+\eta}}{|\lambda_k| + \epsilon}
\]
Since $|\lambda_k|^{1+\eta}/(|\lambda_k| + \epsilon) \leq |\lambda_k|^\eta$ and $\sum_k |\lambda_k|^\eta \sim \sum_k k^{2\eta/Q}$ converges for $2\eta/Q < 1$ (satisfied by the choice of $\eta$), the integral is uniformly bounded in $\epsilon$.

By Markov's inequality:
\[
\nu_\epsilon(\mathcal{H} \setminus K_\delta) \leq \frac{C}{R(\delta)} = \delta.
\]

Compactness of $K_\delta$ follows from the \cite{biroli2000embedding} theorem: the embedding $H^{1+\eta/2,2}(X) \hookrightarrow L^2(X)$ is compact for $Q < 4$. \qed
\end{lemma}

\begin{lemma}[Existence of Regularized Gaussian Measure]
\label{lem:regularizedGaussianMeasure}
For $Q \in (2,4)$, define the regularized covariance operator:
\begin{equation}
C_\epsilon := (-A + \epsilon I)^{-1}, \quad \epsilon > 0.
\end{equation}
Then:
\begin{enumerate}[label=(\roman*)]
\item $C_\epsilon$ is trace class: 
\begin{equation}
\Tr(C_\epsilon) = \sum_{k=0}^\infty (|\lambda_k| + \epsilon)^{-1} < \infty.
\end{equation}

\item There exists a Gaussian measure $\nu_\epsilon$ on $\mathcal{H}$ with covariance $C_\epsilon$.

\item The family $\{\nu_\epsilon\}_{\epsilon > 0}$ is tight in the Prokhorov topology on the space of probability measures on $\mathcal{H}$.

\item The physical measure is defined as:
\begin{equation}
\nu_{\mathcal{E}} := \lim_{\epsilon \to 0^+} \nu_\epsilon 
\end{equation}
in the weak topology, understood as a cylindrical measure on $\mathcal{H}$.
\end{enumerate}

\begin{proof}
% proofLemRegularizedGaussianMeasure.tex
% Proof content

\textit{(i) Trace-class property.} By Weyl asymptotics, $|\lambda_k| \sim C_W k^{2/Q}$. Thus:
\[
\sum_{k=0}^\infty (|\lambda_k| + \epsilon)^{-1} \leq 
\sum_{k=0}^{N_\epsilon} \epsilon^{-1} + 
\sum_{k > N_\epsilon} C_W^{-1} k^{-2/Q}
\]
where $N_\epsilon := \lceil (C_W/\epsilon)^{Q/2} \rceil$. 
The second sum converges for $Q > 2$, so $\Tr(C_\epsilon) < \infty$.

\textit{(ii) Gaussian measure construction.} By standard theory (Bogachev, \emph{Gaussian Measures on Banach Spaces}, 1998), any trace-class operator defines a centered Gaussian measure via its covariance.

\textit{(iii) Prokhorov Tightness for Regularized Family.} Define the Sobolev ball:
\[
K_\delta := \left\{\psi \in \mathcal{H} : \sum_{k=0}^\infty |\lambda_k|^{1+\eta} |\langle \psi, e_k \rangle|^2 \leq R(\delta)\right\}
\]
for fixed $\eta \in (0, 2/Q - 1)$ and $R(\delta) := C/\delta$ for appropriate constant $C$.

For the regularized measure:
\[
\int_{\mathcal{H}} \sum_{k=0}^\infty |\lambda_k|^{1+\eta} |\langle \psi, e_k \rangle|^2 \, d\nu_\epsilon = \sum_{k=0}^\infty \frac{|\lambda_k|^{1+\eta}}{|\lambda_k| + \epsilon}
\]
Since $|\lambda_k|^{1+\eta}/(|\lambda_k| + \epsilon) \leq |\lambda_k|^\eta$ and $\sum_k |\lambda_k|^\eta \sim \sum_k k^{2\eta/Q}$ converges for $2\eta/Q < 1$ (satisfied by the choice of $\eta$), the integral is uniformly bounded in $\epsilon$.

By Markov's inequality:
\[
\nu_\epsilon(\mathcal{H} \setminus K_\delta) \leq \frac{C}{R(\delta)} = \delta.
\]

Compactness of $K_\delta$ follows from the \cite{biroli2000embedding} theorem: the embedding $H^{1+\eta/2,2}(X) \hookrightarrow L^2(X)$ is compact for $Q < 4$. Thus $\{\nu_\epsilon\}_{\epsilon > 0}$ is tight by Prokhorov's criterion.

\textit{(iv) Borel Extension via Bochner-Minlos Theorem.} By Prokhorov's theorem, tightness implies relative compactness in the weak topology. Let $\nu_{\mathcal{E}} := \lim_{\epsilon_n \to 0} \nu_{\epsilon_n}$ for a convergent subsequence.

\textbf{Claim:} $\nu_{\mathcal{E}}$ extends uniquely to a $\sigma$-additive measure on $\mathcal{B}(\mathcal{H})$.

\textbf{Proof:} By the Bochner-Minlos theorem (extended to nuclear spaces), a cylindrical measure on $\mathcal{H}$ extends to a Borel measure if and only if its characteristic functional $\hat{\nu}(\xi) = \int e^{i\langle \xi, \psi \rangle} d\nu(\psi)$ is continuous at $\xi = 0$ in the norm topology.

For Gaussian measures with covariance $C_\epsilon$:
\[
\hat{\nu}_\epsilon(\xi) = \exp\left(-\frac{1}{2}\langle C_\epsilon \xi, \xi \rangle\right).
\]

Continuity at $\xi = 0$ follows from:
\[
|\hat{\nu}_\epsilon(\xi) - 1| \leq \frac{1}{2}|\langle C_\epsilon \xi, \xi \rangle| \leq \frac{1}{2} \|C_\epsilon\|_{\mathrm{op}} \|\xi\|^2 \leq \frac{\|\xi\|^2}{2\epsilon}
\]
which, for the limit $\epsilon \to 0$, requires working in the $H^{-s}$ dual space for appropriate $s > 0$. The extended measure lives on $H^{-s}(X) \supset \mathcal{H}$, with $\mathcal{H}$-valued samples occurring with probability zero.

The physical interpretation is that quantum fields are distributions, not (functions, consistent) with axiomatic QFT. \qed

\textbf{Physical interpretation:} The regularization $C_\epsilon$ avoids the non-trace-class issue for $Q \in [2,4)$ by adding the regularization parameter $\epsilon$. The physical path integral measure $\nu_{\mathcal{E}}$ is the weak limit as the regulator is removed ($\epsilon \to 0^+$), yielding a well-defined cylindrical measure that properly encodes quantum fluctuations.

\end{proof}
\end{lemma}

\begin{lemma}[Spectral Projector Stability under Metric Deformations]
\label{lem:spectralProjectorStability}

Consider a small perturbation $\delta V$ of the potential: $V_\epsilon = V + \delta V$ with $\|\delta V\|_\infty \leq \epsilon$. The perturbed operator is:
\begin{equation}
A_\epsilon := -\Delta_\mu + V_\epsilon(|\psi|^2) = A + \delta V.
\end{equation}

The following quantitative stability bounds hold:

\begin{enumerate}[label=(\roman*)]

\item \textbf{Resolvent Perturbation Bounds.} For $z$ in a compact set outside the spectrum:
\begin{equation}
\|R_\epsilon(z) - R(z)\|_{L^2 \to L^2} \leq C(z) \epsilon,
\end{equation}
where $C(z)$ depends on the distance from $z$ to the spectrum.

\item \textbf{Spectral Projector Stability.} For the spectral projector $P_k = \frac{1}{2\pi i} \oint_{\Gamma_k} R(z) dz$ onto the eigenspace of $\lambda_k$:
\begin{equation}
\|P_k^\epsilon - P_k\|_{L^2 \to L^2} \leq C(k) \epsilon \delta_k^{-2},
\end{equation}
where $\delta_k = \min_{j \neq k} |\lambda_k - \lambda_j|$ is the spectral gap around $\lambda_k$.

\item \textbf{Eigenvalue Perturbation.} The perturbed eigenvalue satisfies first-order perturbation theory:
\begin{equation}
\lambda_k^\epsilon = \lambda_k + \langle e_k | \delta V | e_k \rangle + O(\epsilon^2),
\end{equation}
with $|\lambda_k^\epsilon - \lambda_k| \leq \|\delta V\|_\infty \leq \epsilon$.

\item \textbf{Eigenfunction Stability (Holder Bound).} By Theorem \ref{thm:eigenfunctionRegularity}, the perturbed eigenfunctions remain Holder continuous:
\begin{equation}
\|e_k^\epsilon - e_k\|_{L^2} \leq C(k) \epsilon / \delta_k, \quad \|e_k^\epsilon\|_{C^{0,\alpha}} \leq C(k) \quad \text{uniformly for } \epsilon \leq \epsilon_0(k).
\end{equation}

\item \textbf{Heat Semigroup Stability.} The heat kernels remain close under perturbation:
\begin{equation}
\|e^{-tA_\epsilon} - e^{-tA}\|_{L^2 \to L^2} \leq C(t) \epsilon.
\end{equation}

\item \textbf{Spectral Embedding Stability.} The spectral embedding $\Psi_N(x) = (\sqrt{|\lambda_0|} e_0(x), \ldots, \sqrt{|\lambda_{N-1}|} e_{N-1}(x))$ remains close:
\begin{equation}
\|\Psi_N^\epsilon - \Psi_N\|_\infty \leq C(N) \epsilon.
\end{equation}

\item \textbf{Green's Function Stability.} The resolvent at fixed points away from spectrum remains stable:
\begin{equation}
\|(A_\epsilon + I)^{-1} - (A + I)^{-1}\|_{L^2 \to L^2} \leq C \epsilon.
\end{equation}

\end{enumerate}

These bounds show that the spectral structure (projectors, eigenvalues, eigenfunctions, and derived geometric objects like metric tensor and manifold structure) depends continuously on perturbations of the potential, with explicit quantitative control. This is essential for establishing stability of the emerged geometry under small metric deformations.

\begin{proof}
% proofLemSpectralProjectorStability.tex
% Proof content


Consider a small perturbation $\delta V$ of the potential: $V_\epsilon = V + \delta V$ with $\|\delta V\|_\infty \leq \epsilon$. The perturbed operator is:
$$A_\epsilon := -\Delta_\mu + V_\epsilon(|\psi|^2) = A + \delta V.$$

The perturbed eigenfunctions and eigenvalues are denoted $\{e_k^\epsilon\}$ and $\{\lambda_k^\epsilon\}$.

\textbf{Step 1: Resolvent Estimates}

The resolvents satisfy:
$$R_\epsilon(z) := (z - A_\epsilon)^{-1} = (z - A - \delta V)^{-1},$$
with the resolvent identity:
$$R_\epsilon(z) = R(z) - R(z) \delta V R_\epsilon(z).$$

By Neumann series expansion (for $\|\delta V R(z)\| < 1$):
$$R_\epsilon(z) = \sum_{n=0}^\infty (-1)^n [R(z) \delta V]^n R(z).$$

For $z$ outside the spectrum of $A_\epsilon$ and $A$, with $|\delta V| \leq \epsilon$:
$$\|R_\epsilon(z) - R(z)\| \leq \frac{\|R(z)\|^2 \epsilon}{1 - \|R(z) \delta V\|} \leq C \|R(z)\|^2 \epsilon.$$

\textbf{Step 2: Spectral Projector Perturbation}

The spectral projector onto the eigenspace of $\lambda_k$ is:
$$P_k = \frac{1}{2\pi i} \oint_{\Gamma_k} R(z) dz,$$
where $\Gamma_k$ is a contour surrounding $\lambda_k$ in the complex plane, enclosing only $\lambda_k$ among all eigenvalues.

The perturbed projector is:
$$P_k^\epsilon = \frac{1}{2\pi i} \oint_{\Gamma_k} R_\epsilon(z) dz.$$

By contour integration and resolvent bounds:
$$\|P_k^\epsilon - P_k\| \leq C \epsilon \max_{z \in \Gamma_k} \|R(z)\|^2 \leq C \epsilon \delta_k^{-2},$$
where $\delta_k$ is the spectral gap around $\lambda_k$ (distance to nearest other eigenvalue).

\textbf{Step 3: Eigenvalue Perturbation}

The first-order eigenvalue perturbation is:
$$\lambda_k^\epsilon = \lambda_k + \langle e_k | \delta V | e_k \rangle + O(\epsilon^2).$$

By Holder regularity of eigenfunctions (Theorem \ref{thm:eigenfunctionRegularity}), if $\delta V \in L^\infty$:
$$|\lambda_k^\epsilon - \lambda_k| \leq \|\delta V\|_\infty \leq \epsilon.$$

\textbf{Step 4: Eigenfunction Perturbation}

The first-order eigenfunction perturbation uses the resolvent:
$$e_k^\epsilon = e_k + \sum_{j \neq k} \frac{\langle e_j | \delta V | e_k \rangle}{\lambda_k - \lambda_j} e_j + O(\epsilon^2).$$

By spectral gap bounds: if $\min_{j \neq k} |\lambda_k - \lambda_j| \geq \delta_k > 0$, then:
$$\|e_k^\epsilon - e_k\|_{L^2} \leq C \epsilon / \delta_k.$$

Furthermore, by Holder regularity, eigenfunctions satisfy $e_k^\epsilon \in C^{0,\alpha}(X)$ uniformly for small $\epsilon$:
$$\|e_k^\epsilon\|_{C^{0,\alpha}} \leq C(k) \quad \text{(uniform in } \epsilon \text{ for } \epsilon \leq \epsilon_0(k)).$$

\textbf{Step 5: Quantitative Stability Bounds}

For observables depending on spectral data (heat semigroup, spectral embedding, Green's functions):

\begin{enumerate}
\item \textbf{Heat Semigroup Stability:}
$$\|e^{-tA_\epsilon} - e^{-tA}\|_{L^2 \to L^2} \leq C(t) \epsilon.$$

\item \textbf{Spectral Embedding Stability:}
$$\|\Psi_N^\epsilon - \Psi_N\|_\infty \leq C(N) \epsilon,$$
where $\Psi_N(x) = (\sqrt{|\lambda_0|} e_0(x), \ldots, \sqrt{|\lambda_{N-1}|} e_{N-1}(x))$.

\item \textbf{Green's Function Stability:}
$$\|(A_\epsilon + I)^{-1} - (A + I)^{-1}\|_{L^2 \to L^2} \leq C \epsilon.$$

\end{enumerate}

These bounds show that the spectral structure (projectors, eigenvalues, eigenfunctions) depends continuously on perturbations to the potential, with explicit quantitative control. This is essential for establishing that small variations in $V$ do not destabilize the emergent manifold structure or metric. \qed

\end{proof}

\end{lemma}

