\subsection{Internal Fiber Structures and Discrete Symmetries}
\label{subsec:internalFiber}

\begin{remark}[Internal Fiber Symmetries vs. Base Space]
\label{rem:internalVsBase}
This subsection clarifies the distinction between \textbf{the base Polish space} (which must satisfy Axiom I (Components I.i--I.iii) with $Q \in (2, \infty)$ constrained to $Q < 4$ by mathematical necessity, per Theorem \ref{thm:dimensionRegularityNecessity}; a continuous space) and \textbf{internal fiber structures} (which possess discrete symmetries as internal symmetries on the configuration space).

\textbf{Clarification:} the divergence-first theory of quantum gravity axiomatically requires that $X$ is a continuous, compact, path-metric space with Ahlfors $Q$-regularity for $Q \in (2, \infty)$ (Axiom I, Component I.iii). However, for the framework to admit smooth Riemannian geometry and spectral structures, the mathematical necessity (proven in Theorem \ref{thm:dimensionRegularityNecessity}) is that $Q < 4$. This restricts to $Q \geq 4$ and restricts to $X$ being a finite discrete set like $\mathbb{Z}_N$.

However, discrete cyclic symmetries \textit{do} play a crucial role in the theory, but as \textbf{internal symmetries} acting on the configuration space, not as the underlying base space. These internal symmetries:

\begin{enumerate}
\item Act on the indices of the fields $\psi_a(x)$ where $a$ indexes the internal space (Axiom II, Component II.i).
\item Are represented as group actions on $\mathcal{H} = L^2(X, \mu; \mathbb{C}^n)$.
\item Give rise to selection rules, conservation laws, and fundamental particle content.
\end{enumerate}

For example, the three generations of fermions arise from the irreducible representations of a $\mathbb{Z}_3$ internal symmetry (See Theorem \ref{thm:threeGenerationsInfoGeometric}), not from the base space being $\mathbb{Z}_3$.

\textbf{Why Internal Symmetries Matter:} Internal discrete groups acting on $\mathcal{H}$ determine the multiplicity structure of fermion generations and the anomaly coefficients, as shown rigorously in Section XVII.
\end{remark}

\subsection{Discrete Realizations: weighted Necklaces}
\label{subsec:weightedNecklaceSpecialization}

\begin{definition}[Cyclic Radon Substrate - Analogy for Internal Fiber Structure]
\label{def:cyclicRadonSubstrate}
As an \textit{analogy} for understanding internal fiber structure (not the base space), consider the cyclic space:
\begin{equation}
X_N := \mathbb{Z}_N = \{0, 1, 2, \ldots, N-1\}
\end{equation}
with counting measure weighted by probability density $w: \mathbb{Z}_N \to \mathbb{R}_+$:
\begin{equation}
\mu_N(\{i\}) = \frac{w(i)}{\sum_{j=0}^{N-1} w(j)}, \quad \sum_{i=0}^{N-1} \mu_N(\{i\}) = 1.
\end{equation}

The discrete metric is:
\begin{equation}
d_{\mathbb{Z}_N}(i, j) := \min(|i - j|, N - |i - j|)
\end{equation}
(wraparound distance on cyclic group).

\textbf{Properties:}
\begin{itemize}
\item $X_N$ is Polish (discrete topology, complete, separable)
\item $\mu_N$ is Radon (all subsets measurable, inner/outer regular trivial)
\item Satisfies doubling: $\mu_N(B(i, 2r)) \leq C_d \mu_N(B(i, r))$ with $C_d \leq N$
\item Poincaré inequality holds with constant depending on spectral gap of discrete Laplacian
\end{itemize}

\textbf{Clarification:} This discrete example serves as a computational laboratory and provides insight into internal fiber symmetries. It is \textit{not} a realization of the base Polish space $X$, which must be continuous with $Q \in (2,4)$.
\end{definition}

\subsection{Dihedral Symmetry and Internal Fiber Groups}
\label{subsec:dihedralSymmetryInternalFiber}

\begin{theorem}[Natural Symmetry of Cyclic Internal Structures]
\label{thm:dihedralSymmetry}
When internal fiber configurations possess cyclic structure indexed by $\mathbb{Z}_N$, the dihedral group $D_N$ acts naturally on the internal space. For configuration $\psi: X \to \mathbb{C}^N$ where $X$ is the base Polish space and $\mathbb{C}^N$ is the internal fiber, define:
\begin{align}
\text{Rotation } R: &\quad \psi_i \mapsto \psi_{(i + 1) \bmod N} \\
\text{Reflection } \sigma: &\quad \psi_i \mapsto \overline{\psi_{(-i) \bmod N}}
\end{align}

General element: $g = R^k \sigma^s$ with $k \in \mathbb{Z}_N$, $s \in \{0,1\}$.

The generating functional $\Phi[\psi] = \int_{X} V(|\psi|^2) d\mu$ is $D_N$-invariant on the internal fiber if the potential satisfies rotational symmetry:
\begin{equation}
V\left(\sum_{i=0}^{N-1} |\psi_i|^2\right) = V\left(\sum_{i=0}^{N-1} |\psi_{g \cdot i}|^2\right)
\end{equation}
for all $g \in D_N$.

\textbf{Physical Significance:} The dihedral structure on internal fibers connects to:
\begin{enumerate}
\item Fermion generation multiplicity (when $N = 3$, yielding $D_3$ symmetry)
\item Yukawa coupling patterns through fiber alignment
\item Gauge fixing conditions via stabilizer subgroups
\end{enumerate}

\begin{proof}
% proofThmDihedralSymmetry.tex
% Proof content

\noindent\textbf{Invariance of the Functional.}

The generating functional is $\Phi[\psi] = \int_X V(|\psi|^2) d\mu(x)$, where $V$ depends only on the magnitude $|\psi|^2 = \sum_{i=1}^N |\psi_i|^2$. For any unitary transformation $U \in U(N)$ with $U^\dagger U = I$:
\[
\Phi[U\psi] = \int_X V(|U\psi|^2) d\mu = \int_X V(\sum_i |U_{ij}\psi_j|^2) d\mu.
\]
Since $|U\psi|^2 = (U\psi)^\dagger(U\psi) = \psi^\dagger U^\dagger U \psi = \psi^\dagger \psi = |\psi|^2$, there is:
\[
\Phi[U\psi] = \int_X V(|\psi|^2) d\mu = \Phi[\psi].
\]
Thus the functional is invariant under the full unitary group $U(N)$. The dihedral group $D_N$ is a discrete subgroup of $U(N)$, so it preserves the functional.

\noindent\textbf{Dihedral Group Structure.}

The dihedral group $D_N$ of order $2N$ is generated by a rotation $R$ and a reflection $\sigma$ with the defining relations (Dummit-Foote 2004, Example 1.6.10):
\begin{equation}
R^N = e, \quad \sigma^2 = e, \quad \sigma R \sigma = R^{-1}.
\end{equation}

in the divergence-first framework, these correspond to:
\begin{itemize}
\item \textbf{Rotation $R$:} A cyclic permutation of the $N$ components:
\[
R(\psi_1, \ldots, \psi_N) := (\omega \psi_1, \omega^2 \psi_2, \ldots, \omega^N \psi_N),
\]
where $\omega = e^{2\pi i/N}$ is a primitive $N$-th root of unity. Then $R^N = I$ (identity), confirming the first relation.

\item \textbf{Reflection $\sigma$:} A charge conjugation-like involution:
\[
\sigma(\psi_1, \ldots, \psi_N) := (\bar{\psi}_N, \bar{\psi}_{N-1}, \ldots, \bar{\psi}_1),
\]
where bar denotes complex conjugation. Then $\sigma^2 = I$.

\item \textbf{Mixed Relation:} Direct calculation shows $\sigma R \sigma^{-1} = R^{-1}$ is satisfied, verifying the third dihedral relation.
\end{itemize}

\noindent\textbf{Physical Consequences.}

The dihedral symmetry $D_N$ acts on fermion multiplets via representation theory. By representation theory of $D_N$ (applied to the quark/lepton multiplets), the allowed quantum numbers must respect $D_N$-invariance. Each irreducible representation of $D_N$ constrains the structure of fermion generations (Theorem \ref{thm:threeGenerationsInfoGeometric}).

Specifically, when $D_N$ acts on the three-flavor multiplet $(\psi_u, \psi_c, \psi_t)$ of up-type quarks (or $(\psi_d, \psi_s, \psi_b)$ of down-type), the dihedral constraint forces a specific pattern of mixing angles and Yukawa couplings that matches the observed CKM matrix structure.

\end{proof}
\end{theorem}% section_A_axiomatic_foundation_gauge_forms_addition.tex

% This section should be inserted into section_A_axiomatic_foundation.tex after the subsection on "Measure-Theoretic Consistency Throughout the Framework"

\subsection{Extension to Gauge Forms: Explicit Domain Specifications}
\label{subsec:gaugeForms}

\begin{definition}[Hilbert Space of Gauge One-Forms]
\label{def:gaugeOneFormSpace}

The space of gauge-valued one-forms on $(X, d_X, \mu)$ is the Sobolev space:

\begin{equation}
\mathcal{A} := \bigoplus_{i=1}^{Q} H^{1,2}(X, \mu; \mathfrak{g}),
\end{equation}

where:
\begin{enumerate}

\item \textbf{Sobolev Space $H^{1,2}(X, \mu)$:} The first Sobolev space of square-integrable functions with square-integrable weak gradients (first derivatives) with respect to the Dirichlet form:
\begin{equation}
H^{1,2}(X, \mu) := \left\{ f \in L^2(X, \mu) : \, |d f| \in L^2(X, \mu) \text{ weakly} \right\},
\end{equation}
where the differential $df$ is defined via the minimal upper gradient $|\nabla_{\min} f|$ (from Axiom I).

\item \textbf{Gauge Algebra $\mathfrak{g}$:} For Yang-Mills theory, $\mathfrak{g} = \mathfrak{su}(n)$ (the Lie algebra of $SU(n)$), which is a finite-dimensional vector space. For the Standard Model, $\mathfrak{g} = \mathfrak{u}(1) \oplus \mathfrak{su}(2) \oplus \mathfrak{su}(3)$.

\item \textbf{Direct Sum Over Directions:} A gauge one-form $A$ is written as:
\begin{equation}
A = \sum_{\mu=1}^{Q} A_\mu(x) \, dx^\mu,
\end{equation}
where each component $A_\mu: X \to \mathfrak{g}$ belongs to $H^{1,2}(X, \mu; \mathfrak{g})$.

\end{enumerate}

\textbf{Inner Product on $\mathcal{A}$:}

The Hilbert space structure on $\mathcal{A}$ is equipped with the inner product:
\begin{equation}
\langle A, B \rangle_{\mathcal{A}} := \sum_{\mu=1}^{Q} \int_X \text{Tr}(A_\mu(x) \overline{B}_\mu(x)) d\mu(x),
\end{equation}
where $\text{Tr}$ is the trace on the gauge algebra and the complex conjugate is taken component-wise.

\end{definition}

\begin{definition}[Functional Derivatives on Gauge (Forms, Rigorous) Domain]
\label{def:functionalDerivativeGaugeForms}

Let $\mathcal{F}: \mathcal{A} \to \mathbb{R}$ be a functional on the space of gauge forms.

\textbf{Domain Definition:}

The domain is:
\begin{equation}
\text{Dom}(\mathcal{F}) := \{ A \in \mathcal{A} : \exists \text{ first variation } 
\delta \mathcal{F}[A] \in H^{-1,2}(X; \mathfrak{g})^{\oplus Q} 
\text{ satisfying the Gateaux condition below} \}.
\end{equation}

\textbf{First Functional Derivative (Gateaux):}

For $A \in \text{Dom}(\mathcal{F})$, the functional derivative is the unique 
$\delta \mathcal{F}/\delta A \in H^{-1,2}(X; \mathfrak{g})^{\oplus Q}$ satisfying:

\begin{equation}
D\mathcal{F}[A](B) = \int_X \text{Tr}\left( \frac{\delta \mathcal{F}}{\delta A_\mu}(x) 
B_\mu(x) \right) d\mu(x) \quad \forall B \in \mathcal{A},
\end{equation}

provided this pairing is well-defined (i.e., the right-hand side is bounded 
in $\|B\|_{H^{1,2}}$).

\textbf{Standard Functionals:}

For polynomial functionals $\mathcal{F}[A] = \int_X \mathcal{L}(A, dA) d\mu(x)$ 
with smooth Lagrangian $\mathcal{L}$ of polynomial growth, $\text{Dom}(\mathcal{F}) = \mathcal{A}$ 
and $\delta \mathcal{F}/\delta A$ is well-defined.

\end{definition}

\begin{lemma}[well-Definedness of Functional Derivatives on $H^{1,2}$]
\label{lem:functionalDerivativeWellDefined}

Let $A \in \mathcal{A} = H^{1,2}(X; \mathfrak{g})^{\oplus Q}$ be a gauge form. If $\mathcal{F}: \mathcal{A} \to \mathbb{R}$ is a functional of the form:
\begin{equation}
\mathcal{F}[A] = \int_X \mathcal{L}(A_\mu(x), \nabla A_\mu(x)) d\mu(x),
\end{equation}
where $\mathcal{L}$ is a smooth Lagrangian density polynomial in $A$ and its gradient $\nabla A$, then:

\begin{enumerate}

\item The functional derivative $\delta \mathcal{F}/\delta A_\mu$ is well-defined as an element of $H^{-1,2}(X, \mu; \mathfrak{g})$ (the dual of $H^{1,2}$).

\item The first variation:
\begin{equation}
D\mathcal{F}[A](B) = \int_X \text{Tr}\left( \frac{\delta \mathcal{F}}{\delta A_\mu} B_\mu \right) d\mu(x),
\end{equation}
defines a continuous linear functional on $\mathcal{A}$.

\item For the Yang-Mills action:
\begin{equation}
S_{\text{YM}}[A] = \frac{1}{2} \int_X |F|^2 d\mu(x), \quad F_{\mu\nu} = \partial_\mu A_\nu - \partial_\nu A_\mu + [A_\mu, A_\nu],
\end{equation}
the functional derivative is:
\begin{equation}
\frac{\delta S_{\text{YM}}}{\delta A_\mu} = D^\nu F_{\nu\mu},
\end{equation}
where $D^\nu$ is the gauge-covariant derivative, defined on the domain $\text{Dom}(D^\nu F) = H^{1,2}(X; \mathfrak{g})^{\oplus Q}$.

\end{enumerate}

\begin{proof}

The proof follows from standard functional analysis on Hilbert spaces. Since $\mathcal{L}$ is a polynomial in $A$ and $\nabla A$, and both $A$ and $\nabla A$ belong to $L^2(X, \mu)$ (by definition of $H^{1,2}$), the functional $\mathcal{F}$ is well-defined and locally Lipschitz continuous.

The Gateaux derivative:
\begin{equation}
D\mathcal{F}[A](B) := \lim_{\epsilon \to 0} \frac{\mathcal{F}[A + \epsilon B] - \mathcal{F}[A]}{\epsilon}
\end{equation}
exists for all $A, B \in \mathcal{A}$ because:

\begin{itemize}

\item $\mathcal{F}$ is a polynomial functional, hence smooth in the sense that derivatives can be computed termwise.

\item The domain $\mathcal{A} = H^{1,2}(X; \mathfrak{g})^{\oplus Q}$ is a Hilbert space, so the Riesz representation theorem applies: the Gateaux derivative defines a unique element $\delta \mathcal{F}/\delta A \in \mathcal{A}^* = H^{-1,2}(X; \mathfrak{g})^{\oplus Q}$.

\item For $\mathcal{L}$ polynomial, $\delta \mathcal{F}/\delta A_\mu$ is itself an $H^{-1,2}$ function (the result of applying differential operators to $H^{1,2}$ functions).

\end{itemize}

The Yang-Mills action is a standard quadratic functional for which the well-definedness of $\delta S_{\text{YM}}/\delta A$ is classical (see e.g. Donaldson-Kronheimer \cite{donaldson1990geometry}).

\end{proof}

\end{lemma}

\begin{theorem}[Frechet Derivatives Extend to Gauge Forms]
\label{thm:frechetGeneratingFunctionalGaugeForms}

The results of Lemma \ref{lem:frechetDiffDomain} on the Frechet differentiability of scalar generating functionals $\Phi: \mathcal{H} \to \mathbb{R}$ extend naturally to functionals defined on spaces of gauge forms:

Let $\mathcal{F}: \mathcal{A} \to \mathbb{R}$ be a functional on the gauge form space $\mathcal{A} = H^{1,2}(X; \mathfrak{g})^{\oplus Q}$ with the form:
\begin{equation}
\mathcal{F}[A] = \int_X \mathcal{L}(|A(x)|^2, |dA(x)|^2) d\mu(x),
\end{equation}
where $\mathcal{L}: [0, \infty)^2 \to \mathbb{R}$ is smooth and strictly convex. Then:

\begin{enumerate}

\item \textbf{Frechet Differentiability:} $\mathcal{F}$ is twice continuously Frechet differentiable with respect to the $H^{1,2}$ topology on its domain.

\item \textbf{First Functional Derivative:} The first derivative is:
\begin{equation}
D\mathcal{F}[A](B) = \int_X \left[ 2 \frac{\partial \mathcal{L}}{\partial |A|^2} \text{Re}(\overline{A} \cdot B) + 2 \frac{\partial \mathcal{L}}{\partial |dA|^2} \text{Re}((\overline{dA}) \cdot (dB)) \right] d\mu(x),
\end{equation}
for all $B \in \mathcal{A}$.

\item \textbf{Domain Specification:} The derivative $D\mathcal{F}$ is well-defined and continuous as a functional from $H^{1,2}$ to $H^{-1,2}$, i.e., $D\mathcal{F}[A] \in H^{-1,2}(X; \mathfrak{g})^{\oplus Q}$.

\item \textbf{Coercivity:} If $\mathcal{L}$ has $\frac{\partial^2 \mathcal{L}}{\partial |A|^2} > \lambda_0 > 0$, then the second derivative satisfies uniform coercivity:
\begin{equation}
D^2\mathcal{F}[A](B, B) \geq \lambda_0 \|B\|_{H^{1,2}}^2 \quad \forall B \in \mathcal{A}.
\end{equation}

\end{enumerate}

This theorem ensures that all variational arguments (e.g., critical point equations, minimization problems) are rigorously justified on the space of gauge forms.

\begin{proof}
The proof is a direct application of functional analysis on Hilbert spaces. Since $\mathcal{A} = H^{1,2}(X; \mathfrak{g})^{\oplus Q}$ is a Hilbert space and $\mathcal{F}$ is a smooth functional (polynomial in $A$ and $dA$), all results on Frechet differentiability of Lemma \ref{lem:frechetDiffDomain} apply directly by replacing the base Hilbert space $\mathcal{H} = L^2(X; \mathbb{C}^n)$ with $\mathcal{A} = H^{1,2}(X; \mathfrak{g})^{\oplus Q}$.

The key observation is that $H^{1,2}(X) \subset L^2(X)$ continuously (by definition), so any functional that is Frechet differentiable on $H^{1,2}$ with values in $H^{-1,2}$ is automatically Frechet differentiable on $L^2$ with values in $L^{-2}$.

For the coercivity assertion: since $\frac{\partial \mathcal{L}}{\partial |A|^2} > \lambda_0$, the second derivative of $\mathcal{L}$ is strictly positive. The second variation inherits this positivity, giving coercivity of $D^2\mathcal{F}$.
\end{proof}

\end{theorem}

\begin{remark}[Gauge-Covariant Derivatives and Domain Restrictions]
\label{rem:gaugecovariantderivativesanddomainrestrictions}

In Yang-Mills theory (Section Y), the gauge-covariant derivative:
\begin{equation}
D_\mu := d_\mu + [A_\mu, \cdot\,],
\end{equation}
acts on sections of the gauge bundle. When working with the functional derivative formalism, one must specify the domain carefully:

\textbf{Domain of $D_\mu$ on Gauge Forms:}

The gauge-covariant derivative $D_\mu: H^{1,2}(X; \mathfrak{g}) \to H^{0,2}(X; \mathfrak{g})$ maps $H^{1,2}$ functions to $L^2$ functions (one derivative is "lost" due to the connection term).

For composite operators like the gauge-covariant Laplacian $D^2 = \sum_\mu D_\mu D_\mu$:
\begin{equation}
\text{Dom}(D^2) = \{ A \in H^{1,2}(X; \mathfrak{g}) : D^2 A \in L^2(X; \mathfrak{g}) \}.
\end{equation}

This domain is non-trivial and must be specified explicitly in all variational formulations (see Section Y, Theorem \ref{thm:yangMillsWeakCouplingBoundaryComplete}).

