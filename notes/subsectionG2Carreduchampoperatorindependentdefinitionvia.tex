% Part of sectionGMetricEmergence.tex
\subsection{Carre du Champ Operator - Independent Definition via Minimal Upper Gradient}
\label{subsec:carreDuChampOperatorIndependentDefinitionVia}

\begin{definition}[Carre du Champ via Minimal Upper Gradient - Non-Circular Definition]
\label{def:carreDuChamp}
The Carre du Champ operator is defined as follows:

\noindent\textbf{Post-Spectral Definition (Using Eigenfunctions):}

For eigenfunctions $e_\mu, e_\nu$ from the spectral decomposition (Theorem \ref{thm:laplacianProperties}), define:
\begin{equation}
\Gamma(e_\mu, e_\nu)(x) := \nabla_{\min} e_\mu(x) \cdot \nabla_{\min} e_\nu(x),
\end{equation}
where the product is taken pointwise, and equals zero outside the set where both minimal upper gradients are defined.

This definition:
\begin{enumerate}
\item \textbf{Is Measurable:} By \cite{cheeger1999differentiation}), the minimal upper gradient is measurable for all $u \in H^{1,2}(X)$. Thus products of minimal upper gradients are measurable.

\item \textbf{Requires No Circular Reference:} The definition uses only:
\begin{itemize}
\item Eigenfunctions $\{e_k\}$ (computed from $A$)
\item The minimal upper gradient $\nabla_{\min}$ (derived from Axiom 1, not from $\Gamma$)
\item Pointwise multiplication (an algebraic operation)
\end{itemize}

\item \textbf{Is Integrable:} For $e_\mu, e_\nu \in H^{1,2}(X)$ (since eigenfunctions are in the domain of $A$):
\begin{equation}
\int_X \Gamma(e_\mu, e_\nu) d\mu = \int_X \nabla_{\min} e_\mu \cdot \nabla_{\min} e_\nu \, d\mu \leq \|\nabla_{\min} e_\mu\|_{L^2} \|\nabla_{\min} e_\nu\|_{L^2} < \infty
\end{equation}
by Cauchy-Schwarz and Hölder continuity (Theorem \ref{thm:eigenfunctionRegularity}).

\item \textbf{Hölder Continuity for Eigenfunctions:} For eigenfunctions $e_k, e_\ell \in C^{0,\alpha}(X)$ (Theorem \ref{thm:eigenfunctionRegularity}), the pointwise product $\nabla_{\min} e_k \cdot \nabla_{\min} e_\ell \in C^{0,\beta}(X)$ for some $\beta \in (0, \alpha)$ by standard product estimates on Hölder spaces. Thus:
\begin{equation}
\Gamma(e_k, e_\ell)(x) = \nabla_{\min} e_k(x) \cdot \nabla_{\min} e_\ell(x)
\end{equation}
is well-defined and continuous $\mu$-almost everywhere without any distributional interpretation or reference to the Laplacian.

\item \textbf{General Functions:} For arbitrary $f, g \in H^{1,2}(X)$, define Carre du Champ by:
\begin{equation}
\Gamma(f, g)(x) := \nabla_{\min} f(x) \cdot \nabla_{\min} g(x),
\end{equation}
which is integrable and measurable.
\end{enumerate}
\end{definition}

\begin{lemma}[Pointwise Minimal Upper Gradient Regularity - Cheeger]
\label{lem:canonicalGradientRepresentative}
For $f \in H^{1,2}(X) \cap C^{0,\alpha}(X)$ with $\alpha > 0$, the minimal upper gradient $|\nabla_{\min} f|(x)$ is measurable and satisfies:
\begin{enumerate}[label=(\roman*)]
\item For H\"older-continuous $f$, the minimal upper gradient is Borel measurable and admits a continuous representative on a set of full measure (Cheeger 1999, Theorem 4.21).
\item The pointwise value $|\nabla_{\min} f|(x)$ can be computed as the infinitesimal Lipschitz constant:
\[
|\nabla_{\min} f|(x) = \limsup_{y \to x} \frac{|f(y) - f(x)|}{d_X(x, y)}.
\]
This definition depends only on the metric $d_X$ and $f$, not on any differential operator.
\item For eigenfunctions $e_k \in C^{0,\alpha}(X)$ (Theorem \ref{thm:eigenfunctionRegularity}), this limit exists pointwise, and $|\nabla_{\min} e_k|$ is continuous.
\end{enumerate}

\begin{proof}
By \cite{cheeger1999differentiation}, Theorem 4.21), on metric measure spaces with Poincaré inequalities, H\"older-continuous functions have measurable minimal upper gradients with Borel representatives. The infinitesimal Lipschitz formula is a standard characterization of the minimal upper gradient (\cite{heinonen1998quasiconformal}). Pointwise existence for eigenfunctions follows from Theorem \ref{thm:eigenfunctionRegularity} and standard metric space calculus.
\end{proof}
\end{lemma}

\begin{remark}[Non-Circular Nature of Carre du Champ]
\label{rem:nonCircularCarre}
The definition of $\Gamma$ does \textbf{NOT} invoke the Laplacian using the polarization formula $\Gamma(f, g) = \frac{1}{2}[A(fg) - f(Ag) - g(Af)]$. That formula is a \textit{consequence} of the definition once the Laplacian is constructed. The minimal upper gradient definition is fundamental and non-circular, depending only on $(X, d_X, \mu)$ and pointwise products.
\end{remark}

