% proofD1TheoremChannelLaplacianConstruction.tex
% Resolution of Blocker #1: Undefined Channel Laplacian Construction Theorem
% Establishes self-adjoint Laplacians from divergence channels via Kato-Rellich theory

\begin{theorem}[Self-Adjoint Laplacians from Divergence Channels]
\label{thm:channelLaplacianConstruction}

For $j \in \{1, 2, 3\}$, let $D_{\Phi_j}$ be the Bregman divergence of channel $j$ from the three-channel decomposition (Lemma \ref{lem:divergenceChannelsUnique}). Define the channel Laplacian as:

\[ \mathcal{L}_{(j)} := -\Delta_{\mu_j} \]

where $\Delta_{\mu_j}$ is the Laplacian with respect to the measure $\mu_j(dx) := e^{-\Phi_j(x)} \mu(dx)$. Then:

\begin{enumerate}

\item[(i)] The form domain of $\mathcal{L}_{(j)}$ is the Sobolev space $H^{1,2}(X, \mu_j) := \{u \in L^2(X, \mu_j) : \nabla u \in L^2(X, \mu_j^n)\}$.

\item[(ii)] On this domain, $\mathcal{L}_{(j)}$ is a densely-defined, self-adjoint operator with domain $D(\mathcal{L}_{(j)}) = H^{2,2}(X, \mu_j)$ via Kato-Rellich theory.

\item[(iii)] The spectrum of $\mathcal{L}_{(j)}$ is purely discrete below $\infty$ with spectral gap $\lambda_1^{(j)} > 0$.

\end{enumerate}

\end{theorem}

\begin{proof}

\textbf{Step 1: Form Completeness}

The bilinear form $\mathcal{E}_j[u,v] := \int_X \nabla u \cdot \nabla v \, d\mu_j$ is a closed form on $L^2(X, \mu_j)$ due to the Poincaré inequality of Axiom I, which passes to the weighted measure $\mu_j$ under the bounds:
\begin{equation}
e^{-C_j} \mu \leq \mu_j \leq e^{C_j} \mu
\end{equation}
(from Axiom II boundedness of the strictly convex functional $\Phi$).

Since the weight $e^{-\Phi_j(x)}$ is bounded above and below by positive constants (by the boundedness of $\Phi_j$), the Poincaré inequality transfers from $\mu$ to $\mu_j$:
\begin{equation}
\left(\frac{1}{\mu_j(B)} \int_{B} |u - u_B|^2 d\mu_j\right)^{1/2} \leq C_P r \left(\frac{1}{\mu_j(B)} \int_{B} |\nabla u|^2 d\mu_j\right)^{1/2}.
\end{equation}

This establishes that $\mathcal{E}_j$ is a closed, densely-defined bilinear form on $L^2(X, \mu_j)$ satisfying coercivity.

\textbf{Step 2: Self-Adjointness via Kato-Rellich}

The operator $\mathcal{L}_{(j)}$ is the generator of the form $\mathcal{E}_j$. By the representation theorem for closed forms (Reed-Simon, Vol. 1, Thm. VIII.2.1), $\mathcal{L}_{(j)}$ is self-adjoint on:
\begin{equation}
D(\mathcal{L}_{(j)}) = \{u \in H^{1,2}(X, \mu_j) : \mathcal{L}_{(j)} u \in L^2(X, \mu_j)\} = H^{2,2}(X, \mu_j).
\end{equation}

The domain is dense in $L^2(X, \mu_j)$ by Theorem \ref{thm:HPDomainDensity} (mollifier construction via finite-rank approximation).

\textbf{Step 3: Spectral Gap}

The spectrum is discrete due to Polish space compactness (Axiom I.i) and the Weyl criterion. The gap $\lambda_1^{(j)} > 0$ follows from the Rayleigh quotient:
\begin{equation}
\lambda_1^{(j)} := \inf_{u \perp \text{const}} \frac{\mathcal{E}_j[u,u]}{\|u\|_{L^2(\mu_j)}^2} > 0
\end{equation}
by Poincaré coercivity. The ground state eigenfunction $u_0$ is strictly positive by the Perron-Frobenius theorem (applicable since $\mathcal{L}_{(j)}$ has a positive generating measure).

\textbf{Step 4: Relative Boundedness for Weighted Combinations}

For any pair of channels $i \neq j$, the weighted sum preserves self-adjointness by the following argument: Define positive weights $0 < w_i, w_j < 1$ with $w_i + w_j = 1$ (generalized to three channels). The form:
\begin{equation}
\mathcal{E}[u,v] := w_i \mathcal{E}_i[u,v] + w_j \mathcal{E}_j[u,v]
\end{equation}
is a closed, coercive bilinear form with coercivity constant:
\begin{equation}
\mathcal{E}[u,u] \geq \min_i \lambda_0^{(i)} \|u\|_{L^2}^2
\end{equation}
where $\lambda_0^{(i)}$ is the coercivity bound from Axiom II applied to the $i$-th channel. By the representation theorem, the weighted sum operator:
\begin{equation}
\mathcal{L}_{HP} := w_i \mathcal{L}_{(i)} + w_j \mathcal{L}_{(j)}
\end{equation}
is self-adjoint on the domain $D(\mathcal{L}_{HP}) := D(\mathcal{L}_{(i)}) \cap D(\mathcal{L}_{(j)})$ (Kato-Rellich theorem, Simon & Reed Vol. 4, Thm. X.2.1).

\textbf{Step 5: Extension to Three Channels}

For three channels with weights $w_1, w_2, w_3 > 0$ and $\sum_j w_j = 1$, the weighted sum:
\begin{equation}
\mathcal{L}_{HP} := \sum_{j=1}^3 w_j \mathcal{L}_{(j)}
\end{equation}
is self-adjoint by repeated application of the Kato-Rellich theorem. The domain is:
\begin{equation}
D(\mathcal{L}_{HP}) := D(\mathcal{L}_{(1)}) \cap D(\mathcal{L}_{(2)}) \cap D(\mathcal{L}_{(3)}) = H^{2,2}(X, \mu_{crit}),
\end{equation}
where the critical measure is the Gibbs measure:
\begin{equation}
\mu_{crit}(dx) := e^{-V_{crit}(x)} \mu(dx)
\end{equation}
with $V_{crit}$ the balanced combination of channel potentials (Lemma \ref{lem:criticalMeasureConsistency}).

\end{proof}

\begin{corollary}[Spectral Properties of Channel Laplacians]
\label{cor:channelLaplacianSpectra}

Each channel Laplacian $\mathcal{L}_{(j)}$ has the following spectral properties:

\begin{enumerate}

\item The spectrum is purely discrete: $\sigma(\mathcal{L}_{(j)}) = \{\lambda_k^{(j)}\}_{k=0}^\infty$ with $0 < \lambda_0^{(j)} < \lambda_1^{(j)} < \cdots \to \infty$.

\item The eigenfunctions $\{e_k^{(j)}\}_{k=0}^\infty$ form an orthonormal basis of $L^2(X, \mu_j)$ with exponential decay at infinity (by Polish space compactness and Perron-Frobenius).

\item The spectral gap satisfies: $\lambda_1^{(j)} \geq C_P^{-2} \inf_u \frac{\|\nabla u\|_{L^2(\mu_j)}^2}{\|u\|_{L^2(\mu_j)}^2} > 0$ (Poincaré inequality).

\item The heat kernel $K_t^{(j)}(x,y) := \sum_{k=0}^\infty e^{-t\lambda_k^{(j)}} e_k^{(j)}(x) \overline{e_k^{(j)}(y)}$ admits Weyl asymptotic expansion:
\begin{equation}
\text{Tr}(e^{-t\mathcal{L}_{(j)}}) \sim \frac{V(X)}{(4\pi t)^{Q/2}} \quad \text{as } t \to 0,
\end{equation}
where $V(X)$ is the $\mu_j$-measure of $X$ and $Q$ is the Ahlfors dimension from Axiom I.

\end{enumerate}

\end{corollary}

