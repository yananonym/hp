\subsection{Pathway 1: Divergence-Induced Hilbert Space for Yang-Mills}
\label{subsec:pathway1HilbertSpace}

\subsubsection{Preliminary: Extension of Divergence to Gauge Forms}

\begin{definition}[Gauge One-Form Space and Induced Divergence]
\label{def:gaugeOneFormSpace}

Configuration space of gauge one-forms:
\begin{equation}
\Lambda^1(X; \mathfrak{g}) := \{\mathcal{A} \in L^2(X; T^* X \otimes \mathfrak{g})\}.
\end{equation}

For $\mathcal{A} \in \Lambda^1(X; \mathfrak{g})$, magnitude function:
\begin{equation}
|\mathcal{A}|(x) := \sqrt{\sum_\mu |\mathcal{A}_\mu(x)|^2_{\mathfrak{g}}}.
\end{equation}

Extend generating functional:
\begin{equation}
\Phi[\mathcal{A}] := \int_X V(|\mathcal{A}(x)|^2) d\mu(x).
\end{equation}

Extend divergence:
\begin{equation}
D_\Phi(\mathcal{A} \| \mathcal{B}) := \Phi[\mathcal{A}] - \Phi[\mathcal{B}] - \left\langle \frac{\delta \Phi}{\delta \mathcal{B}}, \mathcal{A} - \mathcal{B} \right\rangle.
\end{equation}

\end{definition}

\begin{lemma}[Carre du Champ for Gauge Forms]
\label{lem:carreGaugeForms}

For $\mathcal{A}, \mathcal{B} \in \Lambda^1(X; \mathfrak{g}) \cap H^{1,2}(X; \mathfrak{g})$:
\begin{equation}
\Gamma(\mathcal{A}, \mathcal{B})(x) := \sum_\mu \langle d\mathcal{A}_\mu(x), d\mathcal{B}_\mu(x) \rangle_{\mathfrak{g}}.
\end{equation}

This is integrable: $\int_X |\Gamma(\mathcal{A}, \mathcal{B})| d\mu < \infty$.

\end{lemma}

\begin{definition}[Spectral Gap and Mass Threshold for Yang-Mills]
\label{def:spectral_gap_yang_mills}

Let $H: \Dom(H) \to \mathcal{H}_{YM}$ be the Yang-Mills Hamiltonian operator (to be constructed in Theorem \ref{thm:yangMillsHilbertStructure}). The spectrum is decomposed as:
\[\sigma(H) = \sigma_p(H) \cup \sigma_{ess}(H),\]
where $\sigma_p$ is the point spectrum (eigenvalues) and $\sigma_{ess}$ is the essential spectrum.

The \textbf{vacuum energy} is defined and normalized as:
\[E_0 := \min \sigma_p(H) = 0\]
(the ground state energy, by convention set to zero for this theory).

The \textbf{first excited mass threshold} is defined as:
\[\Delta := \inf\{E \in \sigma(H) : E > 0\}.\]

This is the \textbf{mass gap} if $\Delta > 0$. Equivalently,
\[\Delta = \inf(\text{Spectrum}(H) \cap (0, \infty)).\]

For Yang-Mills on finite volume $\Lambda \subset \mathbb{R}^4$ with Dirichlet or periodic boundary conditions, the spectrum is purely discrete, so $\Delta = E_1 - E_0 > 0$ is the lowest excitation energy above the vacuum (the energy of the lightest glueball state).

In the thermodynamic limit $\Lambda \nearrow \mathbb{R}^4$, the point spectrum merges into the essential spectrum, but the gap $\Delta > 0$ persists (proven by Theorems \ref{thm:interactionStabilityComplete} and \ref{thm:yangMillsComplete}).

\end{definition}

\begin{remark}[Absence of Zero Modes and Conformal Invariance]
\label{rem:zeroModesConformal}

In conformal field theory, the vacuum is the unique eigenstate with $E = 0$ (see \cite{witten1988quantum}). Yang-Mills theory is non-conformal in four dimensions (conformal invariance exists only in 2D). Therefore, the Yang-Mills Hamiltonian has no zero modes in its spectrum only considering for the vacuum itself.

By Theorem \ref{thm:yangMillsHilbertStructure} and the spectral properties of the divergence-based dynamics (Axiom II), the energy operator $H$ satisfies:
\begin{itemize}
\item A unique ground state $|0\rangle$ with $H|0\rangle = 0$ (the vacuum)
\item A strict mass gap: no states with $0 < E < \Delta$
\end{itemize}

This spectrum structure is enforced by:
\begin{enumerate}
\item The strict convexity of the generating functional $\Phi$ (Axiom II), which controls growth
\item The Ward identities (Theorem \ref{thm:wardIdentitiesAllOrders}), which enforce gauge consistency
\item The divergence-based interaction coupling (divergence consistency), which prevents anomalous zero modes
\end{enumerate}

Consequently, gap closure via zero-mode accumulation is impossible within the divergence-first theory of quantum gravity framework.

\end{remark}

\begin{theorem}[Yang-Mills Hilbert Space Construction from Divergence]
\label{thm:yangMillsHilbertStructure}

From the divergence-extended configuration space $\Lambda^1(X; \mathfrak{g})$, construct the Yang-Mills Hilbert space:

\begin{equation}
\mathcal{H}_{\text{YM}} := \text{Fock space over } L^2(X; T^* X \otimes \mathfrak{g}).
\end{equation}

The inner product is induced by the generating functional $\Phi$ through the Carre du Champ structure.

\textbf{Properties:}

\begin{enumerate}

\item \textbf{Separability:} $\mathcal{H}_{\text{YM}}$ is a separable Hilbert space with countable orthonormal basis via eigenfunctions of the induced Laplacian on one-forms.

\item \textbf{Poincaré group:
\begin{equation}
U(a, \Lambda): \mathcal{H}_{\text{YM}} \to \mathcal{H}_{\text{YM}}, \quad [U(a, \Lambda), H] = 0.
\end{equation}

\item \textbf{Spectral Condition:} The energy-momentum four-vector $P^\mu$ has spectrum in the forward light cone:
\begin{equation}
\text{Spec}(P^\mu) \subseteq V^+ = \{p : p^0 \geq |\mathbf{p}|, \, p^0 \geq 0\}.
\end{equation}

\item \textbf{Vacuum State:} There exists a unique, normalized vacuum state $|0\rangle \in \mathcal{H}_{\text{YM}}$ satisfying:
\begin{equation}
a(\mathbf{k})|0\rangle = 0 \quad \text{for all annihilation operator modes}.
\end{equation}

\end{enumerate}

\end{theorem}

\begin{proof}
\input{proofT3TheoremYangMillsHilbertStructure}
\end{proof}

\subsubsection{Coherent State Resolution for Gauge One-Forms}

\begin{theorem}[Coherent State Resolution for Gauge One-Forms]
\label{thm:coherentStateGaugeFormsResolution}

For gauge one-forms $\mathcal{A} \in \Omega^1(X; \mathfrak{g}) \cap H^{1,2}(X; \mathfrak{g})$, the Dirichlet form associated with the Carre du Champ $\Gamma(\mathcal{A}, \mathcal{B})$ admits a coherent-state resolution of identity:

\[\mathbb{1}_{\mathcal{H}_{YM}} = \int_{\Omega^1 \times \mathbb{R}_+} |\alpha \rangle \langle \alpha| \, \nu(d\alpha),\]

where:
\begin{itemize}
\item $|\alpha \rangle$ are coherent states labeled by (gauge form $\alpha$, rescaling parameter $s \in \mathbb{R}_+$)
\item $\nu$ is a measure on the coherent-state manifold
\item Resolution holds in the weak-operator topology on $\mathcal{H}_{YM}$
\end{itemize}

\begin{proof}

By Lemma \ref{lem:carreGaugeForms}, the Carre du Champ operator $\Gamma(\mathcal{A}, \mathcal{B})$ on one-forms is integrable with respect to the measure $\mu$ on $X$. This integrability induces a positive-semidefinite quadratic form on $H^{1,2}(X; \mathfrak{g})$, the Sobolev space of gauge one-forms with $L^2$ weak derivatives.

By standard Hilbert space theory, this quadratic form extends to a unique self-adjoint operator $A_{\text{gauge}}: \text{Dom}(A_{\text{gauge}}) \to \mathcal{H}_{YM}$ (where $\mathcal{H}_{YM} = L^2(\Omega^1(X; \mathfrak{g}))$ is the Hilbert space of square-integrable gauge one-forms).

The spectral decomposition of $A_{\text{gauge}}$ yields a countable orthonormal basis of eigenfunctions $\{\phi_n^{\text{gauge}}\}_{n=1}^\infty$ with eigenvalues $0 = \lambda_0 < \lambda_1 \leq \lambda_2 \leq \ldots \to \infty$ (by the discrete spectrum assumption from the compactness of the resolvent).

For each point $\alpha \in \Omega^1(X; \mathfrak{g})$ in the space of gauge one-forms, construct coherent states by analogy with scalar theory:

\[|\alpha, s \rangle := \exp\left( \sum_{n=0}^\infty \sqrt{s \lambda_n} \phi_n^{\text{gauge}}(\alpha) a_n^\dagger \right) |0 \rangle,\]

where $a_n, a_n^\dagger$ are bosonic creation/annihilation operators for the $n$-th mode and $|0 \rangle$ is the gauge-form vacuum.

These states satisfy:
\begin{enumerate}
\item \textbf{Continuity:} The map $\alpha \mapsto |\alpha, s \rangle$ is continuous in norm for each fixed $s$.
\item \textbf{Normalization:} $\langle \alpha, s | \alpha, s \rangle = 1$ by construction.
\item \textbf{Resolution:} By Theorem \ref{thm:coherentStateProperties} applied to the Fock space over $L^2(\Omega^1(X; \mathfrak{g}))$, and using standard measure-theoretic arguments, the integral:

\[\mathbb{1} = \int_{\Omega^1 \times \mathbb{R}_+} |\alpha, s \rangle \langle \alpha, s| \, \nu(d\alpha \, ds)\]

converges in the weak-operator topology (and strongly on dense subsets), where $\nu$ is the product of the Lebesgue measure on $\Omega^1(X; \mathfrak{g})$ (in a suitable sense) and the measure $ds$ on $\mathbb{R}_+$.
\end{enumerate}

This establishes the resolution of identity for gauge one-forms, extending the scalar coherent-state machinery to vector-valued fields required for Yang-Mills theory.

\end{proof}

\end{theorem}

% =========================================================================
% FREE YANG-MILLS MASS GAP
% =========================================================================

\begin{theorem}[Free Yang-Mills Theory Has Positive Mass Gap]
\label{thm:freeYangMillsMassGap}

Consider the free Yang-Mills Hamiltonian (no interactions, no matter coupling):
\begin{equation}
H_0 = \int_X \left(\frac{1}{2}|\mathcal{E}|^2 + \frac{1}{2}|\mathcal{B}|^2\right) d\mu,
\end{equation}
where $\mathcal{E} = \partial_t \mathcal{A} - \nabla \Phi$ is the electric field and $\mathcal{B} = \text{curl}(\mathcal{A})$ is the magnetic field.

The spectrum satisfies:

\begin{enumerate}

\item \textbf{Ground State:} Unique ground state $|0\rangle$ with energy $E_0 = 0$ (the vacuum).

\item \textbf{Spectral Gap:} There exists a positive gap:
\begin{equation}
\Delta_0 := \inf\{\lambda \in \text{Spec}(H_0) : \lambda > 0\} > 0.
\end{equation}

\item \textbf{Discrete Spectrum Below Continuum:} Below the continuum threshold, the spectrum is discrete and $\Delta_0$-separated from the vacuum.

\end{enumerate}

\end{theorem}

\begin{proof}
\input{proofT3TheoremFreeYangMillsMassGap}
\end{proof}

% =========================================================================
% MOLLIFIED FIELD OPERATOR CONVERGENCE
% =========================================================================

\begin{lemma}[Mollified Field Operator Convergence in Strong Operator Topology]
\label{lem:mollifiedFieldOperatorConvergence}

For creation/annihilation operators $\{a_n, a_n^\dagger\}_{n=1}^\infty$ with canonical commutation relations $[a_n, a_m^\dagger] = \delta_{nm}$, define the mollified gauge field:
\begin{equation}
\widehat{\mathcal{A}}_\mu^\epsilon(x, t) := \sum_{n=1}^\infty \rho_\epsilon(\lambda_n) (a_n e^{-i\omega_n t} + a_n^\dagger e^{i\omega_n t}) \phi_n(x),
\end{equation}
where $\rho_\epsilon(\lambda) = \exp(-\epsilon \lambda/2)$ and $\omega_n = \sqrt{\lambda_n}$.

For any finite-particle state $|\psi\rangle \in \bigcup_{N=0}^\infty \mathcal{F}_N(\mathcal{H})$ (finite particle number), the limit
\begin{equation}
\lim_{\epsilon \to 0^+} \widehat{\mathcal{A}}_\mu^\epsilon(x, t) |\psi\rangle =: \widehat{\mathcal{A}}_\mu(x, t) |\psi\rangle
\end{equation}
exists in norm, and the convergence is uniform in time $t$ on any finite interval $[0, T]$.

The limiting operator $\widehat{\mathcal{A}}_\mu(x, t)$ defines a tempered distribution on test functions $f \in \mathcal{S}(\mathbb{R}^4)$ in the Schwartz sense.

\begin{proof}
\input{proofT3LemmaMollifiedFieldOperatorConvergence}
\end{proof}

\end{lemma}

% =========================================================================
% FIELD OPERATORS
% =========================================================================

\begin{theorem}[Yang-Mills Field Operators and Tempered Distributions]
\label{thm:yangMillsFieldOperators}

The electric and magnetic fields can be represented as operator-valued distributions:
\begin{equation}
\widehat{\mathcal{E}}_i(f), \quad \widehat{\mathcal{B}}_i(f) \quad \text{for } f \in \mathcal{S}(\mathbb{R}^4),
\end{equation}
satisfying:

\begin{enumerate}

\item \textbf{Tempered Distribution Property:} For $f \in \mathcal{S}(\mathbb{R}^4)$ (Schwartz space),
\begin{equation}
f \mapsto \widehat{\mathcal{E}}_i(f) \quad \text{is continuous in Schwartz topology}.
\end{equation}

\item \textbf{Commutation Relations:} Canonical commutation relations are satisfied:
\begin{equation}
[\widehat{\mathcal{E}}_i(\mathbf{x}), \widehat{\mathcal{B}}_j(\mathbf{y})] = i \epsilon_{ijk} \partial_k \delta(\mathbf{x} - \mathbf{y}) \cdot \mathbb{I}.
\end{equation}

\item \textbf{Gauge Covariance:} Under gauge transformations $\mathcal{A} \to g^{-1} \mathcal{A} g + g^{-1} \partial g$,
\begin{equation}
\widehat{\mathcal{A}}_\mu \to g^{-1} \widehat{\mathcal{A}}_\mu g + g^{-1} \partial_\mu g.
\end{equation}

\end{enumerate}

\end{theorem}

\begin{proof}
\input{proofT3TheoremYangMillsFieldOperators}
\end{proof}

% =========================================================================
% FUNCTIONAL INTEGRAL AND \cite{osterwalderSchrader1973axioms} AXIOMS
% =========================================================================

\input{proofT3TheoremOSPositivityBargMeasure}

\begin{theorem}[Yang-Mills Functional Integral and \cite{osterwalderSchrader1973axioms} Axioms]
\label{thm:yangMillsFunctionalIntegral}

The Yang-Mills theory constructed above satisfies all five \cite{osterwalderSchrader1973axioms} (OS) axioms for Euclidean QFT (equivalently, Wightman axioms for Lorentzian theory via analytic continuation):

\begin{enumerate}

\item \textbf{OS0 (Hilbert Space):} $\mathcal{H}_{\text{YM}}$ is a separable Hilbert space (proven in Theorem \ref{thm:yangMillsHilbertStructure}).

\item \textbf{OS1 (Relativistic Covariance):} The field transforms covariantly under the Poincaré group via the unitary representation $U(a, \Lambda)$.

\item \textbf{OS2 (Cyclicity and Uniqueness):} The vacuum state $|0\rangle$ is cyclic, meaning $\{\phi(f)|0\rangle : f \in \mathcal{S}(\mathbb{R}^4)\}$ is dense in $\mathcal{H}_{\text{YM}}$, and the vacuum is unique.

\item \textbf{OS3 (Locality/Commutativity):} Fields at spacelike-separated points commute:
\begin{equation}
[\widehat{\mathcal{A}}_\mu(\mathbf{x}), \widehat{\mathcal{A}}_\nu(\mathbf{y})] = 0 \quad \text{for } (\mathbf{x} - \mathbf{y})^2 < 0.
\end{equation}

\item \textbf{OS4 (Spectral Condition):} The energy spectrum is bounded below and contained in the forward light cone (proven in Theorem \ref{thm:yangMillsHilbertStructure}).

\end{enumerate}

\end{theorem}

\begin{proof}
\input{proofT3TheoremYangMillsFunctionalIntegral}
\end{proof}

% =========================================================================
% PART 2: UNCONDITIONAL PROOF
% =========================================================================

